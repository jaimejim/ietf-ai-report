\documentclass{article}
\usepackage[utf8]{inputenc}
\usepackage{hyperref}
\usepackage{geometry}
\usepackage{graphicx}
\usepackage{lmodern} 
\geometry{a4paper, margin=1.25in}
\title{\textbf{IETF 121 Meeting Report}}
\author{}
\date{}
\begin{document}
\maketitle

\noindent\textit{Disclaimer: This report is generated by an AI system and may contain inaccuracies. It is intended for informational purposes only and should be verified against official sources.}


\tableofcontents
\newpage

\section{6lo Working Group (6lo)}

\subsection{Attendees}
\subsubsection{Overview}
The 6lo Working Group meeting was attended by representatives from prominent organizations such as Apple, Huawei Technologies, Juniper Networks, and the U.S. Department of Defense, among others. The total attendance was approximately 40 participants, reflecting a diverse mix of academia, industry, and independent contributors.

\subsection{Meeting Discussions}

\subsubsection{Introduction and Draft Status}
Chairs Shwetha Bhandari and Carles Gomez opened the session with an overview of the agenda and the status of current drafts. The session included no significant comments on the agenda or working group drafts.

\subsubsection{IPv6 Neighbor Discovery Prefix Registration}
Pascal Thubert presented the \href{https://datatracker.ietf.org/doc/html/draft-ietf-6lo-prefix-registration-04}{draft-ietf-6lo-prefix-registration-04}. The discussion centered around the role of Neighbor Discovery (ND) in prefix allocation, with participants debating its overlap with DHCP functionalities. Concerns were raised about the necessity of multiple protocols achieving similar outcomes, highlighting the need for clarity in the draft regarding its unique use case.

\subsubsection{Path-Aware Semantic Addressing for LLNs}
Luigi Iannone discussed the \href{https://datatracker.ietf.org/doc/html/draft-ietf-6lo-path-aware-semantic-addressing-08}{draft-ietf-6lo-path-aware-semantic-addressing-08}. The presentation was well-received with no further comments, indicating consensus on the proposed approach.

\subsubsection{Generic Address Assignment Option for 6LoWPAN ND}
The \href{https://datatracker.ietf.org/doc/html/draft-ietf-6lo-nd-gaao-00}{draft-ietf-6lo-nd-gaao-00} was debated, focusing on its energy efficiency compared to DHCPv6. Participants discussed the implications of defining new protocols versus reusing existing ones, with a consensus on the need for further data to support claims of energy efficiency.

\subsubsection{Transmission of SCHC-compressed Packets over IEEE 802.15.4}
Carles Gomez presented the \href{https://datatracker.ietf.org/doc/html/draft-ietf-6lo-schc-15dot4-07}{draft-ietf-6lo-schc-15dot4-07}. The draft received no additional comments, suggesting broad agreement with the proposed transmission method.

\subsubsection{Transmission of IPv6 Packets over Short-Range OWC}
Younghwan Choi introduced the \href{https://datatracker.ietf.org/doc/html/draft-ietf-6lo-owc-02}{draft-ietf-6lo-owc-02}, with discussions on the need for signaling optional compression. The session concluded with plans for further collaboration with IEEE.

\subsubsection{Milestones and Charter Discussion}
The meeting concluded with a discussion on milestones and potential updates to the working group's charter. The chairs emphasized the importance of aligning future work with the evolving landscape of 6lo technologies, with plans to continue discussions on the mailing list.

Meeting materials can be accessed \href{https://example.com/meeting-materials}{here}.

\newpage

\section{6MAN Working Group (6MAN)}

\subsection{Attendees Overview}
\subsubsection{Prominent Companies and Institutions}
The meeting was attended by representatives from prominent companies and institutions such as Huawei Technologies, Google, Cisco Systems, Ericsson, and the University of Liege. The total attendance was approximately 80 participants.

\subsection{Meeting Discussions}

\subsubsection{The IPv6 VPN Service Destination Option}
The presentation by Ron Bonica focused on the \href{https://datatracker.ietf.org/doc/html/draft-ietf-6man-vpn-dest-opt}{draft-ietf-6man-vpn-dest-opt}, which proposes a simple method for conveying service information from ingress to egress without requiring changes to the control plane. The discussion highlighted the experimental nature of the draft and its potential simplicity compared to SRv6, with feedback requested on the security considerations.

\subsubsection{Deprecation of The IPv6 Router Alert Option}
Ron Bonica also discussed the \href{https://datatracker.ietf.org/doc/html/draft-ietf-6man-deprecate-router-alert}{draft-ietf-6man-deprecate-router-alert}, which aims to phase out the IPv6 Router Alert Option. The conversation centered around the implications for existing protocols and the potential for inter-domain issues, with the consensus leaning towards a gradual deprecation.

\subsubsection{Improving SLAAC Robustness to Flash Renumbering Events}
Richard Patterson presented the \href{https://datatracker.ietf.org/doc/html/draft-ietf-6man-slaac-renum}{draft-ietf-6man-slaac-renum}, which seeks to enhance the robustness of Stateless Address Autoconfiguration (SLAAC) against flash renumbering events. The discussion emphasized the need for clearer guidelines on configuration changes and the importance of addressing potential security vulnerabilities.

\subsubsection{IPv6 Node Requirements}
Tim Winters introduced the \href{https://datatracker.ietf.org/doc/html/draft-clw-6man-rfc8504-bis}{draft-clw-6man-rfc8504-bis}, which updates IPv6 node requirements. Key points included the necessity of clarifying multihoming scenarios and the potential shift towards making DHCPv6 a mandatory requirement, reflecting the evolving needs of enterprise networks.

\subsubsection{Internet Control Message Protocol (ICMPv6) Reflection}
Ron Bonica's presentation on \href{https://datatracker.ietf.org/doc/html/draft-mh-6man-icmpv6-reflection}{draft-mh-6man-icmpv6-reflection} explored the utility of ICMPv6 reflection for debugging NATs. The session raised concerns about security implications and the broader impact on NAT deployment strategies.

\subsection{Conclusion and Next Steps}
The meeting concluded with a consensus on several drafts moving forward to Working Group Last Call (WGLC), particularly those addressing SLAAC robustness and IPv6 node requirements. The discussions underscored the importance of aligning technical strategies with emerging network challenges, with a focus on enhancing security and operational efficiency.

Meeting materials are available at \href{https://datatracker.ietf.org/meeting/121/materials/agenda-121-6man-00}{IETF 121 6MAN Meeting Materials}.



\newpage

\section{Authentication and Authorization for Constrained Environments (ACE) [ACE]}

\subsection{Attendees}

\subsubsection{Overview}
The ACE working group meeting was attended by representatives from prominent organizations such as DigiCert, Inria, Uni Bremen TZI, JHU/APL, Ericsson, Cisco Systems, and the U.S. Department of Defense, among others. The total attendance was approximately 25 participants.


\subsection{Meeting Discussions}

\subsubsection{Update from Chairs}
The meeting commenced with an update from the chairs, highlighting the recent publication of RFC 9594, titled \href{https://datatracker.ietf.org/doc/html/draft-ietf-ace-key-groupcomm}{draft-ietf-ace-key-groupcomm}, which focuses on key provisioning for group communication using ACE. The chairs also noted the need for a shepherd writeup for \href{https://datatracker.ietf.org/doc/html/draft-ietf-ace-oscore-gm-admin}{draft-ietf-ace-oscore-gm-admin}.

\subsubsection{WG Documents}
The session included presentations on several working group documents. The \href{https://datatracker.ietf.org/doc/html/draft-ietf-ace-coap-est-oscore}{draft-ietf-ace-coap-est-oscore} was discussed, with updates based on recent reviews and ongoing GitHub issues. The \href{https://datatracker.ietf.org/doc/html/draft-ietf-ace-edhoc-oscore-profile}{draft-ietf-ace-edhoc-oscore-profile} presentation focused on clarifications and next steps, including the encoding of access token requests and responses as CBOR. The \href{https://datatracker.ietf.org/doc/html/draft-ietf-ace-group-oscore-profile}{draft-ietf-ace-group-oscore-profile} discussion emphasized fine-grained access control and the prevention of ambiguous situations in group memberships.

\subsubsection{For Adoption}
The draft \href{https://datatracker.ietf.org/doc/html/draft-tiloca-ace-authcred-dtls-profile}{draft-tiloca-ace-authcred-dtls-profile} was proposed for adoption, aiming to extend the DTLS profile to support alternative formats for authentication credentials. The document is ready for a working group last call.

\subsubsection{AoB}
In the Any Other Business section, Francesca Palombini discussed the publication status of related documents and the implementation of feedback-driven changes.

Meeting materials, including slides and detailed examples, are available at \href{https://datatracker.ietf.org/meeting/121/materials/}{IETF 121 Meeting Materials}.




\newpage

\section{Automated Certificate Management Environment (ACME)}

\subsection{Attendees}

The ACME working group meeting was attended by representatives from prominent companies and institutions, including Nokia, Dell Technologies, Cloudflare, Microsoft, and Google, among others. The total attendance was approximately 60 participants, reflecting a strong interest in the ongoing developments within the ACME working group.

\subsection{Meeting Discussions}

\subsubsection{Document Status}

The chairs provided an update on the status of various documents. The \href{https://datatracker.ietf.org/doc/html/draft-ietf-acme-ari}{draft-ietf-acme-ari} is awaiting publication, while \href{https://datatracker.ietf.org/doc/html/draft-ietf-acme-onion}{draft-ietf-acme-onion} has requested publication. The \href{https://datatracker.ietf.org/doc/html/draft-ietf-acme-device-attestation}{acme device attestation} document has a new version -03 as of August. Discussions also touched on the \href{https://datatracker.ietf.org/doc/html/draft-ietf-acme-scope-dns-challenge}{acme scope DNS challenge}.

\subsubsection{Work Items}

\paragraph{draft-ietf-acme-ari and draft-aaron-acme-profiles (Gable - Remote)}

The presentation for \href{https://datatracker.ietf.org/doc/html/draft-ietf-acme-ari}{draft-ietf-acme-ari} and \href{https://datatracker.ietf.org/doc/html/draft-aaron-acme-profiles}{draft-aaron-acme-profiles} was postponed due to the absence of the presenter.

\paragraph{draft-ietf-acme-dtnnodeid (Sipos - Remote)}

A discussion ensued regarding the naming of the \href{https://datatracker.ietf.org/doc/html/draft-ietf-acme-dtnnodeid}{draft-ietf-acme-dtnnodeid}, with suggestions like bp-nodeid and dtn-nodenid being considered. A short Working Group Last Call (WGLC) will follow the name change to gather any objections.

\paragraph{draft-geng-acme-public-key-00 (Liang Xia - Onsite)}

The presentation of \href{https://datatracker.ietf.org/doc/html/draft-geng-acme-public-key-00}{draft-geng-acme-public-key-00} raised questions about its readiness for Working Group adoption. The presenter expressed a desire for further feedback from the group before proceeding.

\subsubsection{Potential Work Item}

\paragraph{draft-liu-acme-rats (Peter Liu - Onsite)}

The potential work item \href{https://datatracker.ietf.org/doc/html/draft-liu-acme-rats}{draft-liu-acme-rats} was introduced, with initial ideas being discussed by Mike O and MCR. A draft is expected to be published soon to elaborate on these concepts.

\subsubsection{AOB}

The meeting concluded with an open discussion period, allowing participants to raise additional topics or concerns. This segment facilitated a broader exchange of ideas and potential future directions for the working group.

Meeting materials can be accessed at \href{https://meetecho.ietf.org/lite/?group=acme}{Meetecho} and \href{https://notes.ietf.org/notes-ietf-121-acme}{Notes}.




\newpage

\section{Adaptive DNS Discovery (ADD)}

\subsection{Attendees}
\subsubsection{Overview}
The ADD working group meeting was attended by representatives from prominent companies and institutions, including Cisco, Comcast-NBCUniversal, Meta Platforms, Inc., Verisign, and Apple. The total attendance was over 60 participants, reflecting a diverse range of stakeholders in the DNS community.

\subsection{Meeting Discussions}

\subsubsection{Handling Encrypted DNS Server Redirection}
Tommy Jensen presented the \href{https://datatracker.ietf.org/doc/html/draft-ietf-add-encrypted-dns-server-redirection}{draft-ietf-add-encrypted-dns-server-redirection}, discussing options for handling encrypted DNS server redirection. The group considered three options: doing nothing, adding a simple flag, or incorporating a key with data. The discussion highlighted the need for simplicity and clarity, with some participants advocating for a straightforward flag to facilitate implementation.

\subsubsection{DNS Resolver Information Key for DNSSEC Validation}
Stephane Bortzmeyer introduced the \href{https://datatracker.ietf.org/doc/html/draft-bortzmeyer-resinfo-dnssecval}{draft-bortzmeyer-resinfo-dnssecval}. The conversation centered on the necessity of a clear specification for DNSSEC validation keys. Participants debated the merits of a simple flag versus more detailed information, with a consensus leaning towards a minimalistic approach to expedite deployment.

\subsubsection{DNS Resolver Information Key for DNS64}
Florian Obser discussed the \href{https://datatracker.ietf.org/doc/html/draft-fobser-resinfo-dns64}{draft-fobser-resinfo-dns64}. The group acknowledged the importance of pushing forward with DNS64 information keys, emphasizing the potential benefits for network operators and end-users.

\subsubsection{Architectural Directions Discussion}
The session concluded with a discussion on architectural directions, particularly focusing on the interim work regarding encrypted DNS forwarders. The group recognized the need for continued exploration outside the working group to refine the approach before formal adoption as a working group draft.

Meeting materials and further details can be accessed via the \href{https://datatracker.ietf.org/meeting/121/session/add}{meeting materials link}.




\newpage

\section{ALLDISPATCH Hybrid Meeting @ IETF 121 (ALLDISPATCH)}

\subsection{Attendees}
\subsubsection{Overview}
The ALLDISPATCH meeting was attended by representatives from prominent companies and institutions such as Google, Cisco, Cloudflare, and Fastmail, with a total attendance of over 200 participants. The diverse group included experts from academia, industry, and government agencies, reflecting a wide range of interests and expertise.

\subsection{Meeting Discussions}

\subsubsection{Standards Processes}
Presenter: Rich Salz (onsite) \\
The discussion focused on updating the \href{https://datatracker.ietf.org/doc/html/draft-rsalz-2418bis}{IETF Working Group Guidelines and Procedures} and \href{https://datatracker.ietf.org/doc/html/draft-rsalz-2026bis}{The Internet Standards Process}. The consensus was to charter a focused working group to address outdated elements and consider broader changes, emphasizing community participation before finalizing updates.

\subsubsection{The IETF Chair May Delegate}
Presenter: Lars Eggert (onsite) \\
Lars Eggert proposed a draft on delegation by the IETF Chair, \href{https://datatracker.ietf.org/doc/html/draft-eggert-ietf-chair-may-delegate}{The IETF Chair May Delegate}. The discussion highlighted the need for a working group to explore implications and interactions with other roles, suggesting a potential combination with the standards processes update.

\subsubsection{High Assurance DIDs with DNS}
Presenter: Jesse Carter (remote) \\
The draft \href{https://datatracker.ietf.org/doc/html/draft-carter-high-assurance-dids-with-dns}{High Assurance DIDs with DNS} was deemed more suitable for W3C, given its focus on DIDs and resolution methods, which align better with W3C's scope.

\subsubsection{ALFA 2.0 - The Abbreviated Language for Authorization}
Presenter: Theo Dimitrakos (onsite) \\
The proposal for \href{https://datatracker.ietf.org/doc/html/draft-brossard-alfa-authz}{ALFA 2.0} led to suggestions for a Birds of a Feather (BoF) session to engage the community and explore potential integration with OAuth or other relevant groups.

\subsubsection{Identifying and Authenticating Home Servers: Requirements and Solution Analysis}
Presenter: Dan Wing (remote) \\
The draft \href{https://datatracker.ietf.org/doc/html/draft-rbw-home-servers-00}{Identifying and Authenticating Home Servers} was proposed for a BoF to further explore the problem space, with considerations for IRTF involvement due to its research-oriented nature.

\subsubsection{Update IDMEFv1}
Presenter: Gilles Lehmann (remote) \\
The update to \href{https://datatracker.ietf.org/doc/html/draft-lehmann-idmefv2}{IDMEFv2} was discussed with concerns about its adoption. The proposal was to pursue AD sponsorship contingent on demonstrating significant demand and addressing competition from existing frameworks.

\subsubsection{Two Secevent Drafts}
Presenter: Aaron Parecki (onsite) \\
The drafts on \href{https://datatracker.ietf.org/doc/html/draft-deshpande-secevent-http-multi-push}{Multi-Push-Based Security Event Token Delivery} and \href{https://datatracker.ietf.org/doc/html/draft-tulshibagwale-saag-pushpull-delivery}{Push And Pull Based Security Event Token Delivery} led to the decision to reopen the secevent working group to address unresolved issues and involve the HTTP directorate.

\subsubsection{A File Format to Aid in Consumer Privacy Enforcement, Research, and Tools}
Presenter: Louise Van der Peet (onsite) \\
The draft \href{https://datatracker.ietf.org/doc/html/draft-colwell-privacy-txt}{A File Format to Aid in Consumer Privacy} prompted discussions on its practicality and potential overlap with W3C efforts. A non-working group mailing list was suggested for further exploration.

\subsubsection{DKIM2: Why DKIM Needs Replacing}
Presenter: Bron Gondwana (onsite) \\
Time constraints limited the discussion on \href{https://datatracker.ietf.org/doc/html/draft-gondwana-dkim2-motivation}{DKIM2}. It will be addressed in the mailmaint working group later in the week.

Meeting materials are available at \href{https://notes.ietf.org/notes-ietf-121-alldispatch}{Notes IETF-121 ALLDISPATCH}.



\newpage

\section{ASDF Working Group (ASDF)}

\subsection{Attendees}
\subsubsection{Overview}
The ASDF Working Group meeting was attended by representatives from prominent organizations such as Ericsson, Cisco Systems, and ETRI, among others. In total, there were 24 participants via Meetecho and 12 in the room, highlighting a strong interest and engagement from the industry and academia.

\subsection{Meeting Discussions}

\subsubsection{WG Status Update}
The meeting commenced with a status update on the Working Group's progress. The IETF Last Call has been completed, but the IESG review is still pending. This update set the stage for further discussions on the group's ongoing projects and future directions.

\subsubsection{SDF NonAffordance Discussion}
The discussion on SDF NonAffordance was led by MJK, focusing on the integration of non-affordance information and instance graphs into the SDF framework. The conversation explored the potential for transforming SDF to RDF, considering the need for relations and hyperlinks. The group deliberated on the concept of "declared properties" and the necessity of a registry for keywords, aiming to enhance the SDF's applicability in digital twin scenarios. For more details, refer to the \href{https://datatracker.ietf.org/doc/html/draft-ietf-asdf-sdf}{draft-ietf-asdf-sdf}.

\subsubsection{NIPC Hackathon Report and 03 Update}
BB presented the hackathon report, highlighting the translation of Matter/SDF models and the integration of NIPC with SDF. The discussion emphasized the need for more examples in the \href{https://datatracker.ietf.org/doc/html/draft-ietf-asdf-nipc-03}{draft-ietf-asdf-nipc-03} and considered the implications of using SDF files for property and event registration. The group acknowledged the potential for further hackathon activities to refine the NIPC framework and discussed the possibility of renaming NIPC to better reflect its scope.

\subsubsection{Future Meeting Plans}
The group discussed plans for future meetings, including potential design team meetings and an interim meeting in mid-January. The next IETF meeting in Bangkok was also considered, with discussions on aligning meeting times to accommodate global participants.

\subsubsection{Any Other Business}
EL presented a scenario involving CoAP over NIPC over BLE, illustrating the complexities of spontaneous registration and message tunneling. The group considered alternative protocols such as HTTP long-poll or WebSockets to streamline the process, emphasizing the need for further discussion and exploration of efficient communication models.

Meeting materials and additional notes can be accessed at \href{https://notes.ietf.org/notes-ietf-121-asdf}{this link}.



\newpage

\section{Audio/Video Transport Core Maintenance (avtcore) Working Group}

\subsection{Attendees}
\subsubsection{Overview}
The meeting was attended by representatives from prominent companies and institutions including Microsoft, Google, Nokia, Ericsson, and Netflix, with a total attendance of over 40 participants. Key attendees included Bernard Aboba from Microsoft, Jonathan Lennox from 8x8/Jitsi, and Harald Alvestrand from Google.

\subsection{Meeting Discussions}

\subsubsection{Preliminaries}
The session began with a review of the meeting protocols and the status of current drafts. Notably, three documents are in the AUTH48 stage, aiming to align language between frame-marking and VP9 payload formats. The RTP payload format for SFRAME has expired, and contributors are needed to revive it.

\subsubsection{HEVC Profile for WebRTC}
Bernard Aboba presented on the \href{https://datatracker.ietf.org/doc/html/draft-ietf-avtcore-hevc-webrtc}{draft-ietf-avtcore-hevc-webrtc}. Discussions focused on the level-ID asymmetry limitation in video payload formats, with suggestions to use sendonly and recvonly m-lines as a potential fix. Bernard plans to draft a proposal based on the feedback received.

\subsubsection{RTP over QUIC}
The \href{https://datatracker.ietf.org/doc/html/draft-ietf-avtcore-rtp-over-quic}{draft-ietf-avtcore-rtp-over-quic} was discussed by Mathis Engelbart, Joerg Ott, and Spencer Dawkins. Recent updates include RTCP clarifications and progress in interoperability testing. Further work on SDP by Spencer and Victor is ongoing.

\subsubsection{RTP Payload for V-DMC}
Hyunsik Yang introduced the \href{https://datatracker.ietf.org/doc/html/draft-hsyang-avtcore-rtp-vdmc}{draft-hsyang-avtcore-rtp-vdmc}, highlighting the need for a new RTP payload format for V-DMC. Discussions centered on synchronization issues and the necessity of certain parameters, which will be revisited in future updates.

\subsubsection{Absolute Capture Time RTP Header Extension}
Harald Alvestrand presented the \href{https://datatracker.ietf.org/doc/html/draft-alvestrand-avtcore-abs-capture-time}{draft-alvestrand-avtcore-abs-capture-time}. The discussion revolved around the utility of the proposal and its implications for RTP middleboxes. An update is planned, with a working group adoption call to follow.

\subsubsection{SDP Fingerprints for Raw Public Keys in (D)TLS}
Jonathan Lennox discussed the \href{https://datatracker.ietf.org/doc/html/draft-lennox-sdp-raw-key-fingerprints}{draft-lennox-sdp-raw-key-fingerprints}, focusing on challenges with post-quantum ciphersuites and the potential need for a new SDP attribute. The discussion highlighted the need for a dedicated working group for SDP maintenance.

\subsubsection{Wrapup and Next Steps}
The session concluded with a summary of action items and next steps. Key outcomes include drafting proposals for HEVC profile issues, continuing interoperability testing for RTP over QUIC, and updating the Absolute Capture Time draft. The group will also explore forming a working group for SDP maintenance.

Meeting materials are available at \href{https://docs.google.com/presentation/d/1bLGLkbxmi4w8Od5vUU-YTRREF8sEBJP0XBJOTdm4-T8/}{Google Slides}.




\newpage

\section{BGP Enabled Services (BESS) [BESS]}

\subsection{Attendees}
\subsubsection{Overview}
The BESS working group meeting was attended by representatives from prominent companies and institutions such as Cisco, Nokia, Huawei, and Juniper Networks, with a total attendance of over 70 participants. Notable attendees included Mankamana Mishra from Cisco, Richard Foote from Nokia, and Shunwan Zhuang from Huawei.

\subsection{Meeting Discussions}

\subsubsection{Working Group Draft Updates}
The session began with updates on several working group drafts. The \href{https://datatracker.ietf.org/doc/html/draft-ietf-bess-evpn-vpws-seamless-01}{draft-ietf-bess-evpn-vpws-seamless-01} was highlighted as ready for last call, with Ali Sajassi noting its overdue status. Discussions on the \href{https://datatracker.ietf.org/doc/html/draft-ietf-bess-weighted-hrw-01}{draft-ietf-bess-weighted-hrw-01} suggested minor modifications to enhance detail. The \href{https://datatracker.ietf.org/doc/html/draft-ietf-bess-secure-evpn-01}{draft-ietf-bess-secure-evpn-01} sparked debate over data plane modifications, with a consensus to merge efforts with NVO3.

\subsubsection{Individual Drafts}
Ali Sajassi's \href{https://datatracker.ietf.org/doc/html/draft-sajassi-bess-rfc8317bis-02}{draft-sajassi-bess-rfc8317bis-02} was deemed ready for adoption, with Jeffrey Haas urging email communication for further engagement. The \href{https://datatracker.ietf.org/doc/html/draft-lrss-bess-evpn-group-policy-01}{draft-lrss-bess-evpn-group-policy-01} prompted discussions on visibility within routing workgroups, with suggestions to expedite the 7348 bis process. The \href{https://datatracker.ietf.org/doc/html/draft-rabnag-bess-evpn-anycast-aliasing-02}{draft-rabnag-bess-evpn-anycast-aliasing-02} was noted for its proprietary implementations, with calls for operational considerations.

\subsubsection{Prospective Actions}
The meeting concluded with a focus on merging drafts and preparing them for adoption calls. The \href{https://datatracker.ietf.org/doc/html/draft-burdet-bess-evpn-fast-reroute-08}{draft-burdet-bess-evpn-fast-reroute-08} is set to merge with another draft, while the \href{https://datatracker.ietf.org/doc/html/draft-sajassi-bess-evpn-first-hop-security-03}{draft-sajassi-bess-evpn-first-hop-security-03} and \href{https://datatracker.ietf.org/doc/html/draft-sajassi-bess-evpn-umr-mobility-02}{draft-sajassi-bess-evpn-umr-mobility-02} were both ready for adoption. The discussions underscored the need for strategic planning in integrating MPLS and SRv6 capabilities.

Meeting materials can be accessed at \href{https://www.ietf.org/proceedings/121/bess.html}{IETF 121 BESS Meeting Materials}.



\newpage

\section{Benchmarking Methodology Working Group (bmwg)}

\subsection{Attendees}

The meeting was attended by representatives from prominent companies and institutions such as Cisco Systems, Huawei Technologies, Telefonica Innovacion Digital, and ETH Zürich, with a total attendance of 25 participants.

\subsection{Meeting Discussions}

\subsubsection{Multiple Loss Ratio Search}

Maciek Konstantynowicz and Vratko Polak presented updates on the \href{https://datatracker.ietf.org/doc/html/draft-ietf-bmwg-mlrsearch}{draft-ietf-bmwg-mlrsearch}. The authors highlighted significant changes and examples, indicating readiness for WG Last Call. Gabor Lencse volunteered to review the draft, requesting additional time.

\subsubsection{A YANG Data Model for Network Tester Management}

Vladimir Vassilev discussed the \href{https://datatracker.ietf.org/doc/html/draft-ietf-bmwg-network-tester-cfg}{draft-ietf-bmwg-network-tester-cfg}, focusing on hackathon results and draft updates. The model's flexibility to accommodate future RFCs was affirmed, with ongoing efforts to integrate with traffic generators like T-REX.

\subsubsection{Considerations for Benchmarking Network Performance in Containerized Infrastructures}

Minh-Ngoc Tran presented the \href{https://datatracker.ietf.org/doc/html/draft-ietf-bmwg-containerized-infra}{draft-ietf-bmwg-containerized-infra}, receiving feedback from RFC 8204 authors. The authors are preparing for a Working Group Last Call.

\subsubsection{Benchmarking Methodology for Segment Routing}

Paolo Volpato updated on the \href{https://datatracker.ietf.org/doc/html/draft-ietf-bmwg-sr-bench-meth}{draft-ietf-bmwg-sr-bench-meth}, with discussions on security and public test applications. Feedback from Carsten Rossenhoevel is anticipated.

\subsubsection{Recommendations for Using Multiple IP Addresses in Benchmarking Tests}

Gabor Lencse introduced the \href{https://datatracker.ietf.org/doc/html/draft-lencse-bmwg-multiple-ip-addresses}{draft-lencse-bmwg-multiple-ip-addresses}, seeking WG adoption. Feedback and reviews are encouraged, with Boris Khasanov volunteering for review.

\subsubsection{SRv6 Service Benchmarking Guideline}

Xuesong Geng presented the \href{https://datatracker.ietf.org/doc/html/draft-geng-bmwg-srv6-service-guideline}{draft-geng-bmwg-srv6-service-guideline}, incorporating feedback from SRv6OPS and adding a new co-author. The draft focuses on lab-based testing.

\subsubsection{Benchmarking Methodology for Source Address Validation}

Libin Liu discussed updates to the \href{https://datatracker.ietf.org/doc/html/draft-chen-bmwg-savnet-sav-benchmarking}{draft-chen-bmwg-savnet-sav-benchmarking}, with plans for a hackathon at the next IETF meeting. The draft is directed towards BMWG discussions.

\subsubsection{Calibration of Measured Time Values between Network Elements}

Luis M. Contreras introduced the \href{https://datatracker.ietf.org/doc/html/draft-contreras-bmwg-calibration}{draft-contreras-bmwg-calibration}, exploring its application in traffic engineering without bias. Discussions on potential integration with hybrid methods like IOAM are ongoing.

\subsubsection{Characterization and Benchmarking Methodology for Power in Networking Devices}

Romain Jacob presented the \href{https://datatracker.ietf.org/doc/html/draft-cprjgf-bmwg-powerbench}{draft-cprjgf-bmwg-powerbench}, emphasizing the need for power benchmarking in networking devices. The draft will be updated for the next meeting, with offline comments welcomed.

Meeting materials are available at \href{https://datatracker.ietf.org/meeting/121/materials/slides-121-bmwg-ietf-121-bmwg-chairs-slides-00}{this link}.



\newpage

\section{Calendaring Extensions (CALEXT) [CALEXT]}

\subsection{Attendees}

\subsubsection{Overview}
The CALEXT working group meeting at IETF121 in Dublin was attended by representatives from prominent organizations such as Ericsson, Fastmail, Data Transfer Initiative, Bedework Commercial SVC, and DENIC eG, among others. The total attendance was approximately 20 participants, including key contributors like Daniel Migault from Ericsson and Robert Stepanek from Fastmail.

\subsection{Meeting Discussions}

\subsubsection{Tasks and Subscription Upgrade}
The discussion on the \href{https://datatracker.ietf.org/doc/html/draft-ietf-calext-ical-tasks}{ical-tasks} and subscription upgrade drafts concluded that both documents are ready for progression. Mike Douglass noted an issue regarding the percent complete metric, which will be addressed in a resubmitted draft. The group agreed that Mike will submit the revised documents shortly, and Bron Gondwana will initiate a new Working Group Last Call (WGLC) due to the elapsed time since the last review.

\subsubsection{iTip with Participants}
The iTip protocol discussion highlighted the need for mapping adjustments to accommodate participant properties, as detailed in the \href{https://datatracker.ietf.org/doc/html/draft-ietf-calext-itip-participant}{draft-ietf-calext-itip-participant}. The group emphasized the importance of maintaining compatibility with existing ORGANIZER/ATTENDEE structures while exploring extensions for multiple owners. The consensus was to prioritize this draft to facilitate references in related documents. Bron will call for adoption of this draft on the mailing list.

\subsubsection{JSCalendar/iCalendar}
Robert Stepanek presented on the JSCalendar and iCalendar interoperability, noting the cessation of efforts on jscalendarbis. The focus is now on ensuring successful data conversion and implementation experiments, with a potential last call at IETF122. The group discussed the possibility of relaxing charter constraints to improve data model mapping reliability. Orie Steele supported the idea of publishing an experimental RFC to encourage implementation and interop testing. Robert will lead further discussions on the mailing list to define success criteria and implementation milestones.

\subsection{Meeting Materials}
The meeting materials, including slides and draft documents, are available at \href{https://datatracker.ietf.org/meeting/121/materials/agenda-121-calext}{IETF121 CALEXT Materials}.

\subsection{Milestones}
The group briefly reviewed updated milestones but deferred additional business to future discussions due to time constraints.




\newpage

\section{Computing-Aware Traffic Steering (CATS) WG Agenda - IETF 121}

\subsection{Attendees Overview}

The meeting was attended by representatives from prominent companies and institutions such as Huawei, Cisco Systems, China Mobile, Ericsson, and Nokia, with a total attendance of over 70 participants. The diverse representation underscored the broad interest and collaborative effort in advancing computing-aware traffic steering solutions.

\subsection{Meeting Discussions}

\subsubsection{CATS Use Cases \& Requirements}

The session on CATS Use Cases and Requirements focused on consolidating existing use cases and clarifying the derivation of requirements. Key discussions included the need for a "forwarding considerations" section to prevent routing loops, as highlighted by Jeffrey Haas. The group also debated the inclusion of 5G edge use cases, with a poll indicating moderate support. The session emphasized the importance of aligning requirements with practical deployment scenarios, as detailed in the \href{https://datatracker.ietf.org/doc/html/draft-ietf-cats-usecases-requirements}{draft-ietf-cats-usecases-requirements}.

\subsubsection{CATS Framework}

Cheng Li presented the CATS Framework, which aims to provide a structured approach to integrating computing services with network operations. Discussions highlighted the potential for architectural adjustments to accommodate new use cases, such as on-path computing for content delivery. The framework's adaptability was seen as crucial for future-proofing CATS deployments, as outlined in the \href{https://datatracker.ietf.org/doc/html/draft-ietf-cats-framework}{draft-ietf-cats-framework}.

\subsubsection{CATS Metrics Discussion}

The metrics discussion, led by Kehan Yao, focused on defining and describing metrics that can effectively guide traffic steering decisions. The group explored the balance between generic and specific metrics, with an emphasis on ensuring compatibility with existing routing protocols. The session concluded with an action plan to refine metric definitions, as detailed in the \href{https://datatracker.ietf.org/doc/html/draft-ysl-cats-metric-definition}{draft-ysl-cats-metric-definition}.

\subsubsection{Flash Teasers}

The flash teaser session provided brief insights into ongoing work related to CATS security considerations and hierarchical loop prevention. These presentations, including \href{https://datatracker.ietf.org/doc/html/draft-wang-cats-security-considerations}{draft-wang-cats-security-considerations}, highlighted emerging challenges and potential solutions in the CATS domain.

\subsubsection{Open Discussion \& Next Steps}

The meeting concluded with an open discussion on the next steps, emphasizing the need for continued collaboration and refinement of the CATS framework and metrics. Participants were encouraged to contribute to ongoing drafts and discussions to ensure robust and scalable solutions.

Meeting materials and further details can be accessed at \href{https://datatracker.ietf.org/meeting/121/session/cats}{IETF 121 CATS Session}.




\newpage

\section{Concise Binary Object Representation (CBOR) Working Group [CBOR]}

\subsection{Attendees}

\subsubsection{Overview}

The CBOR session at IETF 121 was attended by representatives from notable organizations including SECOM CO., LTD., Futurewei Technologies, RISE Research Institutes of Sweden, Arm, Comcast, Verisign, and Cloudflare, among others. The total attendance was approximately 30 participants, reflecting a diverse mix of industry and academic stakeholders.

\subsection{Meeting Discussions}

\subsubsection{Intro and Agenda Review}

The session commenced with an introduction and agenda review led by Christian Amsüss. The agenda included updates on current documents and a decision to defer the discussion on \texttt{cddl-modules} to a future interim meeting due to time constraints.

\subsubsection{Document Status and Hackathon Report}

Christian Amsüss presented a report on the Hackathon activities, focusing on the implementation of Packed CBOR. The discussion highlighted the ease of implementing decompression in Python, while noting the absence of constrained implementations in C, which remains a future goal.

\subsubsection{edn-literals Discussion}

Carsten Bormann led a detailed discussion on \href{https://datatracker.ietf.org/doc/html/draft-ietf-cbor-edn-literals}{draft-ietf-cbor-edn-literals}, emphasizing the need for a diagnostic notation form for specific CBOR tags. The conversation explored the syntax for text and byte strings, with a focus on potential extensions and the integration of comments within prefixed strings.

\subsubsection{cbor-cde Presentation}

Laurence Lundblade presented on \href{https://datatracker.ietf.org/doc/html/draft-ietf-cbor-cde}{draft-ietf-cbor-cde}, discussing serialization options and the importance of deterministic encoding for test cases. The session underscored the need for application-level rules to ensure consistent encoding across implementations.

\subsubsection{dns-cbor Proposal}

Martine Lenders introduced the \href{https://datatracker.ietf.org/doc/draft-lenders-dns-cbor/}{draft-lenders-dns-cbor}, proposing the use of CBOR for DNS message compression. The presentation covered the benefits of CBOR-packed encoding and the potential for resource record set optimization. The working group considered the draft's adoption, with a consensus to involve DNSOP expertise.

\subsubsection{Interim Meeting Dates}

The group proposed interim meeting dates leading up to IETF 122, maintaining a bi-weekly cadence to continue discussions and document progress.

Meeting materials and slides are available at \href{https://datatracker.ietf.org/meeting/121/materials/slides-121-cbor-chairs-slides-01.pdf}{CBOR Session Materials}.

\subsection{Conclusion}

The CBOR session at IETF 121 facilitated significant progress on multiple drafts, with discussions pointing towards strategic shifts in document handling and encoding practices. The group aims to refine these drafts further in upcoming interim meetings, ensuring robust and interoperable CBOR implementations.



\newpage

\section{CCAMP Working Group (CCAMP)}

\subsection{Attendees}
\subsubsection{Overview}
The meeting was attended by representatives from prominent companies and institutions, including Huawei Technologies, Nokia, Ericsson, Cisco, and Telefonica Innovacion Digital, with a total attendance of over 40 participants. The diverse representation underscored the collaborative nature of the working group, fostering a rich exchange of ideas and expertise.

\subsection{Meeting Discussions}

\subsubsection{Administrivia and WG Status}
The session commenced with an update on the working group's status, milestones, and charter, presented by the chairs. Discussions highlighted the need for alignment with new contributors on the \href{https://datatracker.ietf.org/doc/html/draft-ietf-ccamp-tsvmode-signalling}{tsvmode-signalling draft} and the ongoing work on the \href{https://datatracker.ietf.org/doc/html/draft-ietf-ccamp-pluggable-usecase-gap-analysis}{pluggable-usecase-gap-analysis} draft, which is undergoing revisions to address issues raised during the adoption call.

\subsubsection{YANG Model for Optical Interface Parameters}
Gabriele Galimberti presented the \href{https://datatracker.ietf.org/doc/html/draft-ietf-ccamp-dwdm-if-param-yang}{YANG model for managing optical interface parameters}. The discussion focused on the model's stability and potential integration with pluggable attributes. The alignment with RFC9093bis and optical impairment models was noted, with suggestions for text realignment.

\subsubsection{WDM Tunnel YANG Data Model}
Aihua Guo introduced the \href{https://datatracker.ietf.org/doc/html/draft-ietf-ccamp-wdm-tunnel-yang}{YANG Data Model for WDM Tunnels}. The presentation was concise, with no additional discussion points raised, indicating consensus on the draft's current state.

\subsubsection{Modelling Optical Pluggables}
Reza Rokui discussed the \href{https://datatracker.ietf.org/doc/html/draft-rokui-ccamp-actn-wdm-pluggable-modelling}{modelling of optical pluggables}. The conversation revolved around the lifecycle of related documents and the logical separation between operator-driven use cases and vendor-driven artifacts. The need for converting Google Docs into draft text was emphasized to facilitate broader accessibility and collaboration.

\subsubsection{Fine Grain Optical Transport Network YANG Models}
Yanxia Tan presented the \href{https://datatracker.ietf.org/doc/html/draft-tan-ccamp-fgotn-yang}{YANG Data Models for fine grain Optical Transport Network}. The augmentation of topology and tunnel models was discussed, with suggestions to consolidate augmentations within a single draft to streamline documentation.

\subsubsection{GMPLS Applicability for Optical Transport Networks}
Yi Lin's presentation on the \href{https://datatracker.ietf.org/doc/html/draft-lin-ccamp-gmpls-fgotn-applicability}{applicability of GMPLS} for fine grain Optical Transport Networks was brief, with no further discussion, indicating a general agreement on the draft's direction.

\subsubsection{Performance Management Streaming YANG Model}
Bin Yeong Yoon introduced the \href{https://datatracker.ietf.org/doc/html/draft-yoon-ccamp-pm-streaming}{YANG data model for Performance Management Streaming}. The discussion highlighted the importance of differentiating PM collection rates and the potential for splitting the document into generic and technology-specific sections.

\subsubsection{AI-based Network Management Agent}
Xing Zhao presented the \href{https://datatracker.ietf.org/doc/html/draft-zhao-nmop-network-management-agent}{AI-based Network Management Agent} concepts. The dialogue focused on the integration of NMA with existing controllers and the potential for cross-technology applications. The need for further exploration of intent-based interfaces and alignment with existing frameworks was acknowledged.

Meeting materials are available at \href{https://datatracker.ietf.org/meeting/121/materials/agenda-121-ccamp}{IETF 121 CCAMP Meeting Materials}.



\newpage

\section{Congestion Control Working Group (CCWG)}

\subsection{Attendees}

\subsubsection{Overview}
The CCWG meeting was attended by representatives from prominent companies and institutions such as Netflix, Google, Apple, Ericsson, and Huawei, with a total attendance of 72 participants. The diverse representation underscored the broad interest and collaborative effort in advancing congestion control mechanisms.

\subsection{Meeting Discussions}

\subsubsection{Chair Slides}
The chairs provided updates on the rechartering process, emphasizing the use of GitHub for document collaboration. Discussions highlighted the need for representative test cases, particularly for environments like LEO satellites. Christian Huitema and Mohit Tahilani contributed insights on the importance of maintaining test suites and leveraging ns-3 simulations. The chairs encouraged community involvement in defining scenarios and maintaining test suites, emphasizing the need for real-world applicability.

\subsubsection{BBRv3}
Neal Cardwell presented the \href{https://datatracker.ietf.org/doc/html/draft-ietf-ccwg-bbr}{draft-ietf-ccwg-bbr}, focusing on community collaboration via GitHub for issue tracking and contributions. Key discussions revolved around potential changes to pacing gain and the integration of ECN. The community was encouraged to contribute through testing and feedback, with an emphasis on balancing innovation with real-world applicability.

\subsubsection{SEARCH: Slow-start Exit At Right CHokepoint}
Mark Claypool introduced the \href{https://datatracker.ietf.org/doc/html/draft-chung-ccwg-search}{draft-chung-ccwg-search}, proposing a new slow-start algorithm for TCP and QUIC. The discussion centered on optimizing the algorithm's efficiency and exploring modular slow-start exit conditions. The community expressed interest in testing and implementing the proposal, with a call for adoption initiated.

\subsubsection{Increase of the Congestion Window when the Sender is Rate-Limited}
Michael Welzl discussed the \href{https://datatracker.ietf.org/doc/html/draft-welzl-ccwg-ratelimited-increase}{draft-welzl-ccwg-ratelimited-increase}, seeking adoption for minor changes aimed at improving congestion window adjustments for rate-limited senders. The proposal received positive feedback, with plans for an adoption call.

\subsubsection{SCReAMv2}
Ingemar Johansson presented \href{https://datatracker.ietf.org/doc/html/draft-johansson-ccwg-rfc8298bis-screamv2}{draft-johansson-ccwg-rfc8298bis-screamv2}, focusing on media congestion control. While the draft garnered interest, discussions highlighted the need to assess the working group's capacity to handle multiple concurrent projects. The chairs encouraged ongoing contributions and discussions within the group.

\subsection{Conclusion}
The CCWG meeting underscored the collaborative spirit and technical depth of the discussions, with significant contributions from industry and academia. The meeting materials are available at \href{https://datatracker.ietf.org/meeting/121/materials/slides-121-ccwg}{IETF 121 CCWG Meeting Materials}. The group is poised to advance congestion control strategies, balancing innovation with practical deployment considerations.



\newpage

\section{Content Delivery Network Interconnection (CDNI) [CDNI]}

\subsection{Attendees}
\subsubsection{Overview}
The CDNI working group meeting was attended by representatives from prominent companies and institutions such as Verizon, Comcast, Ericsson, Broadpeak, and Huawei, totaling 30 participants. Notable attendees included Sanjay Mishra from Verizon, Chris Lemmons from Comcast, and Francesca Palombini from Ericsson.

\subsection{Meeting Discussions}

\subsubsection{Introductory Remarks}
The meeting commenced with the chairs, Kevin J. Ma, Chris Lemmons, and Sanjay Mishra, introducing the agenda. A key highlight was the announcement that the \href{https://datatracker.ietf.org/doc/html/rfc9677}{RFC9677} on CDNI Metadata for Delegated Credentials has been published.

\subsubsection{Document Status and Reviews}
Sanjay Mishra provided a status update on documents under IESG review, including the \href{https://datatracker.ietf.org/doc/html/draft-ietf-cdni-capacity-insights-extensions-10}{draft-ietf-cdni-capacity-insights-extensions-10}. The draft addresses open comments and awaits IESG Telechat review. Discussions emphasized the importance of addressing third-party dependencies in drafts, particularly those not based on open standards, such as HashiCorp Vault and AWSv4 auth.

\subsubsection{Rechartering Discussion}
A significant portion of the meeting was dedicated to discussing the potential rechartering of the working group to include new drafts like \href{https://datatracker.ietf.org/doc/html/draft-goldstein-processing-stages-metadata-02}{draft-goldstein-processing-stages-metadata-02} and \href{https://datatracker.ietf.org/doc/html/draft-power-metadata-expression-language-02}{draft-power-metadata-expression-language-02}. The group debated the scope of the charter, particularly regarding the inclusion of general-purpose templating languages and the creation of content.

\subsubsection{Working Group Draft Updates}
Jay Robertson and Alan Arolovitch presented updates on the \href{https://datatracker.ietf.org/doc/html/draft-ietf-cdni-ci-triggers-rfc8007bis-15}{draft-ietf-cdni-ci-triggers-rfc8007bis-15}, highlighting significant changes and the need for further reviews. The draft aims for a Working Group Last Call (WGLC) by IETF 122.

\subsubsection{Named Footprints and Edge Control Metadata}
Alan Arolovitch discussed the \href{https://datatracker.ietf.org/doc/html/draft-ietf-cdni-named-footprints-00}{draft-ietf-cdni-named-footprints-00}, proposing a RESTful interface for named footprints. Alfonso Silóniz presented updates on the \href{https://datatracker.ietf.org/doc/html/draft-ietf-cdni-edge-control-metadata-02}{draft-ietf-cdni-edge-control-metadata-02}, which is nearing readiness for WGLC.

\subsubsection{Open MIC Session}
The session included discussions on the CDNI Control Interface HTTP REST API, request routing interface extensions, and cache management interface extensions. Guillaume Bichot and Alan Arolovitch presented proposals for new REST APIs to enhance CDNI capabilities, emphasizing the need for comprehensive reviews and alignment with existing standards.

Meeting materials are available at \href{https://datatracker.ietf.org/meeting/121/agenda}{IETF 121 Agenda}.






\newpage

\section{Constrained RESTful Environments (CoRE) WG [CoRE]}

\subsection{Attendees}

\subsubsection{Overview}
The meeting was attended by representatives from prominent organizations such as Ericsson, Huawei Ireland Research Center, RISE Research Institutes of Sweden, and Tampere University, among others. In total, there were over 40 participants, reflecting a diverse range of expertise and interests in the field of constrained environments.

\subsection{Meeting Discussions}

\subsubsection{Intro, Agenda, Status}
The session began with an introduction and agenda overview by the chairs. Key updates included the status of various documents, such as the \href{https://datatracker.ietf.org/doc/html/draft-ietf-core-corr-clar}{draft-ietf-core-corr-clar}, which is a unique document type aimed at collecting corrections and clarifications. The group plans to hold interim meetings to process these issues, with a focus on amplification and 0-RTT topics.

\subsubsection{CORECONF}
Carsten Bormann presented on CORECONF, highlighting the progress on documents like \href{https://datatracker.ietf.org/doc/html/draft-ietf-core-comi}{draft-ietf-core-comi} and \href{https://datatracker.ietf.org/doc/html/draft-ietf-core-yang-library}{draft-ietf-core-yang-library}. The discussion centered on simplifying the management of YANG modules for constrained servers, with contributions from Zhuoyao Lin offering potential solutions to existing challenges.

\subsubsection{Constrained Resource Identifiers}
The presentation on \href{https://datatracker.ietf.org/doc/html/draft-ietf-core-href}{draft-ietf-core-href} addressed the ongoing implementation work and the handling of zone identifiers. The group considered maintaining zone identifiers in the model without specifying their URI model relationship, allowing for future developments.

\subsubsection{Conditional Attributes for Constrained RESTful Environments}
Bill Silverajan discussed the \href{https://datatracker.ietf.org/doc/html/draft-ietf-core-conditional-attributes}{draft-ietf-core-conditional-attributes}, which enhances CoAP observation by allowing clients to influence notification conditions. The document is nearing completion, with plans to address remaining editorial issues promptly.

\subsubsection{DNS over CoAP (DoC) "Bundle"}
Martine Lenders presented updates on the \href{https://datatracker.ietf.org/doc/html/draft-ietf-core-dns-over-coap}{draft-ietf-core-dns-over-coap} and related documents. The group is preparing for a Working Group Last Call, pending resolution of references to the \href{https://datatracker.ietf.org/doc/html/draft-ietf-core-corr-clar}{draft-ietf-core-corr-clar}.

\subsubsection{CoAP Transport Indication}
Christian Amsüss introduced the \href{https://datatracker.ietf.org/doc/html/draft-ietf-core-transport-indication}{draft-ietf-core-transport-indication}, which aims to standardize how different transports for CoAP are used. The discussion included the potential for a unified \_coap service and the handling of IP literals with extra data.

\subsubsection{A Publish-Subscribe Architecture for CoAP}
Jaime Jiménez presented the \href{https://datatracker.ietf.org/doc/html/draft-ietf-core-coap-pubsub}{draft-ietf-core-coap-pubsub}, which defines a publish-subscribe model for CoAP. The document is ready for a Working Group Last Call, with ongoing work in the ACE group to develop a compatible pub-sub profile.

\subsubsection{Group Communication for CoAP}
Esko Dijk discussed the \href{https://datatracker.ietf.org/doc/html/draft-ietf-core-groupcomm-bis}{draft-ietf-core-groupcomm-bis}, which updates the experimental RFC 7390. The document is ready for another Working Group Last Call, focusing on group communication features and proxy operations.

\subsubsection{Proxy Operations for CoAP Group Communication}
Marco Tiloca presented the \href{https://datatracker.ietf.org/doc/html/draft-ietf-core-groupcomm-proxy}{draft-ietf-core-groupcomm-proxy}, which defines proxy operations for CoAP group communication. The document includes updates on error handling and the use of structured field values for HTTP header fields.

\subsubsection{Key Update for OSCORE (KUDOS)}
Rikard Höglund discussed the \href{https://datatracker.ietf.org/doc/html/draft-ietf-core-oscore-key-update}{draft-ietf-core-oscore-key-update}, focusing on the exchange of nonces for deriving new OSCORE Security Contexts. The document addresses compatibility with ACE profiles and emphasizes the importance of timely key updates.

\subsubsection{Constrained Application Protocol over Bundle Protocol}
Carles Gomez presented the \href{https://datatracker.ietf.org/doc/html/draft-gomez-core-coap-bp}{draft-gomez-core-coap-bp}, which explores CoAP over the Bundle Protocol. The discussion included message aggregation and the introduction of a Payload-Length option, with considerations for future refinements.

\subsubsection{Stateless OSCORE}
Christian Amsüss introduced the \href{https://datatracker.ietf.org/doc/html/draft-amsuess-core-stateless-oscore}{draft-amsuess-core-stateless-oscore}, which enables OSCORE to operate without storing per-peer keys. The approach is suitable for safe operations and seeks feedback on potential use cases.

Meeting materials can be accessed at \href{https://datatracker.ietf.org/meeting/121/session/core}{Meeting Materials}.



\newpage

\section{COSE (IETF 121)}

\subsection{Attendees}

\subsubsection{Overview}

The COSE working group meeting was attended by representatives from prominent organizations such as Microsoft, NSA, Nokia, and Ericsson, with a total attendance of over 50 participants. Key institutions included the University of Applied Sciences Bonn-Rhein-Sieg, SECOM CO., LTD., and the Bundesdruckerei.

\subsection{Meeting Discussions}

\subsubsection{Opening Remarks}

The meeting commenced with opening remarks from the chairs, highlighting the publication of RFC 9596 and RFC 9597. The chairs proposed moving current drafts to publication, receiving no objections from attendees.

\subsubsection{Post-Quantum Signatures}

Mike Prorock presented on \href{https://datatracker.ietf.org/doc/html/draft-ietf-cose-dilithium}{draft-ietf-cose-dilithium}, \href{https://datatracker.ietf.org/doc/html/draft-ietf-cose-sphincs-plus}{draft-ietf-cose-sphincs-plus}, and \href{https://datatracker.ietf.org/doc/html/draft-ietf-cose-falcon}{draft-ietf-cose-falcon}, recommending separate drafts for pre-hashed and non-pre-hashed versions. The community was encouraged to contribute to implementations and testing, particularly for Sphincs and Falcon.

\subsubsection{Hash Envelope}

Jon Geater, stepping in for Steve Lasker, discussed \href{https://datatracker.ietf.org/doc/html/draft-steele-cose-hash-envelope}{draft-steele-cose-hash-envelope}, focusing on the challenges of large payloads and the need for explicit identifiers for hashed payloads. The discussion emphasized the importance of early allocation and the need for further work on security considerations.

\subsubsection{Hybrid Public Key Encryption (HPKE)}

Hannes Tschofenig presented \href{https://datatracker.ietf.org/doc/html/draft-ietf-cose-hpke}{draft-ietf-cose-hpke}, addressing the complexities of nested structures and the evolution of HPKE to improve interoperability. The discussion highlighted the need for a structure that relies on out-of-band material, with suggestions to consult the JOSE group for further insights.

\subsubsection{PQC Hybrid HPKE}

In \href{https://datatracker.ietf.org/doc/html/draft-reddy-cose-jose-pqc-hybrid-hpke}{draft-reddy-cose-jose-pqc-hybrid-hpke}, Hannes Tschofenig explored the necessity of post-quantum cryptography and hybrid transitions. The document includes algorithms chosen based on TLS work, with a call for comments on adoption.

\subsubsection{CEK HKDF SHA256}

The \href{https://datatracker.ietf.org/doc/html/draft-tschofenig-cose-cek-hkdf-sha256}{draft-tschofenig-cose-cek-hkdf-sha256} was presented as a response to a plaintext recovery attack. Volunteers were sought for reviewing the draft before calling for adoption, with discussions on the broader implications for COSE.

\subsubsection{Merkle Mountain Range Proofs}

Robin Bryce introduced \href{https://datatracker.ietf.org/doc/html/draft-bryce-cose-merkle-mountain-range-proofs}{draft-bryce-cose-merkle-mountain-range-proofs}, proposing a new log format for COSE Receipts. The draft aims to establish a registry for new formats, with a call for reviews to ensure consistency with related profiles.

\subsubsection{Cometre CCF Profile}

Henk Birkholz presented \href{https://datatracker.ietf.org/doc/html/draft-birkholz-cose-cometre-ccf-profile}{draft-birkholz-cose-cometre-ccf-profile}, focusing on the integration of COSE with CCF for confidential attestations. The need for harmonization with MMR was acknowledged, with a request for reviewers before adoption.

\subsection{Conclusion}

The meeting concluded with no additional business, and the chairs outlined the next steps, including calls for adoption and further reviews. Meeting materials can be accessed at \href{https://datatracker.ietf.org/meeting/121/session/cose}{IETF 121 COSE Meeting Materials}.



\newpage

\section{Delegation Extensions Working Group (DELEG)}

\subsection{Attendees Overview}
\subsubsection{Attendance}
The DELEG working group meeting was attended by representatives from prominent organizations such as Cloudflare, Verisign, ICANN, and Microsoft, among others. In total, there were over 100 participants, reflecting a diverse and engaged audience from various sectors of the internet infrastructure community.

\subsection{Meeting Discussions}

\subsubsection{Requirements Document}
\textbf{Presenter: Tale}

The \href{https://datatracker.ietf.org/doc/draft-wirelela-deleg-requirements/}{draft-wirelela-deleg-requirements} was discussed, highlighting the adoption of a new taxonomy and addressing minor issues. The document is nearing readiness for a Working Group Last Call (WGLC), indicating significant progress towards consensus on foundational requirements.

\subsubsection{Delegation Models for DELEG}
\textbf{Presenter: Paul Hoffman}

The session explored three delegation models: Parent delegates, Child delegates, and Mixed. The discussion emphasized the need to focus on resolver requirements without over-designing for secondary use cases. The consensus leaned towards the "Parent delegates" model, though considerations about DNSSEC implications were raised, suggesting a need for further exploration of trust models.

\subsubsection{DELEG Records: Omnibus vs. Discrete}
\textbf{Presenter: Paul Hoffman}

A key debate centered on whether to use an omnibus RRtype or discrete RRtypes for DELEG records. The omnibus approach offers simplicity and performance benefits, while discrete types provide flexibility. Feedback from developers was solicited to assess the feasibility of each approach, with a preference emerging for the omnibus method due to its potential for streamlined implementation.

\subsubsection{How DELEG and \_deleg Meet the Requirements}
\textbf{Presenter: Philip Homburg}

The comparison between \href{https://datatracker.ietf.org/doc/html/draft-wesplaap-deleg}{draft-wesplaap-deleg} and \href{https://datatracker.ietf.org/doc/html/draft-homburg-deleg-incremental-deleg-00}{draft-homburg-deleg-incremental-deleg-00} highlighted that both drafts meet most hard and soft requirements. However, \_deleg offers advantages in adding new zones without authoritative updates, while DELEG provides a more straightforward query process.

\subsubsection{Comparison: DELEG vs. \_deleg}
\textbf{Presenter: Petr Špaček}

The discussion compared the architectural purity of DELEG with the expediency of \_deleg. DELEG was noted for its optimal query count and necessary DNSSEC changes, whereas \_deleg was recognized for its ease of deployment without authoritative changes. The session concluded with a preference for DELEG due to its long-term architectural benefits, despite the immediate deployment advantages of \_deleg.

\subsection{Conclusion and Next Steps}
The meeting concluded with a consensus to proceed with a WGLC for the requirements document, with an emphasis on ensuring comprehensive feedback. The group expressed a preference for the DELEG approach, reflecting a strategic choice for long-term technical alignment. Future actions will focus on refining the DELEG model and addressing DNSSEC-related considerations.

Meeting materials and detailed slides are available at \href{https://datatracker.ietf.org/meeting/121/materials/slides-121-deleg-comparison-draft-wesplaap-deleg-01-vs-draft-homburg-deleg-incremental-deleg-00-00}{this link}.



\newpage

\section{Deterministic Networking (DetNet) [DetNet]}

\subsection{Attendees Overview}
\subsubsection{Participants}
The DetNet session was attended by representatives from prominent companies and institutions such as Cisco Systems, Ericsson, Huawei Technologies, Juniper Networks, and the University of Oxford, with a total attendance of approximately 50 participants.

\subsection{Meeting Discussions}

\subsubsection{Intro, WG Status, Draft Status}
The session began with an introduction and update on the working group status by the chairs. Discussions centered on the categorization of solutions, with a focus on aligning the taxonomy document with the functional categorization. Lou Berger emphasized the need for a single standards track solution per category, suggesting a strategic shift towards more experimental and informational documents to explore novel areas.

\subsubsection{DetNet Controller Plane Framework}
Xuesong Geng presented the \href{https://datatracker.ietf.org/doc/html/draft-ietf-detnet-controller-plane-framework}{draft-ietf-detnet-controller-plane-framework}. The discussion highlighted the need for clarifications on encryption policies and protocol extensions. It was agreed that alignment with the RAW architecture is necessary, with potential updates to ensure document consistency.

\subsubsection{RAW Architecture}
Pascal Thubert discussed the \href{https://datatracker.ietf.org/doc/html/draft-ietf-raw-architecture}{draft-ietf-raw-architecture}, focusing on the integration of RCPF and LCPF functions. The need for clear documentation of proposed changes was emphasized, with a call for contributions to ensure alignment with the DetNet framework.

\subsubsection{Dataplane Enhancement Taxonomy}
Jinoo Joung presented the \href{https://datatracker.ietf.org/doc/html/draft-ietf-detnet-dataplane-taxonomy}{draft-ietf-detnet-dataplane-taxonomy}. The session explored the categorization of functional characteristics and the potential for a separate analysis document. The importance of aligning with existing RFCs and models was noted, with a focus on reducing the number of solution categories.

\subsubsection{Latency Guarantee with Stateless Fair Queuing}
Jinoo Joung introduced the \href{https://datatracker.ietf.org/doc/html/draft-joung-detnet-stateless-fair-queuing}{draft-joung-detnet-stateless-fair-queuing}, discussing the inclusion of transmission and propagation latencies. The need for clarity on service rates and overprovisioning was highlighted.

\subsubsection{Deadline Based Deterministic Forwarding}
Shaofu Peng presented the \href{https://datatracker.ietf.org/doc/html/draft-peng-detnet-deadline-based-forwarding}{draft-peng-detnet-deadline-based-forwarding}, with no additional comments from the attendees.

\subsubsection{Mechanism to Control Jitter Caused by Policing in DetNet}
Shaofu Peng discussed the \href{https://datatracker.ietf.org/doc/html/draft-peng-detnet-policing-jitter-control}{draft-peng-detnet-policing-jitter-control}. Collaboration was proposed to address overlaps with existing frameworks, aiming for a unified approach.

\subsubsection{Resilient Cycle Queuing and Forwarding}
Rubing Liu presented the \href{https://datatracker.ietf.org/doc/html/draft-liu-detnet-rcqf}{draft-liu-detnet-rcqf}, which was well-received without further comments.

\subsubsection{Data Unit Groups for DetNet-Enabled Networks}
Sebastian Robitzsch introduced the \href{https://datatracker.ietf.org/doc/html/draft-rc-detnet-data-unit-groups}{draft-rc-detnet-data-unit-groups}, discussing the integration of IPv6 extension headers. The session concluded with a call for coordination with other working groups to ensure comprehensive coverage of use cases.

Meeting materials are available at \href{https://datatracker.ietf.org/meeting/121/session/detnet}{DetNet Session Materials}.



\newpage

\section{DMM Working Group (DMM)}

\subsection{Attendees}

\subsubsection{Overview}
The DMM Working Group session was attended by representatives from prominent companies and institutions such as Cisco, Huawei, SoftBank, Telefonica Innovacion Digital, and Tohoku University, among others. The total attendance was approximately 35 participants.

\subsection{Meeting Discussions}

\subsubsection{Mobility Aware Transport Network Slicing for 5G}
The draft \href{https://datatracker.ietf.org/doc/draft-ietf-dmm-tn-aware-mobility}{draft-ietf-dmm-tn-aware-mobility} was discussed, highlighting its readiness for Working Group Last Call. Key changes include transitioning from 'EP\_Transport' in 3GPP terms to ACaaS in IETF GTP terms, with updates to the YANG model. The discussion emphasized that the 3GPP-related slicing architecture remains unchanged, and the model's scope now extends to gNB. This draft is poised to significantly influence the integration of transport network slicing in 5G environments.

\subsubsection{MUP Architecture for DMM}
The \href{https://datatracker.ietf.org/doc/draft-mhkk-dmm-mup-architecture}{draft-mhkk-dmm-mup-architecture} was presented, focusing on a pluggable user-plane architecture for mobile service systems. The discussion compared this draft with RFC-9433, noting its broader applicability. The group debated whether the draft should pursue a standards track or remain informational, ultimately deciding on the former due to its foundational role in defining standard protocols.

\subsubsection{Mobile Traffic Steering}
The \href{https://datatracker.ietf.org/doc/draft-liebsch-dmm-mts}{draft-liebsch-dmm-mts} was reviewed, with discussions on its relationship to CATS and its general applicability beyond specific mobile systems. The draft's structure, including use cases and deployment options, was outlined. The session explored the draft's implications for traffic steering per-flow or per-session, particularly in relation to the N6 interface.

\subsubsection{Computing Aware Traffic Steering Consideration for MUP}
The \href{https://datatracker.ietf.org/doc/html/draft-dcn-dmm-cats-mup}{draft-dcn-dmm-cats-mup} was discussed, focusing on centralized versus distributed CATS-MUP models. Concerns about routing instability due to evolving CATS metrics were raised, with suggestions to engage with the BGP team for further insights.

\subsubsection{SRH Reduction for SRv6 End.M.GTP6.E Behavior}
The \href{https://datatracker.ietf.org/doc/draft-kawakami-dmm-srv6-gtp6e-reduced}{draft-kawakami-dmm-srv6-gtp6e-reduced} was presented, with discussions on the allocation of SR-endpoint behavior codepoints. The chairs recommended continuing discussions on the mailing list and suggested an informational draft as a suitable next step.

\subsubsection{Introducing the 6G Reset Initiative}
David Lake introduced the 6G Reset Initiative, advocating for a broader technological scope beyond 3GPP, including WiFi and fixed wireless technologies. This initiative aims to redefine the 6G landscape by integrating diverse technological advancements.

Meeting materials are available at \href{https://meetings.conf.meetecho.com/ietf121/?group=dmm&short=dmm&item=1}{Meetecho} and \href{https://notes.ietf.org/notes-ietf-121-dmm}{Notes}.




\newpage

\section{DNS Operations (DNSOP) Working Group}

\subsection{Attendees Overview}
\subsubsection{Prominent Companies and Institutions}
The DNSOP Working Group session was attended by representatives from notable organizations such as Verisign, NLnet Labs, ISC, APNIC, Google, and ICANN, among others. The total attendance was approximately 120 participants, reflecting a diverse range of expertise and interest in DNS operations.

\subsection{Meeting Discussions}

\subsubsection{Generalized Notify}
Peter Thomassen presented the \href{https://datatracker.ietf.org/doc/html/draft-ietf-dnsop-generalized-notify}{draft-ietf-dnsop-generalized-notify}, which aims to enhance DNS notification mechanisms. The discussion focused on the potential for a Working Group Last Call (WGLC), with participants emphasizing the need for clarity in implementation guidelines to ensure interoperability across different DNS systems.

\subsubsection{Domain Verification Techniques}
Shivan Sahib discussed the \href{https://datatracker.ietf.org/doc/html/draft-ietf-dnsop-domain-verification-techniques}{draft-ietf-dnsop-domain-verification-techniques}. The dialogue highlighted concerns about the applicability of domain control validation (DCV) methods, with suggestions to improve the document's structure for better guidance. The consensus was to refine the draft to address both point-in-time checks and long-term validation strategies.

\subsubsection{DNSSEC Algorithm Recommendations}
Wes Hardaker introduced the \href{https://datatracker.ietf.org/doc/html/draft-ietf-dnsop-rfc8624-bis}{draft-ietf-dnsop-rfc8624-bis}, proposing updates to DNSSEC cryptographic algorithm recommendations. The session underscored the importance of balancing implementation flexibility with clear guidance for administrators. The group agreed on the necessity of further discussions to refine the recommendations.

\subsubsection{IPv6 Transport Guidelines}
Tobias Fiebig presented the \href{https://datatracker.ietf.org/doc/html/draft-momoka-dnsop-3901bis}{draft-momoka-dnsop-3901bis}, focusing on DNS IPv6 transport operational guidelines. The debate centered on the challenges of IPv6 deployment and the need for comprehensive documentation to address operational issues. Participants expressed support for continued work on this draft, recognizing its potential to facilitate smoother IPv6 transitions.

\subsubsection{Deprecation of DNS64}
Nick Buraglio's \href{https://datatracker.ietf.org/doc/html/draft-buraglio-deprecate7050}{draft-buraglio-deprecate7050} was discussed, advocating for the deprecation of DNS64 in certain environments. The conversation highlighted the draft's relevance in non-mobile contexts, with a call for engagement with mobile operators to assess broader applicability. The group acknowledged the draft's potential to streamline DNS operations in specific scenarios.

\subsection{Meeting Materials}
All meeting materials, including slides and detailed minutes, are available at \href{https://notes.ietf.org/notes-ietf-121-dnsop}{this link}.

The discussions in the DNSOP Working Group sessions were pivotal in shaping the future direction of DNS operations, with a focus on enhancing security, interoperability, and operational efficiency. The outcomes of these discussions are expected to contribute significantly to the ongoing evolution of DNS standards and practices.



\newpage

\section{Extensions for Scalable DNS Service Discovery (dnssd)}

\subsection{Attendees}
\subsubsection{Overview}
The meeting was attended by representatives from prominent companies and institutions such as Google, Apple, Meta Platforms, Inc., Microsoft, and Cisco, with a total attendance of over 50 participants.

\subsection{Meeting Discussions}

\subsubsection{Publication Update on SRP and Update Lease}
Ted Lemon provided a status update on the \href{https://datatracker.ietf.org/doc/draft-ietf-dnssd-srp/}{draft-ietf-dnssd-srp} and \href{https://datatracker.ietf.org/doc/draft-ietf-dnssd-update-lease/}{draft-ietf-dnssd-update-lease}, which are in the final stages of AUTH48 processing. The process requires another round of Working Group review due to complexities in the RFC Editor process.

\subsubsection{Multiple QTYPEs}
Ray Bellis discussed the \href{https://datatracker.ietf.org/doc/draft-ietf-dnssd-multi-qtypes/}{draft-ietf-dnssd-multi-qtypes}, addressing feedback and seeking readiness for Working Group Last Call (WGLC). The discussion highlighted the need for implementation feedback and potential examples to clarify the draft.

\subsubsection{MDNS Conflict Resolution Using Time Since Received}
The \href{https://datatracker.ietf.org/doc/draft-tllq-tsr/}{draft-tllq-tsr} was presented, focusing on practical results and API issues. The group considered adopting TSR as a prerequisite for the Advertising Proxy, with discussions on potential API changes.

\subsubsection{DNS Push Additional Records}
The \href{https://datatracker.ietf.org/doc/draft-tlmd-push-dnssd-additional/}{draft-tlmd-push-dnssd-additional} was discussed, emphasizing the efficiency of delivering multiple related records in a single DNS message. The group debated the conditions under which excessive data might be returned and the need for client expression of additional data requests.

\subsubsection{Compressing SRP for Constrained Networks}
Abtin Keshavarzian presented on compressing SRP messages for constrained networks, particularly on Thread. The proposal aims to reduce message sizes significantly, with discussions on the applicability of the compression technique across different contexts.

\subsubsection{SVCB and HTTPS for DNSSD Service Instances}
Gautam Akiwate introduced a proposal for using SVCB records to communicate multiple protocol variants for service instances. The group considered the implications for DNS-SD services and the potential for a simpler, more efficient implementation.

\subsubsection{Advertising Proxy Open Issue Discussion}
Due to time constraints, this discussion was deferred. However, the group acknowledged the need to address open issues in the \href{https://datatracker.ietf.org/doc/draft-ietf-dnssd-advertising-proxy/}{draft-ietf-dnssd-advertising-proxy}.

Meeting materials are available \href{https://datatracker.ietf.org/meeting/121/materials/slides-121-dnssd-srp-compression-for-constrained-networks-00}{here}.




\newpage

\section{Delay/Disruption Tolerant Networking (dtn) WG}

\subsection{Attendees Overview}
\subsubsection{Prominent Institutions and Attendance}
The meeting was attended by representatives from notable organizations such as Johns Hopkins University Applied Physics Laboratory, NASA, The MITRE Corporation, Juniper Networks, and Huawei, among others. In total, there were 40 participants, reflecting a diverse range of expertise and interest in Delay/Disruption Tolerant Networking.

\subsection{Meeting Discussions}

\subsubsection{Introduction, Note Well \& Milestones}
The session began with the chairs providing an overview of the working group's milestones and administrative notes. Discussions included the status of documents in the RFC editor queue and the adoption of new drafts such as \href{https://datatracker.ietf.org/doc/html/draft-ietf-dtn-amp}{draft-ietf-dtn-amp}.

\subsubsection{4838bis}
Ed Birrane led a discussion on the relevance of RFC4838 in 2024, questioning which elements should be reaffirmed or updated. The group considered whether an IETF WG should pursue a refreshed architecture document, given the evolution since 2007. Alberto Montilla supported the idea, highlighting the need for alignment with current standards.

\subsubsection{DTNMA Next Steps}
Jenny Cao, Justin Ethier, and Brian Sipos presented the next steps for DTNMA, now an RFC9675. The discussion focused on the adoption of related drafts such as \href{https://datatracker.ietf.org/doc/html/draft-ietf-dtn-amm}{draft-ietf-dtn-amm} and the development of an open-source library. The need for integrating spacecraft autonomy models into network management was emphasized.

\subsubsection{Other Proposed Work}
Brian Sipos introduced several drafts, including \href{https://datatracker.ietf.org/doc/html/draft-sipos-dtn-eid-pattern}{draft-sipos-dtn-eid-pattern} and \href{https://datatracker.ietf.org/doc/html/draft-sipos-dtn-udpcl}{draft-sipos-dtn-udpcl}. The discussion covered the potential of UDPCL for multicast capabilities and the implications for extensibility and interoperability.

\subsubsection{IMC Orange Book}
Joshua Deaton discussed the experimental Interplanetary Multi-Destination Communication (IMC) within CCSDS, which aims to handle multiple destinations without duplicates. The conversation touched on the technical challenges of URI scheme restrictions and the potential for IANA registration.

\subsubsection{"Custody Transfer" to Historic}
Rick Taylor proposed re-evaluating the concept of Custody Transfer (CT) in DTN, suggesting it might not fully address reliability issues. The group debated the merits of CT and its role in ensuring end-to-end reliability, with input from Scott Burleigh and others.

\subsubsection{Interpeer Architecture}
Jens Finkhaeuser presented on the Interpeer Architecture, exploring its potential for low-latency, high-bandwidth applications. The discussion considered the overlap with existing DTN protocols and the possibility of using BP as a convergence layer.

Meeting materials are available at \href{https://www.ietf.org/proceedings/121/slides/slides-121-dtn-00.pdf}{IETF 121 dtn Meeting Materials}.



\newpage

\section{Device Usage and Location Tracking (DULT) Working Group [DULT]}

\subsection{Attendees}

\subsubsection{Overview}

The DULT Working Group meeting was attended by representatives from prominent companies and institutions, including Apple, Cisco Systems, Juniper Networks, and Technology, among others. A total of 40 participants were present, contributing to a diverse and comprehensive discussion.

\subsection{Meeting Discussions}

\subsubsection{Threat Model Discussion}

The session commenced with a discussion on the \href{https://datatracker.ietf.org/doc/html/draft-ietf-dult-threat-model}{draft-ietf-dult-threat-model}, led by Maggie Delano. Key points included the scope of easily-concealable devices and the relevance of consumer-grade technology. The dialogue emphasized the necessity to document system gaps and trade-offs, particularly concerning non-compliant tags and their detection.

\subsubsection{Accessory Protocol Presentation}

Brent Ledvina presented the \href{https://datatracker.ietf.org/doc/html/draft-ietf-dult-accessory-protocol}{draft-ietf-dult-accessory-protocol}, focusing on the integration of terminology within the document. Discussions highlighted the potential for a standalone terminology document to ease maintenance and the importance of aligning documents within a cluster for coherent publication. The session also covered the applicability statement and the necessity of sound maker requirements for accessibility.

\subsubsection{Technical Specifications and Proposed Requirements}

The meeting further explored technical specifications and proposed requirements, with a focus on the practicality of implementing measurable metrics. The conversation underscored the challenges of defining concrete numbers for tracking scenarios and the implications of remote disabling features. Participants debated the balance between technical feasibility and user privacy, considering the diverse environments in which devices operate.

\subsubsection{Next Steps and Action Items}

The group concluded with a consensus to refine the document organization and integrate feedback from reviewers. Future actions include drafting a table of assessment parameters and exploring the feasibility of customizability in tracking devices. The discussions are expected to guide the working group's strategic direction, potentially influencing industry standards for device tracking and privacy.

Meeting materials are available at \href{https://datatracker.ietf.org/meeting/121/materials/slides-121-dult-dult-wg-accessory-protocol-slides-00}{this link}.



\newpage

\section{EMAILCORE (EC)}

\subsection{Attendees}
\subsubsection{Overview}
The EMAILCORE working group meeting was attended by representatives from prominent companies and institutions such as Apple, Fastmail, ICANN, Isode Limited, Meta Platforms, Inc., and Futurewei Technologies, among others. The total attendance was approximately 20 participants.

\subsection{Meeting Discussions}

\subsubsection{Review of SMTP (rfc5321bis)}
The primary focus of the meeting was the review of remaining SecDir and DnsDir comments on the \href{https://datatracker.ietf.org/doc/html/draft-ietf-emailcore-rfc5321bis-32}{draft-ietf-emailcore-rfc5321bis-32}. Key issues discussed included the wording of certain sections, the classification of RFC 821, and the handling of chained CNAMEs. The group reached a consensus on several textual changes, particularly in security considerations, and agreed to defer some decisions to the IESG. The discussions highlighted the importance of balancing technical precision with practical implementation concerns, especially regarding SMTP extensions like STARTTLS.

\subsubsection{Applicability Statement for IETF Core Email Protocols}
The session also covered the \href{https://datatracker.ietf.org/doc/html/draft-ietf-emailcore-as-12}{draft-ietf-emailcore-as-12}, focusing on its role in providing guidance on the use of core email protocols. The group debated the inclusion of specific security recommendations and the potential impact on existing implementations. The consensus was to ensure that the document remains a useful reference without imposing unnecessary burdens on implementers.

\subsubsection{Outcomes and Next Steps}
The meeting concluded with a plan to incorporate the agreed changes into the drafts and to prepare for an interim meeting in early December. The group aims to finalize the drafts for IESG review, with a strategic decision to potentially align the submission of the SMTP and Applicability Statement drafts. This alignment is expected to streamline the review process and enhance the coherence of the guidance provided to the community.

Meeting materials and further details can be accessed via \href{https://meetings.conf.meetecho.com/ietf121/?group=emailcore&short=&item=1}{Meetecho} and \href{https://notes.ietf.org/notes-ietf-121-emailcore}{meeting notes}.



\newpage

\section{EMU Working Group (EMU)}

\subsection{Attendees}

The EMU Working Group meeting at IETF 121 was attended by representatives from prominent companies and institutions such as Cisco Systems, Huawei, Google, and the University of Murcia, with a total attendance of over 50 participants. Notable attendees included Alan DeKok from InkBridge, Jan-Frederik Rieckers from DFN-Verein, and Joe Salowey from Venafi.

\subsection{Meeting Discussions}

\subsubsection{eap.arpa}

The discussion on \href{https://datatracker.ietf.org/doc/html/draft-ietf-emu-eap-arpa}{draft-ietf-emu-eap-arpa} highlighted recent updates, including added text on security and consistency. The Working Group Last Call (WGLC) has concluded, and the document awaits further actions pending an ACME draft.

\subsubsection{TLS-POK}

The \href{https://datatracker.ietf.org/doc/html/draft-ietf-emu-bootstrapped-tls}{draft-ietf-emu-bootstrapped-tls} is currently in WGLC, with all comments addressed. The main challenge discussed was the lack of end-to-end implementation on commercial TLS stacks, which is anticipated to be straightforward but time-consuming.

\subsubsection{EAP-EDHOC}

The \href{https://datatracker.ietf.org/doc/html/draft-ietf-emu-eap-edhoc}{draft-ietf-emu-eap-edhoc} discussion focused on the need for external connection identifiers to correlate EDHOC messages during EAP authentication sessions. The group debated the necessity of handling fragmentation within EAP methods, with insights from Alan DeKok and Rafael Marin-Lopez.

\subsubsection{EAP-FIDO}

The \href{https://datatracker.ietf.org/doc/html/draft-ietf-emu-eap-fido}{draft-ietf-emu-eap-fido} is undergoing a name change and has seen updates in its proof-of-concept implementation. Discussions centered on the structure of FIDO challenges and the implications of using binary formats versus JSON, with considerations for middle-person attack prevention.

\subsubsection{EAP-PPT}

The \href{https://datatracker.ietf.org/doc/html/draft-sawant-eap-ppt}{draft-sawant-eap-ppt} aims to provide anonymous network access using Privacy Pass tokens. Key points included the introduction of key material generation to mitigate middle-person attacks and the need for channel binding specifications. Feedback was sought on deployment considerations and potential abuse prevention mechanisms.

\subsection{Meeting Materials}

All meeting materials are available at \href{https://datatracker.ietf.org/meeting/121/materials/agenda-121-emu}{IETF 121 EMU Meeting Materials}.






\newpage

\section{High Performance Wide Area Network (HPWAN) BoF [HPWAN]}

\subsection{Attendees Overview}
The HPWAN BoF session was attended by representatives from prominent companies and institutions such as Google, Huawei, China Mobile, Netflix, and Microsoft, among others. The total attendance was approximately 150 participants, reflecting a broad interest across industry and academia.

\subsection{Meeting Discussions}

\subsubsection{Introduction and Goals}
The session began with an introduction by the chairs, Tim Chown and Gorry Fairhurst, who outlined the goals of the BoF and provided a definition of High Performance Wide Area Networks (HPWAN). The definition was accepted without objections, setting the stage for focused discussions on the topic.

\subsubsection{State of the Art Congestion Control}
Michael Welzl presented on the current state of congestion control technologies. The discussion highlighted the need for network feedback mechanisms to aid applications in adjusting windowing to prevent congestion. The conversation touched on the applicability of DCTCP and BBRv2 in HP-WAN contexts, with insights shared on the challenges of implementing these technologies in public Internet scenarios. For further details, refer to \href{https://datatracker.ietf.org/doc/html/draft-ietf-hpwan-congestion-control}{draft-ietf-hpwan-congestion-control}.

\subsubsection{High-Volume Content Mover}
Michał Zasadziński from Google discussed the Effingo platform, emphasizing its use of BBR for latency-sensitive transfers. The presentation sparked dialogue on fairness and traffic shaping, with insights into Google's approach to maximizing throughput while maintaining proportional fairness.

\subsubsection{R\&E Operator Insights}
Tim Chown shared experiences from CERN's data transfers, focusing on predictable traffic patterns and the potential use of multicast for data distribution. The discussion explored the challenges of optimizing data flows and the role of QoS mechanisms in research and education networks.

\subsubsection{Public Operator Applications}
Kehan Yao from China Mobile presented on HP-WAN applications, discussing the reliability of protocols like iWARP and the potential for RDMA over WAN. The session underscored the need for lightweight, reliable solutions in high-performance network environments.

\subsubsection{Open Technical Issues}
Daniel Huang led a discussion on open technical issues, focusing on the need for network feedback signals and the importance of narrowing down use cases. The dialogue highlighted the potential for coordination between transport and routing layers to enhance HP-WAN services.

\subsubsection{Open Discussion and Conclusions}
The open discussion addressed gaps in current HP-WAN solutions and the potential for IETF contributions. Participants expressed interest in exploring multi-path techniques and MTU discovery, with a consensus on the need for collaborative efforts to address these challenges.

\subsubsection{Next Steps}
The session concluded with a series of votes indicating strong support for IETF involvement in HP-WAN topics. Participants expressed willingness to contribute and review work, with discussions on whether existing groups could accommodate the transport-related issues identified.

Meeting materials, including slides and detailed notes, are available at \href{https://example.com/hpwan-materials}{HPWAN Meeting Materials}.



\newpage

\section{HTTPAPI (IETF 121)}

\subsection{Attendees}

The session was attended by representatives from prominent companies and institutions, including Akamai, Microsoft, Cloudflare, Ericsson, Nokia, and Fastmail, with a total attendance of over 30 participants.

\subsection{Meeting Discussions}

\subsubsection{Privacy Concerns: Exposing API Keys via HTTP}

Mike Bishop led a discussion on the privacy implications of exposing API keys through HTTP. Marius highlighted that some authenticated requests, like AWS's key-id and signature method, do not contain private material. The conversation explored potential mitigations, such as using hash(tls-exporter, credential) to prevent credential reuse. This discussion is crucial for enhancing security measures in API communications.

\subsubsection{HTTPbis Signature Work}

Justin Richer introduced RFC 9421, focusing on end-to-end signature work compared to TLS's hop-by-hop approach. The session emphasized the utility of signatures in verifying message integrity and authenticating software clients. The discussion underscored the importance of maintaining HTTP semantics across hops, which could significantly impact secure communications.

\subsubsection{Digests-Fields Problem Types}

Marius Kleidl presented on defining digest-specific problem types as instances of RFC 9457 messages. The dialogue included considerations of potential oracle attacks and the necessity of high-level problem definitions. This work aims to improve error signaling in HTTP APIs, which is vital for robust client-server interactions.

\subsubsection{REST Media Types}

Roberto Polli summarized the current state and challenges of REST media types. The discussion included the possibility of registering media types for complete OAS documents, which could streamline API documentation and integration processes.

\subsubsection{Ratelimit-Headers}

Darrel Miller, speaking as a co-author, discussed enhancements to the \href{https://datatracker.ietf.org/doc/html/draft-ietf-httpapi-ratelimit-headers}{draft-ietf-httpapi-ratelimit-headers}, including multiple named policies and quota-units. The session explored the utility of partition keys and the need for a registry and syntax rules, which are pivotal for managing API rate limits effectively.

\subsubsection{Idempotency-Key-Header}

The session concluded with discussions on the \href{https://datatracker.ietf.org/doc/html/draft-ietf-httpapi-idempotency-key-header}{draft-ietf-httpapi-idempotency-key-header}. Consensus was reached that idempotency services are out of scope, but participants were encouraged to review open issues in the repository.

\subsection{Meeting Materials}

For all slides and detailed materials, please refer to the \href{https://www.ietf.org/proceedings/121/httpapi.html}{meeting materials}.




\newpage

\section{HTTP Working Group (HTTPBIS)}

\subsection{Attendees}
\subsubsection{Overview}
The HTTP Working Group meeting at IETF 121 saw participation from a diverse range of companies and institutions, including Cloudflare, Google, Apple, Microsoft, and Ericsson, with a total attendance of over 50 participants. Notable attendees included Mark Nottingham from Cloudflare, Yoav Weiss from Shopify, and Tommy Pauly from Apple.

\subsection{Meeting Discussions}

\subsubsection{Resumable Uploads}
Marius Kleidl presented the \href{https://datatracker.ietf.org/doc/html/draft-ietf-httpbis-resumable-upload}{draft-ietf-httpbis-resumable-upload}, highlighting successful interoperability tests during the hackathon. Discussions centered around potential issues with the OPTIONS header for detecting upload limits, with suggestions to possibly downgrade it from a MUST to a SHOULD if practical implementation proves challenging. The draft is nearing readiness for Working Group Last Call (WGLC).

\subsubsection{QUERY Method}
Mike Bishop discussed the \href{https://datatracker.ietf.org/doc/html/draft-ietf-httpbis-safe-method-w-body}{draft-ietf-httpbis-safe-method-w-body}, focusing on the syntax of "Accept-Query" and the use of structured fields. The consensus was to proceed with the current approach and prepare for Last Call once final adjustments are made.

\subsubsection{Cache Groups}
Mark Nottingham provided updates on the \href{https://datatracker.ietf.org/doc/html/draft-ietf-httpbis-cache-groups}{draft-ietf-httpbis-cache-groups}, noting minor textual changes to enhance clarity. The draft is close to completion, with a call for feedback from cache vendors to ensure comprehensive review.

\subsubsection{Incremental HTTP Messages}
Kazuho Oku introduced the \href{https://datatracker.ietf.org/doc/html/draft-kazuho-httpbis-incremental-http}{draft-kazuho-httpbis-incremental-http}, which aims to address buffering issues in intermediaries. The group discussed the potential for signaling incremental delivery preferences and the need for further exploration of bidirectional signaling.

\subsubsection{The HTTP Wrap Up Capsule}
Lucas Pardue presented the \href{https://datatracker.ietf.org/doc/html/draft-schinazi-httpbis-wrap-up}{draft-schinazi-httpbis-wrap-up}, which received strong support for adoption. The draft addresses the need for a mechanism to signal the end of a connection, akin to a GOAWAY for inner connections.

\subsubsection{Guidance for HTTP Capsule Protocol Extensibility}
Lucas Pardue discussed the \href{https://datatracker.ietf.org/doc/html/draft-pardue-capsule-ext-guidance}{draft-pardue-capsule-ext-guidance}, emphasizing the importance of clear guidance on handling unknown capsules. The group considered the potential for a registry to manage capsule behaviors.

\subsubsection{Cookie Eviction}
Yoav Weiss proposed a new mechanism for cookie eviction, addressing the limitations of the current method of setting cookies with past expiry dates. The proposal was well-received, with suggestions to integrate it into the upcoming cookie revision.

\subsubsection{AD-Requested Feedback}
Feedback was solicited on the \href{https://datatracker.ietf.org/doc/draft-ietf-netconf-http-client-server/}{draft-ietf-netconf-http-client-server}, with concerns raised about the practicality of HTTP version configuration in client applications. The group agreed on the need for further discussion with the draft's authors.

\subsubsection{Template-Driven CONNECT for TCP}
Ben Schwartz presented the \href{https://datatracker.ietf.org/doc/draft-ietf-httpbis-connect-tcp/}{draft-ietf-httpbis-connect-tcp}, which sparked debate on the necessity of supporting both capsule and non-capsule protocols. The consensus leaned towards consolidating on a single protocol approach.

\subsubsection{Security Considerations for Optimistic Use of HTTP Upgrade}
Ben Schwartz discussed the \href{https://datatracker.ietf.org/doc/draft-ietf-httpbis-optimistic-upgrade/}{draft-ietf-httpbis-optimistic-upgrade}, focusing on security implications and the need for clear guidance on handling proxy authentication failures.

\subsubsection{No-Vary-Search}
Jeremy Roman introduced the \href{https://datatracker.ietf.org/doc/draft-ietf-httpbis-no-vary-search/}{draft-ietf-httpbis-no-vary-search}, which aims to improve cache efficiency. The group discussed naming conventions and the importance of alignment between web and browser implementers.

\subsubsection{The IP Geolocation HTTP Client Hint}
Ciara McMullin presented the \href{https://datatracker.ietf.org/doc/draft-pauly-httpbis-geoip-hint/}{draft-pauly-httpbis-geoip-hint}, highlighting privacy considerations and the potential for misuse. The group agreed on the need for a requirements process to guide future geolocation work.

Meeting materials are available \href{https://notes.ietf.org/notes-ietf-121-httpbis}{here}.



\newpage

\section{IABOPEN @ IETF 121 (IABOPEN)}

\subsection{Attendees}

The IABOPEN session at IETF 121 saw participation from a diverse group of attendees, including representatives from prominent companies and institutions such as Apple, Ericsson, Huawei, Nokia, and the Internet Society. The total attendance was approximately 80 participants, reflecting a broad interest in the topics discussed.

\subsection{Meeting Discussions}

\subsubsection{Welcome \& Status Update - Chairs}

The session commenced with a welcome and status update by the chairs, Tommy Pauly and Matthew Bocci. They highlighted recent developments and ongoing initiatives within the IAB, including the publication of \href{https://datatracker.ietf.org/doc/rfc9614/}{RFC 9614} and the progress of the \href{https://datatracker.ietf.org/doc/draft-iab-bias-workshop-report/}{draft-iab-bias-workshop-report}. The chairs also introduced the concept of a 'new work help desk' to facilitate discussions on emerging work areas within the IETF.

\subsubsection{Liaison RFCs Update - Mirja Kühlewind \& Suresh Krishnan}

Mirja Kühlewind and Suresh Krishnan provided an update on liaison RFCs, emphasizing the need for clarity in engagement with other standards development organizations (SDOs). The discussion underscored the challenges of adapting IETF processes to accommodate the unique roles of liaisons, with a focus on improving response mechanisms to inbound liaison statements.

\subsubsection{ITU-T Liaison Update - Scott Mansfield}

Scott Mansfield presented an update on ITU-T liaisons, highlighting ongoing collaborations and the importance of maintaining effective communication channels. The session acknowledged the evolving nature of the ITU and the potential for increased synergy between the ITU and IETF, particularly in areas like YANG model scaling and NEMOPS workshops.

\subsubsection{NEMOPS Workshop Update - Dhruv Dhody}

Dhruv Dhody briefed attendees on the upcoming NEMOPS workshop, detailing submission deadlines and workshop logistics. The workshop aims to explore network operations and management challenges, with a focus on fostering collaboration and innovation in this critical area.

\subsubsection{AI-CONTROL Workshop Summary - Suresh Krishnan}

Suresh Krishnan summarized the AI-CONTROL workshop, which explored the intersection of AI and network control. The session discussed the draft report \href{https://datatracker.ietf.org/doc/draft-iab-ai-control-report/}{draft-iab-ai-control-report} and potential IETF work in this domain. The workshop emphasized the need for coordinated efforts to address AI governance and policy challenges.

\subsubsection{Updates on the Global Digital Compact - Olaf Kolkman}

Olaf Kolkman provided insights into the Global Digital Compact (GDC), discussing the implications of endorsement and the strategic importance of participating in global digital governance dialogues. The session highlighted the need for coherent messaging from the technical community and the potential impact of the GDC on future internet governance frameworks.

Meeting materials are available at \href{https://datatracker.ietf.org/meeting/121/materials/slides-121-iabopen-chair-slides-04}{IAB Open Meeting Materials}.



\newpage

\section{IntArea Working Group (IntArea)}

\subsection{Attendees}
The IntArea Working Group meeting was attended by representatives from prominent companies and institutions such as Cisco, Google, Huawei, Microsoft, and Juniper Networks, with a total attendance of over 70 participants. Notable attendees included Juan Carlos Zuniga from Cisco, Wassim Haddad from Ericsson, and Tommy Pauly from Apple.

\subsection{Meeting Discussions}

\subsubsection{Agenda Bashing and Document Status Updates}
The meeting commenced with a brief agenda bashing and updates on the status of working group documents. It was noted that several drafts, inactive for years, would be declared "WG dead." Carlos J. Bernardos inquired about a specific draft of interest, and Eric Vyncke emphasized the importance of continued collaboration on adopted documents.

\subsubsection{Communicating Proxy Configurations in Provisioning Domains}
Tommy Pauly presented the \href{https://datatracker.ietf.org/doc/html/draft-ietf-intarea-proxy-config-02}{draft-ietf-intarea-proxy-config-02}. Discussions centered around the handling of FTP schemes and the inclusion of authentication points, with consensus that certain elements should not be part of this document. The debate highlighted the need for careful consideration of information duplication with DNS.

\subsubsection{Using Dummy IPv4 Address and Node Identification Extensions for IP/ICMP Translators}
Jen Linkova discussed the use of dummy IPv4 addresses and node identification extensions, referencing the \href{https://datatracker.ietf.org/doc/html/draft-equinox-v6ops-icmpext-xlat-v6only-source-00}{draft-equinox-v6ops-icmpext-xlat-v6only-source-00} and \href{https://datatracker.ietf.org/doc/html/draft-ietf-intarea-extended-icmp-nodeid-00}{draft-ietf-intarea-extended-icmp-nodeid-00}. Ron Bonica raised concerns about the lack of a length attribute in the extension header, suggesting a draft to address this issue.

\subsubsection{Stateless Reverse Traceroute}
Rolf Winter presented the \href{https://datatracker.ietf.org/doc/html/draft-heiwin-intarea-reverse-traceroute-stateless-03}{draft-heiwin-intarea-reverse-traceroute-stateless-03}. The discussion focused on the challenges of introducing new ICMP messages due to legacy middleboxes and the potential for using extended echo requests. The need for backward compatibility and the exploration of state-carrying packet approaches were emphasized.

\subsubsection{Web Proxy Auto Discovery Next Generation}
Josh Cohen introduced the \href{https://datatracker.ietf.org/doc/html/draft-joshco-wpadng-02}{draft-joshco-wpadng-02}, which was met with no immediate questions, indicating general acceptance or the need for further review.

\subsubsection{The Multicast Application Ports}
Nate Karstens discussed the \href{https://datatracker.ietf.org/doc/html/draft-karstens-pim-multicast-application-ports-01}{draft-karstens-pim-multicast-application-ports-01}, proposing the assignment of adjacent ports for multicast applications. The idea received positive feedback, though concerns about specific port availability were raised.

\subsubsection{MKA over IP}
Hooman Bidgoli presented the \href{https://datatracker.ietf.org/doc/html/draft-hb-intarea-eap-mka-00}{draft-hb-intarea-eap-mka-00}, emphasizing the need for simultaneous progress with IEEE developments. The concept of using a unique key per flow was discussed, with interest in its potential applications.

\subsubsection{Analog Blockers to Wide Employment of Jumbo MTUs}
Matt Mathis briefly introduced a tentative title addressing the challenges of deploying Jumbo MTUs in the production Internet, suggesting further exploration of this topic.

\subsubsection{Additional Topics (Time Permitting)}
Ron Bonica and David Lamparter presented additional drafts, including the \href{https://datatracker.ietf.org/doc/html/draft-bonica-intarea-icmp-exten-hdr-len-00}{draft-bonica-intarea-icmp-exten-hdr-len-00} and \href{https://datatracker.ietf.org/doc/html/draft-equinox-intarea-dhcpv4-route4via6-00}{draft-equinox-intarea-dhcpv4-route4via6-00}, focusing on ICMP extension header length fields and DHCPv4 options for IPv4 routes with IPv6 nexthops, respectively.

Meeting materials are available at \href{https://www.ietf.org/proceedings/121/intarea.html}{IETF 121 IntArea Meeting Materials}.



\newpage

\section{IP Performance Metrics (IPPM) [IPPM]}

\subsection{Attendees}
\subsubsection{Overview}
The IPPM session was attended by representatives from prominent companies and institutions such as Apple, Cisco, Huawei Technologies, Ericsson, and the University of Liege, among others. The total attendance was approximately 80 participants, reflecting a diverse range of expertise and interest in performance metrics.

\subsection{Meeting Discussions}

\subsubsection{Intro by Chairs}
The session commenced with a welcome note and an overview of the agenda. The chairs highlighted the status of \href{https://datatracker.ietf.org/doc/html/draft-ietf-ippm-ioam-data-integrity}{draft-ietf-ippm-ioam-data-integrity}, calling for additional reviewers to ensure comprehensive feedback.

\subsubsection{draft-ietf-ippm-capacity-protocol}
Presented by R. Geib, this draft sparked discussions on implementation status and procedural aspects of the last call. The open broadband initiative's involvement was noted, with a suggestion for further review by Tommy Pauly.

\subsubsection{draft-ietf-ippm-hybrid-two-step and draft-ietf-ippm-asymmetrical-pkts}
G. Mirsky presented these drafts, prompting a debate on the terminology used, specifically the shift from "asymmetrical packets" to "asymmetrical traffic." The discussion underscored the importance of precise language in technical documentation.

\subsubsection{draft-ietf-ippm-stamp-ext-hdr}
R. Gandhi's presentation included a proposal to integrate active tools with hybrid methods for hop-by-hop measurements, with a suggestion to include an informative reference to related work.

\subsubsection{draft-olden-ippm-qoo}
B. Tiegen's presentation on Quality of Observation (QoO) metrics led to a lively discussion on the applicability of these metrics to real-world scenarios, such as video conferencing, and their potential impact on user experience.

\subsubsection{draft-ietf-ippm-responsiveness}
S. Cheshire discussed the challenges of measuring network responsiveness, emphasizing the need for metrics that accurately reflect application behavior. The conversation highlighted the balance between network and application measurements.

\subsubsection{draft-ietf-ippm-alt-mark-deployment}
G. Fioccola's draft focused on deployment strategies for alternate marking, with an invitation to collaborate on integrating hybrid two-step methodologies.

\subsection{Proposed Work}

\subsubsection{draft-ydt-ippm-alt-mark-yang}
G. Fioccola presented a YANG model for alternate marking, inviting further discussion on its applicability and potential for generating operational insights.

\subsubsection{draft-fioccola-ippm-on-path-active-measurements}
This draft, also by G. Fioccola, explored on-path active measurements, with updates on related drafts and their implications for network performance assessment.

\subsubsection{draft-zhang-ippm-stamp-mp}
L. Zhang's presentation addressed the challenges of ensuring comprehensive measurement coverage, with feedback on the feasibility of current device capabilities.

Meeting materials are available at \href{https://datatracker.ietf.org/meeting/121/materials.html}{IETF 121 Meeting Materials}.

\subsection{Conclusion}
The IPPM session at IETF 121 highlighted significant advancements in performance metrics, with discussions pointing towards a more integrated approach to network measurement and analysis. The proposed drafts and collaborative efforts suggest a promising direction for future developments in this field.



\newpage

\section{IP Security Maintenance and Extensions (IPsecME) WG}

\subsection{Attendees}

The IPsecME working group meeting was attended by representatives from prominent companies and institutions such as Huawei, Ericsson, NTT DOCOMO, Dell Technologies, and the UK NCSC, with a total attendance of over 50 participants.

\subsection{Meeting Discussions}

\subsubsection{Anti Replay Notification}

Wei Pan presented the \href{https://datatracker.ietf.org/doc/html/draft-pan-ipsecme-anti-replay-notification}{draft-pan-ipsecme-anti-replay-notification}, addressing the issue of operators disabling Anti-replay protection, leading to packet drops. The discussion highlighted the need for notifying peers about the disabling of this protection and considered the implications on security, especially in contexts like 3GPP where Anti-replay is mandated.

\subsubsection{Enhanced Encapsulating Security Payload}

Steffen Klassert introduced the \href{https://datatracker.ietf.org/doc/html/draft-klassert-ipsecme-eesp}{draft-klassert-ipsecme-eesp}, proposing a new protocol to replace the current ESP. The draft aims to optimize packet formats and introduce TLV options, potentially saving bytes in tunnel mode. The group discussed the draft's potential to modernize IPsec, with suggestions to refine the negotiation of session IDs and flow IDs.

\subsubsection{Sha 3}

Ben Salter discussed the \href{https://datatracker.ietf.org/doc/html/draft-salter-ipsecme-sha3}{draft-salter-ipsecme-sha3}, advocating for the use of SHA3 across IPsec implementations. The draft proposes a new PRF+ function to streamline cryptographic operations. The group debated the relevance of SHA 224 and the transition to KMAC, emphasizing the need for efficient cryptographic practices.

\subsubsection{FrodoKEM}

Wang Guilin presented the \href{https://datatracker.ietf.org/doc/html/draft-wang-hybrid-kem-ikev2-frodo}{draft-wang-hybrid-kem-ikev2-frodo}, focusing on improving IKEv2's efficiency in handling packet loss. The discussion centered on the limitations of current retransmission methods and the potential benefits of separating IKE and ESP processes.

\subsubsection{IKEv2 Negotiation for BEET mode}

Antony Antony's \href{https://datatracker.ietf.org/doc/html/draft-antony-ipsecme-iekv2-beet-mode}{draft-antony-ipsecme-iekv2-beet-mode} was discussed, with a call for working group adoption. The draft aims to enhance IKEv2 negotiations, and the group agreed to further discussions on the mailing list.

\subsubsection{Encrypted ESP Echo Protocol}

Antony Antony also presented the \href{https://datatracker.ietf.org/doc/html/draft-antony-ipsecme-encrypted-esp-ping}{draft-antony-ipsecme-encrypted-esp-ping}, which proposes enhancements to ESP echo protocols. The group considered the draft's utility in complementing IKEv2's DPD and discussed potential conflicts and solutions.

\subsubsection{PQC Auth}

Scott Fluhrer introduced the \href{https://datatracker.ietf.org/doc/html/draft-reddy-ipsecme-ikev2-pqc-auth}{draft-reddy-ipsecme-ikev2-pqc-auth}, seeking working group adoption. The draft focuses on integrating post-quantum cryptography into IKEv2 authentication processes, with an emphasis on swift adoption and implementation.

\subsubsection{PQT Hybrid Auth}

Jun Hu's \href{https://datatracker.ietf.org/doc/html/draft-hu-ipsecme-pqt-hybrid-auth}{draft-hu-ipsecme-pqt-hybrid-auth} was discussed, highlighting the draft's approach to hybrid authentication. The group debated the use of PPK and the implications of context strings in the authentication process.

\subsubsection{Lightweight auth for IP Header}

Linda Dunbar presented the \href{https://datatracker.ietf.org/doc/html/draft-dunbar-secdispatch-ligthtweight-authenticate}{draft-dunbar-secdispatch-ligthtweight-authenticate}, which was redirected from the Dispatch group to IPsecME for expert feedback. The draft proposes lightweight authentication for IP headers, with discussions on its applicability in gateway authentication and potential integration with existing security mechanisms.

Meeting materials are available at \href{https://meetings.conf.meetecho.com/ietf121/?group=ipsecme&short=&item=1}{IETF 121 IPsecME Meeting Materials}.



\newpage

\section{IVY Working Group (IVY)}

\subsection{Attendees}

\subsubsection{Overview}

The IVY Working Group session was attended by representatives from prominent companies and institutions such as Huawei, Nokia, Cisco Systems, and Deutsche Telekom, among others. The total attendance was approximately 40 participants, reflecting a diverse and engaged audience.

\subsection{Meeting Discussions}

\subsubsection{Introduction}

The session commenced with an introduction and update on the working group's status. A key point raised was the importance of responding to liaison statements, as silence is considered consent. This emphasizes the need for active communication within the IETF community.

\subsubsection{A YANG Data Model for Network Inventory}

The discussion, led by Chaode Yu, centered on the \href{https://datatracker.ietf.org/doc/html/draft-ietf-ivy-network-inventory-yang-03}{draft-ietf-ivy-network-inventory-yang-03}. Key topics included the modeling of ports and breakouts, and the potential need to update RFC 8348 to better align with current network modeling needs. The dialogue highlighted the necessity of coordinating with other working groups to ensure compatibility and avoid dependencies that could delay progress.

\subsubsection{A YANG Network Data Model of Network Inventory Software Extensions}

Bo Wu led the discussion on the \href{https://datatracker.ietf.org/doc/html/draft-wzwb-ivy-network-inventory-software-03}{draft-wzwb-ivy-network-inventory-software-03}. The group debated the scope of software inventory, particularly the inclusion of software images versus instances. The consensus leaned towards separating software updates from inventory to maintain clarity and focus.

\subsubsection{Evolving the ALMO/DMALMO Model Towards License/Entitlement Management}

Camilo Cardona facilitated the discussion on evolving the ALMO/DMALMO model, referencing \href{https://datatracker.ietf.org/doc/html/draft-palmero-ivy-ps-almo-02}{draft-palmero-ivy-ps-almo-02} and \href{https://datatracker.ietf.org/doc/html/draft-palmero-ivy-dmalmo-02}{draft-palmero-ivy-dmalmo-02}. The group debated the definition and modeling of features versus capabilities, with a focus on simplifying the model while ensuring it remains useful for managing entitlements.

\subsubsection{A YANG Data Model for Passive Network Inventory}

Aihua Guo presented the \href{https://datatracker.ietf.org/doc/html/draft-ygb-ivy-passive-network-inventory-00}{draft-ygb-ivy-passive-network-inventory-00}, prompting discussions on the integration of passive devices and cables within the inventory model. The group considered the potential need for separate models to address different network scenarios, while also discussing the relationship between inventory and topology.

\subsection{Meeting Materials}

All session materials, including slides and notes, are available at \href{https://datatracker.ietf.org/meeting/121/session/ivy}{IETF 121 IVY Session Materials}.




\newpage

\section{JMAP Working Group (JMAP)}

\subsection{Attendees}
\subsubsection{Overview}
The JMAP working group meeting was attended by representatives from prominent companies and institutions, including Fastmail, Apple, Huawei, and Meta Platforms, Inc., with a total attendance of 22 participants. Notable attendees included Bron Gondwana from Fastmail, Phillip Tao from Apple, and Murray Kucherawy from Meta Platforms, Inc.

\subsection{Meeting Discussions}

\subsubsection{With IESG}
The session with the IESG focused on the progress of JMAP sharing, calendars, and contacts. The sharing document is nearing publication, currently in the Auth48 stage. Discussions on calendars are ongoing, with comments to be addressed in the IESG review phase. Contacts are already with the editors for finalization.

\subsubsection{Portability and Tasks}
Joris Baum presented on the motivation for enhancing data portability through JMAP, emphasizing its suitability for a generic API. The discussion highlighted the need for extensions to facilitate data migration. Hans-Jörg Happel raised the importance of OAuth context in JMAPACCESS, suggesting further discussions on the mailing list. The group aims to publish a new minimal profile document soon. The tasks discussion noted the diversity in task management systems and the need for outreach to encourage adoption. Joris plans to update related specifications, integrating feedback and internal tasks.

\subsubsection{EmailPush}
Neil Jenkins introduced the concept of filtering push notifications to receive only relevant messages. This involves passing a filter to the server to trigger notifications based on specific criteria. The group plans to publish the emailpush document, with Bron Gondwana leading the call for adoption.

\subsubsection{Filenode}
The discussion on Filenode, led by Neil Jenkins, explored the potential for supporting features akin to WebDAV, with custom extensions for additional functionalities. The group considered the implications of changing blobId to a consistent `id` and the possibility of revision history for nodes. Hans-Jörg Happel shared insights from existing implementations, highlighting the need for features like modification time setting and file hashes. The group plans to discuss these aspects further post-adoption.

\subsubsection{AOB}
No additional business was discussed.

\subsubsection{Milestones}
The milestones were reviewed and updated to reflect the current progress and future goals.

Meeting materials are available at \href{https://example.com/jmap-meeting-materials}{JMAP Meeting Materials}.



\newpage

\section{Lightweight Authenticated Key Exchange (LAKE) [LAKE]}

\subsection{Attendees}
\subsubsection{Overview}
The LAKE working group meeting was attended by representatives from prominent companies and institutions such as Cisco Systems, Ericsson, Google, Huawei, and the University of Murcia. In total, the meeting saw participation from over 50 attendees, reflecting a diverse and engaged audience.

\subsection{Meeting Discussions}

\subsubsection{Presentation: \href{https://datatracker.ietf.org/doc/html/draft-ietf-lake-authz}{draft-ietf-lake-authz}}
Geovane Fedrecheski presented updates on the draft, highlighting new strategies for supporting EDHOC reverse message flow and discussing implementation results on platforms like nRF52833. The discussion emphasized the separation of authorization and onboarding processes, with plans to update the draft to support both flows.

\subsubsection{Presentation: \href{https://datatracker.ietf.org/doc/html/draft-ietf-lake-edhoc-impl-cons}{draft-ietf-lake-edhoc-impl-cons}}
Marco Tiloca discussed improvements in trust models and guidelines for EDHOC with CoAP. The presentation focused on clarifying exceptions to no-learning rules and enhancing security considerations. Feedback was solicited for further refinement of the draft.

\subsubsection{Presentation: \href{https://datatracker.ietf.org/doc/html/draft-song-lake-ra}{draft-song-lake-ra}}
Yuxuan Song introduced a restructured draft with new EAD items for remote attestation. The presentation covered mutual authentication updates and evaluation metrics, with a call for working group adoption.

\subsubsection{Presentation: \href{https://datatracker.ietf.org/doc/html/draft-amsuess-core-edhoc-grease}{draft-amsuess-core-edhoc-grease}}
Christian Amsüss proposed using extension points to prevent protocol ossification, as advised by RFC 9170. The draft was considered complete, pending working group decision on adoption.

\subsubsection{Presentation: \href{https://datatracker.ietf.org/doc/html/draft-lopez-lake-edhoc-psk}{draft-lopez-lake-edhoc-psk}}
Elsa López Pérez presented on pre-shared key authentication, detailing message flow and performance metrics. The draft seeks working group adoption to proceed with formal analysis.

\subsubsection{Presentation: \href{https://datatracker.ietf.org/doc/html/draft-tiloca-lake-app-profiles}{draft-tiloca-lake-app-profiles}}
Marco Tiloca discussed the need for EDHOC application profiles to manage negotiable parameters. The draft suggests initial profiles and seeks feedback for further development.

\subsubsection{Presentation: \href{https://datatracker.ietf.org/doc/html/draft-serafin-lake-ta-hint}{draft-serafin-lake-ta-hint}}
Göran Selander presented on trust anchor hints, proposing integration with application profiles. The draft's necessity was questioned, with a suggestion to incorporate its concepts into existing profiles.

Meeting materials are available at \href{https://datatracker.ietf.org/meeting/121/materials/slides-121-lake}{IETF 121 LAKE Meeting Materials}.




\newpage

\section{LAMPS Working Group (LAMPS)}

\subsection{Attendees}
\subsubsection{Overview}
The LAMPS Working Group meeting at IETF 121 was attended by representatives from prominent organizations such as Google, NSA, Ericsson, and Cisco Systems, among others, with a total attendance of over 70 participants. The diverse representation underscores the broad interest and collaborative effort in advancing cryptographic standards.

\subsection{Meeting Discussions}

\subsubsection{Recently Published RFCs}
The group reviewed the recently published RFCs, including \href{https://datatracker.ietf.org/doc/html/draft-ietf-lamps-ocsp-nonce-update}{draft-ietf-lamps-ocsp-nonce-update} (RFC 9654), which updates the OCSP nonce handling, and \href{https://datatracker.ietf.org/doc/html/draft-ietf-lamps-x509-policy-graph}{draft-ietf-lamps-x509-policy-graph} (RFC 9618), which introduces a new policy graph for X.509 certificates. These publications mark significant progress in enhancing security protocols.

\subsubsection{RFC Editor Queue}
Several drafts are currently in the RFC Editor queue, including \href{https://datatracker.ietf.org/doc/html/draft-ietf-lamps-e2e-mail-guidance}{draft-ietf-lamps-e2e-mail-guidance}, which awaits the advancement of the header-protection draft. The discussion highlighted the interdependencies between drafts and the strategic importance of progressing them in tandem.

\subsubsection{With IESG}
The draft \href{https://datatracker.ietf.org/doc/html/draft-ietf-lamps-cert-binding-for-multi-auth}{draft-ietf-lamps-cert-binding-for-multi-auth} is under review, with discussions focusing on resolving existing comments. The \href{https://datatracker.ietf.org/doc/html/draft-ietf-lamps-header-protection}{draft-ietf-lamps-header-protection} was debated for its implications on email security, particularly concerning spoofing and phishing attacks.

\subsubsection{Active PKIX-related Documents}
The group discussed the \href{https://datatracker.ietf.org/doc/html/draft-ietf-lamps-dilithium-certificates}{draft-ietf-lamps-dilithium-certificates} and \href{https://datatracker.ietf.org/doc/html/draft-ietf-lamps-kyber-certificates}{draft-ietf-lamps-kyber-certificates}, emphasizing the need for interoperability testing and consensus on pre-hash usage. The discussions are pivotal in shaping the future of post-quantum cryptography standards.

\subsubsection{Active S/MIME-related Documents}
The \href{https://datatracker.ietf.org/doc/html/draft-ietf-lamps-cms-kyber}{draft-ietf-lamps-cms-kyber} and \href{https://datatracker.ietf.org/doc/html/draft-ietf-lamps-cms-sphincs-plus}{draft-ietf-lamps-cms-sphincs-plus} drafts were reviewed, with a focus on ensuring compliance with emerging cryptographic requirements and enhancing the robustness of S/MIME protocols.

\subsubsection{Special Topic: EUF-CMA for CMS SignedData}
A special topic discussion on EUF-CMA for CMS SignedData explored the implications of cryptographic security proofs on the CMS framework, highlighting the need for rigorous validation processes.

\subsubsection{Under Consideration for Adoption}
Drafts such as \href{https://datatracker.ietf.org/doc/html/draft-wang-lamps-root-ca-cert-rekeying}{draft-wang-lamps-root-ca-cert-rekeying} and \href{https://datatracker.ietf.org/doc/html/draft-harvey-cfrg-mtl-mode}{draft-harvey-cfrg-mtl-mode} were considered for adoption, reflecting the group's proactive approach in addressing emerging security challenges.

Meeting materials are available at \href{https://www.ietf.org/proceedings/121/lamps.html}{LAMPS WG Meeting Materials}.




\newpage

\section{Link State Routing (LSR) Working Group [LSR]}

\subsection{Attendees Overview}
\subsubsection{Participants}
The meeting was attended by representatives from prominent companies and institutions, including Cisco, Nokia, Juniper Networks, Huawei Technologies, and ZTE, with a total attendance of over 60 participants.

\subsection{Meeting Discussions}

\subsubsection{Advertising Infinity Links in OSPF}
Liyan Gong presented the draft \href{https://datatracker.ietf.org/doc/html/draft-ietf-lsr-ospf-ls-link-infinity}{draft-ietf-lsr-ospf-ls-link-infinity}, which proposes using infinity link metrics in OSPF to align with ISIS and simplify adoption by avoiding extended LSA support. The discussion highlighted backward compatibility and the potential for a Working Group Last Call (WGLC).

\subsubsection{Flexible Algorithms Exclude Node}
The draft \href{https://datatracker.ietf.org/doc/html/draft-gong-lsr-flex-algo-exclude-node}{draft-gong-lsr-flex-algo-exclude-node} was debated, with differing opinions on its necessity. While some argued it simplifies deployment by excluding categories of nodes, others saw no operational advantage. Further clarification was requested on the mailing list.

\subsubsection{IS-IS Distributed Flooding Reduction}
Tony P presented \href{https://datatracker.ietf.org/doc/html/draft-ietf-lsr-distoptflood}{draft-ietf-lsr-distoptflood}, focusing on leaderless algorithm enablement. The discussion centered on the independence of algorithm enablement from the algorithm itself, with a consensus call planned to determine the Working Group's interest in pursuing leaderless signaling.

\subsubsection{Optional IS-IS Fragment Timestamping}
The draft \href{https://datatracker.ietf.org/doc/html/draft-rigatoni-lsr-isis-fragment-timestamping}{draft-rigatoni-lsr-isis-fragment-timestamping} was proposed for WG adoption. The use case involving Traffic Engineering (TE) was discussed, with requests for further elaboration.

\subsubsection{IGP Flex Soft Dataplane}
Peter Psenak discussed \href{https://datatracker.ietf.org/doc/html/draft-ginsberg-lsr-flex-soft-dataplane}{draft-ginsberg-lsr-flex-soft-dataplane}, emphasizing its necessity for multicast. The discussion included requests for additional examples to clarify the limitations of current algorithm encodings.

\subsubsection{Source Prefix Advertisement for Intra-domain SAVNET}
Lancheng Qin presented drafts \href{https://datatracker.ietf.org/doc/html/draft-li-savnet-source-prefix-advertisement}{draft-li-savnet-source-prefix-advertisement} and \href{https://datatracker.ietf.org/doc/html/draft-li-lsr-igp-based-intra-domain-savnet}{draft-li-lsr-igp-based-intra-domain-savnet}. The discussion focused on the applicability of IGP for intra-domain SAVNET, with concerns about misconfiguration and the need for better-defined use cases.

\subsubsection{Intra-domain SAVNET Support via IGP}
Shengnan Yue's presentation on \href{https://datatracker.ietf.org/doc/html/draft-cheng-lsr-adv-savnet-capbility}{draft-cheng-lsr-adv-savnet-capbility} and \href{https://datatracker.ietf.org/doc/html/draft-cheng-savnet-intra-domain-sav-igp}{draft-cheng-savnet-intra-domain-sav-igp} was met with skepticism regarding its necessity within the LSR. The consensus was to await the stabilization of the SAVNET intra-domain architecture document before proceeding.

\subsubsection{Open Discussions}
The session concluded with open discussions, addressing various technical queries and clarifications. The chairs emphasized the importance of continued dialogue on the mailing list to refine proposals and align on strategic directions.

Meeting materials are available at \href{https://datatracker.ietf.org/meeting/121/session/lsr}{IETF 121 LSR Session Materials}.



\newpage

\section{Link State Vector Routing (LSVR)}

\subsection{Attendees}
\subsubsection{Overview}
The LSVR working group meeting was attended by representatives from prominent companies and institutions, including Cisco Systems, Huawei Technologies, Juniper Networks, and Arrcus, Inc. A total of 18 participants contributed to the discussions, reflecting a diverse range of expertise and perspectives.

\subsection{Meeting Discussions}

\subsubsection{Usage and Applicability of Link State Vector Routing in Data Centers}
The session began with a discussion on the \href{https://datatracker.ietf.org/doc/html/draft-ietf-lsvr-applicability-13}{draft-ietf-lsvr-applicability-13}, focusing on its usage in data centers. The working group concluded that the document is ready to be sent to the IESG for publication, pending any final comments. This marks a significant step towards standardizing LSVR's applicability, potentially enhancing data center network efficiency.

\subsubsection{Proposed Update to BGP Link-State SPF NLRI Selection Rules}
Jie Dong presented the \href{https://datatracker.ietf.org/doc/html/draft-dong-lsvr-bgp-spf-selection-01}{draft-dong-lsvr-bgp-spf-selection-01}, which proposes updates to the BGP Link-State SPF NLRI selection rules. The discussion highlighted the need to avoid redundant advertisements, which could optimize network performance. The group debated the implications of sequence number handling and agreed to consider this document as an optional optimization, potentially influencing future BGP implementations.

\subsubsection{Applying BGP-LS Segment Routing Extensions to BGP-LS SPF}
Li Zhang introduced the \href{https://datatracker.ietf.org/doc/html/draft-li-lsvr-bgp-spf-sr-00}{draft-li-lsvr-bgp-spf-sr-00}, exploring the application of BGP-LS Segment Routing extensions to BGP-LS SPF. The group discussed the integration of MPLS and SRv6 capabilities, with a consensus to further evaluate the draft's alignment with the current charter. This discussion could lead to significant advancements in segment routing, enhancing routing flexibility and efficiency.

\subsubsection{Future Work and Recharter Considerations}
The meeting concluded with an open mic session, where participants discussed potential topics for future work and recharter considerations. The integration of SR extensions into LSVR was highlighted as a promising area for development, suggesting a strategic shift towards more robust and scalable routing solutions.

Meeting materials are available at \href{https://datatracker.ietf.org/meeting/121/session/lsvr}{IETF 121 LSVR Meeting Materials}.



\newpage

\section{MAILMAINT Working Group (MAILMAINT)}

\subsection{Attendees}
\subsubsection{Overview}
The MAILMAINT working group meeting was attended by representatives from prominent companies and institutions, including Fastmail, Yahoo, Apple Inc., ICANN, and Meta Platforms, Inc. The total attendance was approximately 50 participants, reflecting a diverse range of stakeholders in the email technology ecosystem.

\subsection{Meeting Discussions}

\subsubsection{Active Drafts}
The session began with discussions on active drafts. John Levine presented the \href{https://datatracker.ietf.org/doc/html/draft-ietf-mailmaint-expires}{draft-ietf-mailmaint-expires}, which faced skepticism regarding implementation from major providers like Yahoo. The group debated the relevance of the "Expires" header, with some participants noting existing implementations and others questioning its practical utility. The consensus was to seek further commitment from implementers before proceeding to last call.

David Weekly introduced the \href{https://datatracker.ietf.org/doc/html/draft-ietf-mailmaint-wrong-recipient}{draft-ietf-mailmaint-wrong-recipient}, proposing a mechanism for handling misdirected emails. The discussion highlighted the potential to combine this with existing unsubscribe mechanisms, though concerns about semantic clarity and implementation challenges were raised.

Ben Bucksch's presentation on \href{https://datatracker.ietf.org/doc/html/draft-ietf-mailmaint-autoconfig}{draft-ietf-mailmaint-autoconfig} emphasized the need for standardized email client configuration. The draft aims to streamline client setup through DNS-based service discovery, with a focus on implicit TLS for security.

\subsubsection{Proposed Work}
Arnt Gulbrandsen discussed the \href{https://datatracker.ietf.org/doc/draft-gulbrandsen-smtputf8-syntax}{SMTPUTF8 address syntax} and \href{https://datatracker.ietf.org/doc/draft-gulbrandsen-smtputf8-nice-addresses}{Nice Email Addresses for SMTPUTF8}, advocating for a more constrained UTF8 syntax in email addresses. The proposal received support for its potential to simplify internationalization while maintaining flexibility.

Neil Jenkins presented on \href{https://datatracker.ietf.org/doc/draft-jenkins-mail-keywords}{IMAP/JMAP keyword registration}, aiming to document existing practices and enhance interoperability. The discussion underscored the importance of clear semantic definitions to facilitate widespread adoption.

The \href{https://datatracker.ietf.org/doc/draft-jenkins-oauth-public}{OAuth Profile for Open Public Clients} was also introduced by Neil Jenkins, addressing the need for a standardized OAuth implementation for email clients. The proposal seeks to enhance security and interoperability, though concerns about its scope and complexity were noted.

\subsubsection{Topics of Interest}
Bron Gondwana's presentation on \href{https://datatracker.ietf.org/doc/draft-gondwana-dkim2-motivation}{DKIM2} explored the limitations of the current DKIM standard and proposed a framework for its replacement. The discussion highlighted the need for improved security and reliability in email authentication.

\subsection{Meeting Materials}
Meeting materials, including slides and detailed notes, are available at \href{https://www.ietf.org/proceedings/121/mailmaint.html}{IETF 121 MAILMAINT materials}.

The discussions during the MAILMAINT session were pivotal in shaping the future direction of email standards, with a focus on enhancing security, interoperability, and user experience. The outcomes suggest a strategic shift towards more robust and user-friendly email protocols, with several drafts poised for further development and potential adoption.



\newpage

\section{MASQUE Working Group (MASQUE)}

\subsection{Attendees}
\subsubsection{Overview}
The MASQUE Working Group meeting was attended by representatives from prominent companies and institutions such as Google, Apple, Cisco, Ericsson, and Meta Platforms, Inc., with a total attendance of over 80 participants.

\subsection{Meeting Discussions}

\subsubsection{QUIC-Aware Proxying Using HTTP}
Tommy Pauly and Eric Rosenberg presented updates from version 3 to 4 of the \href{https://datatracker.ietf.org/doc/draft-ietf-masque-quic-proxy/}{draft-ietf-masque-quic-proxy}. The discussion highlighted the utility of 'blocked signals' for detecting issues in QUIC, with a consensus on filing a new issue for further exploration. The group agreed that the active attack on scramble does not necessitate protocol changes but requires documentation. A show of hands indicated significant engagement with the draft, suggesting robust interest and implementation efforts.

\subsubsection{Proxying Listener UDP in HTTP}
Abhijit Singh discussed mostly editorial changes in the \href{https://datatracker.ietf.org/doc/draft-ietf-masque-connect-udp-listen/}{draft-ietf-masque-connect-udp-listen}. The path forward includes achieving interoperability between implementations before proceeding to Working Group Last Call (WGLC). The session underscored the need for closing issues and conducting interop tests, with multiple participants indicating aspirational implementations.

\subsubsection{Proxying Ethernet in HTTP}
Alejandro Sedeño addressed the \href{https://datatracker.ietf.org/doc/draft-ietf-masque-connect-ethernet/}{draft-ietf-masque-connect-ethernet}, focusing on layer separation and congestion control. The discussion revealed a preference for documenting behavior when congestion occurs, with a suggestion to seek early review from INTAREA. The dialogue emphasized the complexity of congestion control in non-IP applications, reflecting broader implications for MASQUE's technical strategy.

\subsubsection{DNS Configuration for Proxying IP in HTTP}
David Schinazi presented updates aligning the \href{https://datatracker.ietf.org/doc/draft-ietf-masque-connect-ip-dns/}{draft-ietf-masque-connect-ip-dns} with SVCB documents. The conversation centered on the challenges of internal and search domains, with a suggestion to align with IKEv2 split DNS practices. The group agreed to initiate discussions on the dnsops mailing list to refine the draft's approach to DNS configurations.

Meeting materials are available at \href{https://datatracker.ietf.org/meeting/121/materials/masque}{IETF 121 MASQUE Materials}.



\newpage

\section{MBONED (MBONED)}

\subsection{Attendees}
\subsubsection{Overview}
The MBONED session at IETF 121 in Dublin saw participation from a diverse group of industry leaders and institutions, including Juniper, Cisco, Nokia, Jisc, Garmin, H3C, Akamai, China Mobile, Apple, NICT, Futurewei USA, Deutsche Telekom, Cloudflare, UCLouvain, and TU Berlin. The total attendance was approximately 35 participants.

\subsection{Meeting Discussions}

\subsubsection{Status of WG Items}
The session began with a review of the working group's current items. The chairs highlighted that \href{https://datatracker.ietf.org/doc/html/draft-ietf-mboned-multicast-yang-model}{draft-ietf-mboned-multicast-yang-model} and \href{https://datatracker.ietf.org/doc/html/draft-ietf-mboned-redundant-ingress-failover}{draft-ietf-mboned-redundant-ingress-failover} are ready for Working Group Last Call (WGLC). Additionally, \href{https://datatracker.ietf.org/doc/html/draft-zzhang-mboned-non-source-routed-sr-mcast}{draft-zzhang-mboned-non-source-routed-sr-mcast} was noted as stable and mature, with a request for WG adoption to be made on the mailing list.

\subsubsection{Adaptive Unicast to Multicast Forwarding}
Yisong Liu presented \href{https://datatracker.ietf.org/doc/html/draft-liu-mboned-adaptive-utom}{draft-liu-mboned-adaptive-utom}, discussing the potential advantages of multicast over traditional CDN methods. The discussion highlighted the need for more detailed use-case scenarios and coordination between sending servers and networks. The conversation was encouraged to continue on the mailing list due to time constraints.

\subsubsection{IPv6 Multicast in BSV Blockchain Network}
Jake Jones explored the use of IPv6 multicast within the BSV Blockchain Network, emphasizing its role in facilitating communication between network nodes. The session considered the potential for drafting a document on multicast deployment use-cases within the MBONED charter.

\subsubsection{Flexicast Extensions for QUIC}
Louis Navarre discussed \href{https://datatracker.ietf.org/doc/html/draft-navarre-quic-flexicast}{draft-navarre-quic-flexicast}, focusing on the integration of multicast as an optimization for QUIC. The feasibility of implementing this in browsers was debated, with concerns about privacy and the complexity of current QUIC implementations.

\subsubsection{Multicast QUIC}
Max Franke's presentation on \href{https://datatracker.ietf.org/doc/html/draft-jholland-quic-multicast-05}{draft-jholland-quic-multicast-05} addressed the potential for multicast as a service. The discussion centered on the demand for standardization and the challenges of proprietary solutions by companies.

\subsubsection{Survey on the State of SSM Support}
Max Franke also led a survey on SSM support, noting the lack of API calls for JOIN\_SOURCE\_GROUP on various operating systems. The discussion was cut short and will continue on the mailing list.

\subsubsection{Bandwidth Aware Multicast}
The session briefly touched on bandwidth-aware multicast, with references to existing work in the area. Further details were deferred due to time limitations.

\subsubsection{Optimizing Multicast Traffic Distribution on the Local LAN}
Nate Karstens and Joseph Huang introduced a problem statement regarding multicast traffic distribution, with a more detailed discussion scheduled for the PIM WG session.

Meeting materials, including notes and recordings, are available at \href{https://notes.ietf.org/notes-ietf-121-mboned#}{Etherpad Notes}, \href{https://zulip.ietf.org/#narrow/stream/101-mboned/topic/ietf-121}{Chat Log}, and \href{https://meetecho-player.ietf.org/playout/?session=IETF121-MBONED-20241106-1500}{Full Session Recording}.



\newpage

\section{MIMI (IETF 121)}

\subsection{Attendees}

The MIMI working group meeting was attended by representatives from prominent companies and institutions such as Google, Cloudflare, NTT, Deutsche Telekom, and the Federal Office for Information Security, among others. The total attendance was 40 participants.

\subsection{Meeting Discussions}

\subsubsection{MIMI Protocol}

Richard Barnes presented the \href{https://www.ietf.org/archive/id/draft-ietf-mimi-protocol-02.html}{MIMI Protocol}, focusing on the necessity of an update path for GCE due to potential security implications. The discussion highlighted the need for a security rationale to prevent vulnerabilities. Jonathan Hoyland suggested maintaining distinct proposals to avoid binding MIMI directly to MLS properties, which Rohan Mahy agreed to consider.

\subsubsection{Content Format}

Rohan Mahy discussed the \href{https://www.ietf.org/archive/id/draft-ietf-mimi-content-04.html}{Content Format}, addressing concerns about message franking and its implications for message integrity. The dialogue emphasized the importance of ensuring that franking schemes are secure, with suggestions for further review by the CFRG.

\subsubsection{Room Policy}

Rohan Mahy also presented on \href{https://datatracker.ietf.org/doc/draft-mahy-mimi-room-policy/01/}{Room Policy}, exploring role-based capabilities and enforcement by hubs. The group debated the need for capability negotiation and the potential for adoption, with a poll indicating general support for the draft.

\subsubsection{Discovery Requirements}

Femi Olumofin led the discussion on \href{https://www.ietf.org/archive/id/draft-interop-mimi-discovery-requirements-01.html}{Discovery Requirements}, focusing on the mapping of CSIP to MSP and the implications for privacy and security. The conversation underscored the complexity of ensuring consistent and secure mappings, with suggestions for further refinement of the requirements.

\subsubsection{Metadata Minimization}

Konrad Kohbrok presented on \href{https://datatracker.ietf.org/doc/draft-kohbrok-mimi-metadata-minimalization/}{Metadata Minimization}, emphasizing the need to limit metadata exposure and discussing potential strategies for achieving this goal.

Meeting materials are available at \href{https://datatracker.ietf.org/meeting/121/session/mimi}{IETF 121 MIMI Session}.

The discussions concluded with a consensus on the need for further iterations on drafts and potential interim meetings to maintain momentum, with a focus on addressing identified gaps and refining proposals for adoption.



\newpage

\section{Machine Learning for Audio Coding (mlcodec)}

\subsection{Attendees}
The meeting was attended by representatives from prominent companies and institutions, including Amazon, Cisco, Google, Netflix, Ericsson, and Meta Platforms, Inc., among others. In total, there were 18 attendees, showcasing a diverse range of expertise and interest in the field of machine learning for audio coding.

\subsection{Meeting Discussions}

\subsubsection{Opus Extension Mechanism}
Timothy Terriberry presented the \href{https://datatracker.ietf.org/doc/html/draft-ietf-mlcodec-opus-extension}{draft-ietf-mlcodec-opus-extension}, which proposes an extension mechanism for Opus. The discussion highlighted the potential benefits of this mechanism for future extensions, despite concerns about middleboxes needing to parse these extensions. The consensus was to proceed with the proposed changes, which could facilitate more flexible audio coding solutions.

\subsubsection{Deep REDundancy}
Jean-Marc Valin introduced the \href{https://datatracker.ietf.org/doc/html/draft-ietf-mlcodec-opus-dred}{draft-ietf-mlcodec-opus-dred}, focusing on redundancy in audio coding. The dialogue centered on the archival of files and the inclusion of SHA-256 hashes in the RFC to ensure data integrity. This approach aims to enhance the robustness of audio coding by providing reliable redundancy mechanisms.

\subsubsection{Speech Coding Enhancements}
Jan Buethe discussed \href{https://datatracker.ietf.org/doc/html/draft-buethe-opus-speech-coding-enhancement}{draft-buethe-opus-speech-coding-enhancement}, which seeks to improve speech coding techniques. The working group acknowledged the potential of these enhancements to be included in future versions of libopus, emphasizing the importance of maintaining flexibility for future advancements in speech coding technology.

\subsubsection{Scalable Quality Extension}
Jean-Marc Valin proposed a new work item under \href{https://datatracker.ietf.org/doc/html/charter-ietf-mlcodec}{charter-ietf-mlcodec deliverable \#4}, focusing on scalable quality extensions. The discussion explored the feasibility of implementing multiple layers and the potential for scaling to lossless audio. The proposal was well-received, and a draft will be developed for further consideration.

\subsubsection{Optional Simplifications}
The final discussion, led by Jean-Marc Valin, revolved around \href{https://datatracker.ietf.org/doc/html/charter-ietf-mlcodec}{charter-ietf-mlcodec new deliverable \#5}, which suggests optional simplifications for audio coding. The group debated the implications of these changes on conformance tests and the potential for signaling these simplifications. The proposal aims to streamline audio coding processes without compromising compatibility.

Meeting materials are available at \href{https://meetings.conf.meetecho.com/ietf121/?session=33539}{this link}.



\newpage

\section{Messaging Layer Security (MLS)}

\subsection{Attendees}
The MLS Working Group session saw participation from over 50 individuals, representing leading organizations such as Apple, Cisco, Google, and the UK NCSC. Distinguished attendees included Raphael Robert from Phoenix RD, Britta Hale from NPS, and Richard Barnes from Cisco.

\subsection{Meeting Discussions}

\subsubsection{Chairs' Update}
Chairs Sean Turner and Nick Sullivan provided an update on the working group's progress, highlighting the absence of agenda bashes and setting the stage for the day's discussions.

\subsubsection{MLS Extensions}
Raphael Robert presented updates on the \href{https://datatracker.ietf.org/doc/html/draft-ietf-mls-extensions}{draft-ietf-mls-extensions}, emphasizing the numerous extensions under development. The discussion focused on the necessity of these extensions for enhancing the protocol's flexibility and security.

\subsubsection{Negotiation Mechanisms}
Rohan Mahy discussed the need for negotiation mechanisms within the MLS protocol, addressing concerns raised by Richard Barnes regarding the stack's awareness of extensions. The proposal aims to ensure that the stack can perform necessary functions for applications, with a potential shift towards virtual interims for further exploration.

\subsubsection{Combiners (Post-Quantum)}
Britta Hale introduced an alternative proposal for post-quantum combiners, advocating for a hybrid approach that balances simplicity and security. The discussion underscored the importance of addressing non-repudiation and authenticity, with a consensus on the need for further exploration at the HPKE level.

\subsubsection{Cipher Suites}
Rohan Mahy briefly mentioned the ongoing maturation of drafts related to plain ML-KEM and mixed cipher suites, indicating that these developments are crucial for the protocol's evolution.

\subsubsection{AppSync/GCEDiff}
Richard Barnes presented on the \href{https://datatracker.ietf.org/doc/html/draft-barnes-mls-appsync}{draft-barnes-mls-appsync}, expressing a preference for the GCEDiff approach. The discussion explored potential overlaps with encrypted group contexts and the need for collaboration on future versions.

\subsubsection{Associated Parties}
Konrad Kohbrok discussed the concept of associated parties within MLS, proposing a mini-MLS key schedule. The dialogue highlighted the proposal's security implications and the necessity for simplicity in implementation.

\subsubsection{Light Clients}
Richard Barnes addressed the challenges and security considerations associated with light clients, emphasizing the need for thorough security reviews to prevent potential vulnerabilities.

\subsubsection{Splittable Commits}
Jöel Mularczyk sought feedback on the \href{https://datatracker.ietf.org/doc/html/draft-mularczyk-mls-splitcommit}{draft-mularczyk-mls-splitcommit}, discussing the security proofs and potential benefits of the proposed approach.

\subsubsection{Semi-Private Messaging}
Rohan Mahy explored the concept of semi-private messaging, suggesting its integration with associated parties to enhance privacy and security.

\subsubsection{Additional Wire Formats}
Raphael Robert proposed additional wire formats, garnering support from attendees. The discussion focused on the necessity of separate secret trees for forward secrecy.

Meeting materials, including slides and notes, are available at \href{https://datatracker.ietf.org/meeting/121/materials/slides-121-mls-chairs-slides-00}{meeting materials}.

\subsection{Next Steps}
The working group plans to initiate adoption calls for the PQ Combiners proposal, reflecting a strategic shift towards enhancing the protocol's resilience against quantum threats. These developments are expected to significantly contribute to the field by ensuring robust security measures in the evolving landscape of messaging protocols.




\newpage

\section{MODPOD - Moderation Procedures (MODPOD)}

\subsection{Attendees}
\subsubsection{Overview}
The meeting was attended by representatives from prominent organizations such as Google, Mozilla, Ericsson, and Cisco, among others. In total, there were 52 participants, including key figures from the IETF Administration LLC and various academic institutions.

\subsection{Meeting Discussions}

\subsubsection{Chairs Introduction}
The session commenced with an introduction by the chairs, outlining the working group's input documents and goals. A significant focus was placed on addressing the cumbersome nature of current PR actions and exploring whether moderation procedures could be streamlined to be more discreet and efficient. Roman Daniliw emphasized the critical nature of these discussions for the IETF.

\subsubsection{Document Update by Eliot Lear}
Eliot Lear provided an update on the document status, highlighting the consensus on approaches that allow for consistent moderation across venues. The authors of \href{https://datatracker.ietf.org/doc/html/draft-ecahc-moderation}{draft-ecahc-moderation} and \href{https://datatracker.ietf.org/doc/html/draft-lear-bcp83-replacement}{draft-lear-bcp83-replacement} are considering merging the best elements of each. Despite differences, both drafts advocate for a moderation team with broad discretion and propose a graduated response and transparency model to avoid binary outcomes.

\subsubsection{Discussion}
The group expressed support for the direction of the drafts and their potential consolidation. Key points included:
- Concerns about the scope of combining documents, particularly if replacing BCP83, which focuses on mailing list bans, might require separate documentation.
- The role of the Ombudsteam was questioned, with Roman noting that updating RFC 7776 is out of scope.
- A consensus emerged that overly detailed procedures could be counterproductive, advocating instead for defining responsibilities and allowing discretion.
- The utility of various moderation tools was acknowledged, with emphasis on warnings and rapid response.
- Concerns were raised about potential biases in moderation by chairs or mod-teams.
- The possibility of moderation as an LLC function was debated, with a preference for community-led moderation but with routine tasks potentially delegated to the LLC.
- The idea of a process experiment was discussed, though it was agreed that document work should precede any experimental implementation.

\subsubsection{Consensus Call}
A consensus call was conducted to determine support for using the three documents as a starting point. The results were 35 in favor, 1 against, and 4 with no opinion. A further call to consolidate using the ecahc draft as a baseline resulted in 22 in favor, 1 against, and 14 with no opinion. The conclusion was that either approach could proceed with participant support.

Meeting materials are available at \href{https://datatracker.ietf.org/meeting/121/materials}{IETF 121 Meeting Materials}.


\newpage

\section{Media Operations (MOPS) [MOPS]}

\subsection{Attendees Overview}

The MOPS session at IETF 121 in Dublin saw participation from a diverse group of industry leaders and academic institutions, with a total attendance of 35 individuals. Notable organizations represented included Akamai, Huawei, Meta Platforms Inc., Comcast, Cisco, and the Technical University of Munich. This diverse representation underscores the collaborative effort to address media operations challenges across different sectors.

\subsection{Meeting Discussions}

\subsubsection{Working Group Documents}

The session began with an update on the status of recent working group documents. A new document titled \href{https://datatracker.ietf.org/doc/html/draft-deen-mops-network-overlay-impacts}{draft-ietf-mops-network-overlay-impacts} was introduced by Sanjay Mishra. The discussion highlighted the need for specificity in identifying companies and services involved in network overlay issues, while balancing privacy concerns. The potential for suggesting APIs to address these challenges was also explored, although the group is not chartered for protocol development.

\subsubsection{Common Access Token (CAT)}

Will Law presented on the Common Access Token (CAT), emphasizing its role in enhancing content selection options. The presentation sparked a dialogue on the integration of geohash claims and privacy considerations, with references to ongoing work in the COSE group, as detailed in \href{https://datatracker.ietf.org/doc/draft-lemmons-cose-composite-claims}{draft-lemmons-cose-composite-claims}. The discussion underscored the importance of CAT in the context of media operations, particularly its potential application in the `moq` protocol.

\subsubsection{SVTA Update}

Glenn Deen provided an update on the Streaming Video Technology Alliance (SVTA), noting its involvement with the Metaverse Standard Forums. The session highlighted the SVTA's role in advancing immersive video standards, with Sanjay Mishra confirming his leadership in the SVTA Immersive Video Group. This collaboration is poised to influence future media operations standards significantly.

\subsubsection{Any Other Business (AoB)}

The meeting concluded with no additional business, allowing for a timely adjournment.

Meeting materials, including presentation slides, are available at \href{https://datatracker.ietf.org/meeting/121/materials/slides-121-mops-mops-chair-slides-ietf-121-01}{MOPS Chair Slides}.




\newpage

\section{Media Over QUIC (MoQ) [MoQ]}

\subsection{Attendees}
\subsubsection{Overview}
The MoQ working group session was attended by representatives from prominent companies and institutions, including Google, Cisco, Ericsson, Meta, and Apple, among others. In total, there were over 80 participants, showcasing a diverse range of expertise and interest in the development of media transport over QUIC.

\subsection{Meeting Discussions}

\subsubsection{Administrivia (Chairs)}
The session commenced with a brief overview of the meeting logistics and agenda, facilitated by the Chairs. The group discussed potential incentives for meeting scribes and reviewed the "Note Well" and "Note Really Well" notices to ensure professional collaboration. Meeting materials are available \href{https://datatracker.ietf.org/meeting/121/materials/slides-121-moq-chair-slides-00}{here}.

\subsubsection{Interop (Mike / Mathis)}
Mathis presented the results of recent interoperability tests, highlighting the need for additional tests involving relays. The discussion emphasized the importance of media interop tests, with suggestions to use WARP and catalog formats as baselines. The session underscored the necessity of addressing interoperability issues, particularly with LOC, to enhance the robustness of the protocol.

\subsubsection{MoQT Updates Since Vancouver (Ian)}
Ian provided an update on the MoQT protocol, focusing on significant changes since the last meeting. A key update involved the use of track namespaces as tuples in the Announce mechanism, enhancing the protocol's flexibility. For detailed changes, refer to the \href{https://datatracker.ietf.org/meeting/121/materials/slides-121-moq-moq-transport-updates-since-vancouver-00}{presentation slides}.

\subsubsection{JOIN (Will Law)}
Will proposed redefining the SUBSCRIBE mechanism to focus solely on future objects, introducing a new JOIN API. The discussion explored the implications of this change, with participants debating the merits of FETCH versus SUBSCRIBE for different use cases. The proposal received significant interest, with a majority expressing support for further exploration.

\subsubsection{Priority (Alan / Victor)}
Alan and Victor discussed updates to MoQT priorities based on implementation experiences. The group debated two proposals: maintaining explicit subscription priority or eliminating track priority in favor of subscriber and subgroup priorities. The consensus leaned towards the latter, with a show of hands indicating strong support for proposal 2. Further review and discussion are planned for the next session.

\subsubsection{WARP + Catalog Merge (Will)}
Will discussed the integration of WARP and catalog formats, referencing \href{https://datatracker.ietf.org/doc/html/draft-law-moq-warpstreamingformat}{draft-law-moq-warpstreamingformat} and \href{https://datatracker.ietf.org/doc/html/draft-ietf-moq-catalogformat}{draft-ietf-moq-catalogformat}. The merge aims to streamline media interoperability testing, with plans to include examples in the Warp draft.

\subsubsection{SWITCH (Will)}
The SWITCH proposal was evaluated for its potential to enhance current specifications. Participants expressed interest in comparing it to existing capabilities, with suggestions to document any gaps. The discussion highlighted the need for a separate priority for new tracks, indicating a strategic shift towards more flexible media handling.

\subsubsection{Timestamps (Ian)}
Ian's presentation on timestamps sparked a debate on the appropriate abstraction for time in media transport. The group agreed on the need for a clear definition, with plans to create a pull request detailing the proposed changes. The discussion emphasized the importance of distinguishing between transport and media-specific timestamps.

\subsubsection{LOC (Mo)}
Mo presented the LOC draft, available \href{https://datatracker.ietf.org/doc/draft-mzanaty-moq-loc/}{here}, which was well-received. The group was encouraged to continue discussions on the mailing list or GitHub to refine the proposal.

\subsubsection{Draft Files (Cullen)}
Due to time constraints, Cullen's presentation on draft files was postponed. The draft is accessible \href{https://datatracker.ietf.org/doc/draft-jennings-moq-file/}{here} for those interested in reviewing and providing feedback.

Overall, the sessions highlighted significant advancements and ongoing challenges in the development of media transport over QUIC, with a clear focus on enhancing interoperability and refining protocol specifications.



\newpage

\section{IETF 121 MPLS WG Meeting (MPLS)}

\subsection{Attendees}
The meeting was attended by representatives from prominent companies and institutions including Deutsche Telekom, ZTE Corporation, Ericsson, Huawei Technologies, University of Tuebingen, Cisco Systems, Juniper Networks, and Nokia, among others. The total attendance was approximately 60 participants.

\subsection{Meeting Discussions}

\subsubsection{WG Status Update (Agenda Bashing)}
The session commenced with a status update from the Working Group Chairs. Discussions focused on collaboration efforts to consolidate the IOAM-dex documents into a unified draft, highlighting the importance of streamlined documentation for future developments.

\subsubsection{Stateless MNA-based Egress Protection (SMEP)}
Presented by Fabian Ihle, this session explored the potential of stateless MNA-based egress protection. Key discussions revolved around the applicability of local protection in SR-MPLS and the scalability of repair mechanisms. The draft can be accessed at \href{https://datatracker.ietf.org/doc/draft-ihle-mpls-mna-stateless-egress-protection-00}{draft-ihle-mpls-mna-stateless-egress-protection-00}.

\subsubsection{Update on MNA Implementation Experience in P4}
Fabian Ihle shared insights on the implementation of MNA in P4, addressing challenges related to parsing and overhead in PSD. The dialogue underscored the complexity of integrating AD in PSD and the implications for future P4 implementations.

\subsubsection{Performance Measurement Using STAMP for Segment Routing Networks}
Rakesh Gandhi presented on using STAMP for performance measurement in segment routing networks. The discussion highlighted the dependency on the MNA-HDR draft and the potential need to classify the document as experimental. The draft is available at \href{https://datatracker.ietf.org/doc/draft-ietf-spring-stamp-srpm-16}{draft-ietf-spring-stamp-srpm-16}.

\subsubsection{LSP Ping for SR Path Segment Identifier with MPLS Data Planes}
Xiao Min discussed the LSP Ping for SR Path Segment Identifier, focusing on synchronization challenges between ingress and egress. The need for mechanisms to ensure state awareness at the headend was emphasized. The draft can be found at \href{https://datatracker.ietf.org/doc/draft-ietf-mpls-spring-lsp-ping-path-sid-02}{draft-ietf-mpls-spring-lsp-ping-path-sid-02}.

\subsubsection{Deterministic Networking Specific MNA}
Greg Mirsky presented on deterministic networking-specific MNA, noting the potential for combining options for different deployments and the importance of feedback from the DETNET WG. The draft is accessible at \href{https://datatracker.ietf.org/doc/draft-varmir-mpls-detnet-mna-00}{draft-varmir-mpls-detnet-mna-00}.

Meeting materials, including slides, are available at \href{https://datatracker.ietf.org/meeting/121/session/mpls/}{IETF 121 MPLS Session Materials}.




\newpage

\section{Network Configuration Protocol (NETCONF) [NETCONF]}

\subsection{Attendees}
\subsubsection{Overview}
The NETCONF working group session was attended by 52 participants, representing a diverse array of prominent companies and institutions including Cisco Systems, Huawei, Nokia, and Bell Canada. The session was chaired by Kent Watsen and Per Andersson.

\subsection{Meeting Discussions}

\subsubsection{List Pagination for YANG-driven Protocols}
Qin Wu led the discussion on \href{https://datatracker.ietf.org/doc/html/draft-ietf-netconf-list-pagination-04}{draft-ietf-netconf-list-pagination-04}, focusing on the implementation status and remaining open issues. The draft aims to standardize pagination mechanisms for YANG-driven protocols, addressing feedback from YANG Doctor reviews. The group discussed the necessity of allowing a limit of zero and the implications of locale reporting for ordered lists. The next steps include closing open issues and preparing the documents for publication.

\subsubsection{Transaction ID Mechanism for NETCONF}
Jan Lindblad and Roque Gagliano presented updates on the \href{https://datatracker.ietf.org/doc/html/draft-ietf-netconf-transaction-id-05}{draft-ietf-netconf-transaction-id-05} and related trace context extensions. These drafts propose mechanisms for transaction identification and trace context propagation in NETCONF and RESTCONF. The session concluded with a consensus to proceed to Working Group Last Call (WGLC) for these drafts, highlighting their maturity and partial implementation.

\subsubsection{NETCONF Private Candidates}
James Cumming discussed the \href{https://datatracker.ietf.org/doc/html/draft-ietf-netconf-privcand-04}{draft-ietf-netconf-privcand-04}, which introduces independent candidate datastores per session. The group debated the YANG modeling approaches and the need for source datastore specification in commit operations. The preferred approach is to use augmentations for YANG models, supporting non-NMDA clients, and finalizing solutions for NMDA clients.

\subsubsection{Using NETCONF over QUIC Connection}
Per Andersson presented the \href{https://datatracker.ietf.org/doc/html/draft-dai-netconf-quic-netconf-over-quic-06}{draft-dai-netconf-quic-netconf-over-quic-06}, advocating for the adoption of NETCONF over QUIC to address TCP's limitations. The draft promises reduced latency and improved connection management, particularly beneficial in deep-space applications. The session gauged interest for an adoption call, receiving positive feedback.

\subsubsection{Augmented-by Addition into the IETF-YANG-Library}
Zhuoyao Lin remotely discussed the \href{https://datatracker.ietf.org/doc/html/draft-lincla-netconf-yang-library-augmentedby-01}{draft-lincla-netconf-yang-library-augmentedby-01}, which aims to enhance real-time knowledge of YANG module dependencies. The draft received support for its incremental approach, with a call for adoption to be initiated.

Meeting materials, including notes and slides, are available at \href{https://datatracker.ietf.org/meeting/120/session/netconf}{NETCONF Session Materials}.



\newpage

\section{Network Configuration (NETCONF) Working Group [NETCONF]}

\subsection{Attendees}
\subsubsection{Overview}
The NETCONF Working Group session was attended by representatives from prominent companies and institutions, including Cisco Systems, Nokia, Huawei, Ericsson, and Juniper Networks. The total attendance was 49 participants, reflecting a strong interest from both industry and academia.

\subsection{Meeting Discussions}
\subsubsection{Session Introduction and WG Status}
The session began with an introduction by the chairs, Kent Watsen and Per Andersson, who provided an update on the working group's status. They highlighted the recent publication of several RFCs, including RFCs 9641 to 9646, and discussed ongoing efforts such as the NETCONF interim for YANG notification specification gaps. The chairs encouraged community participation, particularly in advancing NETCONF-next and RESTCONF-next initiatives.

\subsubsection{Chartered Items}
\paragraph{NETCONF over QUIC}
Marc Blanchet led the discussion on \href{https://datatracker.ietf.org/doc/html/draft-ietf-netconf-over-quic-01}{draft-ietf-netconf-over-quic-01}, emphasizing its necessity for deep space communications. Key issues include the handling of notifications and RPC calls over QUIC streams. A proof-of-concept implementation is underway, with plans to address remaining issues before the next IETF meeting.

\paragraph{NETCONF Private Candidates}
Robert Wills presented \href{https://datatracker.ietf.org/doc/html/draft-ietf-netconf-privcand-05}{draft-ietf-netconf-privcand-05}, which simplifies the candidate datastore approach. The discussion focused on whether to add resolution-mode as an option to \texttt{<commit>}, with the authors recommending against it for orthogonality reasons.

\paragraph{Transaction ID Mechanism for NETCONF}
Jan Lindblad discussed \href{https://datatracker.ietf.org/doc/html/draft-ietf-netconf-transaction-id-07}{draft-ietf-netconf-transaction-id-07}, noting that the latest version addresses all received comments. The working group is considering concluding the Working Group Last Call (WGLC).

\paragraph{List Pagination for YANG-driven Protocols}
Qin Wu presented three related drafts, including \href{https://datatracker.ietf.org/doc/html/draft-ietf-netconf-list-pagination-05}{draft-ietf-netconf-list-pagination-05}. The discussion highlighted the use of \texttt{if-feature} for RESTCONF and capabilities for NETCONF, with further discussion planned on the mailing list.

\paragraph{YANG Groupings for UDP Clients and UDP Servers}
Alex Huang Feng discussed \href{https://datatracker.ietf.org/doc/html/draft-ietf-netconf-udp-client-server-05}{draft-ietf-netconf-udp-client-server-05}, focusing on default port settings and the potential inclusion of UDP-DTLS groupings. A poll indicated consensus to focus solely on UDP, and the draft is ready for WGLC.

\paragraph{UDP-based Transport for Configured Subscriptions}
Alex Huang Feng also led the discussion on \href{https://datatracker.ietf.org/doc/html/draft-ietf-netconf-udp-notif-15}{draft-ietf-netconf-udp-notif-15}, addressing open issues related to default ports and client-server capabilities. The draft is stable, but further transport review is needed before WGLC.

\paragraph{Subscription to Distributed Notifications}
The discussion on \href{https://datatracker.ietf.org/doc/html/draft-ietf-netconf-distributed-notif-10}{draft-ietf-netconf-distributed-notif-10} indicated readiness for WGLC, although recent discussions have been limited.

\paragraph{External Trace ID for Configuration Tracing}
Jean Quilbeuf presented \href{https://datatracker.ietf.org/doc/html/draft-ietf-netconf-configuration-tracing-03}{draft-ietf-netconf-configuration-tracing-03}, which focuses on configuration tracing using client-id attributes. The draft is ready for WGLC.

\paragraph{Augmented-by Addition into the IETF-YANG-Library}
Zhuoyao Lin discussed \href{https://datatracker.ietf.org/doc/html/draft-ietf-netconf-yang-library-augmentedby-01}{draft-ietf-netconf-yang-library-augmentedby-01}, aimed at addressing reverse dependencies in YANG modules. The draft is ready for WGLC.

\subsubsection{Non-Chartered Items}
\paragraph{YANG Groupings for QUIC Clients and QUIC Servers}
Per Andersson presented \href{https://datatracker.ietf.org/doc/html/draft-andersson-netconf-quic-client-server-01}{draft-andersson-netconf-quic-client-server-01}, seeking WG adoption. The draft defines reusable YANG groupings for QUIC, with a focus on TLS1.3.

\paragraph{YANG Notification Transport Capabilities}
Thomas Graf introduced \href{https://datatracker.ietf.org/doc/html/draft-netana-netconf-yp-transport-capabilities-00}{draft-netana-netconf-yp-transport-capabilities-00}, which defines new properties for transport protocol capabilities. The draft is under consideration for WG adoption.

\paragraph{Extensible YANG Model for YANG-Push Notifications}
Alex Huang Feng discussed \href{https://datatracker.ietf.org/doc/html/draft-netana-netconf-notif-envelope-00}{draft-netana-netconf-notif-envelope-00}, which proposes an opt-in for new headers to bypass XML-specific limitations. The draft is ready for WG adoption.

\paragraph{YANG-Push Operational Data Observability Enhancements}
Rob Wilton presented \href{https://datatracker.ietf.org/doc/html/draft-wilton-netconf-yp-observability-00}{draft-wilton-netconf-yp-observability-00}, aiming to simplify YANG-Push implementations. The draft is exploring options for extending or creating new RFCs.

\paragraph{Collector Implementation of HTTPS-Notif}
Bharadwaja MeherRushi Chittapragada shared insights from implementing HTTPS-Notif, highlighting challenges with XML namespaces and YANG module linkages.

Meeting materials, including slides and notes, are available at \href{https://datatracker.ietf.org/meeting/121/session/netconf}{NETCONF Session Materials}.



\newpage

\section{Network File System Version 4 Working Group (NFSv4)}

\subsection{Attendees}
\subsubsection{}
The meeting was attended by representatives from prominent organizations including Hammerspace Inc, FreeBSD Project, Carnegie Mellon University, NETAPP, Broadcom, Nokia, Huawei Technologies, and Meetecho. In total, there were 15 participants.

\subsection{Meeting Discussions}

\subsubsection{Chairs}
The session commenced with a welcome note and a reminder of the "Note Well" policies. Interim meetings are scheduled to resume on November 20, occurring bi-weekly. Participants were encouraged to post updates on documents in progress to the mailing list to ensure they are being reviewed.

\subsubsection{Internationalization}
David Noveck discussed the need for coordination on document reviews, emphasizing the importance of advancing working group documents. The \href{https://datatracker.ietf.org/doc/html/draft-ietf-nfsv4-internationalization-11}{draft-ietf-nfsv4-internationalization-11} is in the working group last call.

\subsubsection{Flex File v2 Erasure Encoding}
Thomas Haynes presented on client-side erasure coding, aiming to enhance scaling and performance on parallel writes. Concerns were raised about write amplification and metadata consistency. The discussion highlighted the need for further feedback on the design.

\subsubsection{Adding Uncacheable Attr}
Haynes proposed mechanisms to enforce access checks and avoid client-side caching, akin to O\_DIRECT. The discussion focused on performance implications and ensuring semantic consistency across clients.

\subsubsection{Recursive Attributes}
Rijesh Parambattu introduced new operations to efficiently handle recursive attribute changes in directory trees. The proposal includes both synchronous and asynchronous operations, with discussions on handling partial failures and scaling challenges.

\subsubsection{POSIX ACL}
Rick Macklem discussed the challenges of aligning NFS Version 4 ACLs with POSIX standards. The proposal involves a "TRUE\_FORM" approach, with ongoing debates about scope and implementation complexities.

\subsubsection{ACL Redux}
David Noveck provided updates on aligning ACLs with POSIX ACL drafts, addressing unresolved issues and seeking consensus on document adoption.

\subsubsection{Authentication \& Authorization}
Christopher Inacio explored updates and approaches to authentication and authorization, considering extensions to GSSAPI and comparisons with existing frameworks like Fed FS (RFC8000).

\subsubsection{5661bis / Security}
Noveck highlighted the accumulation of errata over 15 years and the need to progress on the \href{https://datatracker.ietf.org/doc/html/draft-ietf-nfsv4-rfc5661bis-09}{draft-ietf-nfsv4-rfc5661bis-09}. Discussions centered on document management challenges and the necessity for a clear path forward.

\subsubsection{Any Other Business}
The session concluded with a call for any additional business, emphasizing the importance of reaching consensus on document adoption and progression.

Meeting materials are available at \href{https://example.com/meeting-materials}{Meeting Materials}.



\newpage

\section{Network Management Operations (nmop) WG Agenda - IETF 121}

\subsection{Attendees Overview}
\subsubsection{Prominent Companies and Institutions}
The meeting was attended by representatives from major companies and institutions such as Huawei, Cisco, Nokia, Telefonica, and Deutsche Telekom, among others. The total attendance was approximately 70 participants, reflecting a diverse and engaged group of stakeholders in the network management domain.

\subsection{Meeting Discussions}

\subsubsection{Digital Map: Concepts and Requirements}
Presenter: Olga Havel

The session focused on defining the concept of a "Digital Map" for network topology modeling. Discussions highlighted the need for clear navigation between different network layers and abstraction levels. The group considered the implications of multi-layer topologies and the potential for a unified approach to modeling. For further details, refer to the \href{https://datatracker.ietf.org/doc/html/draft-ietf-nmop-digital-map-concept}{draft-ietf-nmop-digital-map-concept}.

\subsubsection{YANG-Push to Message Broker Integration}
Presenter: Thomas Graf

This presentation outlined an architecture for integrating YANG-Push with message brokers, emphasizing the importance of feedback from implementers and operators. The discussion underscored the challenges of on-change notifications and the need for a comprehensive approach to data encoding and validation. The draft document can be accessed at \href{https://datatracker.ietf.org/doc/html/draft-ietf-nmop-yang-message-broker-integration}{draft-ietf-nmop-yang-message-broker-integration}.

\subsubsection{Anomaly Detection and Incident Management}
Presenter: Qin Wu

The session addressed the development of a YANG module for incident management, focusing on the alignment of incident notifications with anomaly detection frameworks. The need for a dedicated document to harmonize terminology and structures was discussed. Relevant materials include the \href{https://datatracker.ietf.org/doc/html/draft-ietf-nmop-network-incident-yang}{draft-ietf-nmop-network-incident-yang}.

\subsubsection{Flash Teasers: Innovative Concepts}
Presenters: Robert Peschi, Xing Zhao, Rob Wilton, Diego Lopez

Brief presentations introduced innovative concepts such as a YANG Template Framework and AI-based Network Management Agents. These teasers aimed to spark interest and discussion on emerging technologies in network management.

Meeting materials, including slides and additional resources, are available \href{https://datatracker.ietf.org/meeting/121/materials/slides-121-nmop-sessa-chairs-slides-session-1-01}{here}.

\subsection{Next Steps}
The working group identified several key areas for further exploration, including the refinement of the Digital Map concept and the integration of YANG-Push with message brokers. The discussions set the stage for potential shifts in strategy, particularly in enhancing interoperability and scalability of network management solutions. The group will continue to engage with stakeholders to refine these approaches and contribute to the broader IETF objectives.




\newpage

\section{OAuth Working Group (OAuth WG)}

\subsection{Attendees Overview}
\subsubsection{Prominent Companies and Institutions}
The OAuth WG meeting was attended by representatives from major companies and institutions such as Microsoft, Cisco Systems, Okta, and the Georgia Institute of Technology. The total attendance was 91 participants.

\subsection{Meeting Discussions}

\subsubsection{Token Status List}
The discussion on the \href{https://datatracker.ietf.org/doc/html/draft-ietf-oauth-status-list}{draft-ietf-oauth-status-list} focused on enabling token issuers to communicate dynamic status for longer-lived tokens. The group debated the architectural implications of dropping the unsigned option to simplify the specification. The consensus leaned towards maintaining a secured container for tokens, with considerations for existing ecosystems using JOSE/COSE.

\subsubsection{Attestation-based Client Authentication}
The \href{https://datatracker.ietf.org/doc/html/draft-ietf-oauth-attestation-based-client-auth}{draft-ietf-oauth-attestation-based-client-auth} was reviewed, highlighting the restructuring to focus solely on client attestation. Discussions included the use of nonces and the potential for a common nonce solution, with emphasis on maintaining context-specific nonces to avoid security risks.

\subsubsection{Transaction Tokens}
The \href{https://datatracker.ietf.org/doc/html/draft-ietf-oauth-transaction-tokens}{draft-ietf-oauth-transaction-tokens} aims to manage transactions within multi-workload environments. The group explored the need for discovery mechanisms and the implications of batch processing on transaction token lifecycles. The discussion underscored the importance of context and authorization details in transaction tokens.

\subsubsection{OAuth Identity and Authorization Chaining Across Domains}
The \href{https://datatracker.ietf.org/doc/html/draft-ietf-oauth-identity-chaining}{draft-ietf-oauth-identity-chaining} was presented, addressing identity chaining across domains. Key points included the use of the "requested\_cnf" claim and the challenges of sender-constrained tokens. The group plans to merge the current pull request and aims for a working group last call before the next IETF meeting.

\subsubsection{First Party Apps}
The \href{https://datatracker.ietf.org/doc/html/draft-ietf-oauth-first-party-apps}{draft-ietf-oauth-first-party-apps} was discussed to improve user experience by leveraging OAuth's existing ecosystem. The group considered the inclusion of passkeys and the potential for a separate extension document. The need for reviewers was emphasized to advance the draft.

\subsubsection{SD-JWT and SD-JWT-VC}
The \href{https://datatracker.ietf.org/doc/html/draft-ietf-oauth-selective-disclosure-jwt}{draft-ietf-oauth-selective-disclosure-jwt} and \href{https://datatracker.ietf.org/doc/html/draft-ietf-oauth-sd-jwt-vc}{draft-ietf-oauth-sd-jwt-vc} drafts were reviewed, focusing on selective disclosure and verifiable credentials. The group discussed the removal of DIDs and the simplification of media types to avoid conflicts, with a consensus to proceed with the proposed changes.

Meeting materials can be accessed at \href{https://datatracker.ietf.org/meeting/121/materials/slides-121-oauth-oauth-identity-and-authorization-chaining-across-domains-00}{IETF 121 OAuth Meeting Materials}.




\newpage

\section{Oblivious HTTP Application Intermediation (OHAI) Working Group [OHAI]}

\subsection{Attendees Overview}

The OHAI working group meeting was attended by representatives from prominent companies and institutions, including Apple, Cloudflare, Google, Mozilla, and Cisco, among others. The total attendance was approximately 40 participants, reflecting a diverse range of expertise and interest in the ongoing development of HTTP privacy enhancements.

\subsection{Meeting Discussions}

\subsubsection{Chunked OHTTP Presentation by Tommy Pauly}

Tommy Pauly presented the latest updates on the \href{https://datatracker.ietf.org/doc/html/draft-ohai-chunked-ohttp}{draft-ohai-chunked-ohttp}, which is now in its third revision. The discussion highlighted a new dependency on the incremental header field, a draft under the HTTPbis working group. The need for a thorough security analysis was emphasized, particularly concerning the potential privacy implications of interleaving responses. The group debated the necessity of buffering by intermediaries and the implications for privacy and security. There was consensus on the need for explicit guidance to mitigate risks associated with timing attacks and response chunking.

\subsubsection{Key Issues and Next Steps}

Several key issues were identified, including the negotiation of media types and the maximum chunk size, with a proposal to align the latter with TLS standards at 16K. The group agreed on the importance of addressing replayability and interactivity concerns, with a focus on ensuring robust privacy protections. The meeting concluded with a call for further comments and reviews to refine the draft and address outstanding issues.

Meeting materials are available \href{https://datatracker.ietf.org/meeting/121/materials/slides-121-ohai-chair-slides-00}{here}.




\newpage

\section{OpenPGP WG (IETF 121)}

\subsection{Attendees Overview}

The OpenPGP Working Group meeting was attended by representatives from prominent organizations such as ACLU, Red Hat, Bundesdruckerei, Canadian Centre for Cyber Security, and NSA, among others. The total attendance was approximately 40 participants, reflecting a diverse mix of industry leaders and academic institutions.

\subsection{Meeting Discussions}

\subsubsection{OpenPGP Interoperability Test Suite Status}

Justus Winter presented the current status of the \href{https://tests.sequoia-pgp.org/}{OpenPGP Interoperability Test Suite}. The discussion focused on the potential migration of the test suite to a new platform, \href{https://tests.openpgp.org}{tests.openpgp.org}, with participants encouraged to share their opinions on the mailing list.

\subsubsection{Post-Quantum Cryptography in OpenPGP}

Aron Wussler led a detailed discussion on the \href{https://datatracker.ietf.org/doc/draft-ietf-openpgp-pqc/}{Post-Quantum Cryptography in OpenPGP} draft. Key points included the challenges of achieving NIST compliance with hybrid cryptographic constructions and the potential use of the LAMPS combiner to simplify the process. The dialogue underscored the complexities of aligning cryptographic approaches across working groups, with a consensus to continue discussions on the mailing list to resolve compliance issues.

\subsubsection{OpenPGP Key Replacement}

Andrew Gallagher discussed the \href{https://datatracker.ietf.org/doc/draft-ietf-openpgp-replacementkey/}{OpenPGP Key Replacement} draft, focusing on the transition from v4 to v6 keys. The group debated the inclusion of preferred key server (PKS) information, ultimately deciding to simplify the draft by excluding PKS references. The conversation highlighted the importance of balancing simplicity with functionality in key management protocols.

\subsubsection{Persistent Symmetric Keys in OpenPGP}

Daniel Huigens presented on \href{https://datatracker.ietf.org/doc/draft-ietf-openpgp-persistent-symmetric-keys/}{Persistent Symmetric Keys in OpenPGP}, exploring the technical implications of integrating persistent symmetric keys within the OpenPGP framework. The discussion addressed potential risks associated with v4 secret keys and the necessity of restricting certain features to v6 keys to ensure security.

\subsubsection{Stateless OpenPGP Update}

Due to time constraints, the update on \href{https://datatracker.ietf.org/doc/draft-dkg-openpgp-stateless-cli}{Stateless OpenPGP} by Daniel Kahn Gillmor was not presented. Participants were encouraged to review the draft independently and provide feedback.

\subsection{Meeting Materials}

All meeting materials, including slides and detailed notes, are available at \href{https://datatracker.ietf.org/meeting/121/materials/}{IETF 121 Meeting Materials}.



\newpage

\section{Operations and Management Area Working Group (OPSAWG)}

\subsection{Attendees}
\subsubsection{Overview}
The OPSAWG meeting was attended by representatives from prominent companies and institutions such as Cisco, Huawei, Ericsson, and Deutsche Telekom, with a total attendance of over 70 participants. The diverse group included industry leaders and academic researchers, fostering a collaborative environment for discussing advancements in network operations and management.

\subsection{Meeting Discussions}

\subsubsection{Agenda Bashing \& Introduction}
Chairs Joe Clarke and Benoît Claise initiated the session by discussing the rechartering of the working group to align with current objectives. They highlighted the transition of MUD-related work to the IOTOPS group, emphasizing the need for updated charter goals.

\subsubsection{An Information Model for Packet Discard Reporting}
John Evans presented the \href{https://datatracker.ietf.org/doc/html/draft-ietf-opsawg-discardmodel}{draft-ietf-opsawg-discardmodel}, focusing on the development of a YANG-based information model. The discussion centered on the potential for standardizing data models to enhance interoperability across different network environments.

\subsubsection{A Data Manifest for Contextualized Telemetry Data}
Jean Quilbeuf introduced the \href{https://datatracker.ietf.org/doc/draft-ietf-opsawg-collected-data-manifest/}{draft-ietf-opsawg-collected-data-manifest}, which aims to provide a structured approach for telemetry data collection. The group considered the implications of platform identification and the draft's readiness for last-call.

\subsubsection{Export of GTP-U Information in IPFIX}
Dan Voyer discussed the \href{https://datatracker.ietf.org/doc/draft-ietf-opsawg-ipfix-gtpu/}{draft-ietf-opsawg-ipfix-gtpu}, proposing enhancements to IPFIX for GTP-U data export. The draft is poised for progression, with a call for document shepherds to facilitate its advancement.

\subsubsection{A YANG Data Model for Network Diagnosis}
Victor López presented the \href{https://datatracker.ietf.org/doc/draft-contreras-opsawg-scheduling-oam-tests/}{draft-contreras-opsawg-scheduling-oam-tests}, which outlines a YANG model for scheduling OAM tests. The group expressed interest in adopting the work, recognizing its potential to streamline network diagnostics.

\subsubsection{Publishing End-Site Prefix Lengths}
Randy Bush's presentation on \href{https://datatracker.ietf.org/doc/html/draft-gasser-opsawg-prefix-lengths}{draft-gasser-opsawg-prefix-lengths} and \href{https://datatracker.ietf.org/doc/html/draft-ymbk-opsawg-rpsl-extref}{draft-ymbk-opsawg-rpsl-extref} sparked a debate on the necessity of prefix granularity for IPv6. The discussion highlighted the balance between operational needs and potential redundancy with existing RIR solutions.

\subsubsection{SAV-based Anti-DDoS Architecture}
Mingzhe Xing introduced the \href{https://datatracker.ietf.org/doc/draft-cui-savnet-anti-ddos/}{draft-cui-savnet-anti-ddos}, seeking feedback on its applicability within OPSAWG. The group encouraged building a community of interest to further develop the architecture.

\subsubsection{A YANG Data Model for Network Element Threat Surface Management}
Liang Xia discussed \href{https://datatracker.ietf.org/doc/draft-hu-opsawg-network-element-tsm-yang/}{draft-hu-opsawg-network-element-tsm-yang} and \href{https://datatracker.ietf.org/doc/draft-hu-opsawg-sec-config-yang/}{draft-hu-opsawg-sec-config-yang}, focusing on security configuration checks. The group noted the potential overlap with MUD and called for further discussion.

\subsubsection{Joint Exposure of Network and Compute Information}
Jordi Ros presented \href{https://datatracker.ietf.org/doc/draft-rcr-opsawg-operational-compute-metrics/}{draft-rcr-opsawg-operational-compute-metrics}, emphasizing the need for standardized metrics for infrastructure-aware service deployment. The dialogue explored the draft's alignment with CATS and its broader applicability.

\subsubsection{Intent-Based Security Management Automation}
Jaehoon Paul Jeong's presentation on \href{https://datatracker.ietf.org/doc/draft-jeong-opsawg-security-management-automation/}{draft-jeong-opsawg-security-management-automation} and related drafts sought feedback on standardizing security management automation. The group encouraged continued discussion to clarify the drafts' objectives.

\subsubsection{PCAP Document Status}
Michael Richardson updated on the status of \href{https://datatracker.ietf.org/doc/draft-ietf-opsawg-pcaplinktype/}{draft-ietf-opsawg-pcaplinktype}, \href{https://datatracker.ietf.org/doc/draft-ietf-opsawg-pcap/}{draft-ietf-opsawg-pcap}, and \href{https://datatracker.ietf.org/doc/draft-ietf-opsawg-pcapng/}{draft-ietf-opsawg-pcapng}, noting IANA's approval and the need for further reviews.

Meeting materials are available at \href{https://www.ietf.org/proceedings/121/agenda/agenda-121-opsawg-00}{IETF 121 OPSAWG Agenda}.




\newpage

\section{Path Computation Element Working Group (PCE WG)}

\subsection{Attendees}
\subsubsection{Overview}
The meeting was attended by representatives from prominent companies and institutions such as Cisco Systems, Nokia, Huawei, ZTE Corporation, and Juniper Networks, with a total attendance of over 40 participants. The diverse representation underscored the collaborative nature of the working group and the broad interest in the topics discussed.

\subsection{Meeting Discussions}

\subsubsection{Introduction}
The session began with administrivia and agenda bashing, followed by a status update on the working group's progress. Key discussions included the response to the \href{https://datatracker.ietf.org/liaison/1963/}{ETSI liaison}, focusing on the use of GMPLS and PCEP for fgOTN control. The chairs emphasized the importance of community feedback on the mailing list to shape the group's direction.

\subsubsection{Segment Routing}
\textbf{PCEP Extensions for Circuit Style Policies:} Samuel Sidor presented the \href{https://datatracker.ietf.org/doc/html/draft-ietf-pce-circuit-style-pcep-extensions}{draft-ietf-pce-circuit-style-pcep-extensions}, highlighting the dependencies with SPRING. The discussion centered on timing the Working Group Last Call (WGLC) appropriately to align with related drafts.

\textbf{SRv6 Policy SID List Optimization:} Zafar Ali discussed the \href{https://datatracker.ietf.org/doc/html/draft-ali-pce-srv6-policy-sid-list-optimization}{draft-ali-pce-srv6-policy-sid-list-optimization}, focusing on the need for a section on MSD for validations. The debate revolved around the encoding benefits and the necessity of explicit signaling flags.

\subsubsection{Stateful PCE}
\textbf{Updating Open Message Content:} Andrew Stone presented the \href{https://datatracker.ietf.org/doc/html/draft-stone-pce-update-open}{draft-stone-pce-update-open}, exploring the potential of borrowing concepts from BGP for dynamic capabilities. The group discussed the complexities of state changes and the possibility of a soft reset mechanism.

\textbf{LSP State Reporting Extensions:} Samuel Sidor's \href{https://datatracker.ietf.org/doc/html/draft-sidor-pce-lsp-state-reporting-extensions}{draft-sidor-pce-lsp-state-reporting-extensions} was examined, with discussions on BSID fallback and the implications of transit eligibility signaling.

\subsubsection{Others}
\textbf{PCEP Extension for Bounded Latency:} Quan Xiong presented the \href{https://datatracker.ietf.org/doc/html/draft-xiong-pce-detnet-bounded-latency}{draft-xiong-pce-detnet-bounded-latency}, with feedback on aligning with the DetNet RFC information model and clarifying ERO interactions.

\textbf{Using PCEP over QUIC:} Tingting Han discussed the \href{https://datatracker.ietf.org/doc/html/draft-yang-pce-pcep-over-quic}{draft-yang-pce-pcep-over-quic}, prompting a debate on the necessity and benefits of QUIC for PCEP. The group suggested further exploration of performance metrics to justify the transition.

Meeting materials, including slides and recordings, are available at \href{https://datatracker.ietf.org/meeting/121/session/pce}{IETF 121 PCE Session Materials}.




\newpage

\section{Post-Quantum Use and Implementation Practices (PQUIP) WG}

\subsection{Attendees}
The PQUIP working group session at IETF 121 in Dublin was attended by representatives from prominent companies and institutions including Google, Cisco Systems, Verisign, NSA, and Huawei, among others. The total attendance was over 100 participants, reflecting a strong interest in post-quantum cryptography developments.

\subsection{Meeting Discussions}

\subsubsection{Current Document Status}
The session began with updates on the status of several key documents. The \href{https://datatracker.ietf.org/doc/html/draft-ietf-pquip-pqt-hybrid-terminology}{Terminology for Post-Quantum Traditional Hybrid Schemes} draft has been submitted to the IESG for publication, though it requires additional work to address final comments. The \href{https://datatracker.ietf.org/doc/html/draft-ietf-pquip-hybrid-signature-spectrums}{Hybrid Signature Spectrums} draft is in WG Last Call, with a call for more feedback to refine the document. The \href{https://datatracker.ietf.org/doc/html/draft-ietf-pquip-pqc-engineers}{Post-Quantum Cryptography for Engineers} draft has completed its WG Last Call, with discussions focusing on the inclusion of normative language and the need for technical depth in explaining cryptographic proofs.

\subsubsection{Previously Discussed Topics}
Further discussions are needed on the \href{https://datatracker.ietf.org/doc/html/draft-vaira-pquip-pqc-use-cases}{Post-Quantum Cryptography Migration Use Cases} and \href{https://datatracker.ietf.org/doc/html/draft-wiggers-hbs-state}{Hash-based Signatures: State and Backup Management}. These topics will continue to be explored on the mailing list to gather more insights and consensus.

\subsubsection{PQC in Certificates at the Hackathon}
An update was provided on the progress made during the Hackathon, emphasizing the practical implementation of PQC in certificates. The presentation highlighted the challenges and potential solutions identified during the event.

\subsubsection{FIPS Issues with Deploying ML-KEM and ML-DSA}
Mike Ounsworth led a discussion on the challenges of deploying ML-KEM and ML-DSA under FIPS guidelines. Quynh Dang from NIST shared insights into ongoing internal discussions about seed management and API preferences, with a draft expected before the February seminar. The session underscored the need for collaboration with OASIS to address these issues effectively.

\subsubsection{PQC Algorithm Commonality Across the IETF}
The session concluded with a discussion on the commonality of PQC algorithms across IETF working groups. Paul Wouters noted the overlap with CFRG and the need for a rechartering discussion in January 2025 to address any remaining gaps. Joe Harvey suggested that PQUIP could play a role in ensuring consistency across drafts that reference each other.

Meeting materials can be accessed at \href{https://datatracker.ietf.org/meeting/121/session/pquip}{IETF 121 PQUIP Session Materials}.



\newpage

\section{RADEXT Working Group (RADEXT)}

\subsection{Attendees Overview}

\subsubsection{Attendees}

The RADEXT Working Group meeting was attended by representatives from prominent organizations such as Cisco, Cloudflare, and the NSA, among others. The total attendance was approximately 30 participants, including key contributors from Radiator Software, Vitrifi Limited, and the Technical University of Munich.

\subsection{Meeting Discussions}

\subsubsection{Administrivia and WG Status}

The meeting commenced with a brief administrivia session led by the chairs, Margaret Cullen and Valery Smyslov, covering the agenda and IPR statements. The chairs emphasized the importance of adhering to the Note Well guidelines and encouraged participants to engage actively in the discussions.

\subsubsection{WG Documents}

Alan DeKok provided a status update on the working group's documents, highlighting the progress on the \href{https://datatracker.ietf.org/doc/html/draft-ietf-radext-radiusdtls-bis}{draft-ietf-radext-radiusdtls-bis}. Discussions focused on the need for further testing and potential revisions to address session resumption issues. The group considered a WG last call on deprecating insecure practices, with suggestions to streamline the document by moving explanatory content to an appendix.

\subsubsection{(Datagram) Transport Layer Security (D)TLS Encryption for RADIUS}

Janfred Rieckers presented updates on the \href{https://datatracker.ietf.org/doc/html/draft-ietf-radext-radiusdtls-bis}{draft-ietf-radext-radiusdtls-bis}, addressing open issues such as proxying and load balancing considerations. The group discussed potential countermeasures for the selfie attack and the importance of explicit specifications for DTLS records. The presentation concluded with a call for further reviews and an interim meeting to expedite progress.

\subsubsection{Related Topics}

Mark Grayson introduced a draft on the RADIUS Connect-Info attribute for Wi-Fi networks, \href{https://datatracker.ietf.org/doc/html/draft-grayson-connectinfo}{draft-grayson-connectinfo}. The discussion centered on the need for a consistent framework to handle diverse wireless environments and the potential for increased adoption of RADIUS without bilateral agreements. Feedback was solicited on the draft's scope and implementation experiences.

Sri Gundavelli discussed RADIUS attributes for NS/EP services, emphasizing the growing role of Wi-Fi in mission-critical communications. The presentation outlined proposed attributes for capability indication and subscription info, aiming to enhance traffic prioritization and service authorization workflows.

\subsubsection{Open Mic}

The open mic session provided an opportunity for attendees to voice additional comments and questions, fostering a collaborative environment for addressing any remaining concerns or suggestions.

\subsubsection{Closing}

The meeting concluded with a summary of the discussions and a reminder of the next steps, including the potential for an interim meeting to maintain momentum on key drafts.

Meeting materials are available at \href{https://www.ietf.org/proceedings/121/radext.html}{IETF 121 RADEXT Meeting Materials}.



\newpage

\section{Remote Attestation Procedures (RATS) [RATS]}

\subsection{Attendees Overview}
\subsubsection{Attendees}
The RATS working group meetings were attended by representatives from prominent companies and institutions such as Cisco Systems, Intel, Huawei, Siemens, and Microsoft, among others. The total attendance was approximately 70 participants, reflecting a diverse and engaged group of stakeholders in the field of remote attestation.

\subsection{Meeting Discussions}

\subsubsection{Concise Reference Integrity Manifest}
Yogesh Deshpande presented updates on the \href{https://datatracker.ietf.org/doc/html/draft-ietf-rats-corim}{draft-ietf-rats-corim}, focusing on clarifications and improvements to the Security Version Number and Appraisal Claim Set definitions. The discussion highlighted the integration of CoRIM with SCITT and its potential to enhance software bill of materials (SBoMs) through reference APIs. The meeting underscored the importance of community feedback to refine the draft.

\subsubsection{Conceptual Message Wrappers}
Thomas Fossati discussed the \href{https://datatracker.ietf.org/doc/html/draft-ietf-rats-msg-wrap}{draft-ietf-rats-msg-wrap}, emphasizing the need for ergonomic abstractions in recursive message wrapping. The presentation covered recent editorial updates and security considerations, with a focus on resolving issues related to CBOR tag registration. The draft is preparing for Working Group Last Call (WGLC) exit, with a new version forthcoming.

\subsubsection{Attester Groups for Remote Attestation}
Jun Zhang introduced the concept of attester groups, proposing extensions to the RATS architecture to accommodate composite and layered attesters. The dialogue explored potential conflicts with existing posture assessment architectures and the need for further discussion on integrating swarm attestation concepts.

\subsubsection{Handling Multiple Verifiers in RATS Architecture}
Jun Zhang presented on the necessity of supporting multiple verifiers within the RATS architecture, as outlined in \href{https://datatracker.ietf.org/doc/html/draft-zhang-rats-multiverifiers}{draft-zhang-rats-multiverifiers}. The discussion focused on resilience and the challenges of implementing Byzantine fault tolerance in verifier management.

\subsubsection{Evidence Carrying Protocols}
Michael Richardson highlighted the absence of a standardized protocol for evidence carrying in RFC 9334, proposing a comprehensive list of existing protocols that handle evidence and attestation results. The session invited contributions from the community to expand this list, acknowledging the broad scope of potential protocols.

\subsubsection{PKIX Evidence}
Hannes Tschofenig, represented by Mike Ounsworth, discussed the challenges of encoding EAT claims into DER format, as detailed in \href{https://datatracker.ietf.org/doc/html/draft-ietf-rats-pkix-evidence}{draft-ietf-rats-pkix-evidence}. The presentation proposed splitting the document to address encoding and key attestation architecture separately, with an emphasis on leveraging existing media types for nesting.

\subsubsection{Verifiable Service Mesh}
Ramki Krishnan presented a framework for integrating attestation into service mesh architectures, addressing scalability and verifier provisioning challenges. The discussion acknowledged overlaps with multi-verifier and posture-assessment work, suggesting further architectural refinement.

\subsubsection{Security Considerations of Attested TLS}
Muhammad Usama Sardar emphasized the need for distinct protocol and attestation nonces in attested TLS, proposing a new document to update RFC 9334 with extended security considerations. The session highlighted the orthogonality of security requirements to interaction models.

\subsubsection{RATS Endorsements}
Dave Thaler reviewed updates to the \href{https://datatracker.ietf.org/doc/html/draft-ietf-rats-endorsements}{draft-ietf-rats-endorsements}, focusing on trust establishment and endorsement timeliness. The document is poised for Working Group Last Call, with recent revisions enhancing its security considerations.

Meeting materials and additional resources are available at \href{https://datatracker.ietf.org/meeting/121/materials/agenda-121-rats-00}{IETF RATS Meeting Materials}.



\newpage

\section{Registration Protocols Extensions (REGEXT)}

\subsection{Attendees}
\subsubsection{Overview}
The REGEXT working group meeting was attended by representatives from prominent organizations such as ICANN, Verisign, GoDaddy, and ARIN, with a total attendance of over 50 participants. The diverse representation underscored the importance of the discussions and the collaborative effort required to advance the group's objectives.

\subsection{Meeting Discussions}

\subsubsection{Adopted Work Presentations}
The session began with a presentation on \href{https://datatracker.ietf.org/doc/draft-ietf-regext-rdap-extensions/}{RDAP Extensions} by Andy Newton. The discussion focused on the need for consensus on extension styles and the implications of non-compliance. The dialogue highlighted the challenges in updating standards and the potential for informational drafts to guide implementation.

\subsubsection{New Work Presentations}
Several new proposals were discussed, including an \href{https://datatracker.ietf.org/doc/draft-brown-rdap-ttl-extension/}{RDAP Extension for DNS TTL Values} by Gavin Brown, which raised concerns about data consistency and user confusion. The \href{https://datatracker.ietf.org/doc/draft-loffredo-regext-epp-over-http/}{EPP over HTTPS} and \href{https://datatracker.ietf.org/doc/draft-yao-regext-epp-quic/}{EPP over QUIC} implementation experiences presented by James Gould prompted discussions on protocol preferences and potential security implications.

\subsubsection{Existing Work Status Updates}
The group reviewed the status of existing work, including the \href{https://datatracker.ietf.org/doc/draft-ietf-regext-epp-ttl/}{EPP mapping for DNS TTL values} and \href{https://datatracker.ietf.org/doc/draft-ietf-regext-epp-delete-bcp/}{Best Practices for Deletion of Domain and Host Objects in EPP}. These documents are progressing through the IESG review process, with discussions focusing on prioritizing workload and ensuring alignment with broader IETF goals.

\subsubsection{Milestones and Priorities}
A strategic discussion on milestones emphasized the need to balance new adoption requests with existing commitments. The group acknowledged the limited active participation and deliberated on prioritizing work that aligns with the most pressing industry needs.

\subsection{Meeting Materials}
All meeting materials, including slides and session recordings, are available at \href{https://datatracker.ietf.org/meeting/121/materials/}{IETF 121 Meeting Materials}.

The discussions in this meeting are expected to influence future technical directions, particularly in enhancing RDAP functionalities and refining EPP protocols. The outcomes will contribute significantly to the ongoing evolution of internet registration protocols, ensuring they remain robust and adaptable to emerging challenges.



\newpage

\section{Routing in Fat Trees (RIFT) [RIFT]}

\subsection{Attendees}
\subsubsection{Overview}
The RIFT working group session was attended by representatives from prominent companies and institutions, including Nvidia, Huawei, Juniper Networks, ISC, ZTE Corporation, Bloomberg, and the University of Tuebingen, among others. The total attendance was 20 participants.

\subsection{Meeting Discussions}

\subsubsection{WG Status}
Chairs Jeff Tantsura and Jeffrey Zhang provided an update on the working group's status, emphasizing the importance of tracking milestones. Discussions highlighted the separation of SR-MPLS and SRv6 extensions into distinct documents, underscoring the group's commitment to clear and organized progress.

\subsubsection{Draft: Auto IS-IS Integration}
Jordan Head presented the \href{https://datatracker.ietf.org/doc/html/draft-ietf-rift-auto-is-is-00}{draft-head-rift-auto-is-is-00}, focusing on integrating IS-IS and EVPN without reinventing existing protocols. The discussion centered on customer demand for flat network architectures with flood reflection capabilities. The draft's next iteration will include more detailed content, with adoption anticipated post-update.

\subsubsection{Draft: SRv6 Extensions}
Changwang Lin discussed the \href{https://datatracker.ietf.org/doc/html/draft-ietf-rift-srv6-extensions}{draft-cheng-rift-srv6-extensions}, addressing the need for use cases and requirements to justify the draft. The conversation explored adaptive routing and traffic engineering capabilities, with suggestions to include specific deployment scenarios. The draft aims to maintain simplicity by focusing on static traffic engineering.

\subsubsection{Draft: Multicast Enhancements}
Jeffrey Zhang presented the \href{https://datatracker.ietf.org/doc/html/draft-ietf-rift-multicast-02}{draft-zzhang-rift-multicast-02}, proposing multicast tree pre-setup for AI/ML networking. The discussion considered the potential for flood reduction and the use of distributed algorithms for full coverage. The draft suggests innovative approaches to support in-substrate computation.

Meeting materials are available at \href{https://datatracker.ietf.org/group/rift/about/}{RIFT Working Group Information}.



\newpage

\section{Routing Over Low Power and Lossy Networks (ROLL)}

\subsection{Attendees}
\subsubsection{Overview}
The ROLL working group meeting was attended by representatives from prominent institutions such as Nokia, Juniper Networks, Vrije Universiteit Brussel, and the U.S. Department of Defense, among others. The total attendance was approximately 20 participants, reflecting a diverse mix of industry and academic stakeholders.

\subsection{Meeting Discussions}

\subsubsection{WG Status - Introduction}
The session began with an introduction by the working group chairs, Ines and Aris, who provided an overview of the current status and minor rechartering efforts. The discussion emphasized the need to align with existing working group drafts, with a consensus to keep the rechartering scope minimal. John highlighted the importance of moving quickly and suggested deferring any major charter revisions for future discussions.

\subsubsection{Controlling Secure Network Enrollment in RPL Networks}
Ines presented the status of the \href{https://datatracker.ietf.org/doc/html/draft-ietf-roll-enrollment-priority-11}{draft-ietf-roll-enrollment-priority-11}. The group addressed minor issues and outlined next steps, focusing on resolving these through collaborative efforts. The draft aims to enhance secure network enrollment processes, which is crucial for the scalability and security of RPL networks.

\subsubsection{RPL DAG Metric Container Node State and Attribute Object Type Extension}
Aris discussed the \href{https://datatracker.ietf.org/doc/html/draft-koutsiamanis-roll-nsa-extension-02}{draft-koutsiamanis-roll-nsa-extension-02}, which proposes extensions to the RPL DAG metric container. The presentation highlighted the goal of implementing minimal changes while maintaining feature independence. Pascal expressed willingness to review the draft, underscoring its potential to optimize network performance.

\subsubsection{Mode of Operation Extension}
The session continued with a discussion on the \href{https://datatracker.ietf.org/doc/html/draft-ietf-roll-mopex-07}{draft-ietf-roll-mopex-07}. Dominique offered assistance in addressing open issues, emphasizing the draft's relevance in completing capabilities for RPL networks. The group acknowledged the importance of this extension in enhancing operational flexibility.

\subsubsection{Open Floor}
The meeting concluded with an open floor session where Pascal raised questions about DAO projections. John indicated that Jim has taken responsibility for this task, with a contingency plan for John to assume it if necessary. The discussion also touched on the need to identify consumers of RPLv2, highlighting the ongoing efforts to integrate DAO projections into broader network strategies.

Meeting materials are available at \href{https://example.com/meeting-materials}{Meeting Materials}.




\newpage

\section{RESTful Provisioning Protocol (RPP)}

\subsection{Attendees}
\subsubsection{Overview}
The RPP BoF session at IETF 121 in Dublin saw participation from a diverse group of stakeholders, including representatives from Verisign, Microsoft, ICANN, and the Estonian Internet Foundation, among others. The total attendance was approximately 70 participants, reflecting significant interest from both industry and academia.

\subsection{Meeting Discussions}

\subsubsection{Welcome and Introduction}
The session commenced with a welcome from the co-chairs, Darrel Miller and Andy Newton, who outlined the history and purpose of the BoF. The introductory slides can be accessed \href{https://datatracker.ietf.org/meeting/121/materials/slides-121-rpp-introduction-and-agenda-slides-02}{here}.

\subsubsection{EPP and ICANN}
Gavin Brown presented on the current state of the Extensible Provisioning Protocol (EPP) and its contractual obligations under ICANN. The discussion highlighted the potential deployment of a new provisioning protocol within the gTLD namespace. Relevant documents include \href{https://datatracker.ietf.org/doc/html/std-69}{STD 69}. The presentation slides are available \href{https://datatracker.ietf.org/meeting/121/materials/slides-121-rpp-background-on-epp-icann-00}{here}.

\subsubsection{RPP Motivations and Current Work}
Pawel Kowalik discussed the motivations for a RESTful provisioning protocol, emphasizing the need to prevent future fragmentation by standardizing RPPs-like APIs. The presentation underscored the ongoing work and can be reviewed \href{https://datatracker.ietf.org/meeting/121/materials/slides-121-rpp-ietf-121-rpp-bof-motivation-current-work-in-the-area-01}{here}.

\subsubsection{Technical Choices}
Timo Vohmar elaborated on the technical choices made in implementing a RESTful protocol for domain provisioning in .ee, which has been in use since 2014. The slides are accessible \href{https://datatracker.ietf.org/meeting/121/materials/slides-121-rpp-restful-epp-ee-00}{here}.

\subsubsection{Drafts and Requirements}
Maarten Wullink led a discussion on the requirements for RPP and current drafts, including the \href{https://datatracker.ietf.org/doc/draft-wullink-rpp-json/}{draft-wullink-rpp-json}. The session explored trade-offs in implementation styles and the importance of transition strategies. The requirements document is available \href{https://github.com/SIDN/ietf-wg-rpp-charter/blob/main/requirements.md}{here}.

\subsubsection{Charter Discussion}
The co-chairs facilitated a discussion on the proposed charter for a potential working group, emphasizing the need for coordination with REGEXT. The charter draft can be reviewed \href{https://github.com/SIDN/ietf-wg-rpp-charter/blob/main/rpp-charter.md}{here}. Key points included the necessity of a non-replacement strategy for EPP and the potential for a new provisioning protocol to foster ecosystem growth.

\subsubsection{Conclusions and Next Steps}
The session concluded with a call for further input on the charter and an acknowledgment of the need for continued collaboration. Participants were encouraged to provide feedback and engage in the ongoing development of the RPP framework.

Meeting materials, including all presentation slides, are available \href{https://datatracker.ietf.org/meeting/121/materials/}{here}.



\newpage

\section{Routing Area Working Group (RTGWG)}

\subsection{Attendees}
\subsubsection{Overview}
The RTGWG meeting was attended by representatives from prominent companies and institutions such as Cisco, Huawei, Juniper Networks, Nokia, and Ericsson, with a total attendance of over 100 participants. The meeting materials can be accessed at \href{https://datatracker.ietf.org/meeting/121/session/rtgwg}{RTGWG Meeting Materials}.

\subsection{Meeting Discussions}

\subsubsection{TI-LFA, BGP-PIC, and SR ULoop}
Ahmed Bashandy presented updates on \href{https://datatracker.ietf.org/doc/html/draft-ietf-rtgwg-segment-routing-ti-lfa}{draft-ietf-rtgwg-segment-routing-ti-lfa}, \href{https://datatracker.ietf.org/doc/html/draft-ietf-rtgwg-bgp-pic}{draft-ietf-rtgwg-bgp-pic}, and \href{https://datatracker.ietf.org/doc/html/draft-bashandy-rtgwg-segment-routing-uloop}{draft-bashandy-rtgwg-segment-routing-uloop}. Discussions highlighted the need for clarity on mandatory features and terminology consistency with existing RFCs. The group emphasized the importance of addressing feedback to ensure alignment with the working group's objectives.

\subsubsection{SR Based Loop-free Implementation}
Lijie Deng discussed the \href{https://datatracker.ietf.org/doc/html/draft-deng-rtgwg-sr-loop-free}{draft-deng-rtgwg-sr-loop-free}, focusing on documenting microloop scenarios. Feedback suggested referencing existing documents to enhance the draft's utility and questioned the necessity of the draft given its informational nature.

\subsubsection{Path-aware Remote Protection Framework}
Yisong Liu and Changwang Lin presented the \href{https://datatracker.ietf.org/doc/html/draft-liu-rtgwg-path-aware-remote-protection}{draft-liu-rtgwg-path-aware-remote-protection}. The discussion centered on the need for protocol independence and the draft's applicability to spine-leaf topologies. Suggestions included clarifying the correlation between router ID and next hop.

\subsubsection{Destination/Source Routing}
Shu Yang's presentation on \href{https://datatracker.ietf.org/doc/html/draft-ietf-rtgwg-dst-src-routing-revive}{draft-ietf-rtgwg-dst-src-routing-revive} was acknowledged as important work, with no significant objections raised.

\subsubsection{Deep Collaboration between Application and Network}
Xinxin Yi introduced the \href{https://datatracker.ietf.org/doc/html/draft-zhang-rtgwg-collaboration-app-net}{draft-zhang-rtgwg-collaboration-app-net}, which sparked interest in expanding network capabilities to cloud applications. The group encouraged further detailed presentations to explore this collaboration.

\subsubsection{The Challenges and Requirements for Routing in Computing Cluster Network}
Yizhou Li and Fengkai Li discussed the \href{https://datatracker.ietf.org/doc/html/draft-li-rtgwg-computing-network-routing}{draft-li-rtgwg-computing-network-routing}, highlighting the potential of hybrid routing to reduce configuration complexity. The group suggested considering IGP solutions and referencing LSVR for further development.

\subsubsection{In-Network Congestion Notification}
Zongpeng Du presented the \href{https://datatracker.ietf.org/doc/html/draft-du-rtgwg-in-network-congestion-notification}{draft-du-rtgwg-in-network-congestion-notification}, which received no major comments, indicating general acceptance or the need for further review.

\subsubsection{Adaptive Routing Framework}
Changwang Lin and Rui Zhuang's \href{https://datatracker.ietf.org/doc/html/draft-cheng-rtgwg-adaptive-routing-framework}{draft-cheng-rtgwg-adaptive-routing-framework} prompted questions about traffic congestion avoidance, with a recommendation to continue discussions on the mailing list.

\subsubsection{Generalized IPv6 Tunnel (GIP6)}
Xinxin Yi, Zhenbin Lin, and Qiangzhou Gao presented the \href{https://datatracker.ietf.org/doc/html/draft-li-rtgwg-gip6-protocol-ext-requirements}{draft-li-rtgwg-gip6-protocol-ext-requirements} and \href{https://datatracker.ietf.org/doc/html/draft-li-rtgwg-generalized-ipv6-tunnel/04}{draft-li-rtgwg-generalized-ipv6-tunnel}, which concluded without significant feedback, suggesting either consensus or the need for further examination.

\subsubsection{Advertising Router Information}
Jeffrey Zhang's \href{https://datatracker.ietf.org/doc/html/draft-zzhang-rtgwg-router-info}{draft-zzhang-rtgwg-router-info} presentation raised concerns about terminology, specifically the use of "flooding," prompting a review of language to ensure clarity and precision.

Overall, the RTGWG meeting facilitated robust discussions on various routing technologies, with a focus on refining drafts to align with industry standards and addressing technical challenges. The outcomes suggest a continued evolution of routing protocols to enhance network efficiency and interoperability.



\newpage

\section{Secure Asset Transfer Protocol (SATP) Working Group [satp]}

\subsection{Attendees}
The SATP Working Group meeting was attended by representatives from prominent organizations such as IBM, Intel, Huawei, and Cisco Systems, with a total attendance of 30 participants. Notable attendees included Venkatraman Ramakrishna from IBM, Thomas Hardjono from MIT, and Wes Hardaker from USC/ISI and ICANN Board.

\subsection{Meeting Discussions}

\subsubsection{Chair Introduction}
The meeting commenced with an introduction by the chairs, Wes Hardaker and Claire Facer, who reminded attendees of the IETF process and called for note takers.

\subsubsection{SATP Architecture Draft Review}
Thomas Hardjono presented the \href{https://datatracker.ietf.org/doc/html/draft-ietf-satp-architecture}{draft-ietf-satp-architecture}, highlighting updates to the architecture diagram. Feedback was received from John, suggesting potential updates contingent on changes to the core protocol.

\subsubsection{SATP Core Draft Review}
The \href{https://datatracker.ietf.org/doc/html/draft-ietf-satp-core}{draft-ietf-satp-core} was discussed, focusing on updates to version 05. A significant point of discussion was the inclusion of an identifier in the stage-1 message to ensure credential verification. The group debated the necessity of a general-purpose reject message, with insights from Denis Avrilionis on multi-phase commit protocols.

\subsubsection{SATP Use Cases Draft Review}
Venkatraman Ramakrishna reviewed the \href{https://datatracker.ietf.org/doc/html/draft-ietf-satp-usecases}{draft-ietf-satp-usecases}, which outlines scenarios such as international trade and decentralized finance. The group emphasized the need for new use cases that explore novel scenarios. Denis Avrilionis proposed integrating digital representations in supply chains, with Peter Yee volunteering for further reviews.

\subsubsection{Case Study: South Korea CBDC Pilot Project}
A case study on the South Korea CBDC pilot project was presented, exploring the use of SATP for cross-border payments. The project utilized the Hyperledger Cacti SATP connector, demonstrating successful functionality in a demo environment. The group discussed the challenges of evaluating SATP's safety and its suitability for cross-border payments.

\subsubsection{Overview of the IETF Process}
The chairs outlined the need for document shepherds and announced a forthcoming four-week last call for all three SATP documents.

\subsubsection{Next Steps for SATP}
The discussion on next steps focused on the overarching goal of network architecture with gateways. Rama highlighted the need for asset transfer within a larger workflow, including network discovery and cross-network query capabilities. The group considered developing two documents: one for view address definitions and another for a request/response protocol. Denis Avrilionis suggested starting with the stage-0 draft and addressing asset profiles.

Meeting materials are available at \href{https://datatracker.ietf.org/meeting/121/materials.html}{IETF 121 Meeting Materials}.



\newpage

\section{Source Address Validation Networking (SAVNET) [SAVNET]}

\subsection{Attendees}
\subsubsection{Overview}
The SAVNET working group meeting was attended by representatives from prominent companies and institutions including Cisco Systems, China Telecom, Huawei, and the Georgia Institute of Technology, among others. The total attendance was approximately 70 participants.

\subsection{Meeting Discussions}

\subsubsection{Intra-domain SAVNET Architecture}
Lancheng Qin presented the \href{https://datatracker.ietf.org/doc/draft-ietf-savnet-intra-domain-architecture/}{draft-ietf-savnet-intra-domain-architecture}, which received support from Peter Psenak. Joel Halpern raised concerns about incremental deployment, suggesting the need for clarity in the documentation. The discussion emphasized defining full deployment before showcasing incremental benefits.

\subsubsection{Source Prefix Advertisement for Intra-domain SAVNET}
Lancheng Qin also presented the \href{https://datatracker.ietf.org/doc/draft-li-savnet-source-prefix-advertisement/}{draft-li-savnet-source-prefix-advertisement}. The presentation proceeded without questions, indicating general consensus or clarity on the topic.

\subsubsection{Intra-domain SAVNET Support via IGP \& BGP}
Shengnan Yue discussed the \href{https://datatracker.ietf.org/doc/draft-cheng-savnet-intra-domain-sav-igp-03}{draft-cheng-savnet-intra-domain-sav-igp-03} and \href{https://datatracker.ietf.org/doc/draft-cheng-savnet-intra-domain-sav-bgp-01}{draft-cheng-savnet-intra-domain-sav-bgp-01}. The debate focused on handling FRR cases and asymmetric routing, with suggestions for further offline discussions and additional text in the draft to address these scenarios.

\subsubsection{General Source Address Validation Capabilities}
Mingqing Huang presented the \href{https://datatracker.ietf.org/doc/draft-huang-savnet-sav-table/07/}{draft-huang-savnet-sav-table}. Discussions highlighted computational limitations as a challenge, with Aijun Wang suggesting raising the issue on the mailing list for broader input.

\subsubsection{Remote Measurement of Outbound Source Address Validation Deployment}
Shuai Wang's presentation on the \href{https://datatracker.ietf.org/doc/draft-wang-savnet-remote-measurement-osav/}{draft-wang-savnet-remote-measurement-osav} concluded without questions, suggesting alignment with the group's expectations.

\subsubsection{Inter-domain Source Address Validation (SAVNET) Architecture}
Libin Liu introduced the \href{https://datatracker.ietf.org/doc/draft-wu-savnet-inter-domain-architecture/}{draft-wu-savnet-inter-domain-architecture}. Sriram acknowledged previous comments and raised issues about data source prioritization and security, which will be addressed in future solutions.

\subsubsection{Update on the BAR-SAV Draft}
K. Sriram updated on the \href{https://datatracker.ietf.org/doc/draft-ietf-sidrops-bar-sav/}{draft-ietf-sidrops-bar-sav}. The discussion revolved around prefix management and potential efficiency improvements, with suggestions for further exploration on the mailing list.

\subsubsection{BGP Operations for Inter-domain SAVNET}
Xueyan Song presented the \href{https://datatracker.ietf.org/doc/draft-song-savnet-inter-domain-bgp-ops/}{draft-song-savnet-inter-domain-bgp-ops}, focusing on multi-homing scenarios. Aijun Wang recommended comparing this with existing inter-domain architectures for a comprehensive understanding.

\subsubsection{Source Address Validation Enhanced by Network Controller}
Tian Tong discussed the \href{https://datatracker.ietf.org/doc/draft-tong-savnet-sav-enhanced-by-controller}{draft-tong-savnet-sav-enhanced-by-controller}. The presentation highlighted the need for a detailed solution to address centralized control challenges, with plans to refine the approach offline.

\subsection{Meeting Materials}
Meeting materials are available at \href{https://datatracker.ietf.org/meeting/121/session/savnet}{IETF 121 SAVNET Session Materials}.




\newpage

\section{Static Context Header Compression (SCHC) Working Group (SCHC)}

\subsection{Attendees}
\subsubsection{Overview}
The SCHC working group meeting was attended by representatives from prominent institutions such as IMT Atlantique, Cisco, RISE Research Institutes of Sweden, and Concordia University, with a total attendance of over 40 participants.

\subsection{Meeting Discussions}

\subsubsection{Administrivia}
The session commenced with administrative updates, including a review of the working group's draft status and a discussion on the IP protocol number and Ethertype, which have been moved from the INTAREA. The importance of aligning with the architecture document was emphasized, with a consensus to seek early allocation for protocol numbers.

\subsubsection{ICMPv6 Draft}
Laurent Toutain presented the \href{https://datatracker.ietf.org/doc/html/draft-ietf-schc-icmpv6-compression}{draft-ietf-schc-icmpv6-compression}, focusing on ICMPv6 compression. The discussion highlighted the need for more generic echo functions and the potential for proxy-ping actions, which will be defined in the architecture document.

\subsubsection{SCHC Rule Format for FEC in Fragmentation}
Alexander Pelov discussed the \href{https://datatracker.ietf.org/doc/html/draft-pelov-schc-fragmentation-fec-rule-format}{draft-pelov-schc-fragmentation-fec-rule-format}, proposing updates to the draft. The presentation underscored the flexibility of FEC operations and their applicability in diverse network conditions, including deep space communications.

\subsubsection{Updating RFC 8824}
Marco Tiloca presented updates on the \href{https://datatracker.ietf.org/doc/html/draft-ietf-schc-8824-update}{draft-ietf-schc-8824-update}. The session focused on aligning terminology with ISO standards and considering new features for consistency and performance improvements.

\subsubsection{Deep Space Communications}
Marc Blanchet provided insights into the Deep Space BoF, discussing the implications for SCHC in space communications. The session explored the potential for SCHC to support multipath and FEC strategies in space networks, emphasizing the need for collaboration with other protocols like QUIC and CoAP.

\subsubsection{SCHC for Networks Susceptible to Disruptions}
Edgar Ramos presented the \href{https://datatracker.ietf.org/doc/html/draft-ietf-schc-over-networks-prone-to-disruptions}{draft-ietf-schc-over-networks-prone-to-disruptions}, focusing on adoption and repurposing strategies for networks prone to disruptions. The discussion highlighted the role of SCHC proxies in enhancing network resilience.

\subsubsection{Secure and Autonomic Framework for SCHC Context Management in LoRaWAN}
Maryam Hatami introduced a framework for SCHC context management in LoRaWAN, emphasizing its potential for secure and autonomic operations. The presentation sparked interest in further developing this work within the IETF.

\subsubsection{SCHC Action}
Ana Minaburo discussed the \href{https://datatracker.ietf.org/doc/html/draft-minaburo-schc-flow-compression}{draft-minaburo-schc-flow-compression}, proposing a new parameter in the Rule: Action. The session explored its impact on architecture and the necessity for synchronized context management.

Meeting materials are available at \href{https://datatracker.ietf.org/meeting/121/session/schc}{Meeting Materials}.




\newpage

\section{System for Cross-domain Identity Management (SCIM) [SCIM]}

\subsection{Attendees}
\subsubsection{}
The SCIM working group meeting was attended by representatives from prominent companies and institutions such as Cisco Systems, Microsoft, Okta, Amazon Web Services (AWS), and Huawei, among others. The total attendance was approximately 40 participants, reflecting a diverse range of expertise and interest in identity management solutions.

\subsection{Meeting Discussions}

\subsubsection{Chairs Intro}
The meeting commenced with a brief introduction by the chairs, setting the stage for the agenda and highlighting the importance of the discussions to follow.

\subsubsection{SCIM Use Cases}
Pam and Paulo presented on SCIM use cases, aiming to orient implementers on the practical applications of SCIM. The draft, which is seeking adoption, outlines various resource types and attributes, emphasizing orchestrator roles and provisioning domains. The discussion highlighted the need for a common understanding of SCIM actions, with a focus on data directionality and implementation options. The presentation underscored the potential for SCIM to expand its capabilities beyond initial use cases, as detailed in the \href{https://datatracker.ietf.org/doc/html/draft-correia-scim-use-cases}{draft-ietf-scim-use-cases}. The working group showed interest in adopting the draft, with a call for reviewers to refine the document further.

\subsubsection{Device Models}
Eliot provided an update on the SCIM device models, which have undergone multiple reviews and a working group last call. The core device model remains streamlined, with examples including BLE and Zigbee. Recent reviews have led to updates, and an open-source implementation is available. Key issues discussed included the handling of SubjectAltnames in client certificates and the optionality of telemetry endpoints. The next steps involve drafting a new version and conducting a shepherd review, with plans for an extension document on x509 iDevIDs. The \href{https://datatracker.ietf.org/doc/html/draft-ietf-scim-device-model}{draft-ietf-scim-device-model} serves as a reference for these developments.

\subsubsection{Update to Cursor Pagination}
Nancy reported on Cisco's implementation of cursor pagination, which demonstrated interoperability with the server reference implementation at scim.dev. While the server side is incomplete, the exercise confirmed the feasibility of the approach. The discussion also touched on the progress of SCIM events moving towards a shepherd's writeup, indicating imminent multiple implementations.

\subsubsection{Other Business}
The meeting concluded with a call for any additional business, to which there were no further contributions.

Meeting materials and further details can be accessed via the \href{https://datatracker.ietf.org/meeting/121/materials.html}{IETF 121 meeting materials}.



\newpage

\section{Supply Chain Integrity, Transparency, and Trust (SCITT)}

\subsection{Attendees Overview}

The SCITT session at IETF 121 was attended by representatives from prominent organizations such as MITRE Corporation, ITOCHU Techno Solutions, Rakuten, Carnegie Mellon University, Arm, Keio University, Verisign, China Mobile, MIT, Microsoft, Ericsson, Huawei, and many others, totaling over 80 participants.

\subsection{Meeting Discussions}

\subsubsection{Welcome and Introduction}

The session commenced with a brief welcome and introduction, emphasizing the importance of maintaining a collaborative and respectful environment.

\subsubsection{SCITT Overview}

Henk provided an overview of the SCITT architecture, focusing on its role in enhancing integrity, transparency, and accountability within software supply chains. The architecture is ready for Working Group Last Call (WGLC), with the next focus on SCRAPI, the API for interacting with SCITT instances. The discussion highlighted the use of CDDL for payload descriptions and the importance of small COSE receipts for efficient data handling. For more details, attendees were encouraged to review the \href{https://datatracker.ietf.org/doc/html/draft-ietf-scitt-architecture}{draft-ietf-scitt-architecture}.

\subsubsection{Transparency in the News}

Orie discussed the growing importance of transparency in software supply chains, driven by regulatory requirements and industry best practices. SCITT's flexibility in content types and its recognition in key implementation guides were emphasized. The session called for leveraging SCITT as a foundational element for interoperable transparency.

\subsubsection{Recap Since IETF 120}

Steve Lasker recapped the progress since the last meeting, focusing on refining the architecture and preparing it for WGLC. The session noted the responsiveness of the working group to feedback and the stabilization of key architectural elements.

\subsubsection{SCRAPI}

Jon Geater discussed the ongoing development of SCRAPI, highlighting the resolution of issues and the importance of defining consistent APIs for secure environments. The discussion also covered the need for consistency proofs and support for multiple notaries.

\subsubsection{Hackathon Report}

The hackathon demonstrated the practical application of SCRAPI, revealing areas for improvement such as consistent data formats. The session highlighted successful implementations and the potential of SCRAPI in real-world scenarios.

\subsubsection{Next Steps}

The session concluded with a poll indicating readiness for WGLC on the architecture document. The group plans to focus on GitHub workflows and consider interim meetings before IETF 122.

\subsubsection{AOB Open Mic}

No additional questions were raised, indicating a stable working group consensus.

\subsubsection{Wrap-up and Conclusion}

The session wrapped up with acknowledgments of the participants' contributions and a positive outlook on the progress made.

Meeting materials are available at \href{https://www.ietf.org/proceedings/121/scitt.html}{IETF 121 SCITT Meeting Materials}.




\newpage

\section{Standard Communication with Network Elements (SCONE) WG}

\subsection{Attendees}

The SCONE WG meeting was attended by representatives from prominent companies and institutions such as Huawei, Google, Ericsson, Meta, Apple, and Nokia, among others. The total attendance was approximately 120 participants, reflecting a broad interest in the ongoing developments within the working group.

\subsection{Meeting Discussions}

\subsubsection{Welcome and Note Well}

The session was initiated by the chairs, Q. Wu and B. Trammell, who outlined the agenda and emphasized the importance of adhering to the WG's scope as directed by the Area Director. Discussions included clarifications on the WG's charter and its focus on configured limits versus path limitations.

\subsubsection{Discovery of Network Rate-Limit Policies}

M. Boucadair presented the \href{https://datatracker.ietf.org/doc/html/draft-brw-scone-rate-policy-discovery-00}{draft-brw-scone-rate-policy-discovery-00}. The presentation was straightforward, with no immediate clarifying questions from the attendees.

\subsubsection{MASQUE Signaling Extension for Media Bitrate}

M. Ihlar introduced the \href{https://datatracker.ietf.org/doc/html/draft-ihlar-scone-masque-mediabitrate-01}{draft-ihlar-scone-masque-mediabitrate-01}, highlighting its potential applications. Discussions centered on the integration of a common throughput signaling format across multiple protocols.

\subsubsection{TRAIN Protocol}

M. Thomson discussed the \href{https://datatracker.ietf.org/doc/html/draft-thomson-scone-train-protocol-00}{draft-thomson-scone-train-protocol-00}, addressing questions about network node awareness of application types. The topic was deemed suitable for further exploration in subsequent meetings.

\subsubsection{QUIC Version for SCONE}

M. Joras presented the \href{https://datatracker.ietf.org/doc/html/draft-joras-scone-quic-protocol-00}{draft-joras-scone-quic-protocol-00}, focusing on the packet design and its compatibility with existing QUIC versions. Concerns were raised about middlebox compatibility with multi-initial packets.

\subsubsection{Flavors of SCONE and Discussion}

S. Dawkins led a discussion on the various flavors of SCONE, emphasizing the need for a converged solution that considers both client and server support. The conversation touched on throughput advice directionality and the potential for a design team to harmonize SCONE and TRAIN proposals.

\subsubsection{Establishing SCONE and Open Discussion}

The chairs proposed forming design teams to address the information model and protocol aspects separately. However, opinions varied, with some suggesting that the WG's scope was narrow enough to handle these developments without separate teams. The session concluded with plans for virtual interim meetings to further refine the proposals.

\subsection{Meeting Materials}

Meeting materials are available at \href{https://meetings.conf.meetecho.com/ietf121/?session=33576}{IETF 121 SCONE Session}.

\subsection{Lightning Talks}

\subsubsection{draft-shi-scone-rtc-requirement-01}

This talk focused on the real-time communication requirements within the SCONE framework.

\subsubsection{draft-ruan-scone-use-cases-and-requirements-00}

The presentation outlined various use cases and requirements, providing a foundation for future discussions on SCONE's applicability and implementation.




\newpage

\section{SIDROPS (Secure Inter-Domain Routing Operations)}

\subsection{Attendees Overview}
The SIDROPS working group meeting at IETF-121 was attended by representatives from prominent organizations such as RIPE NCC, Cisco, Juniper Networks, and Fastly, among others. In total, the session saw participation from over 50 attendees, reflecting a diverse mix of industry experts and researchers.

\subsection{Meeting Discussions}

\subsubsection{Agenda Bashing and Chair's Slides}
The meeting commenced with a brief agenda bashing session, during which no additional comments or changes were proposed by the attendees.

\subsubsection{Tom Harrison: Manifest Numbers}
Tom Harrison presented the \href{https://datatracker.ietf.org/doc/html/draft-ietf-sidrops-manifest-numbers-02}{draft-ietf-sidrops-manifest-numbers-02}. The discussion highlighted the importance of recommendations on publications, with consensus that these should be finalized before the Working Group Last Call (WGLC).

\subsubsection{Sofía: NRO RPKI Program Update}
Sofía provided an update on the NRO RPKI Program, emphasizing ongoing developments and future plans.

\subsubsection{Job Snijders: Next-Gen RPKI Transport}
Job Snijders discussed the requirements for a next-generation RPKI transport. The session underscored the transactional nature of RRDP compared to RSYNC, with participants supporting the need for a transport requirements document. The conversation also touched on the potential publication of these requirements for future developers, despite no immediate goal for an RFC.

\subsubsection{Job Snijders: Constraining RPKI Trust Anchors}
The \href{https://datatracker.ietf.org/doc/html/draft-snijders-constraining-rpki-trust-anchors-06}{draft-snijders-constraining-rpki-trust-anchors-06} was presented by Job Snijders. Discussions revolved around the consistency of resource records and the potential for an informational RFC to address these issues.

\subsubsection{K. Sriram: ASRA Profile and Verification}
K. Sriram introduced two drafts: \href{https://datatracker.ietf.org/doc/html/draft-geng-sidrops-asra-profile-00}{draft-geng-sidrops-asra-profile-00} and \href{https://datatracker.ietf.org/doc/html/draft-sriram-sidrops-asra-verification-00}{draft-sriram-sidrops-asra-verification-00}. The discussion focused on the challenges of partial deployment and the role of BGPsec in addressing fake link problems.

\subsubsection{Libin: SISPI}
Libin presented the \href{https://datatracker.ietf.org/doc/html/draft-chen-sidrops-sispi-02}{draft-chen-sidrops-sispi-02}, which was briefly discussed.

\subsubsection{Shuhe Wang: Route Partial Visibility}
Shuhe Wang's presentation on \href{https://datatracker.ietf.org/doc/html/draft-wang-sidrops-route-partial-visibility}{draft-wang-sidrops-route-partial-visibility} highlighted the challenges of route visibility and the need for further exploration.

\subsubsection{Shenglin: PSVRO}
Shenglin discussed the \href{https://datatracker.ietf.org/doc/html/draft-jiang-sidrops-psvro-00}{draft-jiang-sidrops-psvro-00}, with questions raised about the coverage of RoA and the need for clarification on multiple ASes.

\subsubsection{YingYing: RPKI Repository Problem Statement}
YingYing presented the \href{https://datatracker.ietf.org/doc/html/draft-li-sidrops-rpki-repository-problem-statement-00}{draft-li-sidrops-rpki-repository-problem-statement-00}, leading to a discussion on the redundancy of RPs and the necessity for operational specifications.

\subsubsection{Jia Zhang: ASPA Egress (If Time Permits)}
Jia Zhang briefly introduced the \href{https://datatracker.ietf.org/doc/html/draft-zhang-sidrops-aspa-egress-00}{draft-zhang-sidrops-aspa-egress-00}, pending time availability.

Meeting materials are available at \href{https://example.com/sidrops-ietf121-materials}{SIDROPS IETF-121 Materials}.



\newpage

\section{Structured Messaging Layer (SML) [sml]}

\subsection{Attendees}

The SML working group meeting was attended by representatives from prominent companies and institutions including Fastmail, audriga GmbH, ICANN, Apple, and the ACLU, with a total attendance of 30 participants.

\subsection{Meeting Discussions}

\subsubsection{Structured Vacation Notices}

Hans-Joerg Happel presented on \href{https://datatracker.ietf.org/doc/html/draft-happel-sml-structured-vacation-notices-01}{draft-happel-sml-structured-vacation-notices-01}, focusing on the inclusion of time zones in vacation notices. The debate centered around whether to enforce UTC or allow user-defined time zones, with opinions divided. The discussion also covered the format for replacement contacts, with a preference for JSON-LD over vCard, and the need for handling multiple absence periods. The consensus leaned towards enhancing the draft to accommodate these features, recognizing the potential complexity of integrating proactive vacation notices akin to a calendaring system.

\subsubsection{Structured Email}

The session on \href{https://datatracker.ietf.org/doc/html/draft-ietf-sml-structured-email-02}{draft-ietf-sml-structured-email-02} led by Hans-Joerg Happel addressed the placement of JSON-LD within MIME structures. Concerns were raised about legacy client compatibility, prompting a call for testing across major email clients like Gmail and Outlook. The group discussed multipart/related and multipart/mixed representations, with a focus on ensuring backward compatibility while leveraging JSON-LD for enhanced email structuring. The need for a robust testing framework was highlighted, with potential contributions from industry stakeholders.

\subsubsection{Use Cases}

Ben Bucksch presented on \href{https://datatracker.ietf.org/doc/html/draft-ietf-sml-structured-email-use-cases-02}{draft-ietf-sml-structured-email-use-cases-02}, exploring diverse applications of structured email. The discussion acknowledged security challenges inherent in the proposed use cases, emphasizing the necessity for coordinated semantics and robust trust/security considerations. The dialogue underscored the transformative potential of structured email while cautioning against the risks of misimplementation.

Meeting materials are available at \href{https://meetings.conf.meetecho.com/ietf121/?group=sml&short=&item=1}{Meetecho} and \href{https://notes.ietf.org/notes-ietf-121-sml}{Notes}.




\newpage


\section{Secure Protocols for the Internet Credential Exchange (SPICE)}

\subsection{Attendees Overview}
\subsubsection{Attendee Summary}
The SPICE BOF at IETF-119 was attended by a diverse group of 144 participants, representing prominent companies and institutions such as Siemens, Cisco Systems, Okta, Google, Apple, Microsoft, and many others.

The discussions during the meeting were centered around the proposed work items and the charter text for the group. The attendees engaged in a lively debate on various topics, including architecture, use cases, SD-CWT, and metadata/capability discovery. The meeting materials can be accessed via the \href{https://datatracker.ietf.org/doc/polls-119-spice-202403191300/}{IETF Datatracker}.

\subsection{Meeting Discussions}

\subsubsection{Architecture}
Henk presented two foundational documents that will inform the group's architecture: \href{https://datatracker.ietf.org/doc/html/draft-steele-spice-transparency-tokens}{draft-steele-spice-transparency-tokens} and \href{https://leifj.github.io/wallet-refarch/draft-johansson-wallet-refarch.html}{draft-johansson-wallet-refarch}. The discussion highlighted the importance of integrating credentials into other protocols and the need for a clear architectural framework.

\subsubsection{Use Cases}
Mike, Brent, and Roy discussed the use cases for SPICE, referencing \href{https://datatracker.ietf.org/doc/html/draft-prorock-spice-use-cases/01}{draft-prorock-spice-use-cases/01}. The conversation underscored the broad applicability of the use cases and their relevance to the work of the group.

\subsubsection{SD-CWT}
Orie introduced the concept of SD-CWT, outlined in \href{https://datatracker.ietf.org/doc/html/draft-prorock-cose-sd-cwt/02}{draft-prorock-cose-sd-cwt/02}, emphasizing its potential performance benefits over JSON-based approaches. The discussion also touched on the importance of security guarantees provided by receipts and the need to maintain conceptual alignment with SD-JWT.

\subsubsection{Meta-data/Capability Discovery}
The topic of metadata and capability discovery was addressed by Orie, with reference to \href{https://datatracker.ietf.org/doc/html/draft-steele-spice-metadata-discovery/01}{draft-steele-spice-metadata-discovery/01}. The debate highlighted the challenges and importance of key discovery and the potential for metadata to become a complex area requiring careful consideration.

\subsubsection{Charter Text Discussion}
The group engaged in a comprehensive discussion on the charter text, with contributions from various attendees. The dialogue reflected a consensus on the need for clarity regarding the group's scope and its relationship with W3C work. The discussion concluded with a poll that favored moving forward with the proposed charter text, including a modification to acknowledge the conceptual security model used in related technologies.

\end{document}



\newpage

\section{SPICE (Selective Privacy and Identity Credentials Engineering) [SPICE]}

\subsection{Attendees Overview}
The SPICE Working Group session at IETF 121 was attended by representatives from prominent companies and institutions such as Mozilla, Cisco Systems, Microsoft, Huawei, and Okta, with a total attendance of over 80 participants. The diverse group included experts in digital identity, security, and privacy, reflecting the broad interest and expertise in the topics discussed.

\subsection{Meeting Discussions}

\subsubsection{Welcome and Introduction to SPICE}
The session began with an introduction to SPICE's mission to address gaps in digital credentials, particularly focusing on secure and private representation of both human and non-human identities. The chairs emphasized the importance of developing profiles tailored to specific use cases, excluding key discovery and new cryptographic primitives. Attendees were encouraged to explore related groups like RATS, OAUTH, and COSE for integration opportunities.

\subsubsection{Selective Disclosure for CBOR Web Tokens (SD-CWT)}
Rohan Mahy presented \href{https://datatracker.ietf.org/doc/html/draft-ietf-spice-sd-cwt}{draft-ietf-spice-sd-cwt}, highlighting a method for selective disclosure of claims in CBOR Web Tokens. The discussion focused on the mechanics of selective disclosure, syntax options for nested structures, and key binding to prevent replay attacks. The approach aims to enhance user-controlled data sharing while maintaining security.

\subsubsection{CBOR and Syntax Optimization}
Orie Steele discussed techniques for handling redactions in CBOR data structures, emphasizing the balance between data size and complexity versus the risk of covert channels. The group debated using unique integers for redactions to prevent hidden communication, leaning towards transparency by limiting redactions to standardized values.

\subsubsection{Global Unique Enterprise Identifiers (GLUE)}
Brent Zundel introduced \href{https://datatracker.ietf.org/doc/draft-zundel-spice-glue-id/}{draft-zundel-spice-glue-id}, proposing a structure for globally unique identifiers for corporate entities. The discussion addressed potential overlaps with existing URN registries and the need for further exploration of GLUE as a formal working item.

\subsubsection{OpenID Connect Standard Claims Registration for CBOR Web Tokens}
Beltram Maldant presented \href{https://datatracker.ietf.org/doc/draft-maldant-spice-oidc-cwt/}{draft-maldant-spice-oidc-cwt}, focusing on aligning OpenID Connect standard claims for CBOR Web Tokens. The proposal aims to register 19 OIDC-defined claims in the CWT registry to enhance efficiency and utility across multiple use cases.

\subsubsection{Audience Q&A and Discussions}
The session included discussions on nested disclosure complexity, differences from SD-JWT standards, and preventing issuer-verifier collusion. Attendees also explored registry structures for GLUE and the potential adoption of GLUE to enable feedback-driven improvements.

\subsection{Speaker Contributions}
- **Rohan Mahy**: Introduced SD-CWT concepts and discussed selective disclosure mechanics.
- **Orie Steele**: Provided insights into CBOR syntax challenges and introduced GLUE concepts.
- **Brent Zundel**: Presented GLUE and addressed audience questions on registry overlap.
- **Mike Prorock**: Voiced concerns about issuer-verifier collusion and syntax standards.

\subsection{Open Questions and Action Items}
1. **CBOR Redactions**: Continue discussions on effective methods for CBOR data redactions.
2. **GLUE Registry Exploration**: Explore registry structures for GLUE identifiers.
3. **Syntax Feedback**: Gather community feedback on syntax options for CBOR environments.
4. **Collusion Prevention**: Develop syntax standards to limit discretionary data control.

Meeting materials are available at \href{https://datatracker.ietf.org/doc/slides-121-spice-chair-slides/}{SPICE Chair Slides}.



\newpage

\section{SPRING (Source Packet Routing in Networking) [SPRING]}

\subsection{Attendees}

The SPRING working group meeting at IETF-121 was attended by representatives from prominent companies and institutions, including Cisco, Huawei, Nokia, ZTE Corporation, and Deutsche Telekom, among others. The total attendance was approximately 100 participants, reflecting a strong interest in the ongoing developments within the SPRING working group.

\subsection{Meeting Discussions}

\subsubsection{SPRING Status - Chairs}

The meeting commenced with an update from the chairs on the rechartering process, emphasizing the importance of working group (WG) engagement in document development. The chairs encouraged participants to actively read and comment on documents to enhance their quality. Discussions highlighted the need for regular updates on draft statuses, with suggestions to adopt practices from other WGs like IPPM for efficient document management.

\subsubsection{Segment Routing IPv6 Security Considerations}

Nick Buraglio presented the \href{https://datatracker.ietf.org/doc/html/draft-ietf-spring-srv6-security}{draft-ietf-spring-srv6-security}, focusing on security considerations for Segment Routing over IPv6 (SRv6). The discussion underscored the necessity of incorporating RFC9602 and ensuring careful wording to address interactions with other prefixes. The WG considered an early security directorate review to refine the threat model and terminologies.

\subsubsection{Circuit Style Segment Routing Policies}

Zafar Ali introduced the \href{https://datatracker.ietf.org/doc/html/draft-ietf-spring-cs-sr-policy}{draft-ietf-spring-cs-sr-policy}, requesting Working Group Last Call (WGLC). The WG was urged to review the draft and provide feedback, with a call for volunteers to shepherd the document.

\subsubsection{Distribute SRv6 Locator by DHCP}

Weiqiang Cheng presented the \href{https://datatracker.ietf.org/doc/html/draft-ietf-spring-dhc-distribute-srv6-locator-dhcp}{draft-ietf-spring-dhc-distribute-srv6-locator-dhcp}, also seeking WGLC. The draft had been reviewed by the DHC WG, and SPRING participants were encouraged to evaluate it from their perspective.

\subsubsection{SRv6 Context Indicator SIDs for SR-Aware Services}

Jiaming Ye discussed the \href{https://datatracker.ietf.org/doc/html/draft-lin-spring-srv6-aware-context-indicator}{draft-lin-spring-srv6-aware-context-indicator}, highlighting renewed interest in service programming. The WG was invited to review related documents and provide feedback, particularly on the complexity and performance implications of mappings towards Context-ID.

\subsubsection{4map6 Segments for IPv4 Service Delivery over IPv6-only Underlay Networks}

Guozhen Dong presented the \href{https://datatracker.ietf.org/doc/html/draft-dong-spring-sr-4map6-segments}{draft-dong-spring-sr-4map6-segments}, which did not elicit comments, suggesting either consensus or the need for further review.

\subsubsection{Encoding Network Slice Identification for SRv6}

Liyan Gong introduced the \href{https://datatracker.ietf.org/doc/html/draft-cheng-spring-srv6-encoding-network-sliceid}{draft-cheng-spring-srv6-encoding-network-sliceid}, which similarly received no immediate feedback, indicating potential acceptance or the necessity for additional scrutiny.

\subsubsection{Link Discovery Protocol (LLDP) Extensions for Segment Routing over IPv6 (SRv6)}

Liyan Gong also presented the \href{https://datatracker.ietf.org/doc/html/draft-gong-spring-lldp-srv6-extensions}{draft-gong-spring-lldp-srv6-extensions}. The discussion focused on the use of LLDP versus DHCP for locator distribution, with considerations of deployment scenarios and protocol ownership by IEEE. The WG was encouraged to continue discussions on the mailing list.

\subsubsection{SRv6 SFC Architecture with SR-aware Functions}

Yuta Fukagawa discussed the \href{https://datatracker.ietf.org/doc/html/draft-watal-spring-srv6-sfc-sr-aware-functions}{draft-watal-spring-srv6-sfc-sr-aware-functions}, addressing the integration of SRv6 OAM and the potential use of BGP-LS for service function configuration. The WG was urged to explore orchestration mechanisms and engage in further offline discussions.

Meeting materials are available \href{https://meetecho-or.ietf.org/client/?session=33036}{here}.




\newpage

\section{SRv6 Operations Working Group (SRv6OPS)}

\subsection{Attendees}
\subsubsection{Overview}
The SRv6OPS session was attended by representatives from prominent companies and institutions, including Bell Canada, China Southern Power Grid, Swisscom, and Verizon, among others. The total attendance was approximately 80 participants, reflecting a diverse mix of industry leaders and technical experts.

\subsection{Meeting Discussions}

\subsubsection{Service Programming at Bell Canada}
Daniel Bernier from Bell Canada presented on the implementation of SRv6 service programming. The discussion highlighted the use of SRv6 Stateless SFC to efficiently steer traffic with dynamically inserted services. Key points included the custom function for mapping multiple network functions under a single SID, enhancing scalability. The presentation also addressed control-plane independence, allowing flexibility in using BGP, Netconf, PCEP, or gRPC for explicit-path SR-Policy creation.

\subsubsection{SRv6 in Smart Grid at China Southern Power Grid}
Jiangang Lu shared insights on deploying SRv6 technology within China Southern Power Grid. The presentation emphasized the operational benefits and specific SRv6 features tailored to meet service requirements. The discussion also touched on the potential for SRv6-MPLS interworking, as outlined in \href{https://datatracker.ietf.org/doc/html/draft-agrawal-spring-srv6-mpls-interworking}{draft-agrawal-spring-srv6-mpls-interworking}.

\subsubsection{Network Analytics Incident at Swisscom}
Thomas Graf discussed an SRv6 network incident at Swisscom, focusing on the role of SRv6 operational metrics in network anomaly detection. The presentation was based on the architecture described in \href{https://datatracker.ietf.org/doc/html/draft-ietf-nmop-network-anomaly-architecture}{draft-ietf-nmop-network-anomaly-architecture}. The session underscored the importance of network observability in managing such incidents.

\subsubsection{SRv6 DC Multi-POD Scenario at Verizon}
Gyan Mishra explored Verizon's IT Multi-POD data center architecture, utilizing SRv6 Next-C-SID. The discussion highlighted the benefits of extending SRv6 fabric to the host level for traffic engineering and steering, addressing challenges like head-of-line blocking. The session concluded with a call for further discussion on the operational guidance of this deployment scenario.

Meeting materials, including slides, video, and chat logs, are available at \href{https://datatracker.ietf.org/meeting/121/session/srv6ops}{IETF Datatracker}.

\end{document}



\newpage

\section{Secure Shell Maintenance (SSHM) WG - IETF 121}

\subsection{Attendees}
The meeting was attended by representatives from prominent organizations such as OpenSSH, Fastly, Google, Cloudflare, and the National Security Agency (NSA), with a total attendance of 35 participants. Notable attendees included Damien Miller from OpenSSH, Job Snijders from Fastly, and Stephen Farrell from Trinity College Dublin.

\subsection{Meeting Discussions}

\subsubsection{Administrivia and Modus Operandi}
The chairs, Job Snijders and Stephen Farrell, presented the proposed modus operandi for the working group. The approach was generally well-received, with no objections raised. Discussions included the consideration of private forks as implementations and the encouragement of feedback through the IETF mailing list rather than direct emails to authors. The possibility of creating a GitHub organization for the group was also mentioned. Meeting materials can be accessed \href{https://datatracker.ietf.org/doc/slides-121-sshm-sshm-chair-slides/}{here}.

\subsubsection{Draft-miller-ssh-agent}
Damien Miller discussed the \href{https://datatracker.ietf.org/doc/html/draft-ietf-sshm-ssh-agent}{draft-miller-ssh-agent}, which aims to document the SSH Agent Protocol. The draft has been in use for two decades, and the discussion focused on whether to encourage extensions and the potential for standardization. Tero Kivinen suggested an extension to limit keys, which was supported by Damien, contingent on further implementation. The next steps involve incorporating feedback and determining the RFC status.

\subsubsection{Deprecating Ciphers}
Theo de Raadt led a discussion on deprecating outdated ciphers, emphasizing the need to balance security with the operational requirements of legacy systems. The group considered documenting ciphers for removal, such as 3DES, and discussed strategies for gradual deprecation. The conversation highlighted the challenges of maintaining compatibility with older devices while ensuring security. The potential impact on upstream crypto libraries and the feasibility of distributing legacy binaries were also debated.

\subsubsection{Other Drafts}
There were no presentations for the drafts on \href{https://datatracker.ietf.org/doc/html/draft-josefsson-ntruprime-ssh}{ntruprime} or \href{https://datatracker.ietf.org/doc/html/draft-kampanakis-curdle-ssh-pq-ke}{PQ-KE} during this session.

\subsubsection{Any Other Business}
Tero Kivinen raised a question regarding the version of the file transfer draft to be used, suggesting that this be discussed further on the mailing list.

The discussions during this session underscored the working group's commitment to evolving the SSH protocol while addressing the complexities of legacy system support and security enhancements. The outcomes suggest a strategic shift towards more secure and standardized implementations, with a focus on community engagement and feedback.



\newpage

\section{Secure Telephony Identity Revisited (STIR) [STIR]}

\subsection{Attendees Overview}
\subsubsection{Attendees}
The STIR Working Group meeting at IETF 121 was attended by representatives from prominent companies and institutions such as Meta Platforms, Inc., Microsoft, Cisco Systems, and Netflix, among others. The total attendance was approximately 30 participants, reflecting a diverse range of stakeholders in the telecommunication and security sectors.

\subsection{Meeting Discussions}

\subsubsection{Certificates}
Jon Peterson presented the \href{https://datatracker.ietf.org/doc/html/draft-ietf-stir-certificates-shortlived-01}{draft-ietf-stir-certificates-shortlived-01}, which mandates the use of x5c and allows x5u for backward compatibility. The group agreed that the root certificate should be omitted from the x5c chain, and this change will be included in the next update before proceeding to Working Group Last Call (WGLC). Chris Wendt discussed the \href{https://datatracker.ietf.org/doc/html/draft-wendt-stir-certificate-transparency-04}{draft-wendt-stir-certificate-transparency-04}, which is now more self-contained and focuses on pre-certificate flow, providing APIs for the STIR/SHAKEN ecosystem.

\subsubsection{VESPER - VErifiable STI Personas}
Chris Wendt introduced the \href{https://datatracker.ietf.org/doc/html/draft-wendt-stir-vesper-02}{draft-wendt-stir-vesper-02}, which proposes extending the STIR architecture with PASSporTs as Selective Disclosure JWTs (SD-JWT). The discussion highlighted the complexity of the three-party architecture and the need for further exploration of use cases. Participants suggested that the proposal might require breaking into smaller components and potentially updating the STIR charter.

\subsubsection{Conclusions}
The group concluded that neither VESPER nor Certificate Transparency drafts are ready for adoption. Further discussions on use cases and potential charter reframing are necessary to align these proposals with the current scope of STIR.

Meeting materials are available at \href{https://www.ietf.org/proceedings/121/stir.html}{IETF 121 STIR Meeting Materials}.

\subsection{Next Steps}
The working group will focus on refining the drafts based on feedback, particularly addressing the omission of the root certificate in the certificate chain and exploring the implications of the VESPER architecture. Future meetings will aim to clarify the value propositions and align these initiatives with the broader goals of the STIR framework.



\newpage

\section{Software Updates for Internet of Things (SUIT) [SUIT]}

\subsection{Attendees Overview}
\subsubsection{Attendance}
The SUIT working group meeting at IETF 121 was attended by representatives from prominent organizations such as Cisco Systems, Cloudflare, Arm Ltd, and the U.S. Department of Defense, among others. The total attendance was approximately 50 participants.

\subsection{Meeting Discussions}

\subsubsection{SUIT Manifest Format}
The \href{https://datatracker.ietf.org/doc/html/draft-ietf-suit-manifest}{draft-ietf-suit-manifest-28} was submitted to the IESG for publication. However, it currently faces two DISCUSS ballot positions that need resolution. Brendan Moran provided updates indicating that recent IANA considerations have been addressed, potentially resolving some of the DISCUSS positions. Further updates are anticipated as the document progresses towards publication.

\subsubsection{SUIT Manifest Extensions for Multiple Trust Domains}
The \href{https://datatracker.ietf.org/doc/html/draft-ietf-suit-trust-domains}{draft-ietf-suit-trust-domains-08} has been submitted for publication. Security AD Deb Cooley completed the AD review, and further clarifications on references were discussed. Brendan Moran and Ken remotely presented updates, emphasizing the importance of maintaining valid examples and addressing implicit indices issues.

\subsubsection{Firmware Encryption with SUIT Manifests}
The \href{https://datatracker.ietf.org/doc/html/draft-ietf-suit-firmware-encryption}{draft-ietf-suit-firmware-encryption-21} is ready for IESG evaluation following reviews and updates. Hannes Tschofenig highlighted the need for additional reviews due to numerous editorial changes. The draft is scheduled for discussion in the upcoming IESG telechat.

\subsubsection{Secure Reporting of Update Status}
The \href{https://datatracker.ietf.org/doc/html/draft-ietf-suit-report}{draft-ietf-suit-report-10} has completed WG Last Call, with encryption clarified as optional. Brendan Moran presented the draft, noting no further comments were received.

\subsubsection{Strong Assertions of IoT Network Access Requirements}
The \href{https://datatracker.ietf.org/doc/html/draft-ietf-suit-mud}{draft-ietf-suit-mud-09} is held for publication pending the clearance of related drafts. Deb Cooley suggested that the chairs inquire about the hold status with Roman Danyliw.

\subsubsection{Mandatory-to-Implement Algorithms for SUIT Manifests}
The \href{https://datatracker.ietf.org/doc/html/draft-ietf-suit-mti}{draft-ietf-suit-mti-08} is ready for publication, with expectations for a -bis version soon. Russ Housley will shepherd this draft.

\subsubsection{Update Management Extensions for SUIT Manifests}
The \href{https://datatracker.ietf.org/doc/html/draft-ietf-suit-update-management}{draft-ietf-suit-update-management-07} awaits WG consensus write-up. Thomas Fossati volunteered as the document shepherd.

\subsubsection{Other Topics}
No additional topics were raised, allowing the meeting to conclude six minutes early.

Meeting materials are available at \href{https://notes.ietf.org/notes-ietf-121-suit}{IETF 121 SUIT Meeting Notes}.



\newpage

\section{TCP Maintenance and Minor Extensions (TCPM)}

\subsection{Attendees}

The TCPM meeting at IETF-121 in Dublin saw participation from key industry players and academic institutions, including Google, TikTok, Ericsson, and Cloudflare, with a total attendance of 44 individuals. Notable attendees included Neal Cardwell from Google, Yoshifumi Nishida from TikTok, and Lars Eggert from Mozilla.

\subsection{Meeting Discussions}

\subsubsection{WG Status Update}

Michael Tüxen provided an update on the Working Group's document status. The slides for this presentation can be accessed \href{https://datatracker.ietf.org/meeting/121/materials/slides-121-tcpm-chair-slides}{here}.

\subsubsection{Update to RFC 6937 Proportional Rate Reduction for TCP}

Neal Cardwell (Google) presented remotely on the updates to \href{https://datatracker.ietf.org/doc/html/draft-ietf-tcpm-prr-rfc6937bis-12}{draft-ietf-tcpm-prr-rfc6937bis}. The document has completed the Working Group Last Call, and the updates are considered minor. Neal will post an updated revision shortly, which will be reviewed by the group.

\subsubsection{Handling of Ghost ACKs}

Michael Tüxen presented on behalf of the authors regarding the \href{https://datatracker.ietf.org/doc/html/draft-ietf-tcpm-tcp-ghost-acks-01}{draft-ietf-tcpm-tcp-ghost-acks}. The presentation proceeded without questions or comments from attendees.

\subsubsection{TCP Extended Data Offset Option}

Yoshifumi Nishida discussed the \href{https://datatracker.ietf.org/doc/html/draft-ietf-tcpm-tcp-edo-14}{draft-ietf-tcpm-tcp-edo}, highlighting challenges such as the need to disable GSO and GRO, which impacts performance. The group debated the practicality of the proposal, with a show of hands indicating limited interest in progressing the document.

\subsubsection{ACK Rate Request}

Carles Gomez presented on the \href{https://datatracker.ietf.org/doc/html/draft-ietf-tcpm-ack-rate-request-06}{draft-ietf-tcpm-ack-rate-request}, emphasizing the importance of pacing to minimize burstiness. The discussion underscored the relevance of the draft to the L4S community and the necessity of careful implementation to avoid network congestion.

\subsubsection{TCP\_REPLENISH\_TIME}

Stuart Cheshire provided an overview of the source-buffering delay problem, suggesting that not attempting to fill the bandwidth-delay-product could be a viable solution. The presentation sparked a brief discussion on alternative approaches to managing network traffic.

Meeting materials are available \href{https://datatracker.ietf.org/meeting/121/materials/slides-121-tcpm}{here}.



\newpage

\section{Traffic Engineering Architecture and Signaling (TEAS) [TEAS]}

\subsection{Attendees Overview}
\subsubsection{Attendance}
The TEAS working group session at IETF 120 was attended by 78 participants, featuring representatives from prominent companies and institutions such as Ericsson, Huawei Technologies, Cisco Systems, Nokia, Telefonica Innovacion Digital, and Juniper Networks.

\subsection{Meeting Discussions}

\subsubsection{Administrivia \& WG Status}
The session commenced with an overview of the working group's current status, presented by the chairs. This segment set the stage for subsequent discussions by outlining the progress and future directions of the group.

\subsubsection{WG Draft Updates}
The chairs provided updates on the working group drafts, emphasizing the need for continued collaboration and input from participants to refine the drafts not currently on the agenda.

\subsubsection{Network Slices for 5G Networks}
Krzysztof Szarkowicz presented on the realization of network slices for 5G networks using current IP/MPLS technologies, referencing the \href{https://datatracker.ietf.org/doc/html/draft-ietf-teas-5g-ns-ip-mpls}{draft-ietf-teas-5g-ns-ip-mpls}. The discussion highlighted consensus on terminology adjustments, specifically the adoption of "underlay transport" to enhance clarity.

\subsubsection{YANG Data Models for NRPs}
Bo Wu discussed the YANG data models for Network Resource Partitions (NRPs), as detailed in \href{https://datatracker.ietf.org/doc/html/draft-ietf-teas-nrp-yang}{draft-ietf-teas-nrp-yang}. The conversation focused on the flexibility of the model and its adaptability to evolving data plane specifications.

\subsubsection{Scalability Considerations for NRPs}
Jie Dong presented scalability considerations for NRPs, referencing \href{https://datatracker.ietf.org/doc/html/draft-ietf-teas-nrp-scalability}{draft-ietf-teas-nrp-scalability}. The dialogue underscored the need for detailed scalability analysis within protocol extensions to ensure robust implementation.

\subsubsection{IETF Network Slice Application in 3GPP 5G}
Luis Miguel Contreras Murillo's presentation on the application of IETF network slices in 3GPP 5G networks, as per \href{https://datatracker.ietf.org/doc/html/draft-ietf-teas-5g-network-slice-application}{draft-ietf-teas-5g-network-slice-application}, proceeded without comments, indicating broad agreement or understanding.

\subsubsection{IETF Network Slice Controller Models}
Luis Miguel Contreras Murillo also discussed the network slice controller and associated data models, detailed in \href{https://datatracker.ietf.org/doc/html/draft-ietf-teas-ns-controller-models}{draft-ietf-teas-ns-controller-models}. The potential merging of related drafts was considered to streamline efforts.

\subsubsection{Applicability of IETF-Defined Models}
The applicability of IETF-defined service and network data models for network slice service management was explored by Luis Miguel Contreras Murillo, referencing \href{https://datatracker.ietf.org/doc/html/draft-barguil-teas-network-slices-instantation}{draft-barguil-teas-network-slices-instantation}. Discussions included the potential merging with existing WG documents to consolidate efforts.

\subsubsection{Network Slice Topology YANG Data Model}
Aihua Guo presented the network slice topology YANG data model, as per \href{https://datatracker.ietf.org/doc/html/draft-liu-teas-transport-network-slice-yang}{draft-liu-teas-transport-network-slice-yang}. The session gauged interest in the topic and considered the draft as a starting point for further exploration.

\subsubsection{5QI to DiffServ DSCP Mapping}
Krzysztof Szarkowicz discussed the mapping of 5QI to DiffServ DSCP for enforcing 5G network slice QoS, detailed in \href{https://datatracker.ietf.org/doc/html/draft-cbs-teas-5qi-to-dscp-mapping}{draft-cbs-teas-5qi-to-dscp-mapping}. The discussion emphasized the practical implementation within existing IP/MPLS frameworks.

\subsubsection{ACTN for Packet Optical Integration}
Paolo Volpato presented on the applicability of ACTN for packet optical integration service assurance, as per \href{https://datatracker.ietf.org/doc/html/draft-poidt-teas-actn-poi-assurance}{draft-poidt-teas-actn-poi-assurance}. The focus was on addressing gaps in device models and enhancing communication between network components.

Meeting materials are available at \href{https://datatracker.ietf.org/meeting/120/session/teas}{TEAS Session Materials}.



\newpage

\section{Traffic Engineering Architecture and Signaling (TEAS) [TEAS]}

\subsection{Attendees}
\subsubsection{Overview}
The TEAS working group session at IETF 121 was attended by representatives from prominent companies and institutions, including Cisco, Huawei, Nokia, Ericsson, and Telefonica, with a total attendance of over 60 participants. The session facilitated discussions on various drafts and technical advancements in traffic engineering and network slicing.

\subsection{Meeting Discussions}

\subsubsection{Administrivia \& WG Status}
The session commenced with an overview of the working group's status, presented by the chairs. Key updates and administrative matters were addressed, setting the stage for subsequent discussions.

\subsubsection{WG Draft Updates}
The chairs provided updates on working group drafts, emphasizing ongoing discussions in the NMOP and Hackathon activities. Notably, the TE topology profile work was highlighted due to its relevance in current discussions. The need for automated profile definitions was debated, with suggestions to engage the NETMOD WG for further exploration.

\subsubsection{YANG Data Model for SR and SR TE Topologies on MPLS Data Plane}
\href{https://datatracker.ietf.org/doc/draft-ietf-teas-yang-sr-te-topo/19/}{draft-ietf-teas-yang-sr-te-topo} was presented by Xufeng Liu. The group agreed that further revisions are necessary before proceeding to Working Group Last Call (WGLC).

\subsubsection{SF Aware TE Topology YANG Model}
Xufeng Liu discussed the \href{https://datatracker.ietf.org/doc/draft-ietf-teas-sf-aware-topo-model/13/}{draft-ietf-teas-sf-aware-topo-model}. Issues with diagram formats were identified, and solutions involving SVG translations were proposed to resolve submission tool limitations.

\subsubsection{Applicability of ACTN to Packet Optical Integration (POI)}
Italo Busi presented \href{https://datatracker.ietf.org/doc/draft-ietf-teas-actn-poi-applicability/12/}{draft-ietf-teas-actn-poi-applicability}. Discussions focused on clarifying model roles, with plans to initiate a mailing list thread for further input.

\subsubsection{IETF Network Slice Application in 3GPP 5G End-to-End Network Slice}
Luis Miguel Contreras Murillo's presentation on \href{https://datatracker.ietf.org/doc/draft-ietf-teas-5g-network-slice-application/03/}{draft-ietf-teas-5g-network-slice-application} led to a decision to seek 3GPP feedback post-revision.

\subsubsection{IETF Network Slice Controller and its Associated Data Models}
The discussion on \href{https://datatracker.ietf.org/doc/draft-ietf-teas-ns-controller-models/03/}{draft-ietf-teas-ns-controller-models} emphasized a strategic pause to align with related document maturity.

\subsubsection{Realizing Network Slices in IP/MPLS Networks}
Tarek Saad presented \href{https://datatracker.ietf.org/doc/draft-ietf-teas-ns-ip-mpls/04}{draft-ietf-teas-ns-ip-mpls}, sparking a debate on NRP Dataplane ID length consistency. The group agreed to seek broader WG input before finalizing.

\subsubsection{Scalability Considerations for Network Resource Partition}
Jie Dong's \href{https://datatracker.ietf.org/doc/draft-ietf-teas-nrp-scalability/06/}{draft-ietf-teas-nrp-scalability} presentation revisited concerns about routing protocol reliance, with plans to refine the document text collaboratively.

\subsubsection{RSVP Cryptographic Authentication, Version 2}
Tony Li discussed \href{https://datatracker.ietf.org/doc/draft-atkinson-teas-rsvp-auth-v2/00/}{draft-atkinson-teas-rsvp-auth-v2}, highlighting the need for coordination with TSVWG and SAAG WG for comprehensive feedback.

\subsubsection{Hackathon Update - IETF TE Topology Profile Graphical Display}
Henry Yu provided insights into the Hackathon outcomes, with plans to extend comparative modeling exercises in future events.

Meeting materials are available at \href{https://datatracker.ietf.org/meeting/121/session/teas}{TEAS Session Materials}.


\newpage


\section{Transport Layer Security Working Group (TLS)}

\subsection{Attendees Overview}
\subsubsection{Overview}
The meeting was attended by representatives from prominent companies and institutions, including Mozilla, Google, Cisco Systems, and Microsoft, with a total attendance of 116 participants.

The discussions centered around the latest updates and proposals for the TLS protocol. Key topics included updates to RFCs 8446 and 8447, Encrypted Client Hello (ECH), and various draft proposals aimed at enhancing the security and efficiency of TLS. The debates were informed by the need to balance security with performance, and the implications of each proposal were carefully considered. Draft documents such as \href{https://datatracker.ietf.org/doc/html/draft-ietf-tls-rfc8446bis}{draft-ietf-tls-rfc8446bis} and \href{https://datatracker.ietf.org/doc/html/draft-ietf-tls-rfc8447bis}{draft-ietf-tls-rfc8447bis} played a significant role in guiding the discussions.

Meeting materials are available through the direct link \href{https://www.ietf.org/proceedings/119/materials/}{IETF 119 Materials}.

\subsection{Meeting Discussions}

\subsubsection{8446/8447 Updates}
The group discussed the processing of errata for RFCs 8446 and 8447, emphasizing the importance of addressing community feedback before advancing the documents.

\subsubsection{ECH Update}
The Encrypted Client Hello (ECH) update was a focal point, with robust discussions on the last call feedback. The consensus was to move forward with the current proposals, acknowledging the need for careful consideration of any potential issues raised.

\subsubsection{Registry Update}
The update on the TLS registry highlighted the smooth processing of registration requests and the importance of keeping the TLS group informed about standardizations within the IETF.

\subsubsection{TLS Hybrid Key Exchange}
The TLS Hybrid Key Exchange draft was debated, with two main issues: awaiting FIPS certification and deciding on the method for combining shared secrets. The group leaned towards a decision that would not disrupt the existing TLS key schedule.

\subsubsection{TLS Obsolete Key Exchange}
The Working Group Last Call (WGLC) for the TLS Obsolete Key Exchange was completed, and the group discussed classifications for key exchanges, with a focus on aligning with RFC 8447bis and ensuring appropriate recommendations for various key exchange methods.

\subsubsection{TLS Formal Analysis}
A proposal was made to formalize the process for triaging formal analysis of TLS 1.3, with the aim of involving experts early in the process to assess the need for formal analysis of new specifications. The group agreed on the importance of this step without allowing it to become a bottleneck.

\subsubsection{Super Jumbo Record Limit}
The proposal to increase the plaintext record size limit in TLS to improve performance was well-received, with the group agreeing to further explore the design details and consider adoption based on performance metrics and security considerations.

\subsubsection{MTLS Flag}
The MTLS Flag proposal, which aims to facilitate the distinction of bots by signaling the availability of a client certificate, was discussed. The group sought more enthusiasm and clearer benefits before moving forward.

\subsubsection{Extended Key Update}
The Extended Key Update draft, aimed at providing forward secrecy for long-lived TLS connections, was debated. The group discussed the merits of application-layer involvement versus complete TLS-layer handling, with a focus on simplifying the design without interrupting communication.

\subsubsection{MLKEM Key Agreement}
Finally, the MLKEM Key Agreement draft was presented, with discussions on the need for a standalone Post-Quantum (PQ) key agreement method. The group considered the option of registering a code point to facilitate early adoption by interested parties.

\end{document}



\newpage

\section{Transport and Services Working Group (TSVWG) WG}

\subsection{Attendees}

\subsubsection{Overview}

The TSVWG session at IETF-121 in Dublin saw participation from a diverse group of attendees, including representatives from prominent companies such as Ericsson, Cisco, Google, and Netflix, as well as academic institutions like the University of Aberdeen and RWTH Aachen University. The total attendance was over 70 participants, reflecting a broad interest in the working group's activities.

\subsection{Meeting Discussions}

\subsubsection{WG Status and Draft Updates}

The session began with updates on the status of various drafts. Recently published RFCs include \href{https://datatracker.ietf.org/doc/html/draft-ietf-tsvwg-ecn-encap-guidelines}{draft-ietf-tsvwg-ecn-encap-guidelines} (RFC 9599) and \href{https://datatracker.ietf.org/doc/html/draft-ietf-tsvwg-rfc6040update-shim}{draft-ietf-tsvwg-rfc6040update-shim} (RFC 9601). Discussions highlighted the progress of drafts with the IESG, such as \href{https://datatracker.ietf.org/doc/html/draft-ietf-tsvwg-multipath-dccp}{draft-ietf-tsvwg-multipath-dccp}, and those beyond Working Group Last Call (WGLC), including \href{https://datatracker.ietf.org/doc/html/draft-ietf-tsvwg-nqb}{draft-ietf-tsvwg-nqb}. The chairs emphasized milestone updates and encouraged participants to review individual drafts ahead of the next session.

\subsubsection{Transport Drafts}

Raffaello Secchi presented on the \href{https://datatracker.ietf.org/doc/html/draft-ietf-tsvwg-careful-resume}{draft-ietf-tsvwg-careful-resume}, focusing on the careful resumption of congestion control. The presentation included performance analysis using QUIC implementations, demonstrating reduced flow completion times across various network conditions. The working group expressed readiness to move the draft to last call, indicating strong support for its advancement.

\subsubsection{WG Differentiated Services: L4S \& NQB}

Jason Livingood provided an update on L4S trials and deployment, noting successful phased deployment by Comcast and the readiness of the technology to scale. The discussion underscored the potential for L4S to enhance user experiences, particularly in live sports streaming. Greg White's presentation on \href{https://datatracker.ietf.org/doc/html/draft-ietf-tsvwg-nqb}{draft-ietf-tsvwg-nqb} confirmed resolution of previous issues, paving the way for document shepherd review.

\subsubsection{SCTP Encryption}

Michael Tuexen discussed \href{https://datatracker.ietf.org/doc/html/draft-ietf-tsvwg-rfc4895-bis}{draft-ietf-tsvwg-rfc4895-bis}, focusing on SCTP authentication chunks and replay protection improvements. The working group agreed to progress the draft, with expectations for stability by the end of 2024. Discussions on DTLS 1.3 for SCTP highlighted a new approach to address IPR concerns, with plans for initial drafts by IETF 122.

\subsection{Meeting Materials}

Meeting materials, including slides and detailed notes, are available at \href{https://datatracker.ietf.org/wg/tsvwg/documents/}{TSVWG Documents}.

\subsection{Next Steps}

The working group will continue to refine drafts and address feedback, with a focus on advancing key documents through the IETF process. Upcoming interop activities and further deployment trials will inform future discussions, ensuring that TSVWG remains at the forefront of transport and services innovation.



\newpage

\section{Time Variant Routing (TVR) Working Group (tvr)}

\subsection{Attendees}

The meeting was attended by representatives from prominent companies and institutions such as Cisco Systems, Nokia, Huawei Technologies, and Telefonica, among others. The total attendance was approximately 40 participants, showcasing a diverse range of expertise and interest in the ongoing developments of the TVR Working Group.

\subsection{Meeting Discussions}

\subsubsection{Introduction, Note Well \& Milestones}

The session commenced with the chairs providing an overview of the administrative details and milestones. The recent publication of RFC9657 was highlighted, and participants were encouraged to consider future work and contributions.

\subsubsection{TVR Applicability}

Li Zhang presented on the applicability of TVR, focusing on the use of data models in Tidal Networks. The discussion emphasized the need for multiple use cases to ensure comprehensive applicability, with participants like Rick Taylor and Daniel King suggesting the inclusion of diverse scenarios. The draft document can be accessed at \href{https://datatracker.ietf.org/doc/draft-wqb-tvr-applicability/}{draft-wqb-tvr-applicability}.

\subsubsection{TVR Requirements}

Daniel King discussed the requirements for TVR, incorporating feedback from previous sessions. The presentation addressed security considerations and the need for clear requirements rather than optional features. The draft is available at \href{https://datatracker.ietf.org/doc/draft-ietf-tvr-requirements/}{draft-ietf-tvr-requirements}.

\subsubsection{ALTO Exposure}

Luis Contreras presented the ALTO Exposure draft, highlighting its alignment with security considerations and the architecture involving schedule requesters and responders. The draft is considered stable, and adoption was discussed. The document can be found at \href{https://datatracker.ietf.org/doc/draft-contreras-tvr-alto-exposure/}{draft-contreras-tvr-alto-exposure}.

\subsubsection{YANG Model}

Yingzhen Qu introduced the YANG model for TVR, which now includes a consistent device model using URIs for node identification. The discussion focused on the granularity of the model and potential extensions for time baselines. The draft is accessible at \href{https://datatracker.ietf.org/doc/draft-ietf-tvr-schedule-yang/}{draft-ietf-tvr-schedule-yang}.

\subsubsection{Experiences with the TVR Schedule Model}

Marc Blanchet shared experiences with the TVR schedule model, particularly in the context of Earth to Mars communications. The discussion touched on the challenges of time synchronization across different celestial bodies, with suggestions to leverage existing standards like CCSDS for time synchronization.

\subsubsection{Open Mic}

The session concluded with an open mic segment, inviting comments on the discussed documents and encouraging further engagement and review from the community.

Meeting materials and further details are available at \href{https://datatracker.ietf.org/meeting/121/materials/agenda-tvr-00}{Meeting Materials}.



\newpage

\section{vCon Working Group (vCon)}

\subsection{Attendees Overview}
\subsubsection{Participants}
The vCon Working Group meeting was attended by representatives from prominent companies and institutions, including Somos, Inc., Meta Platforms, Inc., AIST Japan, Data Transfer Initiative, Cloudflare, and ALAXALA Networks, Corp. The total attendance was 28 participants.

\subsection{Meeting Discussions}

\subsubsection{Draft Discussions}
The session commenced with Rohan Mahy's presentation on \href{https://datatracker.ietf.org/doc/html/draft-mahy-vcon-mimi-messages}{draft-mahy-vcon-mimi-messages}, focusing on the integration of mimi messages within vCon. The discussion highlighted the challenges of managing message history across devices and the implications of metadata changes. Murray Kucherawy raised concerns about the use of mimetype, suggesting potential friction if not revised. The group considered the balance between preserving important fields from mimi in vCon and the need for extensions.

Dan Petrie presented the \href{https://datatracker.ietf.org/doc/draft-ietf-vcon-vcon-container/}{draft-ietf-vcon-vcon-container}, emphasizing use cases in contact centers and messaging. The discussion explored naming conventions, with a consensus on using snake case for vCon. The need for a consistent approach to redaction and the handling of appended data was debated, with suggestions for further exploration in upcoming hackathons.

Diana James introduced the \href{https://datatracker.ietf.org/doc/draft-james-privacy-primer-vcon/}{draft-james-privacy-primer-vcon}, underscoring the importance of privacy by design. The draft aims to guide new actors in understanding privacy principles, with discussions on anonymization and PII masking. The group debated the draft's adoption as a Best Current Practice (BCP) or informational document, recognizing its relevance to the working group.

\subsubsection{Presentation: Privacy by Default}
Steve Lasker presented on Privacy by Default, building on Diana's talk. He proposed integrating privacy by design primitives into vCon, suggesting the inclusion of license, consent lifetime, and data controller identifiers. The presentation sparked a dialogue on the lawful basis for consent and the need to distinguish consent from licensing.

\subsubsection{Wrap Up}
The meeting concluded with a wrap-up by the chairs, summarizing the discussions and outlining the next steps for the working group.

Meeting materials are available at \href{https://datatracker.ietf.org/meeting/121/materials/agenda-121-vcon-00}{IETF 121 vCon Meeting Materials}.



\newpage

\section{Workload Identity in Multi-System Environments (WIMSE)}

\subsection{Attendees}

\subsubsection{Overview}
The WIMSE working group meeting was attended by representatives from prominent companies and institutions, including Microsoft, Google LLC, Meta Platforms, Inc., JPMorgan Chase & Co., and the US NSA, among others. The total attendance was approximately 70 participants, highlighting the significant interest and diverse expertise present in the discussions.

\subsection{Meeting Discussions}

\subsubsection{Welcome and Chair Update}
The chairs, Pieter and Justin, provided updates on their affiliations and emphasized the ongoing work on the WIMSE architecture, OAuth Client Authentication BCP, and Service-to-Service specifications. They encouraged participants to review the \href{https://ietf-wg-wimse.github.io}{working group website} for more information on current documents and initiatives.

\subsubsection{Why are We Here - Brian Campbell}
Brian Campbell presented an overview of the WIMSE workflow, emphasizing the importance of service-to-service authentication and transaction tokens. He highlighted the potential integration of the \href{https://datatracker.ietf.org/doc/html/draft-ietf-wimse-architecture}{architecture draft} with existing solutions like RFC 7523 and the PIKA proposal. The discussion underscored the need for careful consideration of cross-domain token exchange and the potential re-chartering of the working group to address emerging challenges.

\subsubsection{Architecture Update - Joe Salowey}
Joe Salowey discussed updates to the architecture specification, focusing on identity and trust domain definitions. The presentation highlighted the alignment with the Service-to-Service specification and the importance of defining workload identifiers. The group considered the implications of using fully qualified domain names (FQDNs) and discussed potential extensions to the SPIFFE-compatible format.

\subsubsection{Service-to-Service Update - Joe, Brian, Arndt}
The update covered changes to the identity section and the introduction of workload proof tokens. The discussion explored the use of mTLS and application-specific approaches for service-to-service authentication, with considerations for both WPT and HTTPSig. The group debated the merits of supporting multiple authentication methods and the potential interoperability challenges.

\subsubsection{Token Exchange Update - Dean Saxe}
Dean Saxe provided insights into the progress on token exchange, emphasizing the need for profiles to guide the transformation of tokens across domains. The discussion highlighted the challenges of maintaining security and fidelity during token translation and the importance of defining explicit mechanisms for trusted exchanges.

\subsubsection{Informational/BCP Update - Arndt}
Arndt presented the shift from best practices to documenting current practices in workload identity, expanding the scope to include SPIFFE and cloud providers. The presentation outlined common patterns in workload identity management and the security considerations associated with different platforms.

\subsubsection{Authentication Maturity Levels - Ryan Hurst}
Ryan Hurst introduced the concept of authentication maturity levels, aiming to provide a structured framework for organizations to assess and improve their security posture. The discussion focused on the challenges of maintaining relevance over time and the potential for integrating these levels into existing security frameworks.

\subsection{Meeting Materials}
The meeting materials, including slides and detailed notes, are available at \href{https://ietf-wg-wimse.github.io/meetings/ietf-121}{WIMSE IETF 121 Meeting Materials}.




\newpage

\section{WebRTC-HTTP Ingest Signaling Handshake (WISH) [WISH]}

\subsection{Attendees Overview}
The WISH working group meeting was attended by representatives from prominent companies and institutions, including Dolby, Cloudflare, NTT Communications, Ericsson, and Google, among others. In total, there were 25 attendees, highlighting the diverse interest and expertise present in the discussions.

\subsection{Meeting Discussions}

\subsubsection{WHEP Draft Presentation}
Sergio Garcia Murillo from Dolby presented the \href{https://datatracker.ietf.org/doc/draft-ietf-wish-whep/}{draft-ietf-wish-whep}, focusing on the WebRTC-HTTP Egress Protocol (WHEP). The presentation addressed several open issues, including the client/server offer mechanism, server events, and the use of data channels. The group reached a consensus to use data channels for server events, with the understanding that both data channels and server-sent events will remain optional, providing extendable frameworks for future specifications.

\subsubsection{Key Discussion Points}
The discussion highlighted the need for a concrete pull request (PR) for the server offer proposal, which will be revisited on the mailing list. Juliusz Chroboczek suggested deferring section 4.9 on server events into a new experimental draft, while Tim Panton proposed using data channels, which was agreed upon. Francesca Palombini emphasized the importance of an early HTTP Director review. The group also deliberated on the WHEP URL's clarity and the necessity to define failure behaviors to prevent misuse of undefined specifications.

\subsubsection{Outcomes and Next Steps}
The meeting concluded with a commitment to draft a PR for the server offer and to engage HTTP experts for guidance on URL handling. The adoption of data channels marks a strategic shift, potentially enhancing the protocol's flexibility and future-proofing. The next steps involve refining the draft based on these discussions and preparing for further reviews.

Meeting materials are available at \href{https://datatracker.ietf.org/doc/agenda-121-wish/}{IETF Agenda Link}.



\newpage
\end{document}