\documentclass{article}
\usepackage[utf8]{inputenc}
\usepackage{hyperref}
\usepackage{geometry}
\usepackage{graphicx}
\usepackage{lmodern} 
\geometry{a4paper, margin=1in}
\title{IETF Meeting Reports}
\author{Generated by IETF Reporter, programmed by Jaime Jiménez}
\date{\today}
\begin{document}
\maketitle
\tableofcontents
\newpage

\section{IPv6 over Networks of Resource-constrained Nodes (6lo)}

\subsection{Attendees Overview}
\subsubsection{Participants}
The 6lo working group session was attended by 19 participants, including representatives from prominent institutions such as Universitat Politecnica de Catalunya, Ribbon Communications, ETRI, NEC Platforms, Ltd., Cisco, Futurewei, Huawei Technologies France, and Meta. The session was chaired by Shwetha Bhandari and Carles Gomez, with Éric Vyncke serving as the responsible Area Director.

\subsection{Meeting Discussions}

\subsubsection{Introduction and Draft Status}
The session commenced with agenda bashing and administrative tasks, including the distribution of blue sheets and the assignment of scribes. The chairs provided an overview of the current draft status, highlighting the need for updates to RFC 8928 due to overlapping flags in the Extended Address Registration Option (EARO) as identified by Adnan Rashid. Éric Vyncke will investigate the procedure for updating the flag position beyond an erratum, considering the recent deployment status of RFC 9685. Meeting materials are available \href{https://datatracker.ietf.org/meeting/122/materials/slides-122-6lo-6lo-chairs-introduction-01}{here}.

\subsubsection{IPv6 Neighbor Discovery Prefix Registration}
Luigi Iannone, standing in for Pascal Thubert, discussed the main changes in the draft, focusing on energy-saving motivations and the limitations of relying on RIO. Carles Gomez confirmed the completion of the Working Group Last Call (WGLC) and shepherd write-up, with Éric Vyncke advising to request publication, anticipating IESG evaluation by the end of May. The draft can be accessed \href{https://datatracker.ietf.org/doc/html/draft-ietf-6lo-prefix-registration-06}{here}.

\subsubsection{Path-Aware Semantic Addressing for LLNs}
Luigi Iannone acknowledged the contributions of Brian Carpenter and Carles Gomez in refining the draft. A decision was made to hold the document in the working group until the Generic Address Assignment Option (GAAO) draft reaches a similar state, allowing for simultaneous submission. The draft is available \href{https://datatracker.ietf.org/doc/html/draft-ietf-6lo-path-aware-semantic-addressing-11}{here}.

\subsubsection{Generic Address Assignment Option for 6LoWPAN ND}
The discussion centered on the motivations for GAAO compared to DHCPv6/SLAAC. Carles Gomez indicated the need for additional reviews before initiating the WGLC, with plans to request early reviews. The draft can be found \href{https://datatracker.ietf.org/doc/html/draft-ietf-6lo-nd-gaao-02}{here}.

\subsubsection{Transmission of SCHC-compressed Packets over IEEE 802.15.4}
Carles Gomez outlined the main changes aligning with the SCHC architecture and proposed next steps, including protocol stack elaboration and appendix completion. The draft is accessible \href{https://datatracker.ietf.org/doc/html/draft-ietf-6lo-schc-15dot4-09}{here}.

\subsubsection{Transmission of IPv6 Packets over Short-Range OWC}
Younghwan Choi addressed feedback from IEEE802.15 and IANA, with plans to update the security section in future revisions. Éric Vyncke advised revising normative versus informative references. The draft is available \href{https://datatracker.ietf.org/doc/html/draft-ietf-6lo-owc-03}{here}.

\subsubsection{Conclusion}
The session concluded with no additional business, and Carles Gomez thanked participants, expressing anticipation for the next meeting in Madrid. The session adjourned at 13:51.



\newpage

\section{6MAN Working Group (6MAN)}

\subsection{Attendees}
\subsubsection{Overview}
The meeting was attended by 82 participants, including representatives from prominent companies and institutions such as The Boeing Company, Microsoft, Juniper Networks, Ericsson, NOKIA, Check Point Software, ZTE, Cisco, Google, SAP SE, and Deutsche Telekom.

\subsection{Meeting Discussions}

\subsubsection{Internet Control Message Protocol (ICMPv6) Reflection}
Ron Bonica presented the \href{https://datatracker.ietf.org/doc/html/draft-ietf-6man-icmpv6-reflection}{draft-ietf-6man-icmpv6-reflection}, discussing the implementation status and seeking a working group last call. The discussion highlighted the need for a description of alternative methods currently used for debugging purposes, such as sending a copy of the invoking packet back to the sender. The potential security implications of sending back packet content were also debated. The chairs decided to initiate a working group last call.

\subsubsection{Improving the Robustness of Stateless Address Autoconfiguration (SLAAC) to Flash Renumbering Event}
Fernando Gont discussed the \href{https://datatracker.ietf.org/doc/html/draft-ietf-6man-slaac-renum}{draft-ietf-6man-slaac-renum}, focusing on the new default parameters to enhance loss resilience. The discussion emphasized the importance of updating these parameters to prevent device disconnection under packet loss conditions. The group agreed to proceed with a working group last call to gather further feedback.

\subsubsection{IPv6 Node Requirements}
Tim Winters presented the \href{https://datatracker.ietf.org/doc/html/draft-ietf-6man-rfc8504-bis}{draft-ietf-6man-rfc8504-bis}, addressing the inclusion of informational extension header limits. The discussion centered on whether to mention these limits in the document, with some members expressing concerns about setting hard limits. The group decided to review the current text and consider potential updates.

\subsubsection{Clarifying SRv6 SID List Processing}
Adrian Farrel introduced the \href{https://datatracker.ietf.org/doc/html/draft-farrel-6man-sidlist-clarification}{draft-farrel-6man-sidlist-clarification}, aiming to clarify the processing of SRv6 SID lists. The discussion acknowledged the need for updates to ensure consistency with existing standards. The chairs will coordinate with SPRING chairs and ADs to determine the appropriate ownership and status of the document.

\subsubsection{Using Prefix-Specific Link-Local Addresses to Improve SLAAC Robustness}
Jen Linkova discussed the \href{https://datatracker.ietf.org/doc/html/draft-link-6man-gulla}{draft-link-6man-gulla}, which proposes using prefix-specific link-local addresses to enhance SLAAC robustness. The conversation explored the implications for multi-prefix, multi-homing environments and the potential need for DNS and other option invalidation upon renumbering events.

\subsubsection{Send/CGA for OMNI Links}
Fred L. Templin presented the \href{https://datatracker.ietf.org/doc/html/draft-templin-6man-omni3}{draft-templin-6man-omni3}, advocating for the reinstatement of SEND in IPv6 Node Requirements. The group expressed concerns about the practicality and necessity of this inclusion, emphasizing the need for operational experience and implementation feedback.

\subsubsection{Deterministic Routing Header}
Shaofu Peng introduced the \href{https://datatracker.ietf.org/doc/html/draft-pb-6man-deterministic-crh}{draft-pb-6man-deterministic-crh}, proposing a deterministic routing header. The discussion highlighted the need for alignment with existing experimental work and the importance of feedback from the detnet group.

\subsubsection{Problem Statement with Aggregate Header Limit}
Yao Liu presented the \href{https://datatracker.ietf.org/doc/html/draft-liu-6man-aggregate-header-limit-problem}{draft-liu-6man-aggregate-header-limit-problem}, focusing on the security considerations of AHL collection methods. The group agreed on the necessity of refining the draft to address potential reconnaissance risks.

\subsubsection{Extending ICMPv6 for SRv6-related Information Validation}
Yao Liu discussed the \href{https://datatracker.ietf.org/doc/html/draft-liu-6man-icmp-verification}{draft-liu-6man-icmp-verification}, which aims to extend ICMPv6 for SRv6 information validation. The group noted the importance of SPRING's input on the necessity of this extension.

\subsubsection{Terminal Identity Authentication Based on Address Label}
Kexian Liu's presentation on the \href{https://datatracker.ietf.org/doc/html/draft-guan-6man-ipv6-id-authentication}{draft-guan-6man-ipv6-id-authentication} was not conducted as the presenter was unavailable.

Meeting materials are available at \href{https://datatracker.ietf.org/meeting/122/materials/agenda-122-6man}{Meeting Materials}.



\newpage

\section{Authentication and Authorization for Constrained Environments (ACE)}

\subsection{Attendees Overview}

The ACE working group meeting at IETF 122 was attended by 25 participants, including representatives from prominent organizations such as Aiven, TU Dresden, Ericsson, Siemens AG, and RISE Research Institutes of Sweden. The diverse attendance underscored the collaborative effort across academia, industry, and research institutions to advance the group's objectives.

\subsection{Meeting Discussions}

\subsubsection{Message from Chairs}

The meeting commenced with a brief introduction by the chairs, setting the stage for the discussions. The presentation slides are available \href{https://datatracker.ietf.org/meeting/122/materials/slides-122-ace-chair-slides-ace-00}{here}.

\subsubsection{draft-ace-est-oscore (Mališa Vučinić)}

Mališa Vučinić presented updates on the \href{https://datatracker.ietf.org/doc/html/draft-ietf-ace-est-oscore}{draft-ace-est-oscore}, focusing on addressing interoperability issues and certificate handling by reference. The discussion highlighted the need for a media type review to ensure clarity in certificate references, with a preference for cleaner options that enhance protocol efficiency.

\subsubsection{draft-ietf-ace-workflow-and-params (Marco Tiloca)}

Marco Tiloca discussed the \href{https://datatracker.ietf.org/doc/html/draft-ietf-ace-workflow-and-params}{draft-ietf-ace-workflow-and-params}, emphasizing improvements in credential handling and deadlock avoidance. The presentation underscored the importance of aligning with client capabilities to optimize authentication workflows.

\subsubsection{draft-ietf-ace-authcred-dtls-profile (Marco Tiloca)}

The session on the \href{https://datatracker.ietf.org/doc/html/draft-ietf-ace-authcred-dtls-profile}{draft-ietf-ace-authcred-dtls-profile} covered updates to the DTLS profile, focusing on new options and pending registrations. The discussion reinforced the draft's role in enhancing security credential formats.

\subsubsection{draft-ietf-ace-edhoc-oscore-profile (Rikard Höglund)}

Rikard Höglund presented the \href{https://datatracker.ietf.org/doc/html/draft-ietf-ace-edhoc-oscore-profile}{draft-ietf-ace-edhoc-oscore-profile}, detailing alignment with workflows and credential validation requirements. The draft's approach to EDHOC message flows and trust anchor support was discussed, with next steps including codepoint allocations and proof-of-possession enhancements.

\subsubsection{draft-ietf-ace-group-oscore-profile (Rikard Höglund)}

The \href{https://datatracker.ietf.org/doc/html/draft-ietf-ace-group-oscore-profile}{draft-ietf-ace-group-oscore-profile} presentation focused on token request structures and the use of multiple profiles. Discussions explored the complexity of managing multiple security groups and the potential for optimization through token handling.

\subsubsection{draft-ietf-ace-pubsub-profile (Francesca Palombini)}

Francesca Palombini addressed the \href{https://datatracker.ietf.org/doc/html/draft-ietf-ace-pubsub-profile}{draft-ietf-ace-pubsub-profile}, proposing the removal of MQTT content to streamline the draft for CoAP. The session concluded with a consensus on the draft's focus and a proposed new title reflecting its scope.

\subsubsection{AOB}

The meeting concluded with an open floor for any other business, allowing participants to raise additional topics or concerns.

Meeting materials are available \href{https://datatracker.ietf.org/meeting/122/materials/}{here}.



\newpage

\section{ACME Working Group (ACME)}

\subsection{Attendees}

The ACME Working Group meeting at IETF 122 was attended by 45 participants, including representatives from prominent organizations such as ISRG/Let's Encrypt, NSA-CCSS, Dell Technologies, Google, Nokia, and Huawei. The diverse attendance underscored the broad interest and collaborative effort in advancing ACME protocols.

\subsection{Meeting Discussions}

\subsubsection{ACME Renewal Info / Profiles (Gable)}

The discussion on ACME Renewal Info highlighted the progress made since IETF 121, with the document now in the RFC Editor's queue. The ACME Profiles, fully implemented by Let's Encrypt and several clients, are poised for a call-for-adoption, emphasizing the utility of shared account information. This marks a significant step towards standardizing ACME profiles, potentially streamlining certificate management processes.

\subsubsection{DTN Node ID Validation (Sipos)}

The DTN Node ID Validation draft, now ready for Working Group Last Call (WGLC), was discussed with a focus on its readiness following the publication of its dependency, RFC 9713. This advancement is crucial for enhancing secure communications in Delay-Tolerant Networks, reflecting a strategic alignment with broader IETF goals.

\subsubsection{ACME DNS Update (Li)}

Li presented on deploying ACME in 5G core networks, proposing best practices for securely updating DNS records using DNS Update with TSIG or SIG(0). The discussion, which included considerations of security and operational efficiency, highlighted the potential for these methods to reduce barriers in domain management. The dialogue also touched on the implications of CA/B Forum ballots and the utility of standardized APIs.

\subsubsection{ACME-RATS (Liu)}

The ACME-RATS presentation focused on obtaining TLSClientAuth certificates through device attestation. The discussion explored potential design overhauls to accommodate more general attributes in attestations, with plans for a monthly design call to refine these ideas. This initiative aims to enhance mutual TLS security by enforcing client policies.

\subsubsection{JWTClaimConstraints (Wendt)}

Wendt introduced a new identifier type, JWTClaimConstraint, seeking adoption for a draft that extends RFC 8226. This proposal, already presented in the STIR WG, aims to streamline the process of token issuance and challenge fulfillment, potentially impacting how ACME clients interact with token authorities.

\subsubsection{ACME PK Challenges (Wu)}

Wu's presentation on ACME PK Challenges addressed security concerns where ACME Clients might replace public keys. The proposed mechanisms, including WebAuthn and NIZK, aim to ensure key integrity across the ACME flow. This discussion is pivotal for environments where client trust is variable, prompting considerations of broader PKI applications.

\subsubsection{Any Other Business}

\textbf{AutoDiscovery (Ounsworth):} The session concluded with a discussion on autodiscovery drafts, highlighting their relevance in the context of shorter certificate lifetimes considered by the CA/B Forum. The potential for these drafts to mitigate single points of failure in certificate renewal processes was noted, with an invitation for interested parties to continue their development.

Meeting materials are available at \href{https://datatracker.ietf.org/meeting/122/materials.html}{IETF 122 Meeting Materials}.




\newpage

\section{AIPREF Working Group (AIPREF)}

\subsection{Attendees Overview}
\subsubsection{Attendance and Representation}
The AIPREF Working Group meeting at IETF 122 was attended by 98 participants, representing a diverse array of prominent organizations including Google, Apple, Cisco, Cloudflare, Ericsson, and Nokia. Academic institutions such as Saveetha University and Keio University were also present, alongside representatives from regulatory bodies like Ofcom and the UK NCSC.

\subsection{Meeting Discussions}
\subsubsection{Questions for the Group}
The session opened with a discussion on the scope of vocabulary and attachment mechanisms. Jonathan Hoyland raised concerns about the exclusion of authentication and authorization from the scope, while Mirja Kühlewind and Lucas Pardue emphasized the importance of defining trust separately from hints, as seen in other protocols like HTTP. The group agreed that a simple vocabulary scope is crucial for timely delivery, with Suresh Krishnan advocating for a straightforward approach to vocabulary complexity.

\subsubsection{Attachment by Location}
Gary Illyes highlighted scalability issues with using \texttt{robots.txt} for large media sites, suggesting HTTP headers as a more viable alternative. Martin Thomson acknowledged the limitations of \texttt{robots.txt} and supported exploring HTTP headers for better scalability and efficiency.

\subsubsection{Opt-In/Out Mechanisms}
The discussion on opt-in/out mechanisms revealed differing views on default states and the implications of preference expressions. Richard Barnes argued that the default state should remain the status quo, while David Schinazi and others debated the interpretation of opt-in and opt-out signals. The group recognized the need for clear definitions to avoid ambiguity in preference signaling.

\subsubsection{Attachment Mechanisms}
Various attachment mechanisms were considered, including HTTP header fields and JSON formats. Thomas McCarthy-Howe suggested leveraging the vCon WG for describing consent in virtual communications. The group acknowledged the potential complexity of cryptographic binding but agreed to focus on mechanisms with immediate impact.

\subsubsection{Combining Preferences}
Suresh Krishnan identified the reconciliation of multiple preference expressions as a complex task, with a delivery timeline set for August. The group discussed the importance of defining a clear vocabulary to facilitate this process, with Martin Thomson emphasizing the need for detailed discussions on vocabulary specifics.

\subsubsection{Licensing Schemes and Policy Matters}
Licensing schemes were deemed out of scope for initial work, with Suresh Krishnan stressing the importance of communicating the group's focus to policymakers. Farzaneh highlighted the relevance of robots.txt in policy documents, underscoring the need for ongoing dialogue with policymakers.

\subsection{Candidate Drafts}
Martin Thomson presented his draft, which proposes a simple HTTP header and extensions to \texttt{robots.txt}, focusing on usage and training without distinguishing between them. Gary Illyes discussed his draft, which redefines existing headers to accommodate structured fields. The group was encouraged to review Thom Vaughan's draft for additional insights.

\subsection{Open Mic}
Lucas Pardue raised the issue of data posting, suggesting that the design should support bidirectional data flow. Martin Thomson noted that the proposed header field could accommodate this use case. The group debated the utility of expiration times in preference expressions, with Alissa Cooper cautioning against complexity unless justified by compelling use cases.

Meeting materials are available at \href{https://datatracker.ietf.org/meeting/122/materials/slides-122-aipref-chair-slides-00.pdf}{slides-122-aipref-chair-slides-00.pdf}.




\newpage

\section{Audio/Video Transport Core Maintenance (avtcore)}

\subsection{Attendees Overview}
The meeting was attended by representatives from prominent companies and institutions such as Nokia, Samsung, Google, Cisco, and Apple, with a total attendance of 26 participants. Notable attendees included Jonathan Lennox from 8x8/Jitsi, Youngkwon Lim from Samsung, and Harald Alvestrand from Google.

\subsection{Meeting Discussions}

\subsubsection{Preliminaries}
The session began with Jonathan Lennox presenting the Note Well, emphasizing IPR disclosure and meeting conduct. Marius Kleidl was introduced as the new co-chair. The status of various drafts was discussed, with Magnus Westerlund commenting on the RTP payload format registry work. Richard Barnes expressed interest in the \href{https://datatracker.ietf.org/doc/html/draft-ietf-avtcore-sframe}{draft-ietf-avtcore-sframe}, which will be parked for future work.

\subsubsection{RTP Payload for V3C}
Lauri Ilola presented the \href{https://datatracker.ietf.org/doc/html/draft-ietf-avtcore-rtp-v3c}{draft-ietf-avtcore-rtp-v3c}, noting that the second WGLC is ongoing until March 27, with all previous issues resolved except for a new IANA feedback issue.

\subsubsection{RTP over QUIC}
Mathis Engelbart discussed the \href{https://datatracker.ietf.org/doc/html/draft-ietf-avtcore-rtp-over-quic}{draft-ietf-avtcore-rtp-over-quic}, highlighting ongoing interop tests and a pending IANA review issue. The group considered proceeding to WGLC in parallel with SDP work.

\subsubsection{SDP Offer/Answer for RTP over QUIC}
Spencer Dawkins presented updates on the \href{https://datatracker.ietf.org/doc/html/draft-dawkins-avtcore-sdp-roq}{draft-dawkins-avtcore-sdp-roq}, including security considerations and open questions. Feedback from Harald Alvestrand and Jonathan Lennox was discussed, with plans for a -01 version before the next interim meeting.

\subsubsection{RTP Payload for Haptics}
Hyunsik Yang updated on the \href{https://datatracker.ietf.org/doc/html/draft-ietf-avtcore-rtp-haptics}{draft-ietf-avtcore-rtp-haptics}, stating it is ready for WGLC with updates in the SDP sections.

\subsubsection{RTP Payload for V-DMC}
Hyunsik Yang also presented the \href{https://datatracker.ietf.org/doc/html/draft-hsyang-avtcore-rtp-vdmc}{draft-hsyang-avtcore-rtp-vdmc}, inviting reviews and participation, though not yet ready for WG adoption.

\subsubsection{RTP Payload for APV}
Youngkwon Lim discussed the \href{https://datatracker.ietf.org/doc/html/draft-lim-rtp-apv}{draft-lim-rtp-apv}, explaining the new bitstream structure and clarifying questions about marker bits and fragment counters.

\subsubsection{Automatic Corruption Detection}
Erik Språng presented the \href{https://datatracker.ietf.org/doc/draft-sprang-avtcore-corruption-detection}{draft-sprang-avtcore-corruption-detection}, proposing an RTP header extension for error detection, with implementation in Chromium M132+.

\subsubsection{LTR Feedback}
Erik Språng also discussed the \href{https://datatracker.ietf.org/doc/draft-sprang-avtcore-frame-acknowledgement}{draft-sprang-avtcore-frame-acknowledgement}, proposing a header extension for RTCP feedback on video frame decoding status.

\subsubsection{Wrapup and Next Steps}
Jonathan Lennox and Marius Kleidl outlined action items, including parking the sframe draft, initiating WGLC for the RTP over QUIC draft, and starting WGLC for the RTP Haptics draft. The potential for an interim meeting was noted.

Meeting materials are available \href{https://docs.google.com/presentation/d/1E_XfsZbcSMyNl-p3L8jUk2HkD1hfqY35TPzr0aXcV0w/}{here}.




\newpage

\section{BGP Enabled Services (BESS)}

\subsection{Attendees Overview}
\subsubsection{Attendance}
The BESS working group session was attended by 69 participants, including representatives from prominent companies and institutions such as Juniper Networks, Verizon, Nokia, Ericsson, Cisco, and Huawei. The diverse attendance underscores the broad interest and collaborative effort in advancing BGP-enabled services.

\subsection{Meeting Discussions}

\subsubsection{Working Group Update}
The session commenced with an update from the chairs, highlighting the publication of RFC 9746 and RFC 9744 since the last IETF meeting. Six drafts are currently in the RFC editor queue, and the \href{https://datatracker.ietf.org/doc/html/draft-ietf-bess-bgp-srv6-args}{draft-ietf-bess-bgp-srv6-args} is under telechat review. Discussions emphasized the need for active participation to prevent drafts from expiring and the importance of addressing feedback for the \href{https://datatracker.ietf.org/doc/html/draft-ietf-bess-mvpn-ipv6-infras}{bess-mvpn-ipv6-infras} draft, which requires a new solution approach.

\subsubsection{Process: draft-hares-bess-idr-tea-templates}
Susan Hares presented the \href{https://datatracker.ietf.org/doc/html/draft-hares-bess-idr-tea-templates}{draft-hares-bess-idr-tea-templates}, focusing on the use of extended communities for tunnel attributes. The discussion centered around the necessity of adhering to established formats for sub-TLVs, with the aim of streamlining the review process and reducing review time.

\subsubsection{Charter Update}
The chairs discussed the ongoing efforts to update the BESS charter, which has raised questions regarding its scope. The update aims to clarify whether new VPN services fall under BESS's purview, with flexibility being a key consideration. The rechartering process is ongoing, with input from the working group and the AD.

\subsubsection{Working Group Draft Updates}
Linda Dunbar presented the \href{https://datatracker.ietf.org/doc/html/draft-ietf-bess-bgp-sdwan-usage}{draft-ietf-bess-bgp-sdwan-usage}, noting that all comments have been addressed and another Working Group Last Call (WGLC) is needed. Ali Sajassi's \href{https://datatracker.ietf.org/doc/html/draft-ietf-bess-rfc7432bis}{draft-ietf-bess-rfc7432bis} introduced new text based on feedback, with a request for WGLC. Jorge Rabadan's drafts, \href{https://datatracker.ietf.org/doc/html/draft-ietf-bess-evpn-ip-aliasing}{evpn-ip-aliasing} and \href{https://datatracker.ietf.org/doc/html/draft-ietf-bess-evpn-dpath}{evpn-dpath}, are ready for WGLC, with significant community support.

\subsubsection{Individual Rev 0 bis Drafts}
Mankamana Mishra introduced the \href{https://datatracker.ietf.org/doc/html/draft-zzhang-bess-rfc6514bis}{draft-zzhang-bess-rfc6514bis}, emphasizing the need for further discussion and potential surveys to gauge vendor usage. The \href{https://datatracker.ietf.org/doc/html/drafts-sajassi-bess-rfc9251bis}{drafts-sajassi-bess-rfc9251bis} was also presented for the first time, with expectations for more list discussions.

\subsubsection{Individual Drafts}
Jorge Rabadan's \href{https://datatracker.ietf.org/doc/html/draft-sr-bess-evpn-vpws-gateway}{draft-sr-bess-evpn-vpws-gateway} and \href{https://datatracker.ietf.org/doc/html/draft-rabnic-bess-evpn-mcast-eeg}{draft-rabnic-bess-evpn-mcast-eeg} both have existing implementations and are seeking adoption. Jeffery Valley's \href{https://datatracker.ietf.org/doc/html/draft-zzhang-bess-dynamic-overlay-lb}{draft-zzhang-bess-dynamic-overlay-lb} sparked debate on its practical application, with feedback requested to address concerns about load balancing in multi-point connections.

\subsubsection{Future Directions}
The session concluded with discussions on the \href{https://datatracker.ietf.org/doc/html/draft-kriswamy-bess-evpn-perflow-df}{draft-kriswamy-bess-evpn-perflow-df} and \href{https://datatracker.ietf.org/doc/html/draft-wang-bess-l3-accessible-evpn}{draft-wang-bess-l3-accessible-evpn}, focusing on efficient load balancing and the validity of problem statements. The need for offline discussions and further list engagement was emphasized to refine these drafts.

Meeting materials are available at \href{https://www.ietf.org/proceedings/122/bess.html}{IETF 122 BESS Meeting Materials}.



\newpage

\section{BIER Working Group (BIER)}

\subsection{Attendees}
\subsubsection{Overview}
The BIER session was attended by 19 participants, including representatives from prominent companies and institutions such as Nokia, Cisco Systems, Juniper, Ericsson, and Huawei Technologies. Notable attendees included Greg Mirsky from Ericsson, Hooman Bidgoli from Nokia, and Toerless Eckert from Futurewei USA.

\subsection{Meeting Discussions}

\subsubsection{WG Status}
The session began with a status update from the chairs, emphasizing the importance of the OAM requirements draft for BIER Ping, BIER BFD, BIER performance measurements, and BIER MTU discovery drafts. These drafts are ready for the next step, and the shepherd work for OAM requirements and related drafts is anticipated.

\subsubsection{BIER Overlay}
Hooman Bidgoli presented on BIER Overlay, discussing the \href{https://datatracker.ietf.org/doc/html/draft-ietf-bier-pim-signaling}{draft-ietf-bier-pim-signaling} and \href{https://datatracker.ietf.org/doc/html/draft-ietf-bier-mldp-signaling-over-bier}{draft-ietf-bier-mldp-signaling-over-bier}. Key points included the challenges of using JP attributes and the security implications of BIER header information. The discussion highlighted the complexity of implementation and the need for further discussions on mailing lists to resolve open questions.

\subsubsection{EANTC Interop}
The EANTC interoperability testing was reviewed, focusing on the use of ingress replication versus P router replication. The session explored the implications of upstream and downstream label assignments, with a consensus to continue discussions in the MPLS session and mailing lists. The potential for using global labels instead of specific upstream or downstream labels was also considered.

\subsubsection{BIER FRR Discussion}
Toerless Eckert led a discussion on BIER Fast Reroute (FRR), addressing the feasibility and implementation challenges. The dialogue underscored the importance of FRR in multicast scenarios and the potential benefits of TI-LFA for multicast tunneling. The session concluded with a call for further exploration of the draft's experimental or standard track potential.

Meeting materials are available at \href{https://datatracker.ietf.org/meeting/122/materials/agenda-122-bier}{IETF 122 BIER Session Materials}.



\newpage

\section{Benchmarking Methodology Working Group (bmwg)}

\subsection{Attendees Overview}
The meeting was attended by 24 participants, including representatives from prominent organizations such as Huawei Technologies, Cisco Systems, Google, and Telefonica Innovacion Digital. The diverse attendance underscores the broad interest and collaborative effort in advancing benchmarking methodologies.

\subsection{Meeting Discussions}

\subsubsection{Multiple Loss Ratio Search}
Maciek Konstantynowicz and Vratko Polak presented the \href{https://datatracker.ietf.org/doc/html/draft-ietf-bmwg-mlrsearch}{draft-ietf-bmwg-mlrsearch}. The draft has successfully passed the Working Group Last Call (WGLC), with acknowledgments for its valuable contributions to the field. The authors confirmed addressing all idnits in the latest version.

\subsubsection{A YANG Data Model for Network Tester Management}
Vladimir Vassilev discussed the \href{https://datatracker.ietf.org/doc/html/draft-ietf-bmwg-network-tester-cfg}{draft-ietf-bmwg-network-tester-cfg}, emphasizing the importance of YANG data models in automating benchmarking tests. The draft serves as a foundational model that can be extended for comprehensive testing.

\subsubsection{Considerations for Benchmarking Network Performance in Containerized Infrastructures}
Minh-Ngoc Tran presented the \href{https://datatracker.ietf.org/doc/html/draft-ietf-bmwg-containerized-infra}{draft-ietf-bmwg-containerized-infra}, highlighting the relevance of eBPF in industry practices. The draft has garnered interest and feedback from the community, indicating its alignment with current technological trends.

\subsubsection{Benchmarking Methodology for Segment Routing}
Paolo Volpato introduced the \href{https://datatracker.ietf.org/doc/html/draft-ietf-bmwg-sr-bench-meth}{draft-ietf-bmwg-sr-bench-meth}, which is being cross-posted to SPRING and SRV6OPS working groups. The draft addresses the inclusion of more SIDs, reflecting its comprehensive approach to segment routing benchmarking.

\subsubsection{Characterization and Benchmarking Methodology for Power in Networking Devices}
Qin Wu presented the \href{https://datatracker.ietf.org/doc/html/draft-cprjgf-bmwg-powerbench}{draft-cprjgf-bmwg-powerbench}, focusing on methodologies for measuring power consumption in networking devices. Discussions included the feasibility of using 1pps as a baseline for idle traffic, with suggestions for considering higher values.

\subsubsection{Calibration of Measured Time Values between Network Elements}
Luis M. Contreras discussed the \href{https://datatracker.ietf.org/doc/html/draft-contreras-bmwg-calibration}{draft-contreras-bmwg-calibration}, emphasizing the need for precise terminology in timestamping mechanisms. The draft aims to standardize calibration methods, enhancing accuracy and precision in network measurements.

\subsubsection{Benchmarking Methodology for Source Address Validation}
Libin Liu's \href{https://datatracker.ietf.org/doc/html/draft-chen-bmwg-savnet-sav-benchmarking}{draft-chen-bmwg-savnet-sav-benchmarking} was reviewed by Giuseppe due to scheduling conflicts. The draft proposes methodologies for validating source addresses, contributing to improved network security practices.

Meeting materials can be accessed via \href{https://datatracker.ietf.org/meeting/122/materials/slides-122-bmwg-ietf-122-bmwg-chairs-slides-00}{this link}.



\newpage

\section{Computing-Aware Traffic Steering (CATS) WG}

\subsection{Attendees Overview}
The CATS Working Group meeting at IETF 122 was attended by 66 participants, including representatives from prominent companies and institutions such as Ericsson, Huawei, Cisco, Nokia, and China Mobile. The diverse attendance highlighted the broad interest and collaborative efforts in advancing computing-aware traffic steering technologies.

\subsection{Meeting Discussions}

\subsubsection{Agenda Bashing, Introduction, \& WG Status}
The session commenced with an agenda bashing led by the chairs, Adrian Farrel, Mohamed Boucadair, and Peng Liu. The group acknowledged the contributions of Med as a co-chair over the past year. An interim meeting focused on metric discussions was noted for its achievements, setting the stage for further deliberations.

\subsubsection{CATS Framework}
Cheng Li presented the CATS Framework, emphasizing key design issues and the publication plan, targeting submission to the IESG by November 2025. Discussions revolved around encrypted traffic handling, with Med urging deeper exploration of packet handling for encrypted packets. Adrian Farrel encouraged referencing the PCE WG's manageability considerations for guidance. The group agreed not to proceed with WGLC until all issues are resolved, underscoring the priority of document accuracy over meeting aspirational milestones.

\subsubsection{CATS Use Cases \& Requirements}
Kehan Yao provided updates on use case consolidations since IETF 121, with Adrian Farrel highlighting the importance of ensuring no critical requirements are omitted. The discussion included potential overlaps and new requirements from Integrated Sensing and Communications (ISAC) use cases, presented by Carlos J. Bernardos. The group debated the relevance of time synchronization as a CATS-related requirement, with consensus leaning towards refining the use case to better align with CATS objectives.

\subsubsection{CATS Metrics Discussion}
The session on CATS metrics, led by Kehan Yao, reviewed updates since the last interim meeting. The group discussed the classification and aggregation of metrics, with Adrian Farrel suggesting these be incorporated into the framework rather than the metric definition draft. The need for a problem statement to justify the three-level metric classification was also highlighted.

\subsubsection{Open Discussion \& Next Steps}
Time constraints limited the open discussion, but the group acknowledged the need for continued dialogue on unresolved issues and the importance of prioritizing requirements to ensure a robust and actionable framework.

Meeting materials and further details can be accessed at \href{https://datatracker.ietf.org/meeting/122/session/cats}{CATS Meeting Materials}.



\newpage

\section{Concise Binary Object Representation (CBOR) Working Group}

\subsection{Attendees}

The CBOR Working Group meeting was attended by 38 participants, including representatives from prominent organizations such as NASA, Microsoft, Cisco Systems, and RISE Research Institutes of Sweden. Notable attendees included Barry Leiba from Futurewei Technologies, Carsten Bormann from the University of Bremen, and Thomas Fossati from Linaro.

\subsection{Meeting Discussions}

\subsubsection{Agenda Overview}

The meeting commenced with an agenda overview by the chairs, Barry Leiba and Christian Amsüss. The primary focus was on the progression of CBOR-related drafts and the exploration of new proposals.

\subsubsection{Packed CBOR}

Carsten Bormann presented on \href{https://datatracker.ietf.org/doc/html/draft-ietf-cbor-packed}{draft-ietf-cbor-packed}, emphasizing that Packed CBOR is not a data compression method but rather a way to create succinct data structures. The discussion highlighted the allocation of tag numbers and the proposal to use a radical approach that minimizes tag usage, which could significantly streamline data encoding processes.

\subsubsection{dns-cbor}

Martine Lenders introduced the \href{https://datatracker.ietf.org/doc/html/draft-ietf-cbor-dns-cbor}{draft-ietf-cbor-dns-cbor}, which aims to reduce the size of DNS messages using CBOR. The presentation demonstrated a potential 50% reduction in message size and discussed the integration of pre-shared values for efficient name compression. The working group expressed support for adopting this draft.

\subsubsection{EDN / Application-Oriented Literals}

The session on EDN and application-oriented literals, led by Carsten Bormann and Rohan Mahy, focused on the need for a humane representation of data that extends beyond JSON capabilities. The discussion revolved around ensuring compatibility with existing JSON structures while allowing for flexible extensions. The group reached a consensus on several key points, including the handling of Unicode and ASCII representations.

\subsection{Meeting Materials}

Meeting materials, including slides and detailed notes, are available at \href{https://datatracker.ietf.org/meeting/122/materials/slides-122-cbor}{IETF 122 CBOR Meeting Materials}.

\subsubsection{Outcomes and Next Steps}

The meeting concluded with a consensus to adopt the dns-cbor draft and further explore the radical approach for Packed CBOR. The group agreed to continue discussions on the mailing list to refine the EDN draft, with a focus on ensuring robust and flexible data representation. The outcomes of these discussions are expected to influence future CBOR implementations, potentially leading to more efficient data handling in constrained environments.



\newpage

\section{Congestion Control Working Group (CCWG)}

\subsection{Attendees}
\subsubsection{Overview}
The meeting was attended by 78 participants, including representatives from prominent organizations such as Google, Apple, Ericsson, Cloudflare, and Meta. Academic institutions like the National Institute of Technology Karnataka and the University of Oslo were also represented.

\subsection{Meeting Discussions}

\subsubsection{Chair Slides}
The session began with a brief overview by the chairs, setting the stage for the discussions on congestion control advancements and the agenda for the meeting.

\subsubsection{New Tools for Testing Congestion Control and Queue Management Algorithms}
Rati Preethi S and colleagues presented new tools for testing congestion control algorithms. The discussion highlighted the potential integration with QUIC, despite current limitations in ns-3 support. The tools aim to enhance testing diversity, particularly for video delivery and L4S support, which are crucial for compliance with \href{https://datatracker.ietf.org/doc/html/draft-ietf-ccwg-new-tools}{draft-ietf-ccwg-new-tools}.

\subsubsection{BBRv3}
Ian Swett presented updates on BBRv3, focusing on evolving the draft to meet experimental standards. The dialogue centered on the challenges of coexistence with Reno/Cubic and the potential for ECN integration. The group emphasized the need for comprehensive testing and community feedback to refine the draft, as outlined in \href{https://datatracker.ietf.org/doc/html/draft-ietf-ccwg-bbr}{draft-ietf-ccwg-bbr}.

\subsubsection{Rate-limited Senders}
Mohit P. Tahiliani discussed the challenges faced by rate-limited senders, particularly in pacing and fairness. The group acknowledged the need for further research and collaboration to address these issues, as detailed in \href{https://datatracker.ietf.org/doc/html/draft-ietf-ccwg-ratelimited-increase}{draft-ietf-ccwg-ratelimited-increase}.

\subsubsection{SEARCH}
Mark Claypool introduced the SEARCH proposal, which aims to optimize congestion control in variable network conditions. The discussion revealed concerns about initial RTT measurements and their impact on performance, especially in WiFi environments. The proposal is documented in \href{https://datatracker.ietf.org/doc/html/draft-chung-ccwg-search}{draft-chung-ccwg-search}.

Meeting materials, including slides and notes, are available at \href{https://datatracker.ietf.org/meeting/122/materials/agenda-122-ccwg}{Meeting Materials}.




\newpage

\section{Content Delivery Networks Interconnection (CDNI)}

\subsection{Attendees}
The CDNI working group meeting was attended by 27 participants, including representatives from prominent companies and institutions such as Qwilt, Verizon, Streaming Video Technology Alliance (SVTA), Comcast, China Unicom, Rakuten Group Inc, Ericsson, Huawei Technologies Co., Ltd., and Google. The diverse attendance underscores the broad interest and collaborative effort in advancing CDNI standards.

\subsection{Meeting Discussions}

\subsubsection{Introductory Remarks}
Chairs Kevin J. Ma, Chris Lemmons, and Sanjay Mishra opened the meeting with a welcome to the new CDNI Area Director, Gorry Fairhurst, and outlined the agenda. A call for volunteers to assist with document reviews was emphasized, highlighting the collaborative nature of the group's work.

\subsubsection{Charter Updates}
Chris Lemmons presented the status of proposed charter updates, focusing on the introduction of new work items such as the \href{https://datatracker.ietf.org/doc/html/draft-ietf-cdni-metadata-expression-language}{draft-ietf-cdni-metadata-expression-language}. The updates aim to enhance the flexibility and applicability of CDNI interfaces, particularly through a new templating language for metadata expression.

\subsubsection{Metadata Model Drafts}
Glenn Goldstein provided a comprehensive overview of the evolution of CDNI's Metadata interface, now aligning with the SVTA Configuration Interface. Key drafts discussed included \href{https://datatracker.ietf.org/doc/html/draft-power-metadata-expression-language-03}{draft-power-metadata-expression-language-03} and \href{https://datatracker.ietf.org/doc/html/draft-ietf-cdni-client-access-control-metadata-01}{draft-ietf-cdni-client-access-control-metadata-01}, which propose extensions for HTTP2 and HTTP3 support.

\subsubsection{Private Features Metadata}
Arnon Warshavsky led a discussion on the \href{https://datatracker.ietf.org/doc/html/draft-warshavsky-private-features-metadata-02}{draft-warshavsky-private-features-metadata-02}, debating the necessity of standardizing private features. The group reached no consensus, suggesting further iteration as a personal draft or registration through SVTA.

\subsubsection{Protected Secrets and Logging}
Ben Rosenblum discussed updates to the \href{https://datatracker.ietf.org/doc/html/draft-ietf-cdni-protected-secrets-metadata-04}{draft-ietf-cdni-protected-secrets-metadata-04}, emphasizing the need for additional reviews before the next Working Group Last Call (WGLC). The draft clarifies encoding practices and seeks broader community input.

\subsubsection{CDNI CI/T v2 and Named Footprints}
Alan Arolovitch presented updates on \href{https://datatracker.ietf.org/doc/html/draft-ietf-cdni-ci-triggers-rfc8007bis-16}{draft-ietf-cdni-ci-triggers-rfc8007bis-16}, focusing on modernizing language and refining trigger collections. The session concluded with a discussion on \href{https://datatracker.ietf.org/doc/html/draft-ietf-cdni-named-footprints-01}{draft-ietf-cdni-named-footprints-01}, revisiting interface integration decisions.

\subsubsection{Closing Remarks}
The meeting concluded with a call for document shepherds to assist authors in navigating the publishing workflow, emphasizing the importance of community involvement in the standardization process.

Meeting materials are available at \href{https://www.meetecho.com/ietf122/recordings#CDNI}{CDNI Meeting Recordings}.



\newpage

\section{Crypto Forum Research Group (CFRG)}

\subsection{Attendees Overview}

The CFRG session at IETF 122 in Bangkok saw participation from 146 attendees, including representatives from prominent organizations such as Cisco Systems, Cloudflare, Google, and Microsoft. The diverse attendance underscored the broad interest in cryptographic research and its applications.

\subsection{Meeting Discussions}

\subsubsection{Chairs' Update}

Stanislav Smyshlyaev provided an update on the group's activities, including a review of the agenda, the CFRG review panel, and document status updates. The session emphasized the importance of community engagement in reviewing ongoing work and addressing errata.

\subsubsection{KEM Combiners Design Team: Current Status}

Nick Sullivan presented the current status of the KEM Combiners Design Team. The discussion highlighted the complexities of hybrid KEMs compared to PAKE, with a focus on the draft \href{https://datatracker.ietf.org/doc/html/draft-ietf-cfrg-hybrid-kems}{draft-ietf-cfrg-hybrid-kems}. Attendees debated the draft's progress and the need for continued collaboration to refine the design.

\subsubsection{Blind BBS and BBS Pseudonyms}

Vasilis Kalos and Greg Bernstein discussed the development of Blind BBS and BBS Pseudonyms, referencing three drafts and their alignment with existing cryptographic standards. The session addressed implementation concerns and the foundational work from Crypto 2022/23 papers.

\subsubsection{Anonymous Rate-Limited Credentials}

Chris Wood introduced a new draft on Anonymous Rate-Limited Credentials, motivated by the limitations of client-based rate-limiting in Privacy Pass. The proposal aims to streamline existing work and was met with suggestions for incorporating post-quantum considerations.

\subsubsection{Sigma Protocols and Fiat-Shamir}

Michele Orru presented on Sigma protocols and the Fiat-Shamir transformation, emphasizing the practical utility of Mauer proofs for the industry. The session encouraged further exploration of the protocol's applications.

\subsubsection{FrodoKEM}

Patrick Longa provided an update on FrodoKEM, a NIST round 3 alternative, and its inclusion in ISO work alongside ML-KEM and Classic McEliece. The discussion focused on the draft's potential as a viable alternative to ML-KEM.

\subsubsection{Hybrid PQ/T Key Encapsulation Mechanisms}

Deirdre Connolly reviewed the timeline and current state of the Hybrid PQ/T Key Encapsulation Mechanisms document. The session debated the performance implications and the need for consensus on protocol options.

\subsubsection{ECDH-PSI}

Yuchen Wang presented on ECDH-PSI, a protocol for Private Set Intersection. The draft, based on the \href{https://datatracker.ietf.org/doc/html/draft-ietf-cfrg-ecdh-psi}{draft-ietf-cfrg-ecdh-psi}, was updated following feedback from CFRG and PPM-WG members.

\subsubsection{MIMI Franking Mechanism}

Rohan Mahy discussed the MIMI franking mechanism, designed to facilitate abuse reporting across messaging platforms. The session sought input from cryptographic experts to enhance the protocol's interoperability.

\subsubsection{Advantages of NTRU Compared to ML-KEM}

Haruhisa Kosuge compared NTRU and ML-KEM, highlighting trade-offs and industry adoption. The session called for collaboration and further discussion on the list, particularly regarding the potential of NTRU-prime.

Meeting materials and further details are available at \href{https://meetings.conf.meetecho.com/ietf122/?session=33810}{IETF 122 CFRG Session}.



\newpage

\section{Constrained RESTful Environments (CoRE) WG}

\subsection{Attendees}

\subsubsection{Overview}
The CoRE Working Group meeting was attended by 20 participants, including representatives from prominent organizations such as RISE Research Institutes of Sweden, Ericsson, Microsoft, and Akamai. The meeting materials are available at \href{https://datatracker.ietf.org/meeting/122/session/core}{IETF 122 CoRE Session}.

\subsection{Meeting Discussions}

\subsubsection{CoAP Corrections, Clarifications, and Gaps}
Carsten Bormann presented updates on the \href{https://datatracker.ietf.org/doc/draft-ietf-core-corr-clar/}{draft-ietf-core-corr-clar}, focusing on addressing questions and issues identified in CoAP documents. The discussion highlighted the need for an incubator approach to address gaps, suggesting a separate document for specific topics. This approach aims to streamline the process of incorporating necessary changes into the CoAP protocol, potentially leading to more robust implementations.

\subsubsection{CORECONF: YANG CBOR and YANG SIDs}
The session on \href{https://datatracker.ietf.org/doc/rfc9254/}{RFC 9254} and \href{https://datatracker.ietf.org/doc/rfc9595/}{RFC 9595} covered the generation of SID files for YANG models, emphasizing the importance of coordination and tool development. This work is crucial for ensuring consistent and efficient management of YANG models across different implementations, which could significantly enhance interoperability.

\subsubsection{Constrained Resource Identifiers}
The \href{https://datatracker.ietf.org/doc/draft-ietf-core-href/}{draft-ietf-core-href} has reached a significant milestone with the completion of the Working Group Last Call (WGLC). The document aims to make URIs more useful for constrained devices, potentially impacting high-performance environments. The next steps involve deciding whether to submit to the IESG or conduct another WGLC, with a focus on addressing any remaining terminology issues.

\subsubsection{Conditional Attributes for Constrained RESTful Environments}
Jaime Jiménez discussed the \href{https://datatracker.ietf.org/doc/draft-ietf-core-conditional-attributes/}{draft-ietf-core-conditional-attributes}, which is nearing completion. The document addresses complex data structures and mixing query parameters, with a focus on providing a robust toolbox for other specifications. This work could lead to more flexible and powerful RESTful environments for constrained devices.

\subsubsection{DNS over CoAP "Bundle"}
Martine Lenders presented the \href{https://datatracker.ietf.org/doc/draft-ietf-core-dns-over-coap/}{draft-ietf-core-dns-over-coap} and related documents, which have passed WGLC. The discussion emphasized the need for editorial revisions and synchronization with related work, aiming to enhance DNS functionality over CoAP, which could improve network efficiency and reliability in constrained environments.

\subsubsection{A Publish-Subscribe Architecture for CoAP}
Jaime Jiménez introduced the \href{https://datatracker.ietf.org/doc/draft-ietf-core-coap-pubsub/}{draft-ietf-core-coap-pubsub}, highlighting recent updates and open points for discussion. The architecture aims to streamline communication in constrained networks, with potential applications in IoT systems where efficient data distribution is critical.

\subsubsection{CoAP Transport Indication}
Christian Amsüss discussed the \href{https://datatracker.ietf.org/doc/draft-ietf-core-transport-indication/}{draft-ietf-core-transport-indication}, focusing on enabling discovery of alternative transports and planning for new CoAP transports. The document seeks to provide a flexible framework for transport indication, which could facilitate the adoption of CoAP in diverse network environments.

\subsubsection{Observe Notifications as CoAP Multicast Responses}
Marco Tiloca presented updates on the \href{https://datatracker.ietf.org/doc/draft-ietf-core-observe-multicast-notifications/}{draft-ietf-core-observe-multicast-notifications}, which enables multicast notifications for observe requests. The discussion included potential simplifications and the importance of ensuring compatibility with proxy scenarios, which could enhance the scalability and efficiency of CoAP-based systems.

\subsubsection{Key Update for OSCORE (KUDOS)}
Rikard Höglund introduced a redesigned \href{https://datatracker.ietf.org/doc/draft-ietf-core-oscore-key-update/}{draft-ietf-core-oscore-key-update}, which decouples key updates from client/server roles. This simplification could lead to more flexible and secure key management in OSCORE, enhancing the overall security posture of constrained networks.

\subsubsection{CoAP in Space and Over Bundle Protocol (BP)}
Carles Gomez presented the \href{https://datatracker.ietf.org/doc/draft-gomez-core-coap-space/}{draft-gomez-core-coap-space} and \href{https://datatracker.ietf.org/doc/draft-gomez-core-coap-bp/}{draft-gomez-core-coap-bp}, focusing on message aggregation and proxy use in space communication scenarios. The work aims to optimize CoAP for challenging environments, potentially extending its applicability to space communications and other delay-tolerant networks.



\newpage

\section{COSE Working Group (COSE)}

\subsection{Attendees}
The COSE Working Group meeting was attended by 49 participants, including representatives from prominent organizations such as Okta, Yubico, Google, Microsoft, and the NSA. The diverse group of attendees contributed to a rich exchange of ideas and expertise.

\subsection{Meeting Discussions}

\subsubsection{Opening Remarks}
The meeting commenced with opening remarks from the chairs, summarizing the current status of various documents and setting the stage for the discussions to follow.

\subsubsection{Post-Quantum Cryptography Drafts}
Brent Zundel presented on \href{https://datatracker.ietf.org/doc/html/draft-ietf-cose-sphincs-plus}{draft-ietf-cose-sphincs-plus} and \href{https://datatracker.ietf.org/doc/html/draft-ietf-cose-falcon}{draft-ietf-cose-falcon}. The discussion highlighted the need for flexibility in algorithm definitions, with a focus on not limiting to SHA2, and considering SHAKE as well. The group debated the future of Falcon and Dilithium, with consensus leaning towards waiting for final NIST publications before further action.

\subsubsection{GMAC in COSE}
Brian Sipos introduced \href{https://datatracker.ietf.org/doc/html/draft-sipos-cose-gmac-00}{draft-sipos-cose-gmac-00}, emphasizing its potential for hardware acceleration. The draft seeks registry allocations, and volunteers from the group committed to reviewing the document to facilitate its adoption.

\subsubsection{CBOR Encoded Certificates}
John Mattsson discussed \href{https://datatracker.ietf.org/doc/html/draft-ietf-cose-cbor-encoded-cert-12}{draft-ietf-cose-cbor-encoded-cert-12}, noting the accumulation of features over time. The group agreed on the necessity of refining the draft to focus on essential components, with a call for further input on the mailing list.

\subsubsection{Hybrid Public Key Encryption}
Hannes Tschofenig presented \href{https://datatracker.ietf.org/doc/html/draft-ietf-cose-hpke}{draft-ietf-cose-hpke}, seeking reviewers post-alignment. The presentation underscored the importance of packaging cryptographic elements effectively, with commitments from attendees to provide feedback.

\subsubsection{Two-Party Signing Algorithms}
Mike Jones and Emil Lundberg co-presented \href{https://datatracker.ietf.org/doc/html/draft-lundberg-cose-two-party-signing-algs}{draft-lundberg-cose-two-party-signing-algs}. The discussion revolved around the integration with WebAuthn and FIDO2, with concerns about the implications for verifier responsibilities. The group acknowledged the need for broader coordination and input.

\subsubsection{Hash Envelope}
Steve Lasker presented \href{https://datatracker.ietf.org/doc/html/draft-ietf-cose-hash-envelope}{draft-ietf-cose-hash-envelope}, declaring it ready for Working Group Last Call (WGLC) once outstanding issues are resolved. The group expressed support for moving forward, recognizing its potential impact on secure data handling.

\subsection{Meeting Materials}
All meeting materials, including presentation slides, are available at \href{https://datatracker.ietf.org/meeting/122/session/cose}{IETF COSE Meeting Materials}.




\newpage

\section{Delegation Extensions Working Group (DELEG)}

\subsection{Attendees}
\subsubsection{Overview}
The DELEG working group meeting was attended by 97 participants, representing a diverse array of prominent organizations such as Fastly, Verisign, NLnet Labs, Microsoft, and Google. The attendance highlighted the broad interest and engagement from both industry leaders and academic institutions.

\subsection{Meeting Discussions}

\subsubsection{Incrementally Deployable Extensible Delegation for DNS}
Tim Wicinski presented the draft on \href{https://datatracker.ietf.org/doc/html/draft-homburg-deleg-incremental-deleg}{incremental delegation}, focusing on the transition from SVCB to the new DELEG type. The discussion centered on addressing the zone cut problem, with underscore labels being proposed as a best practice for applications and services. The working group debated the long-term implications, questioning the future compatibility of authoritative software. The presentation underscored the need for further testing and refinement, particularly in handling legacy delegation information.

\subsubsection{Extensible Delegation for DNS}
The authors presented their draft on \href{https://datatracker.ietf.org/doc/html/draft-wesplaap-deleg}{extensible delegation}, emphasizing adherence to current DNS semantics to facilitate debugging and implementation. The draft proposes optional QNAME minimalization for performance-sensitive scenarios and introduces an EDNS(0) signal to optimize packet size. The discussion highlighted the challenges of balancing complexity between authoritative and recursive resolvers, with a consensus on the necessity of simplifying deployment for long-term sustainability.

\subsubsection{Working Group Discussion on Drafts}
The working group engaged in a robust discussion on the technical merits and deployment strategies of the proposed drafts. Key questions addressed the incremental deployability of DELEG-aware and non-DELEG-aware servers, and the technical protocol concerns of each draft. The dialogue reflected a strategic focus on designing solutions that are both forward-compatible and understandable to newcomers in the DNS ecosystem. The consensus leaned towards prioritizing simplicity and clarity in the final implementation.

Meeting materials are available at \href{https://ideleg.net/}{ideleg.net}, providing resources for testing and further exploration of the discussed protocols.



\newpage

\section{Deterministic Networking (DetNet)}

\subsection{Attendees Overview}
\subsubsection{Participants}
The DetNet session at IETF 122 saw participation from 60 attendees, including representatives from prominent organizations such as New H3C Technologies, LabN Consulting, University of Oxford, Huawei, Ericsson, Juniper Networks, Cisco, and Google LLC.

\subsection{Meeting Discussions}

\subsubsection{Intro, WG Status, Draft Status}
The session commenced with an introduction by the chairs, providing updates on the working group and draft statuses. Discussions emphasized the need for collaboration with the TSN community to address outstanding questions and align efforts. The chairs encouraged participants to present their experiences and insights at upcoming workshops, aiming to foster a comprehensive understanding of DetNet's integration with TSN.

\subsubsection{Dataplane Enhancement Taxonomy}
Jinoo Joung presented the \href{https://datatracker.ietf.org/doc/draft-ietf-detnet-dataplane-taxonomy/02}{draft-ietf-detnet-dataplane-taxonomy}, which categorizes dataplane enhancements. The taxonomy aims to streamline solutions and identify overlaps. An interim meeting is proposed to consolidate solutions and refine the taxonomy, potentially influencing future DetNet strategies.

\subsubsection{Latency Guarantee with Stateless Fair Queuing}
Jinoo Joung also discussed the \href{https://datatracker.ietf.org/doc/draft-joung-detnet-stateless-fair-queuing/04}{draft-joung-detnet-stateless-fair-queuing}, which proposes a method for ensuring latency guarantees. The presentation did not elicit comments, suggesting consensus or the need for further review.

\subsubsection{Deadline Based Deterministic Forwarding}
Peng Liu presented the \href{https://datatracker.ietf.org/doc/draft-peng-detnet-deadline-based-forwarding/14}{draft-peng-detnet-deadline-based-forwarding}, focusing on deadline-based forwarding mechanisms. The draft was well-received, with no immediate feedback, indicating alignment with current DetNet objectives.

\subsubsection{Flow Aggregation for Enhanced DetNet}
Tianji Jiang and Quan Xiong introduced the \href{https://datatracker.ietf.org/doc/draft-xiong-detnet-flow-aggregation/02}{draft-xiong-detnet-flow-aggregation}, which addresses flow aggregation in multi-domain environments. The chairs highlighted the need for alignment with existing YANG models and RFCs, emphasizing the importance of defining single-domain aggregation before multi-domain considerations.

\subsubsection{Enhanced Use Cases for Scaling Deterministic Networks}
Junfeng Zhao's \href{https://datatracker.ietf.org/doc/draft-zhao-detnet-enhanced-use-cases/02/}{draft-zhao-detnet-enhanced-use-cases} was discussed, focusing on real-world scenarios and requirements. The draft's value was acknowledged, but further clarification on data derivation was requested to enhance its applicability.

\subsubsection{DetNet Multidomain Extensions}
Carlos Bernardos presented the \href{https://datatracker.ietf.org/doc/draft-bernardos-detnet-raw-multidomain/05}{draft-bernardos-detnet-raw-multidomain}, exploring extensions for multi-domain DetNet. The discussion centered on defining domains and the implications of different control and data plane technologies. The draft's evolution from a framework to a solution document was considered, with contributions invited to refine its scope.

\subsection{Conclusion}
The DetNet session at IETF 122 highlighted significant advancements and collaborative efforts in deterministic networking. The discussions underscored the importance of taxonomy alignment, multi-domain considerations, and real-world applicability, setting the stage for future developments in the field. Meeting materials are available at \href{https://datatracker.ietf.org/meeting/122/session/detnet}{DetNet Session Materials}.



\newpage

\section{Dynamic Host Configuration Working Group (DHC)}

\subsection{Attendees}
\subsubsection{Overview}
The meeting was attended by 20 participants, including representatives from prominent companies and institutions such as Cisco, Ericsson, Google, and Akamai Technologies. Notable attendees included Suresh Krishnan from Cisco, Lorenzo Colitti from Google, and Fernando Gont from SI6 Networks.

\subsection{Meeting Discussions}

\subsubsection{Welcome, Agenda Review, and Status Update}
The session commenced with a welcome and agenda review led by Timothy Winters. Discussions included updates on the relay draft, with plans to progress to Working Group Last Call (WLGC) shortly. The \href{https://datatracker.ietf.org/doc/html/draft-ietf-dhc-relay}{draft-ietf-dhc-relay} was highlighted, and updates on the 8415bis draft were discussed, with a minor typo pending correction.

\subsubsection{DHCP Option Concatenation Considerations}
Tommy Jensen presented on DHCP Option Concatenation, focusing on DHCPv4. The presentation was well-received, with strong support for adoption from David Lamparter. The need for further reviews was emphasized, with commitments from David Lamparter, Tom Hill, and Sheng Jiang to provide feedback. The draft can be accessed at \href{https://datatracker.ietf.org/doc/draft-tojens-dhcp-option-concat-considerations/}{draft-tojens-dhcp-option-concat-considerations}.

\subsubsection{Generating Semantically Opaque IPv6 Interface Identifiers}
Fernando Gont discussed a method for generating semantically opaque IPv6 Interface Identifiers using DHCPv6. The presentation sparked a debate on scalability and statelessness, particularly concerning large pool sizes and DECLINE scenarios. The discussion referenced \href{https://datatracker.ietf.org/doc/draft-gont-dhcwg-dhcpv6-iids-00/}{draft-gont-dhcwg-dhcpv6-iids-00}, with considerations for its application to Prefix Delegation (PD).

\subsubsection{Virtual DHCPv6 (8415) Plugfest Results}
Spencer Couture and Ben Patton presented the results of the Virtual DHCPv6 Plugfest, highlighting interoperability achievements and areas for improvement.

\subsubsection{Next Steps and Wrap Up}
The meeting concluded with a discussion on next steps, emphasizing the importance of addressing timezone challenges in future meetings. The working group aims to refine the discussed drafts and prepare for subsequent IETF submissions.

Meeting materials and further details are available at \href{https://www.ietf.org/proceedings/122/dhc.html}{IETF 122 DHC WG Meeting Materials}.




\newpage

\section{Decentralized Internet Infrastructure Research Group (DINRG)}

\subsection{Attendees Overview}
\subsubsection{Attendance and Key Institutions}
The DINRG meeting at IETF 122 saw participation from 88 attendees, including representatives from prominent institutions such as UCLA, Verisign, Cisco, Huawei, and the Internet Society. Notable attendees included Dirk Kutscher from HKUST, Lixia Zhang from UCLA, and Gareth Tyson from Hong Kong University of Science and Technology.

\subsection{Meeting Discussions}

\subsubsection{CoNEXT DIN Workshop Summary}
Dirk Kutscher presented a summary of the CoNEXT'24 DIN Workshop, highlighting discussions on decentralizing communications, naming, and distributed computing. The workshop featured insights from Cory Doctorow on Internet challenges and Michael Karanicolas on legal perspectives. Technical sessions focused on decentralized systems and technologies, emphasizing the need for consensus on the current Internet. For more details, refer to the \href{https://dirk-kutscher.info/events/conferences-workshops/conext-2024-din-report/}{workshop report}.

\subsubsection{DIN Review Since Re-chartering}
Lixia Zhang led a discussion on the re-chartering of DINRG, focusing on structuring discussions around Internet decentralization. Key points included the pros and cons of decentralization, engagement with other communities, and the importance of economic models. The group explored how decentralization impacts society and networks, with references to \href{https://datatracker.ietf.org/doc/html/draft-ietf-dinrg-measurements}{draft-ietf-dinrg-measurements}.

\subsubsection{A Closer Look into IPFS: Accessibility, Content, and Performance}
Ruizhe Shi's presentation examined IPFS, discussing its accessibility, content governance, and performance. The session addressed economic artifacts and the integration of centralized components in decentralized solutions. The discussion raised concerns about content governance, particularly regarding illegal content, and proposed the implementation of blocklists. Presentation slides are available \href{https://datatracker.ietf.org/meeting/122/materials/slides-122-dinrg-a-closer-look-into-ipfs-accessibility-content-and-performance-00}{here}.

\subsubsection{Looking at the Blue Skies of Bluesky}
Gareth Tyson explored the multifaceted nature of Bluesky, encompassing architecture, company, and protocol aspects. Discussions highlighted user control over content and the potential for users to create alternatives. Technical debates covered user control over sockets, the decline of XMPP, and trust mechanisms. Presentation slides can be accessed \href{https://datatracker.ietf.org/meeting/122/materials/slides-122-dinrg-bluesky-00.pdf}{here}.

\subsubsection{Comparison of App Implementations over TCP/IP and NDN}
The session by Tianyuan Yu was suspended due to time constraints, but it aimed to compare application implementations over TCP/IP and NDN, focusing on performance and architectural differences.

Meeting materials and further resources are available \href{https://datatracker.ietf.org/meeting/122/session/dinrg}{here}.



\newpage

\section{DKIM Working Group (DKIM)}

\subsection{Attendees}

The DKIM Working Group meeting at IETF 122 was attended by 36 participants, including representatives from prominent companies and institutions such as Fastmail, Google, Meta Platforms, Inc., ICANN, and Yahoo. Notable attendees included Bron Gondwana from Fastmail, Murray Kucherawy from Meta Platforms, Inc., and Pete Resnick from Episteme Technology Consulting LLC.

\subsection{Meeting Discussions}

\subsubsection{Chair Welcome and Overview}

The meeting commenced with a review of the Note Well slides and an agenda presentation. The chairs emphasized maintaining good conduct and minimizing distractions, while ensuring that the working group's efforts do not disrupt the existing deployed base. The goal is to develop a new protocol, with a preference for incremental improvements where feasible.

\subsubsection{Charter Review}

The chairs reviewed the charter, highlighting the issues and goals that prompted the reconstitution of the working group. Discussions centered around the importance of abuse-resistant techniques and the potential for fragmentation of the email infrastructure. The group acknowledged the need to balance innovation with compatibility, ensuring that any new developments do not inadvertently exclude smaller operators.

\subsubsection{Proposed Documents}

Bron Gondwana presented three proposed documents for adoption: the \href{https://datatracker.ietf.org/doc/draft-gondwana-dkim2-motivation/}{Motivations document}, the \href{https://datatracker.ietf.org/doc/draft-gondwana-dkim2-header/}{Header document}, and the \href{https://datatracker.ietf.org/doc/draft-gondwana-dkim2-modification-alegbra/}{Modifications document}. The discussion highlighted the need for crypto agility and the potential impact of DKIM2 on existing email flows. The group considered the implications of single-recipient transactions and the necessity of addressing interoperability with existing DKIM implementations.

\subsubsection{Open Questions and Next Steps}

The meeting concluded with a discussion on open questions, including the interaction between DKIM2 and existing DKIM protocols, and the need for a capability mechanism to ensure compatibility. The group agreed to conduct calls for adoption for the proposed documents, recognizing that adoption signifies a starting point for further development. The potential for interim meetings before the next IETF meeting in Madrid was also discussed, with a suggestion to initiate discussions on the mailing list to determine their frequency and timing.

Meeting materials can be accessed at \href{https://datatracker.ietf.org/meeting/122/materials/agenda-122-dkim}{IETF 122 DKIM Meeting Materials}.



\newpage

\section{DMM Working Group (DMM)}

\subsection{Attendees}
The DMM Working Group meeting was attended by 49 participants, including representatives from prominent companies and institutions such as Cisco Systems, Rakuten Group Inc, Vodafone, Futurewei, Verizon, Nvidia, SoftBank Corp., and Telefonica Innovacion Digital. The diverse attendance underscores the broad interest and collaborative effort in advancing mobile network technologies.

\subsection{Meeting Discussions}

\subsubsection{Mobility Aware Transport Network Slicing for 5G}
John Kaippallimalil presented the \href{https://datatracker.ietf.org/doc/draft-ietf-dmm-tn-aware-mobility}{draft-ietf-dmm-tn-aware-mobility}, discussing whether it should be classified as an informational or standard document. The consensus leaned towards keeping it informational, pending confirmation by the chairs. Discussions highlighted the need for clarity on the relationship between S-NSSAI and EP\_Transport, with suggestions to elaborate on data structures in the draft. Further reviews were deemed necessary to proceed to the last call.

\subsubsection{Architecture Discussion on SRv6 Mobile User Plane}
Teppei Kamata led the discussion on \href{https://datatracker.ietf.org/doc/draft-ietf-dmm-srv6mob-arch/}{draft-ietf-dmm-srv6mob-arch}, focusing on session aspects beyond routing, such as charging and QoS. While some participants suggested these could be referenced, the general agreement was to keep the draft focused on its core objectives.

\subsubsection{Mobile Traffic Steering}
Marco Liebsch's \href{https://datatracker.ietf.org/doc/draft-liebsch-dmm-mts}{draft-liebsch-dmm-mts} prompted discussions on the need to generalize terminology beyond 3GPP specifics. The draft's framework was noted to apply to 3GPP, but the steering concept should be clarified to differentiate it from routing.

\subsubsection{SRH Reduction for SRv6 End.M.GTP6.E Behavior}
Yuya Kawakami presented the \href{https://datatracker.ietf.org/doc/draft-kawakami-dmm-srv6-gtp6e-reduced}{draft-kawakami-dmm-srv6-gtp6e-reduced}, which sparked debate on whether it should be a standalone draft or integrated with other SRv6 MUP drafts. The draft's focus on SRv6 behavior without altering BGP was emphasized, with calls for further review before adoption.

\subsubsection{MUP Architecture}
Satoru Matsushima's \href{https://datatracker.ietf.org/doc/draft-mhkk-dmm-mup-architecture}{draft-mhkk-dmm-mup-architecture} was considered for adoption, with strong support due to existing implementations. The need for working group participation and reviews was highlighted to ensure progress aligns with the charter.

\subsubsection{Computing Aware Traffic Steering Consideration for MUP}
Minh Ngoc Tran discussed the \href{https://datatracker.ietf.org/doc/html/draft-dcn-dmm-cats-mup}{draft-dcn-dmm-cats-mup}, exploring its applicability across mobile architectures. The draft's potential to benefit from feedback from the CATS community was noted, with volunteers stepping forward for review.

\subsubsection{DMM Centric Summary of 3GPP Workshop on 6G}
Lionel Morand summarized key takeaways from the 3GPP Workshop on 6G, highlighting new services like private networks and V2X, and the complexity of current slicing solutions. The discussions pointed to a need for network simplification and the potential role of AI in network services. The session concluded with reflections on the necessity for a unified data plane approach.

Meeting materials are available at \href{https://meetings.conf.meetecho.com/ietf122/?group=dmm&short=dmm&item=1}{Meetecho} and \href{https://notes.ietf.org/notes-ietf-122-dmm}{Notes}.



\newpage

\section{DNS Operations (DNSOP) Working Group}

\subsection{Attendees Overview}
\subsubsection{Attendance}
The DNSOP Working Group session was attended by 108 participants, including representatives from prominent organizations such as Verisign, NLnet Labs, ICANN, and Apple. The diverse attendance underscored the broad interest and collaborative effort in addressing DNS operational challenges.

\subsection{Meeting Discussions}

\subsubsection{Clarifications on CDS/CDNSKEY and CSYNC Consistency}
Peter Thomassen presented the \href{https://datatracker.ietf.org/doc/html/draft-ietf-dnsop-cds-consistency}{draft-ietf-dnsop-cds-consistency}, which is nearing readiness for Working Group Last Call. The discussion highlighted the importance of ensuring consistency in DNSSEC key management, with a focus on operational clarity and potential adoption timelines.

\subsubsection{Domain Verification Techniques using DNS}
Shumon Huque discussed the \href{https://datatracker.ietf.org/doc/html/draft-ietf-dnsop-domain-verification-techniques}{draft-ietf-dnsop-domain-verification-techniques}, emphasizing the need for clear distinctions between domain control validation and domain authorization. Concerns were raised about security implications and the potential for user confusion, prompting suggestions for further refinement of the draft.

\subsubsection{DNS Filtering Details for Applications}
Mark Nottingham introduced the \href{https://datatracker.ietf.org/doc/html/draft-nottingham-public-resolver-errors}{draft-nottingham-public-resolver-errors}, sparking a debate on the role of DNS filtering in application behavior. The session underscored the necessity of establishing a trust model for DNS error handling and the potential policy implications, with suggestions to engage broader community discussions.

\subsubsection{Collision Free Keytags for DNSSEC}
Shumon Huque's \href{https://datatracker.ietf.org/doc/html/draft-huque-dnsop-keytags}{draft-huque-dnsop-keytags} was discussed, focusing on the operational challenges of keytag collisions in DNSSEC. The group debated the balance between complexity and operational reliability, with consensus on the need for a standardized approach to prevent validation issues.

\subsubsection{A Top-level Domain for Private Use}
Warren Kumari presented the \href{https://datatracker.ietf.org/doc/html/draft-davies-internal-tld}{draft-davies-internal-tld}, advocating for a reserved TLD for private use. The discussion highlighted the potential benefits for network management and security, with calls for further exploration of the technical and policy implications.

Meeting materials, including slides and detailed minutes, are available at \href{https://notes.ietf.org/notes-ietf-122-dnsop}{IETF Notes}.

The discussions in this session reflect a strategic focus on enhancing DNS security and operational efficiency, with several drafts poised for further development and potential adoption. The outcomes suggest a proactive approach to addressing emerging challenges in DNS operations, with a commitment to collaborative problem-solving and innovation.



\newpage

\section{Extensions for Scalable DNS Service Discovery (dnssd)]}

\subsection{Attendees Overview}
The dnssd working group meeting at IETF 122 was attended by 39 participants, including representatives from prominent organizations such as Apple, Cisco, Google, Microsoft, and ISC. This diverse attendance underscores the broad interest and collaborative effort in advancing DNS service discovery technologies.

\subsection{Meeting Discussions}

\subsubsection{RFCs-to-be SRP and Update Lease}
Presenter: Stuart Cheshire

The session focused on significant textual changes in the \href{https://datatracker.ietf.org/doc/html/draft-ietf-dnssd-srp}{draft-ietf-dnssd-srp} and \href{https://datatracker.ietf.org/doc/html/draft-ietf-dnssd-update-lease}{draft-ietf-dnssd-update-lease}. Key discussions revolved around improving readability and clarity, particularly in sections dealing with failure recovery and security considerations. The group agreed on the necessity of plain language to enhance understanding, with specific changes proposed to align with industry best practices on security, especially concerning flooding attacks. These updates are expected to streamline implementation and foster broader adoption.

\subsubsection{Multiple QTYPEs}
Presenter: Ray Bellis

Ray Bellis reported on the hackathon implementations related to \href{https://datatracker.ietf.org/doc/html/draft-ietf-dnssd-multi-qtypes}{draft-ietf-dnssd-multi-qtypes}. The hackathon revealed minor issues, which have since been addressed in the draft. The group discussed the readiness of the draft for Working Group Last Call (WGLC), highlighting the importance of having running code as a positive indicator of maturity.

\subsubsection{Use Cases for Time Since Received}
Presenter: Esko Dijk

Esko Dijk presented various use cases for the Time Since Received (TSR) mechanism, emphasizing its role in conflict resolution and SRP replication. The discussion highlighted the need for further exploration of corner cases, particularly in scenarios involving mobile clients and advertising proxies. The group recognized the potential of TSR to enhance DNS-SD reliability and agreed to continue refining the draft.

\subsubsection{MDNS Conflict Resolution Using TSR}
Presenter: Ted Lemon

Ted Lemon discussed the \href{https://datatracker.ietf.org/doc/html/draft-tllq-tsr}{draft-tllq-tsr}, focusing on its application in mDNS conflict resolution. The presentation underscored the importance of version compatibility and the need for robust implementation strategies to ensure seamless operation across different network environments.

\subsubsection{Advertising Proxy}
Presenter: Ted Lemon

The \href{https://datatracker.ietf.org/doc/html/draft-ietf-dnssd-advertising-proxy}{draft-ietf-dnssd-advertising-proxy} was reviewed, with discussions centering on the necessity of implementation to progress the draft. The group acknowledged the challenges but remained optimistic about the draft's potential to significantly enhance DNS-SD capabilities.

\subsubsection{SRP Replication}
Presenter: Ted Lemon

Ted Lemon presented on \href{https://datatracker.ietf.org/doc/html/draft-ietf-dnssd-srp-replication}{draft-ietf-dnssd-srp-replication}, focusing on ensuring version compatibility and the replication of TSR values. The discussion highlighted the draft's role in improving DNS-SD scalability and reliability.

\subsubsection{Using SVCB with DNSSD}
Presenter: Gautam Akiwate

Gautam Akiwate introduced the \href{https://datatracker.ietf.org/doc/html/draft-gakiwate-dnssd-use-svcb}{draft-gakiwate-dnssd-use-svcb}, which aims to enable services to share ALPN information. The group discussed potential improvements, including the use of Multiple QTYPEs to optimize query efficiency. The draft's potential to streamline service discovery processes was acknowledged, with further refinements anticipated.

Meeting materials are available at \href{https://datatracker.ietf.org/meeting/122/materials}{IETF 122 Meeting Materials}.



\newpage

\section{Delay/Disruption Tolerant Networking (DTN) Working Group}

\subsection{Attendees}
The DTN Working Group meeting at IETF 122 in Bangkok was attended by 47 participants, including representatives from prominent organizations such as Spacely Packets, LLC, King's College London, SAIC/NASA, China Mobile, JHU/APL, and ESA/ESOC. The diverse attendance highlighted the global interest and collaborative efforts in advancing DTN technologies.

\subsection{Meeting Discussions}

\subsubsection{CoAP over BP}
Carles Gomez presented updates on \href{https://datatracker.ietf.org/doc/html/draft-gomez-core-coap-bp-03}{draft-gomez-core-coap-bp-03}, focusing on a new CoAP option for message aggregation and proxying. Discussions centered around namespace translation and the concurrent use of OSCORE and BPSec, with considerations for deployment scenarios.

\subsubsection{DTN Reference Scenarios}
Camillo Malnati and Felix Flentge discussed the development of reference scenarios for BP and its extensions, targeting Earth, Lunar, and Mars environments. The scenarios aim to standardize data formats and will be published as a Yellow Book in CCSDS.

\subsubsection{BPv7 Custody Transfer and Compressed Status Reporting}
Felix Flentge provided insights into opportunistic custody transfer and compressed status reporting. The discussion emphasized the need for efficient bundle sequence IDs and the potential for IETF documentation on signaling and sequencing.

\subsubsection{Interplanetary DNS}
Scott Johnson introduced \href{https://datatracker.ietf.org/doc/html/draft-johnson-dtn-interplanetary-dns-03}{draft-johnson-dtn-interplanetary-dns-03}, proposing a coherent naming structure for IP networks on extraterrestrial bodies. The debate highlighted DNSSEC compatibility and the necessity for a robust naming framework.

\subsubsection{Status of DTNMA}
Jenny Cao updated on multiple drafts, including \href{https://datatracker.ietf.org/doc/html/draft-ietf-dtn-amm-03}{draft-ietf-dtn-amm-03} and \href{https://datatracker.ietf.org/doc/html/draft-ietf-dtn-adm-yang-03}{draft-ietf-dtn-adm-yang-03}, with a focus on YANG model reviews and the integration of ARI drafts.

\subsubsection{Status of WG \& Personal Drafts}
Brian Sipos reviewed several drafts, such as \href{https://datatracker.ietf.org/doc/html/draft-ietf-dtn-bpsec-cose-06}{draft-ietf-dtn-bpsec-cose-06} and \href{https://datatracker.ietf.org/doc/html/draft-ietf-dtn-udpcl-00}{draft-ietf-dtn-udpcl-00}, discussing the stability of BPSec COSE Context and the evolution of UDPCLv2 features.

\subsubsection{Bundle Transfer Protocol - Unidirectional}
Rick Taylor presented \href{https://datatracker.ietf.org/doc/html/draft-taylor-dtn-btpu-00}{draft-taylor-dtn-btpu-00}, highlighting the potential for a common CL protocol for unidirectional, frame-based link layers, emphasizing simplicity and efficiency.

\subsubsection{RFC4838bis}
Ed Birrane discussed the need for updating RFC4838, focusing on DTN and IP integration, QoS, and fragmentation. Community input was encouraged for shaping the BIS document.

\subsubsection{Key Establishment in the Space Environment}
Benjamin Dowling, on behalf of Britta Hale, explored secure key refresh mechanisms in space environments, referencing \href{https://arxiv.org/pdf/2503.06785}{this paper}. The discussion considered the applicability of MLS for BPSec.

\subsubsection{BP QoS Extensions}
Teresa Algarra proposed a QoS block for user-level applications, with discussions on standardizing network QoS blocks for administrative flexibility.

\subsubsection{Open Mic}
The session concluded with an open mic segment, inviting further community engagement and feedback on the discussed topics.

Meeting materials are available at \href{https://www.ietf.org/proceedings/122/dtn.html}{IETF 122 DTN Meeting Materials}.



\newpage

\section{EMAILCORE Working Group (EMAILCORE)}

\subsection{Attendees}
The EMAILCORE working group meeting at IETF 122 in Bangkok was attended by 27 participants, including representatives from prominent companies and institutions such as Apple Inc., Fastmail, ICANN, and Meta Platforms, Inc. Notable attendees included Alexey Melnikov from Isode Limited, Todd Herr, and John Klensin. The complete list of attendees is available in the meeting notes.

\subsection{Meeting Discussions}

\subsubsection{Applicability Statement for IETF Core Email Protocols}
The session began with a detailed review of the \href{https://datatracker.ietf.org/doc/html/draft-ietf-emailcore-as-15}{draft-ietf-emailcore-as-15}, focusing on issues raised during the Working Group Last Call (WGLC). Key discussions revolved around the requirement of STARTTLS for legacy systems, with John Klensin questioning its necessity and Bron Gondwana advocating for a broader approach to confidentiality. Pete Resnick agreed to revise the text to emphasize the opportunistic use of STARTTLS. Further, Ken Murchison was tasked with incorporating DANE and requireTLS into the document, while Barry Leiba will verify the need for additional clarifications. The dialogue underscored the importance of balancing security with practical deployment considerations.

\subsubsection{Status of RFC5321bis-41 and SMTP IANA Cleanup}
Alexey Melnikov and John Klensin provided updates on \href{https://datatracker.ietf.org/doc/html/draft-ietf-emailcore-rfc5321bis-41}{draft-ietf-emailcore-rfc5321bis-41} and \href{https://datatracker.ietf.org/doc/html/draft-melnikov-smtp-iana-cleanup-00}{draft-melnikov-smtp-iana-cleanup-00}. The former will undergo a targeted IETF Last Call focused on security considerations, bypassing further IESG review if successful. The SMTP IANA cleanup was deemed out of scope for EMAILCORE and will be addressed by the MAILMAINT working group. These updates reflect ongoing efforts to refine and secure email protocols, with implications for future interoperability and security enhancements.

Meeting materials, including the agenda and notes, are available at \href{https://notes.ietf.org/notes-ietf-122-emailcore}{this link}.



\newpage

\section{EMU Working Group (EMU)}

\subsection{Attendees Overview}

The EMU Working Group meeting at IETF 122 was attended by 38 participants, including representatives from prominent organizations such as Cisco, NSA, Google, and Nokia. Notable institutions like the University of Murcia and the University of Oviedo were also present, contributing to a diverse and knowledgeable assembly.

\subsection{Meeting Discussions}

\subsubsection{7170bis}

Alan DeKok presented on the \href{https://datatracker.ietf.org/doc/html/draft-ietf-emu-rfc7170bis}{draft-ietf-emu-rfc7170bis}. The discussion highlighted interoperability issues with the current specification. It was proposed to publish 7170bis as TEAPv1 with existing behavior and to develop a TEAPv2 with clearer specifications and pre-publication interoperability testing. This approach aims to streamline the specification by removing non-functional elements, potentially improving security and implementation clarity.

\subsubsection{EAP-EDHOC}

Dan Garcia-Carrillo discussed the \href{https://datatracker.ietf.org/doc/html/draft-ietf-emu-eap-edhoc}{draft-ietf-emu-eap-edhoc}, focusing on its applicability to IoT environments. The need for fast reconnect capabilities was debated, with suggestions to solicit further input from the mailing list. The primary use case revolves around IoT, aligning with EAP method requirements outlined in RFC4017.

\subsubsection{EAP-FIDO}

Janfred Rieckers presented updates on the \href{https://datatracker.ietf.org/doc/html/draft-ietf-emu-eap-fido}{draft-ietf-emu-eap-fido}. The presentation covered ongoing interoperability testing and discussions on cryptographic binding. The group debated the implications of moving away from FIDO standards, particularly concerning the use of binary formats and hash algorithms. The potential for collaboration with W3C and FIDO Alliance was also explored.

\subsubsection{EAP-PPT Charter Update}

Joe Salowey, as WG Chair, introduced the \href{https://datatracker.ietf.org/doc/html/draft-sawant-eap-ppt}{draft-sawant-eap-ppt}. The adoption of this draft is contingent upon rechartering, which would allow immediate integration into the working group's agenda. Discussions included broadening the scope to encompass various protocols and potential provisioning enhancements.

\subsubsection{PQC in EAP-AKA'}

Tiru Reddy presented on the integration of post-quantum cryptography in EAP-AKA', referencing \href{https://datatracker.ietf.org/doc/html/draft-ar-emu-pqc-eapaka}{Draft 1} and \href{https://datatracker.ietf.org/doc/html/draft-ra-emu-pqc-eapaka}{Draft 2}. The work was well-received, with support expressed for its necessity in enhancing security. The potential for implementation in wpa\_supplicant was discussed, highlighting the importance of practical adoption.

Meeting materials are available at \href{https://www.ietf.org/proceedings/122/emu.html}{IETF 122 EMU Meeting Materials}.




\newpage

\section{GREEN (GREEN)}

\subsection{Attendees Overview}
\subsubsection{Attendance and Representation}
The GREEN working group session at IETF 122 was attended by 82 participants, representing a diverse array of prominent organizations including Cisco, Nokia, Deutsche Telekom, Telefonica, and Huawei. The session facilitated a robust exchange of ideas among industry leaders and academic institutions, underscoring the collaborative spirit of the IETF community.

\subsection{Meeting Discussions}
\subsubsection{Use Cases and Requirements}
The session commenced with a discussion on the \href{https://datatracker.ietf.org/doc/draft-bernardos-green-isac-uc/}{Integrated Sensing and Communications (ISAC) use case for GREEN}, presented by Carlos J. Bernardos. The dialogue highlighted the complexity of integrating ISAC within urban environments, emphasizing the need for energy-efficient models. Participants debated the scope of requirements, with a consensus on the necessity for further refinement to align with GREEN's objectives. The \href{https://datatracker.ietf.org/doc/draft-stephan-green-use-cases/}{Use Cases for Energy Efficiency Management} draft, presented by Luis Contreras, prompted discussions on the categorization of use cases and their alignment with the working group's charter.

\subsubsection{Terminology and Metrics Definitions}
Qin Wu presented the \href{https://datatracker.ietf.org/doc/draft-bclp-green-terminology/}{Terminology for Energy Efficiency Network Management} draft, which aims to standardize terms across the field. The session underscored the importance of distinguishing between energy savings and efficiency, with plans to incorporate feedback from the W3C to ensure comprehensive coverage.

\subsubsection{Architectural Components}
Jing Wang introduced the \href{https://datatracker.ietf.org/doc/draft-wang-green-framework/}{Framework for Getting Ready for Energy-Efficient Networking (GREEN)}, which was critiqued for its preliminary nature. Participants called for more detailed content to establish a robust framework. The \href{https://datatracker.ietf.org/doc/draft-lindblad-tlm-philatelist/}{Philatelist framework}, presented by Jan Lindblad, sparked interest due to its potential to integrate telemetry data and time series databases. Discussions focused on its applicability within the GREEN scope, with suggestions to refine its focus to align with the working group's charter.

\subsubsection{Data Models}
The second session featured presentations on data models, including the \href{https://datatracker.ietf.org/doc/draft-opsawg-poweff/}{Power and Energy Efficiency} draft by Jan Lindblad and the \href{https://datatracker.ietf.org/doc/draft-neumann-green-streaming-yang/}{YANG Data Model for Energy Measurements in Streaming Devices} by Christoph Neumann. These models aim to provide foundational structures for energy management in networking, with discussions emphasizing the need for a common base model that can be extended for specific use cases.

\subsubsection{Future Planning}
The sessions concluded with a wrap-up by the chairs, highlighting the need for continued collaboration and refinement of the presented drafts. The working group plans to focus on developing a common model that can serve as a foundation for future work, with interim meetings proposed to maintain momentum.

Meeting materials and additional resources are available at \href{https://notes.ietf.org/notes-ietf-122-green}{Meeting Minutes}.




\newpage

\section{Global Routing Operations Working Group (GROW)}

\subsection{Attendees}
\subsubsection{Overview}
The GROW session at IETF 122 was attended by 50 participants, including representatives from prominent organizations such as RIPE NCC, Red Hat, Cisco, Huawei, and NLnet Labs. The diverse attendance underscored the broad interest and engagement in the ongoing developments within the working group.

\subsection{Meeting Discussions}

\subsubsection{Rechartering Effort}
Mohamed Boucadair presented the ongoing rechartering effort, highlighting the need to update the scope and milestones of the working group. The draft of the new charter is available for review and feedback at \href{https://datatracker.ietf.org/doc/charter-ietf-grow/04-01/}{charter-ietf-grow/04-01}. The discussion emphasized the importance of community input and collaboration with Area Directors to refine the charter.

\subsubsection{BMP TLV Extensions}
Paolo Lucente discussed several drafts related to BMP TLV extensions, including \href{https://datatracker.ietf.org/doc/html/draft-ietf-grow-bmp-tlv}{draft-ietf-grow-bmp-tlv}, \href{https://datatracker.ietf.org/doc/html/draft-ietf-grow-bmp-rel}{draft-ietf-grow-bmp-rel}, and \href{https://datatracker.ietf.org/doc/html/draft-ietf-grow-bmp-tlv-ebit}{draft-ietf-grow-bmp-tlv-ebit}. Key points included the need for IANA code point assignments and the potential for simplifying Stateless Parsing TLV by removing AFI/SAFI. The discussions also touched on timestamping requirements and editorial improvements.

\subsubsection{BMP BGP RIB Statistics}
Mukul Srivastava presented \href{https://datatracker.ietf.org/doc/html/draft-ietf-grow-bmp-bgp-rib-stats}{draft-ietf-grow-bmp-bgp-rib-stats}, focusing on the collection of BGP RIB statistics via BMP. The session highlighted the importance of aligning assignments with the Standards Action of the registry. A Working Group Last Call (WGLC) was suggested as a next step.

\subsubsection{BMP Common Updates and Purge Mechanisms}
Prasad Narasimha introduced drafts on BMP common updates and purge mechanisms, including \href{https://datatracker.ietf.org/doc/html/draft-patki-grow-bmp-common-updates}{draft-patki-grow-bmp-common-updates} and \href{https://datatracker.ietf.org/doc/html/draft-spd-grow-bmp-purge}{draft-spd-grow-bmp-purge}. Discussions revolved around optimizing flag usage and exploring alternative mechanisms such as TLVs. The feedback will inform future revisions and potential Working Group adoption calls.

\subsubsection{BMP Location Peer}
Maxence Younsi presented \href{https://datatracker.ietf.org/doc/html/draft-ietf-grow-bmp-loc-peer}{draft-ietf-grow-bmp-loc-peer}, which aims to enhance BMP's ability to monitor peer locations. The presentation underscored the draft's potential to improve network visibility and operational efficiency.

Meeting materials and presentations are available at \href{https://datatracker.ietf.org/meeting/122/materials.html}{IETF 122 Meeting Materials}.




\newpage

\section{HAPPY Working Group (HAPPY)}

\subsection{Attendees Overview}
\subsubsection{Attendees}
The HAPPY Working Group session was attended by 67 participants, including representatives from prominent organizations such as Apple, Google, Cloudflare, Cisco Systems, and Mozilla. The diverse attendance underscored the broad interest and collaborative efforts in advancing the group's objectives.

\subsection{Meeting Discussions}
\subsubsection{Chair's Introduction}
The session commenced with an introduction by Eric Kinnear (Apple) and Tim Chown (Jisc), who outlined the group's primary focus on updating the core happy eyeballs algorithm. The chairs emphasized the importance of a secondary document on reporting connectivity issues and confirmed the use of GitHub for collaborative work, ensuring transparency and community involvement.

\subsubsection{HEv3 Draft Discussion}
Tommy Pauly and Nidhi Jaju presented the \href{https://datatracker.ietf.org/doc/html/draft-ietf-happy-happyeyeballs-v3}{draft-ietf-happy-happyeyeballs-v3}, highlighting the need for updates due to advancements in DNS and transport technologies like QUIC and SVCB. The discussion centered on refining the client-side algorithm for connection establishment and addressing open issues such as SVCB integration and prioritization strategies. The group debated the implications of server versus client preferences, particularly in security contexts, and considered the potential for interim meetings to resolve outstanding technical challenges.

\subsubsection{Chromium Implementation Status Update}
Kenichi Ishibashi provided an update on Chromium's implementation of HEv3, noting improvements in connection logic and DNS resolution sharing between TCP and QUIC. An early field experiment is planned to assess these changes, with a focus on supporting non-default target names in future updates.

\subsubsection{Reporting Network Errors to Origins}
Nic Jansma and Utkarsh Goel discussed the potential for a draft on network error logging (NEL), aiming to improve feedback mechanisms for network issues masked by happy eyeballs. The conversation touched on privacy concerns and the need for a balanced approach to reporting metrics, with suggestions for collaboration with other IETF groups to refine the proposal.

\subsubsection{Summary and Areas for Coordination}
The session concluded with a call for draft adoption and considerations for virtual interim meetings to address specific topics such as DNS server selection and IPv6 network strategies. The chairs highlighted the importance of coordination with related IETF groups, including v6ops, 6man, and dnsop, to ensure comprehensive and cohesive progress.

Meeting materials, including slides and minutes, are available at \href{https://datatracker.ietf.org/meeting/122/materials/slides-122-happy}{Meeting Materials}.



\newpage

\section{HTTPAPI (HTTPAPI)}

\subsection{Attendees}
\subsubsection{Overview}
The HTTPAPI working group meeting at IETF 122 was attended by 35 participants, including representatives from prominent companies and institutions such as Cloudflare, Microsoft, Akamai, Ericsson, Google LLC, and Apple. The meeting was chaired by Rich, with Francesca Palombini and Mike Bishop serving as Area Directors.

\subsection{Meeting Discussions}

\subsubsection{Document Updates}
The session began with brief updates on documents currently in the RPC queue, including \textit{api-catalog} and \textit{deprecation-header}. The \href{https://datatracker.ietf.org/doc/html/draft-ietf-httpapi-link-hint}{link-hint} document was highlighted for review and feedback, while \href{https://datatracker.ietf.org/doc/html/draft-ietf-httpapi-rest-api-mediatypes}{rest-api-mediatypes} is nearing readiness for a Working Group Last Call (WGLC) pending resolution of two open issues. The \href{https://datatracker.ietf.org/doc/html/draft-ietf-httpapi-ratelimit-headers}{ratelimit-headers} document received substantial feedback, leading to the introduction of new problem types to clarify client rate-limiting issues. Darrel Miller's GitHub implementation was noted as a practical check for feasibility, and further feedback was solicited to potentially advance to WGLC by IETF 123.

\subsubsection{New JSON Schema}
A significant discussion centered around the proposal for a "New JSON Schema," aimed at addressing tooling challenges and enhancing data-binding capabilities compared to existing JSON Schema implementations. The proposal suggests a refactoring approach, introducing additional data types and serialization rules. Concerns were raised about the naming and venue for this initiative, with Clemens Vasters open to leveraging existing drafts for broader acceptance. The chairs and ADs are tasked with determining a strategic path forward.

\subsubsection{Documents Ready for WGLC}
The \href{https://datatracker.ietf.org/doc/html/draft-ietf-httpapi-httpapi-privacy}{httpapi-privacy} document requires further work before proceeding to WGLC. For \href{https://datatracker.ietf.org/doc/html/draft-ietf-httpapi-digest-fields-problem-types}{digest-fields-problem-types}, an early SECDIR review was requested, with discussions on whether to integrate Message Signature problem types, ultimately deciding to keep them separate due to complexity.

\subsubsection{Expired Documents}
The group briefly addressed expired documents, \textit{patch-byterange} and \textit{authentication-link}, inviting interested parties to contact the chairs for potential revival efforts.

Meeting materials are available at \href{https://www.ietf.org/proceedings/122/httpapi.html}{IETF 122 HTTPAPI Meeting Materials}.



\newpage

\section{HTTP Working Group (HTTPBIS)}

\subsection{Attendees Overview}
\subsubsection{Attendance}
The HTTP Working Group session at IETF 122 saw participation from 71 attendees, including representatives from prominent companies and institutions such as Meta, Google, Cloudflare, Microsoft, and Ericsson. Key participants included Julian Reschke, Ben Schwartz, and Marius Kleidl, among others.

\subsection{Meeting Discussions}

\subsubsection{QUERY Method}
The discussion on the \href{https://datatracker.ietf.org/doc/html/draft-ietf-httpbis-safe-method-w-body}{QUERY Method} draft, presented by Julian Reschke, focused on resolving editorial issues and preparing for Working Group Last Call (WGLC). The debate centered around the use of GET parameters for stored queries, with consensus leaning towards treating the entire URL as the resource identity.

\subsubsection{CONNECT-TCP}
Ben Schwartz presented updates on the \href{https://datatracker.ietf.org/doc/html/draft-ietf-httpbis-connect-tcp}{CONNECT-TCP} draft, highlighting the alignment with the Capsule protocol. Discussions explored signaling FIN vs RST, with a preference for maintaining the status quo in HTTP/2 and HTTP/3, while considering future-proofing through Capsules.

\subsubsection{Security Considerations for Optimistic Protocol Transitions}
The \href{https://datatracker.ietf.org/doc/html/draft-ietf-httpbis-optimistic-upgrade}{Security Considerations for Optimistic Protocol Transitions} draft was discussed, emphasizing the need for clients to wait for 2xx responses before forwarding TCP payloads. The group leaned towards using MUST language to enforce connection closure upon rejection, balancing performance and security.

\subsubsection{Resumable Uploads}
Marius Kleidl's presentation on \href{https://datatracker.ietf.org/doc/html/draft-ietf-httpbis-resumable-upload}{Resumable Uploads} focused on recent editorial overhauls and normative changes. The group discussed the method for appending data, with a consensus to retain the current PATCH method approach, moving towards WGLC.

\subsubsection{Secondary Certificate Authentication}
Eric Gorbaty addressed the \href{https://datatracker.ietf.org/doc/html/draft-ietf-httpbis-secondary-server-certs}{Secondary Certificate Authentication} draft, exploring solutions for large exported authenticators in HTTP/2. The group considered using a new stream type, with further investigation needed before finalizing the approach.

\subsubsection{No-Vary-Search}
Domenic Denicola presented the \href{https://datatracker.ietf.org/doc/html/draft-ietf-httpbis-no-vary-search}{No-Vary-Search} draft, noting its implementation in Chrome and Google Search. The group discussed marking the draft as updating RFC9111, with plans for a Working Group Last Call.

\subsubsection{Incremental HTTP Messages}
Kazuho Oku's presentation on \href{https://datatracker.ietf.org/doc/html/draft-kazuho-httpbis-incremental-http}{Incremental HTTP Messages} focused on signaling mechanisms for incremental delivery. The group favored a simple boolean preference, with further refinement needed for hard fail scenarios.

\subsubsection{HTTP Unencoded Digest}
Lucas Pardue discussed the \href{https://datatracker.ietf.org/doc/html/draft-pardue-httpbis-identity-digest}{HTTP Unencoded Digest} draft, addressing concerns about adding a third digest header. The group agreed to continue discussions on the mailing list.

\subsubsection{Delete-Cookie and \_HttpOnly Prefix}
Yoav Weiss presented drafts on \href{https://datatracker.ietf.org/doc/html/draft-deletecookie-weiss-http}{Delete-Cookie} and \href{https://datatracker.ietf.org/doc/html/draft-httponlyprefix-weiss-http}{\_HttpOnly Prefix}, with discussions on integrating these into the broader cookies draft. The group supported merging these proposals into a unified effort.

\subsubsection{Cookies: HTTP State Management Mechanism}
Johann Hofmann's presentation on the \href{https://datatracker.ietf.org/doc/html/draft-annevk-johannhof-httpbis-cookies}{Cookies: HTTP State Management Mechanism} draft highlighted the need for adoption and integration of related proposals. The group planned to initiate an adoption call.

\subsubsection{Critical CH and Accept-CH Frame}
Victor Tan discussed the \href{https://datatracker.ietf.org/doc/html/draft-victortan-httpbis-chr-critical-ch}{Critical CH} and \href{https://datatracker.ietf.org/doc/html/draft-victortan-httpbis-chr-accept-ch-frame}{Accept-CH Frame} drafts, with debates on the implications for privacy and performance. The group emphasized the need for concrete use cases to guide further discussions.

Meeting materials are available \href{https://notes.ietf.org/notes-ietf-122-httpbis}{here}.



\newpage

\section{ICCRG (Internet Congestion Control Research Group)}

\subsection{Attendees Overview}
The ICCRG meeting at IETF 122 was attended by representatives from prominent companies and institutions such as Apple, Google, Ericsson, Huawei, and the University of Oslo, totaling 101 participants. This diverse group brought together experts from academia and industry to discuss advancements and challenges in congestion control.

\subsection{Meeting Discussions}

\subsubsection{LEDBAT++}
The discussion on \href{https://datatracker.ietf.org/doc/html/draft-irtf-iccrg-ledbat-plus-plus-02}{draft-irtf-iccrg-ledbat-plus-plus-02} focused on its readiness for Research Group Last Call. Participants were encouraged to review the document and provide feedback. The consensus was that the draft is mature and ready for broader community input.

\subsubsection{Pacing in Transport Protocols}
Michael Welzl presented updates to \href{https://datatracker.ietf.org/doc/html/draft-welzl-iccrg-pacing}{draft-welzl-iccrg-pacing}, incorporating feedback on SYN/ACK and RTTs. The discussion highlighted the importance of pacing in real-time protocols and its impact on network performance. The group agreed that the draft is ready for adoption, with further refinements to address application-specific pacing deviations.

\subsubsection{LEO Mobility and Congestion Control}
Zeqi Lai's presentation on the effects of LEO satellite mobility on congestion control sparked a debate on bandwidth and delay estimations. The discussion underscored the need for accurate modeling of LEO networks and the challenges in adapting congestion control algorithms to these environments. The group recognized the potential for significant impact on future satellite network designs.

\subsubsection{Improving Cloud Gaming QoS}
Thibault Cholez compared class-based queuing policies with L4S for cloud gaming traffic. The analysis revealed that L4S can enhance QoS, though further testing with alternative AQM targets was suggested. The discussion emphasized the importance of optimizing network policies to support emerging gaming technologies.

\subsubsection{Adapting Home Routers for Low Latency}
Zili Meng presented \href{https://datatracker.ietf.org/doc/html/draft-zhang-iccrg-confucius}{draft-zhang-iccrg-confucius}, focusing on adapting home routers to congestion control reactions. The discussion highlighted the complexity of managing multiple flows and the need for more research on congestion control algorithms to maintain low latency in home networks.

\subsubsection{Networking for AIML Clusters}
Costin Raiciu explored the challenges of networking in AI/ML clusters, emphasizing the need for efficient multipath transport protocols. The presentation introduced the Ultra Ethernet Consortium's efforts to standardize transport for AIML networks. The group acknowledged the potential for these advancements to significantly enhance AI/ML training efficiency.

Meeting materials are available at \href{https://datatracker.ietf.org/meeting/122/materials/iccrg}{ICCRG Meeting Materials}.




\newpage

\section{Information-Centric Networking Research Group (ICNRG)}

\subsection{Attendees Overview}
\subsubsection{Attendance}
The ICNRG meeting was attended by 38 participants, including representatives from prominent institutions such as WiscNet, University of Kentucky, intERLab/AIT, NICT, and Huawei. Notable attendees included Marc Mosko, Ken Calvert, and Lixia Zhang.

\subsection{Meeting Discussions}

\subsubsection{Offline Content Access with NDN and D2D Communication}
Preechai Mekbungwan from intERLab/AIT presented on leveraging Named Data Networking (NDN) for offline content delivery in remote areas using Device-to-Device (D2D) communication. The prototype, built with Google Nearby Connections API, demonstrated a coverage of up to 150 meters. The discussion highlighted the integration of NDN for content discovery and retrieval in multi-hop environments, addressing D2D limitations with proposed NDN modifications. Future plans include real-world deployment in virtual classrooms and file exchange applications. The presentation underscored the importance of ICN in content delivery management, with a research grant from THNICF supporting the initiative.

\subsubsection{FLIC Update}
Marc Mosko provided updates on the \href{https://datatracker.ietf.org/doc/html/draft-ietf-icnrg-flic}{draft-ietf-icnrg-flic}, focusing on non-tree FLICs, security enhancements, and implementation status. The draft is nearing final review, with significant savings demonstrated in de-duplication experiments. A Research Group last call is anticipated soon.

\subsubsection{CCNx Content Object Chunking}
The \href{https://datatracker.ietf.org/doc/html/draft-ietf-icnrg-ccnxchunking}{draft-ietf-icnrg-ccnxchunking} was discussed by Marc Mosko, emphasizing the use of preamble chunks for improved control in identifying the last chunk. The draft is technically complete but requires minor refinements.

\subsubsection{CCNx Versioning}
Marc Mosko also covered the \href{https://datatracker.ietf.org/doc/html/draft-asaeda-icnrg-ccnxcversioning}{draft-asaeda-icnrg-ccnxcversioning}, introducing a T\_version name segment type for querying current content versions. The draft aims to facilitate version discovery and response, with feedback encouraged for further refinement.

\subsubsection{Reflexive Forwarding Update}
Hitoshi Asaeda presented on reflexive forwarding, addressing use cases like IoT sensor polling and peer state synchronization. The \href{https://datatracker.ietf.org/doc/html/draft-irtf-icnrg-reflexive-forwarding}{draft-irtf-icnrg-reflexive-forwarding} explores multi-way handshakes and reflexive interest exchanges, seeking broader reviews and implementations.

\subsubsection{Distributed Micro Service Communication Progress Report}
Aijun Wang discussed the progress of Distributed Micro Service Communication (DMSC), aiming to address service mesh performance overheads. The \href{https://datatracker.ietf.org/doc/bofreq-wang-distributed-micro-services-communicationdmsc}{draft-bofreq-wang-distributed-micro-services-communicationdmsc} outlines plans for a symposium in Beijing and potential Working Group establishment.

\subsubsection{Wrap Up and Next Steps}
The meeting concluded with announcements for upcoming events, including an NDN community meeting in April 2025 and the next ICNRG meeting in Madrid, July 2025.

Meeting materials are available at \href{https://datatracker.ietf.org/meeting/122/session/icnrg}{ICNRG Meeting Materials}.



\newpage

\section{Inter-Domain Routing (IDR) Working Group}

\subsection{Attendees Overview}

The IDR sessions at IETF 122 were attended by representatives from prominent organizations such as Cisco Systems, Juniper Networks, Huawei Technologies, and Deutsche Telekom, among others. The total attendance was 74 participants, reflecting a diverse mix of industry leaders and academic institutions.

\subsection{Meeting Discussions}

\subsubsection{Guidance for Authors on BGP Tunnel Encapsulation Attribute}

Susan Hares presented the \href{https://datatracker.ietf.org/doc/html/draft-hares-idr-bess-tea-templates-00}{draft-hares-idr-bess-tea-templates-00}, emphasizing the importance of early collaboration in drafting sections to avoid late-stage reviews. This guidance aims to streamline the process of writing text for BGP Tunnel Encapsulation Attributes, ensuring clarity and consistency across implementations.

\subsubsection{BGP Link Bandwidth Extended Community}

Reshma Das discussed the \href{https://datatracker.ietf.org/doc/html/draft-ietf-idr-link-bandwidth-11}{draft-ietf-idr-link-bandwidth-11}, highlighting its significance for network operations. The discussion focused on the transitivity of link bandwidth communities and the implications for interoperability. The draft proposes a singleton approach to manage bandwidth across AS boundaries, with further discussions planned to refine the transitivity controls.

\subsubsection{BGP SR Policy Extensions for Path Scheduling}

Li Zhang introduced the \href{https://datatracker.ietf.org/doc/html/draft-zzd-idr-sr-policy-scheduling-08}{draft-zzd-idr-sr-policy-scheduling-08}, which proposes extensions for path scheduling within BGP SR policies. The draft addresses the need for frequent updates without overloading the control plane, suggesting BGP as a viable mechanism for real-time path adjustments. The proposal will undergo further review to align with TVR requirements and SPRING architecture discussions.

\subsubsection{SRv6 Policy SID List Optimization Advertisement}

Zafar Ali presented the \href{https://datatracker.ietf.org/doc/html/draft-ali-idr-srv6-policy-sl-opt-distribution-00}{draft-ali-idr-srv6-policy-sl-opt-distribution-00}, which aims to optimize SID list advertisements in SRv6 policies. The discussion underscored the need for consistency across BGP, PCEP, and YANG extensions, with plans to coordinate with SPRING to ensure alignment.

\subsubsection{Dynamic Capability for BGP-4}

Srihari Sangli discussed the \href{https://datatracker.ietf.org/doc/html/draft-ietf-idr-dynamic-cap-16}{draft-ietf-idr-dynamic-cap-16}, focusing on the introduction of dynamic capabilities in BGP-4. The draft proposes enhancements to manage capability negotiations more effectively, with feedback sought from vendors to ensure backward compatibility and interoperability.

\subsubsection{BGP-Link State Advertisement of Flow Queue}

Jinming Li presented the \href{https://datatracker.ietf.org/doc/html/draft-li-idr-bgpls-flow-queue-00}{draft-li-idr-bgpls-flow-queue-00}, which proposes using BGP-LS for advertising flow queue information. The discussion highlighted the challenges of dynamic queueing and the potential need for alternative approaches like gRPC or netconf for timely updates.

Meeting materials are available at \href{https://datatracker.ietf.org/meeting/122/session/idr/}{IDR Meeting Materials}.



\newpage

\section{IntArea Working Group (IntArea WG)}

\subsection{Attendees Overview}
\subsubsection{Prominent Attendees and Total Attendance}
The IntArea WG meeting was attended by 96 participants, including representatives from major companies and institutions such as Cisco, Ericsson, Apple Inc., Microsoft, Google, and Huawei. Notable attendees included Juan Carlos Zuniga from Cisco, Wassim Haddad from Ericsson, and Tommy Pauly from Apple.

\subsection{Meeting Discussions}

\subsubsection{Agenda Bashing, WG \& Document Status Updates}
The meeting commenced with Juan Carlos Zuniga and Wassim Haddad providing updates on the working group's document status. A call for review was made for the \href{https://datatracker.ietf.org/doc/html/draft-ietf-intarea-tunnels}{draft-ietf-intarea-tunnels}, as the authors have requested a Working Group Last Call (WGLC).

\subsubsection{Communicating Proxy Configurations in Provisioning Domains}
Tommy Pauly, Dragana Damjanovic, and Yaroslav Rosomakho presented updates on \href{https://datatracker.ietf.org/doc/html/draft-ietf-intarea-proxy-config-04}{draft-ietf-intarea-proxy-config-04}. The discussion highlighted the addition of a new co-author and upcoming interoperability testing. A WGLC is anticipated before the Madrid IETF meeting.

\subsubsection{Adding Extensions to ICMP Errors for Originating Node Identification}
Bill Fenner and R. Thomas discussed \href{https://datatracker.ietf.org/doc/html/draft-ietf-intarea-extended-icmp-nodeid-01}{draft-ietf-intarea-extended-icmp-nodeid-01}. Despite interest in adding new data types, the recommendation was to proceed with the current document and extend it as needed. A WGLC was suggested and received positive feedback.

\subsubsection{PROBE: A Utility for Probing Interfaces}
The presentation on \href{https://datatracker.ietf.org/doc/html/draft-ietf-intarea-rfc8335bis-00}{draft-ietf-intarea-rfc8335bis-00} by Bill Fenner and colleagues addressed interoperability issues with existing RFC 8335 implementations. The document includes warnings about specific design mechanisms. A WGLC will be discussed on the mailing list.

\subsubsection{The Multicast Application Port}
Nathan Karstens and team presented \href{https://datatracker.ietf.org/doc/html/draft-karstens-intarea-multicast-application-port-00}{draft-karstens-intarea-multicast-application-port-00}, discussing interactions with the IANA ports team and the implications of using multicast over unicast. The discussion emphasized the need for recommendations to avoid port exhaustion.

\subsubsection{IPv4 Routes with an IPv6 Next Hop}
Warren Kumari and others discussed \href{https://datatracker.ietf.org/doc/html/draft-chroboczek-intarea-v4-via-v6-03}{draft-chroboczek-intarea-v4-via-v6-03}, focusing on adoption and implementation on hosts. The draft received support and suggestions for further guidance on routing differences based on nexthop.

\subsubsection{Safe(r) Limited Domains}
The presentation on \href{https://datatracker.ietf.org/doc/html/draft-wkumari-intarea-safe-limited-domains-04}{draft-wkumari-intarea-safe-limited-domains-04} by Warren Kumari and colleagues emphasized its role as guidance for protocol designers rather than protocol changes.

\subsubsection{EVN6: Mapping of Ethernet Virtual Network to IPv6 Underlay}
Chongfeng Xie and team discussed \href{https://datatracker.ietf.org/doc/html/draft-xls-intarea-evn6-03}{draft-xls-intarea-evn6-03}, with questions about its fit within the IntArea WG versus other groups like NVO3. The document's adoption was considered, with suggestions for additional references.

\subsubsection{Enhancing ICMPv6 Error Message Authentication}
K. Xu and colleagues presented \href{https://datatracker.ietf.org/doc/html/draft-xu-intarea-challenge-icmpv6-00}{draft-xu-intarea-challenge-icmpv6-00}, addressing challenges such as packet loss and state management. The discussion highlighted the need to address potential DoS/amplification issues.

\subsubsection{Proposal for Updates to Guidance on Packet Reordering}
Greg White and team discussed \href{https://datatracker.ietf.org/doc/html/draft-white-intarea-reordering-00}{draft-white-intarea-reordering-00}, exploring interest in contributions and the suitability of IntArea as the right group. The conversation touched on balancing packet holding versus loss and the implications for different traffic types.

Meeting materials are available at \href{https://www.ietf.org/proceedings/122/intarea.html}{IETF 122 IntArea WG Meeting Materials}.



\newpage

\section{Internet of Things Operations (IOTOPS)}

\subsection{Attendees Overview}
\subsubsection{Attendees}
The IOTOPS meeting was attended by 23 participants, including representatives from prominent companies and institutions such as Isode Limited, Sandelman Software Works Inc, Uni Bremen, Orange, Check Point Software, SECOM CO., LTD., RISE Research Institutes of Sweden, Tohoku University, CORE Association, Ericsson, IMT Atlantique, Anatel, TU Dresden, and University of Applied Sciences Bonn-Rhein-Sieg.

\subsection{Meeting Discussions}

\subsubsection{Administrivia}
The meeting commenced with a brief administrative session led by the chairs, Alexey Melnikov and Henk Birkholz. The primary focus was on the \href{https://datatracker.ietf.org/doc/html/draft-ietf-iotops-security-summary-03}{draft-ietf-iotops-security-summary-03}, which summarizes security-enabling technologies for IoT devices. Discussions highlighted the stalled progress of the cTLS document, with participants debating the nature of references to cTLS within the draft. The chairs committed to consulting the TLS Working Group to determine the future of cTLS.

\subsubsection{Recharter Discussion}
The main agenda item was the recharter discussion, which spanned 40 minutes. The conversation centered around refining the working group's scope and ensuring alignment with other groups like SUIT and OPSAWG. Med suggested removing "discovery" from "MUD discovery solutions" and verifying potential overlaps with the SUIT WG. The inclusion of "Terminology" in the charter was proposed to encompass RFC7228bis. Gaetan Feige raised concerns about overlaps with external entities such as CSA/MATTER, prompting a call for additional milestones. MCR proposed transferring certain MUD documents to the IOTOPS WG, indicating a strategic shift in document management.

\subsubsection{Additional Business}
The session concluded with a discussion on the readiness of RFC7228bis for Working Group Last Call (WGLC). MCR confirmed no outstanding issues, while Alexey emphasized the need for authors to address recent review feedback.

Meeting materials are available at \href{https://meetings.conf.meetecho.com/ietf122/?group=iotops&short=&item=1}{Meetecho} and \href{https://notes.ietf.org/notes-ietf-122-iotops}{Notes}.



\newpage

\section{IP Performance Measurement (IPPM)}

\subsection{Attendees}
\subsubsection{Overview}
The IPPM working group meeting was attended by 54 participants, including representatives from prominent organizations such as Cisco, Huawei, Apple, and Comcast. Notable attendees included Thomas Graf from Swisscom, Tanzeela Altaf from the University of Technology Sydney, and Greg Mirsky from Ericsson.

\subsection{Meeting Discussions}

\subsubsection{Agenda Bashing \& Introduction}
The meeting commenced with a brief introduction and agenda review by the chairs, Marcus Ihlar, Tommy Pauly, and Thomas Graf.

\subsubsection{draft-ietf-ippm-ioam-data-integrity}
Presenter: J. Iurman / Chairs

The discussion centered around the proposal for integrity protection of IOAM data fields, focusing on pre-allocated and incremental option-type header fields. Feedback was solicited on the trade-off between versioning and defining header fields in the IANA registry. The document excludes direct export option-type, emphasizing protection for edge-to-edge and proof of transit option-types. For more details, refer to the \href{https://datatracker.ietf.org/doc/html/draft-ietf-ippm-ioam-data-integrity}{draft-ietf-ippm-ioam-data-integrity}.

\subsubsection{draft-ietf-ippm-asymmetrical-pkts}
Presenter: G. Mirsky

The request for Working Group Last Call (WG LC) was considered, with emphasis on gathering more feedback from the group to facilitate the process.

\subsubsection{draft-ietf-ippm-stamp-ext-hdr}
Presenter: R. Gandhi

The discussion highlighted the inclusion of MPLS in the document, with suggestions to split the work if IPv6 progresses more quickly. The consensus was to maintain MPLS, anticipating timely progress from the MPLS working group.

\subsubsection{draft-ietf-ippm-qoo}
Presenter: B. Tiegen

The presentation addressed the adequacy of sample sizes for accurate measurement results. The discussion explored the variability of latency distributions and the potential need for statistical methods to determine sufficient data points. The document's classification as informational was debated, with security considerations suggested before WG LC.

\subsubsection{draft-ietf-ippm-responsiveness}
Presenter: J. Iurman / Chairs

The need for outreach to application communities for feedback was emphasized. The discussion also touched on default values and source buffer management, with suggestions to engage with BMWG and other relevant groups.

\subsubsection{draft-ietf-ippm-alt-mark-deployment}
Presenter: G. Fioccola

The work received support, with discussions on deployment scenarios and potential enhancements.

\subsubsection{draft-ietf-ippm-alt-mark-yang}
Presenter: G. Fioccola

The presentation focused on the YANG data model for alternate marking, with discussions on potential augmentations and operational metrics.

\subsection{Proposed Work}

\subsubsection{draft-white-ippm-stamp-ecn}
Presenter: G. White

The proposal for STAMP ECN extensions was discussed, with emphasis on ensuring compatibility with existing standards and involving the transport area for feedback. The potential for IPPM adoption was considered, contingent on TSVWG approval.

\subsubsection{draft-iuzh-ippm-ioam-integrity-yang}
Presenter: J. Iurman

The presentation covered the YANG model for IOAM integrity, with suggestions for feature statements and conditional leaf additions.

\subsubsection{draft-fioccola-ippm-on-path-active-measurements}
Presenter: G. Fioccola

The work on on-path active measurements was discussed, with considerations for splitting IPv6 and MPLS data planes into separate drafts.

\subsubsection{draft-fz-ippm-on-path-telemetry-yang}
Presenter: G. Fioccola

The discussion focused on the YANG data model for on-path telemetry, with suggestions to explore existing models and potential augmentations.

Meeting materials can be accessed via \href{https://datatracker.ietf.org/meeting/122/materials.html}{IETF 122 Meeting Materials}.




\newpage

\section{IP Security Maintenance and Extensions (IPsecME) WG}

\subsection{Attendees}

The IPsecME working group meeting was attended by 47 participants, including representatives from prominent organizations such as NSA-CCSS, Dell Technologies, secunet Security Networks AG, Huawei, Cisco Systems, and Microsoft. The diverse group of attendees reflects the broad interest and expertise in the field of IP security.

\subsection{Meeting Discussions}

\subsubsection{EESP – Steffen Klassert}

The presentation on Enhanced Encapsulation Security Payload (EESP) by Steffen Klassert, detailed in \href{https://datatracker.ietf.org/doc/html/draft-klassert-ipsecme-eesp}{draft-klassert-ipsecme-eesp}, sparked a robust discussion on the necessity of sequence numbers and the potential for hardware acceleration. Participants debated the balance between flexibility and complexity, with consensus leaning towards adopting EESP as a working group item. The discussion highlighted the need for a modernized protocol that addresses both hardware and software performance improvements.

\subsubsection{IKEv2 Negotiation for EESP – Valery Smyslov}

Valery Smyslov's presentation on IKEv2 negotiation for EESP, as per \href{https://datatracker.ietf.org/doc/html/draft-klassert-ipsecme-eesp-ikev2}{draft-klassert-ipsecme-eesp-ikev2}, focused on key derivation functions (KDFs) and their reliance on SHA-2. The group considered the potential inclusion of SHA-3, reflecting ongoing discussions about cryptographic agility. The proposal was well-received, with plans for parallel adoption calls alongside EESP.

\subsubsection{Use of Variable-Length Output PRFs in IKEv2 – Valery Smyslov}

The introduction of variable-length output pseudorandom functions (PRFs) in IKEv2, as discussed in \href{https://datatracker.ietf.org/doc/html/draft-smyslov-ipsecme-ikev2-prf-plus}{draft-smyslov-ipsecme-ikev2-prf-plus}, was supported for its potential to simplify implementations. The working group expressed interest in adopting this approach, recognizing its minimal complexity increase and potential benefits.

\subsubsection{SA\&TS Payloads Optional – Wei PAN}

Wei PAN's presentation on optimizing rekeys in IKEv2, detailed in \href{https://datatracker.ietf.org/doc/html/draft-ietf-ipsecme-ikev2-sa-ts-payloads-opt}{draft-ietf-ipsecme-ikev2-sa-ts-payloads-opt}, explored various strategies for payload optimization. The group discussed the feasibility of different options, with a focus on achieving interoperability and simplicity.

\subsubsection{Post-quantum Hybrid Key Exchange in the IKEv2 with FrodoKEM – Wang Guilin}

Wang Guilin's work on post-quantum hybrid key exchange using FrodoKEM, as outlined in \href{https://datatracker.ietf.org/doc/html/draft-wang-ipsecme-hybrid-kem-ikev2-frodo}{draft-wang-ipsecme-hybrid-kem-ikev2-frodo}, was met with interest, particularly regarding the number of code points required. The consensus was to streamline the options to facilitate adoption, with recognition of FrodoKEM's relevance in European cryptographic strategies.

\subsubsection{Lightweight Authentication Methods for Encapsulation Headers – Linda Dunbar}

Linda Dunbar's proposal for lightweight authentication methods, as per \href{https://datatracker.ietf.org/doc/html/draft-dunbar-ipsecme-ligthtweight-authenticate}{draft-dunbar-ipsecme-ligthtweight-authenticate}, emphasized the need for efficient security in encapsulation headers. The discussion centered on the appropriate HMAC size, balancing security with performance, and leveraging existing IPsec mechanisms for key exportation.

Meeting materials are available at \href{https://meetings.conf.meetecho.com/ietf122/?group=ipsecme&short=&item=1}{IETF 122 IPsecME Meeting Materials}.



\newpage

\section{IVY Working Group (IVY)}

\subsection{Attendees}
\subsubsection{Overview}
The IVY Working Group session at IETF 122 was attended by 36 participants, including representatives from prominent organizations such as Huawei, Cisco, Nokia, Ericsson, and Telefonica. The diverse attendance underscored the collaborative effort to advance network inventory management standards.

\subsection{Meeting Discussions}

\subsubsection{Introduction}
The session commenced with an introduction and status update of the working group, setting the stage for detailed discussions on network inventory models.

\subsubsection{A YANG Data Model for Network Inventory}
The discussion, led by Aihua Guo, focused on refining the \href{https://datatracker.ietf.org/doc/html/draft-ietf-ivy-network-inventory-yang-05}{draft-ietf-ivy-network-inventory-yang-05}. Key points included the simplification of the base model and the need for clear guidelines on model augmentation. Participants debated the terminology and structure, emphasizing the importance of a straightforward model that accommodates complex inventory scenarios.

\subsubsection{A Network Data Model for Inventory Topology Mapping}
Bo Wu presented the \href{https://datatracker.ietf.org/doc/draft-ietf-ivy-network-inventory-topology-01}{draft-ietf-ivy-network-inventory-topology-01}, highlighting the integration of physical infrastructure mapping. The discussion explored the alignment with existing TE topology frameworks, with consensus on the need for further refinement to ensure comprehensive inventory representation.

\subsubsection{A YANG Module for Entitlement Inventory}
Diego Lopez introduced the \href{https://datatracker.ietf.org/doc/draft-mcd-ivy-entitlement-inventory-00}{draft-mcd-ivy-entitlement-inventory-00}, sparking a debate on the model's scope and its applicability to network and device capabilities. The group discussed the balance between read-only and read-write elements, with a focus on real-world applicability and the potential for future enhancements.

\subsubsection{A YANG Data Model for Passive Network Inventory}
Aihua Guo led the discussion on the \href{https://datatracker.ietf.org/doc/html/draft-ygb-ivy-passive-network-inventory-00}{draft-ygb-ivy-passive-network-inventory-00}, which aims to standardize the representation of passive network components. The session highlighted the need for a generalized approach to encompass various passive devices, ensuring the model's flexibility and utility.

Meeting materials, including slides and notes, are available at \href{https://datatracker.ietf.org/meeting/122/session/ivy}{IVY Session Materials}.




\newpage

\section{JMAP/CALEXT Working Group (JMAP/CALEXT)}

\subsection{Attendees Overview}
The JMAP/CALEXT session at IETF122 in Bangkok was attended by 20 participants, including representatives from prominent organizations such as Fastmail, Apple Inc., and ICANN. Notable attendees included Jim Fenton from Altmode Networks, Bron Gondwana from Fastmail, and Kenneth Murchison from Fastmail.

\subsection{Meeting Discussions}

\subsubsection{JMAP WG Overview}
The JMAP Working Group focused on several key areas, including the status of documents with the IESG and ongoing work on various drafts. The \href{https://datatracker.ietf.org/doc/html/draft-ietf-jmap-webpush-vapid}{webpush-vapid} draft has been published, while the \href{https://datatracker.ietf.org/doc/html/draft-ietf-jmap-calendars}{calendars} draft is with the RFC editor. Discussions highlighted concerns about notification handling and the integration of iMIP with email and calendar systems. Action items included further exploration of JMAP Calendars and spam handling.

\subsubsection{Work in Progress}
The session covered ongoing work on drafts such as \href{https://datatracker.ietf.org/doc/html/draft-ietf-jmap-filenode}{filenode}, \href{https://datatracker.ietf.org/doc/html/draft-ietf-jmap-essential}{essential}, and \href{https://datatracker.ietf.org/doc/html/draft-ietf-jmap-portability}{portability}. Discussions emphasized the need for reference implementations and collaboration with external systems. The group also addressed the potential for new editors to contribute to the progress of these drafts.

\subsubsection{CALEXT WG Overview}
The CALEXT Working Group discussed several drafts, including \href{https://datatracker.ietf.org/doc/html/draft-ietf-calext-jscalendar-icalendar-10}{JSCalendar}, \href{https://datatracker.ietf.org/doc/html/draft-ietf-calext-ical-tasks-12}{iTip using PARTICIPANT only}, and \href{https://datatracker.ietf.org/doc/html/draft-ietf-calext-vpoll-07}{VPOLL}. The focus was on implementation status and interop testing, with actions assigned to ensure progress.

\subsubsection{Future Directions}
The meeting concluded with discussions on potential shifts in strategy, including the adoption of \href{https://datatracker.ietf.org/doc/html/draft-ietf-calext-icalendar-series-03}{Support for Series in iCalendar} and the exploration of jscontact profiles. The group considered renaming the working group to better reflect its scope, emphasizing the integration of calendar and contact extensions.

Meeting materials are available at \href{https://datatracker.ietf.org/meeting/122/session/jmap}{JMAP/CALEXT Meeting Materials}.




\newpage

\section{JOSE Working Group (JOSE)} 

\subsection{Attendees}

The JOSE Working Group meeting at IETF 122 was attended by 24 participants, including representatives from prominent organizations such as NSA-CCSS, Ericsson, Yubico, Google LLC, Microsoft, and Cisco Systems. Notable attendees included Michael Jenkins from NSA-CCSS, John Preuß Mattsson from Ericsson, and John Bradley from Yubico.

\subsection{Meeting Discussions}

\subsubsection{WG Documents @ IESG}

The discussion on \href{https://datatracker.ietf.org/doc/html/draft-ietf-jose-fully-specified-algorithms}{draft-ietf-jose-fully-specified-algorithms} highlighted the completion of addressing comments from the Security AD and the removal of the appendix on fully-specified ECDH. The draft now clarifies that HSS/LMS is not fully specified. A telechat date is being requested, marking progress towards finalization.

\subsubsection{Use of HPKE with JOSE}

The \href{https://datatracker.ietf.org/doc/html/draft-ietf-jose-hpke-encrypt}{draft-ietf-jose-hpke-encrypt} was presented by Tiru Reddy, focusing on updates since IETF 121. Discussions centered around the creation of a header parameter for HPKE Setup info value and the suitability of using "enc":"dir". The group debated the need for updating base specifications and considered forming a new HPKE working group to streamline efforts.

\subsubsection{JSON Web Proof Drafts}

Mike Jones presented on the \href{https://datatracker.ietf.org/doc/html/draft-ietf-jose-json-web-proof}{JSON Web Proofs} and related drafts. Key points included defining CBOR representation for JSON Proof Token (JPT) and addressing cryptographic review comments. Open issues such as the potential abandonment of JSON serialization in favor of JWE compact serialization were discussed, with feedback sought from the working group.

\subsubsection{Deprecation of 'none' and 'RSA1\_5'}

The draft \href{https://datatracker.ietf.org/doc/html/draft-ietf-jose-deprecate-none-rsa15}{draft-ietf-jose-deprecate-none-rsa15} was briefly discussed, with consensus leaning towards deprecation. The chairs called for additional reviews to ensure readiness for WGLC, with volunteers stepping forward to contribute.

\subsubsection{Enhanced JWE Security with Detached AAD}

Tiru Reddy introduced the \href{https://datatracker.ietf.org/doc/html/draft-reddy-jose-detached-aad}{draft-reddy-jose-detached-aad}, proposing a new security mechanism involving detached Additional Authenticated Data (AAD). While the concept was intriguing, concerns were raised about its necessity and potential impact, prompting a cautious approach to further development.

\subsubsection{Other Presentations}

Filip Skokan remotely presented on modern algorithms in the Web Cryptography API, emphasizing the integration of PQ algorithms and the use of JWKs. The presentation underscored the need for feedback on draft specifications to ensure alignment with evolving cryptographic standards.

\subsection{Meeting Materials}

Meeting notes and materials are available at \href{https://notes.ietf.org/notes-ietf-122-jose}{notes.ietf.org/notes-ietf-122-jose}.



\newpage

\section{Key Transparency Working Group (KEYTRANS)}

\subsection{Attendees Overview}
\subsubsection{Participants}
The meeting was attended by 22 individuals, including representatives from notable organizations such as Graz University of Technology, Beyond Identity, AKAYLA, Okta, Transforming Information Security LLC, DHS/CISA, GMO Cybersecurity by Ierae, Inc., ETH Zurich, CISPA Helmholtz Center for Information Security, Cloudflare, and Cisco.

\subsection{Meeting Discussions}

\subsubsection{Changes to the Protocol Document}
Brendan presented significant updates to the protocol document, emphasizing the introduction of a clock label to address the absence of time in the initial draft. This change aims to enhance transparency by mapping tree size to time, thus preventing potential split view attacks. The updated protocol now includes timestamps in log entries, enabling features like maximum lifetime and distinguished log entries for efficient monitoring. The discussion highlighted the need for out-of-band communication to detect discrepancies in timestamps, ensuring clients can identify forked views. For further details, refer to the \href{https://datatracker.ietf.org/doc/html/draft-ietf-keytrans-protocol}{draft-ietf-keytrans-protocol}.

\subsubsection{Self-Balancing Prefix Tree}
The session explored the implications of using a self-balancing prefix tree, which stores key-value pairs and addresses the inefficiencies of using a VRF for non-privacy-sensitive applications. Brendan explained the structural changes that allow users to navigate the tree more efficiently, despite potential complexities. The discussion also touched on the applicability of these changes to DNS domains and the need for a comprehensive list of use cases.

\subsubsection{Formal Analysis of KEYTRANS}
Jonathan Hoyland and Felix Linker presented their formal analysis, conducted in collaboration with Cory Myers and Linard Arquint. The analysis focused on ensuring client agreement on log views, independent of server communication. The team utilized hyper-properties to validate the protocol's commitment scheme and prefix tree. The presentation underscored the challenge of maintaining the document's readability while enhancing its precision. The extensibility of the protocol, particularly for witness configuration, was also discussed. The analysis has been instrumental in refining the draft, as acknowledged by Brendan.

\subsubsection{Next Steps}
The working group plans to incorporate feedback from the formal analysis to further refine the protocol document. Future efforts will focus on enhancing the document's clarity and addressing any remaining protocol issues. Meeting materials can be accessed \href{https://datatracker.ietf.org/meeting/122/materials/slides-122-keytrans-updated-draft-ietf-keytrans-protocol-00}{here} for the protocol changes and \href{https://datatracker.ietf.org/meeting/122/materials/slides-122-keytrans-formal-analysis-and-comments-on-draft-00}{here} for the formal analysis presentation.



\newpage

\section{Lightweight Authenticated Key Exchange (LAKE)}

\subsection{Attendees}

The LAKE working group meeting was attended by 51 participants, including representatives from prominent organizations such as Ericsson, Inria, and Nokia. The diverse attendance underscored the broad interest in advancing secure communication protocols.

\subsection{Meeting Discussions}

\subsubsection{Administrivia}

The session commenced with an overview by the chairs, Mališa Vučinić and Renzo Navas, highlighting the adoption of four new documents since the previous meeting. The chairs noted that the milestones would be updated to reflect these developments, with one being deemed obsolete and thus removed. Meeting materials are available \href{https://datatracker.ietf.org/meeting/122/materials/slides-122-lake-chairs-slides-00}{here}.

\subsubsection{Draft-ietf-lake-authz}

Geovane Fedrecheski presented updates on the \href{https://datatracker.ietf.org/doc/html/draft-ietf-lake-authz}{draft-ietf-lake-authz}, focusing on editorial improvements and the introduction of a reverse flow mechanism to accommodate heterogeneous networks. Discussions during the Hackathon revealed issues with stateless operations, prompting a need for explicit inclusion of EDHOC message\_1. The draft aims to resolve open issues towards a final version.

\subsubsection{Draft-ietf-lake-app-profiles}

Marco Tiloca discussed the \href{https://datatracker.ietf.org/doc/html/draft-ietf-lake-app-profiles}{draft-ietf-lake-app-profiles}, which facilitates configuration agreement through application profiles. Recent updates include editorial fixes and technical changes such as the introduction of new venues for advertising EDHOC capabilities. The draft is progressing towards adding examples and refining security considerations.

\subsubsection{Draft-ietf-lake-edhoc-impl-cons}

Tiloca also covered the \href{https://datatracker.ietf.org/doc/html/draft-ietf-lake-edhoc-impl-cons}{draft-ietf-lake-edhoc-impl-cons}, which addresses implementation considerations for EDHOC. The document has been updated for consistency and alignment with related drafts, with future steps including the addition of security considerations and example certificates.

\subsubsection{Draft-ietf-lake-edhoc-grease}

Christian Amsüss presented on the \href{https://datatracker.ietf.org/doc/html/draft-ietf-lake-edhoc-grease}{draft-ietf-lake-edhoc-grease}, which explores the use of GREASE techniques to ensure the availability of codepoints. The discussion centered on the feasibility of greasing COSE headers and the need for further interoperation testing.

\subsubsection{Draft-ietf-lake-ra}

Yuxuan Song introduced the \href{https://datatracker.ietf.org/doc/html/draft-ietf-lake-ra}{draft-ietf-lake-ra}, focusing on remote attestation over EDHOC. The draft has incorporated feedback and added clarifications on evidence types and continuous attestation. Future work will address error handling and the integration of attestation, authentication, and authorization.

\subsubsection{Draft-ietf-lake-edhoc-psk}

Elsa López Pérez discussed the \href{https://datatracker.ietf.org/doc/html/draft-ietf-lake-edhoc-psk}{draft-ietf-lake-edhoc-psk}, which involves PSK authentication for EDHOC. Key updates include structural changes to prevent reflection attacks and the introduction of session resumption mechanisms. The draft is moving towards formal analysis and further refinement.

\subsubsection{Quantum Resistance Discussion}

John Mattsson led a discussion on quantum resistance, emphasizing the need for EDHOC to adapt to post-quantum cryptography challenges. The conversation highlighted the absence of standardized quantum-resistant key exchange algorithms and the potential for integrating symmetric PSKs. The group agreed to continue discussions while awaiting further developments in COSE.

\subsubsection{EDHOC-Bundle Protocol DTN WG Discussion}

Brian Sipos presented on embedding EDHOC within the Bundle Protocol for delay-tolerant networks. The approach aims to leverage EDHOC for initial authentication, enhancing security without reinventing existing protocols. The group plans to draft a proposal and seek further feedback.

\subsubsection{AOB}

The session concluded with a proposal for an interim meeting before IETF 123 in Madrid, and an update on IEEE's efforts to incorporate EDHOC into 802.15.9. Participants were encouraged to engage with ongoing discussions and contribute to the evolving drafts.



\newpage

\section{LAMPS Working Group (LAMPS)}

\subsection{Attendees}
\subsubsection{Overview}
The LAMPS Working Group meeting at IETF 122 was attended by 62 participants, representing a diverse array of prominent organizations including Cisco Systems, NSA - CCSS, Nokia, Red Hat, and Huawei. The gathering facilitated a robust exchange of ideas and technical insights among experts from these leading institutions.

\subsection{Meeting Discussions}

\subsubsection{Recently Published RFCs}
The group reviewed several recently published RFCs, including \href{https://datatracker.ietf.org/doc/html/draft-ietf-lamps-rfc5990bis}{draft-ietf-lamps-rfc5990bis} (RFC 9690) and \href{https://datatracker.ietf.org/doc/html/draft-ietf-lamps-cms-sha3-hash}{draft-ietf-lamps-cms-sha3-hash} (RFC 9688). These documents reflect advancements in cryptographic message syntax and hash functions, underscoring the group's ongoing commitment to enhancing security protocols.

\subsubsection{RFC Editor Updates}
Discussions included updates on drafts such as \href{https://datatracker.ietf.org/doc/html/draft-ietf-lamps-e2e-mail-guidance}{draft-ietf-lamps-e2e-mail-guidance} and \href{https://datatracker.ietf.org/doc/html/draft-ietf-lamps-rfc5019bis}{draft-ietf-lamps-rfc5019bis}. The focus was on refining end-to-end email security guidance and certificate management protocols, with an emphasis on aligning with emerging security standards.

\subsubsection{Active PKIX-related Documents}
The group explored active PKIX-related drafts, notably \href{https://datatracker.ietf.org/doc/html/draft-ietf-lamps-dilithium-certificates}{draft-ietf-lamps-dilithium-certificates} and \href{https://datatracker.ietf.org/doc/html/draft-ietf-lamps-kyber-certificates}{draft-ietf-lamps-kyber-certificates}. Discussions centered on the integration of post-quantum cryptography into existing frameworks, addressing challenges such as key derivation and algorithm selection. The dialogue highlighted the need for flexibility in security level alignment and the potential impact on future cryptographic standards.

\subsubsection{Active S/MIME-related Documents}
Key discussions on S/MIME-related drafts, including \href{https://datatracker.ietf.org/doc/html/draft-ietf-lamps-cms-kyber}{draft-ietf-lamps-cms-kyber}, focused on the adoption of post-quantum algorithms in secure email communications. The group debated the necessity of aligning security levels across different cryptographic methods, with a consensus on the importance of maintaining robust security postures in light of evolving threats.

\subsubsection{Under Consideration for Adoption}
The meeting also considered new drafts for adoption, such as \href{https://datatracker.ietf.org/doc/html/draft-birglee-lamps-caa-security}{draft-birglee-lamps-caa-security}, which addresses domain control validation vulnerabilities. The proposal aims to enhance security by leveraging DNSSEC and introducing new CAA security tags, potentially influencing future CA/B forum standards.

Meeting materials are available at \href{https://www.ietf.org/proceedings/122/lamps.html}{LAMPS WG Meeting Materials}.




\newpage

\section{Locator/ID Separation Protocol (LISP) Working Group}

\subsection{Attendees}
\subsubsection{Overview}
The meeting was attended by 17 participants, including representatives from prominent organizations such as Amazon, Huawei Technologies, Cisco, and the China Academy of Information and Communications Technology (CAICT). Notable attendees included Padma Pillay-Esnault from Amazon, Luigi Iannone from Huawei Technologies France, and Stig Venaas from Cisco.

\subsection{Meeting Discussions}

\subsubsection{Experiences with LISP Multicast Deployments}
Vengada Prasad Govindan presented on the deployment experiences of LISP Multicast, referencing the \href{https://datatracker.ietf.org/doc/html/draft-vgovindan-lisp-multicast-deploy}{draft-vgovindan-lisp-multicast-deploy}. The discussion centered on the appropriate working group for this document, with consensus leaning towards the LISP WG, while acknowledging potential contributions from PIM and MBONED WGs.

\subsubsection{LISP DDT 8111bis}
Damien Saucez discussed the \href{https://datatracker.ietf.org/doc/html/draft-saucez-8111bis}{draft-saucez-8111bis}, focusing on the absence of priority and weight in Map-Referral messages. Luigi Iannone clarified that these are not used due to the unicast nature of DDT. The need for a description of error handling was highlighted, and WG adoption was supported, pending further discussion on the mailing list.

\subsubsection{LISP NAT Traversal}
The \href{https://datatracker.ietf.org/doc/html/draft-ermagan-lisp-nat-traversal}{draft-ermagan-lisp-nat-traversal} was presented by Damien Saucez, with discussions on backward compatibility of the proposed changes. It was agreed that documenting compatibility issues is necessary. The draft received support for WG adoption, with further deliberations to occur on the mailing list.

\subsubsection{LISP LCAF (8060bis)}
The chairs briefly introduced the \href{https://datatracker.ietf.org/doc/html/draft-retana-lisp-rfc8060bis}{draft-retana-lisp-rfc8060bis}, noting that due to time constraints, the prepared slides by Alvaro Retana were not presented. A work plan and call for adoption will be initiated on the mailing list.

Meeting materials and further details can be accessed via the \href{https://datatracker.ietf.org/meeting/122/materials/agenda-122-lisp-00}{meeting agenda}.



\newpage

\section{Link State Routing (LSR) Working Group}

\subsection{Attendees Overview}
The LSR Working Group session was attended by 66 participants, including representatives from prominent companies and institutions such as Juniper Networks, Cisco Systems, Huawei Technologies, Nokia, and Deutsche Telekom AG. The diverse attendance highlighted the broad industry interest and collaboration in advancing link state routing technologies.

\subsection{Meeting Discussions}

\subsubsection{Flooding Reduction Algorithms Framework}
Shraddha Hegde presented the \href{https://datatracker.ietf.org/doc/draft-prz-lsr-interop-flood-reduction-architecture/}{Flooding Reduction Algorithms Framework}, focusing on the separation of connected components based on algorithm and version differences. The discussion emphasized the need for clear differentiation between versions and algorithms to ensure compatibility and operational efficiency. Tony Li and Tony Przygienda highlighted concerns about version compatibility and suggested treating different versions as separate algorithms to avoid operational complexities.

\subsubsection{Optimized Flooding Leader/Leaderless Discussion}
Les Ginsberg led a discussion on the merits of leader-based versus leaderless flooding algorithms. The consensus leaned towards exploring leaderless approaches due to operational flexibility and reduced complexity. Tony Li and other participants stressed the importance of incremental algorithm transitions to minimize operational risks, with a focus on testing and deployment strategies that accommodate real-world network dynamics.

\subsubsection{IGP Reverse Prefix Metric}
Changwang Lin introduced the \href{https://datatracker.ietf.org/doc/draft-li-lsr-igp-reverse-prefix-metric/}{IGP Reverse Prefix Metric}, which aims to enhance reverse path calculations outside of Flex Algo environments. The discussion with Peter Psenak and Libin Liu explored the applicability of this metric in various network scenarios, including strict reverse path forwarding and traffic engineering.

\subsubsection{IS-IS Extensions for Load Balancing Alternates}
Jie Dong presented the \href{https://datatracker.ietf.org/doc/html/draft-dong-lsr-load-balancing-alternate-00}{IS-IS Extensions for Load Balancing Alternates}, addressing the need for dynamic load balancing to mitigate congestion. Tony Li raised concerns about frequent metric updates and their impact on network stability, suggesting alternative approaches such as tactical traffic engineering.

\subsubsection{LSR YANG Models Update}
Yingzhen Qu provided updates on LSR YANG models, discussing potential splits in the OSPF Augmentation YANG model to enhance modularity and manageability. Bruno Decraene and Acee Lindem highlighted ongoing efforts to align YANG model developments with emerging network requirements.

\subsubsection{Flexible Algorithms for Energy Efficiency}
Jinming Li discussed the \href{https://datatracker.ietf.org/doc/draft-li-lsr-flex-algo-energy-efficiency/}{Flexible Algorithms for Energy Efficiency}, proposing the use of Flex Algo to optimize energy consumption in networks. The session explored the balance between static and dynamic metrics, with a call for more detailed use cases to guide future developments.

Meeting materials are available at \href{https://datatracker.ietf.org/meeting/122/session/lsr}{IETF 122 LSR Session Materials}.




\newpage

\section{Link State Vector Routing (LSVR) Working Group}

\subsection{Attendees Overview}
The LSVR meeting at IETF 122 was attended by 19 participants, including representatives from prominent companies and institutions such as ZTE Corporation, LabN Consulting, Huawei, Cisco, Arrcus, CZ.NIC, Ericsson, Futurewei Technologies, Charter Communications, Deutsche Telekom, Vigil Security, and IIJ Research. The diverse attendance underscored the broad interest and collaborative effort in advancing LSVR technologies.

\subsection{Meeting Discussions}

\subsubsection{Agenda Bashing and Chairs' Slides}
Chairs Acee Lindem and Jie Dong initiated the session by discussing the recent rechartering of the working group, which now includes a Layer-3 discovery and liveness monitoring protocol. The BGP-SPF base and applicability documents have been forwarded to the RFC Editor Queue, marking significant progress in the group's deliverables.

\subsubsection{Discussion about the Liaison from IEEE 802.1}
The session addressed the liaison from IEEE 802.1, emphasizing the need for collaborative discussions between IETF and IEEE. Janos Farkas highlighted the potential for a joint meeting, suggesting the use of IEEE Webex for interim discussions. The group considered extending LLDP for LSVR needs, referencing \href{https://datatracker.ietf.org/doc/html/draft-congdon-lsvr-lldp-tlvs}{draft-congdon-lsvr-lldp-tlvs}. The dialogue underscored the importance of backward compatibility and the need for further meetings to align on protocol requirements.

\subsubsection{Layer-3 Discovery and Liveness}
Randy Bush presented the \href{https://datatracker.ietf.org/doc/html/draft-ietf-lsvr-l3dl-14}{draft-ietf-lsvr-l3dl-14}, focusing on the protocol's ability to announce overlay prefixes, simplifying MAC move scenarios. The discussion explored the distinction between overlay and underlay services, with suggestions to enhance the draft's explanations. The group agreed on the necessity of further updates and concurrent liaison discussions.

\subsubsection{Applying BGP-LS Segment Routing over IPv6 (SRv6) Extensions to BGP-LS-SPF}
Li Zhang introduced \href{https://datatracker.ietf.org/doc/html/draft-li-lsvr-bgp-spf-srv6-01}{draft-li-lsvr-bgp-spf-srv6-01}, proposing the integration of SRv6 extensions into BGP-SPF. The conversation revolved around the reuse of existing TLVs and the potential for flex-algo support. Participants emphasized the need for detailed guidance on TLV usage within BGP-SPF, advocating for a more comprehensive draft before proceeding with an adoption call.

Meeting materials are available at \href{https://datatracker.ietf.org/meeting/122/session/lsvr}{IETF 122 LSVR Session Materials}.



\newpage

\section{MAILMAINT Working Group (MAILMAINT)}

\subsection{Attendees}
\subsubsection{Overview}
The MAILMAINT working group meeting was attended by 33 participants, including representatives from prominent companies and institutions such as Fastmail, Apple Inc., ICANN, Meta Platforms, Inc., and Google. The diverse attendance underscored the broad interest in the group's work on email maintenance and interoperability.

\subsection{Meeting Discussions}

\subsubsection{Active Drafts}
The meeting began with discussions on several active drafts. Ken Murchison presented on the \href{https://datatracker.ietf.org/doc/html/draft-ietf-mailmaint-wrong-recipient}{Wrong Recipient URL} and \href{https://datatracker.ietf.org/doc/html/draft-ietf-mailmaint-messageflag-mailboxattribute}{IMAP/JMAP Keywords}. The group considered enhancing security considerations with privacy warnings, as suggested by Lisa Dusseault, to prevent phishing risks. The \href{https://datatracker.ietf.org/doc/html/draft-ietf-mailmaint-imap-uidbatches}{IMAP UIDBATCHES Extension} was also discussed, with Daniel Eggert and Ken Murchison noting its readiness for Working Group Last Call (WGLC), supported by implementations at Fastmail and Apple.

John Levine's presentation on the \href{https://datatracker.ietf.org/doc/html/draft-ietf-mailmaint-expires}{Expires Message Header Field} sparked debate about its implications for email deletion and user consent. The consensus was to proceed with standardization, emphasizing user control over message handling.

\subsubsection{Proposed Work}
Simon Gougeon introduced the \href{https://datatracker.ietf.org/doc/html/draft-gougeon-imap-webpush}{IMAP WEBPUSH Extension}, which aims to enhance push notification efficiency. The group recognized its potential to improve battery life by filtering unwanted notifications. Hans-Joerg Happel's \href{https://datatracker.ietf.org/doc/html/draft-happel-mailmaint-pdparchive}{Personal Data Portability Archive} proposal was also well-received, highlighting its utility in data migration scenarios.

\subsubsection{Topics of Interest}
Alex Brotman and John Levine discussed \href{https://datatracker.ietf.org/doc/html/draft-brotman-dkim-fbl}{Email Feedback Reports for DKIM Signers}, focusing on balancing transparency with security. The group agreed on the importance of feedback mechanisms for improving email authentication.

\subsubsection{OAuth Profile for Open Public Clients}
Neil Jenkins presented the \href{https://datatracker.ietf.org/doc/html/draft-ietf-mailmaint-oauth-public}{OAuth Profile for Open Public Clients}, addressing concerns about OAuth's complexity and the need for alternative authentication mechanisms. The group acknowledged the necessity of OAuth for current use cases while remaining open to future innovations.

\subsubsection{Mail Autoconfig}
Ben Bucksch's \href{https://datatracker.ietf.org/doc/html/draft-ietf-mailmaint-autoconfig}{Mail Autoconfig} draft was discussed, with emphasis on simplifying user setup processes. The group debated the balance between ease of use and accommodating diverse email configurations.

\subsection{Meeting Materials}
The complete meeting materials are available at \href{https://www.ietf.org/proceedings/122/mailmaint.html}{IETF 122 MAILMAINT Materials}.

\subsection{Outcomes and Next Steps}
The meeting concluded with several key actions: John Levine will incorporate additional considerations into the Wrong Recipient draft, and Alexey Melnikov will implement UIDBATCHES for WGLC. The group will further explore interest in the WEBPUSH extension and data portability archive, while also considering experimental drafts for feedback mechanisms. These steps are anticipated to significantly advance the group's contributions to email technology and interoperability.



\newpage

\section{Mobile Ad-hoc Networks (MANET)}

\subsection{Attendees Overview}
\subsubsection{Participants}
The MANET working group meeting was attended by 26 participants, including representatives from prominent organizations such as TNO, LabN Consulting LLC, SECOM CO., LTD., Aalyria, City, University of London, BIRD | CZ.NIC, The Boeing Company, Ericsson, NICT, and Cisco. The diverse attendance highlighted the broad interest and expertise in the field of mobile ad-hoc networks.

\subsection{Meeting Discussions}

\subsubsection{NS-3 Model of AODVv2, Charlie Perkins}
Charlie Perkins presented the NS-3 model of AODVv2, emphasizing its potential to address the lack of implementations for AODVv2, which has been a barrier to its adoption as a Proposed Standard. The presentation highlighted the ongoing work by teams in Italy and India, with a focus on comparing AODVv1 and AODVv2. The discussion underscored the importance of AODVv2 in both research and practical applications, such as military operations, and the need for further community feedback to advance this work. More details can be found in the \href{https://datatracker.ietf.org/doc/draft-perkins-manet-aodvv2/}{draft-perkins-manet-aodvv2}.

\subsubsection{MANET Internetworking, Fred Templin}
Fred Templin discussed MANET internetworking using AERO/OMNI technologies, which enable MANET nodes to communicate with Internet nodes and use the Internet as a transit network. The presentation detailed the use of IPv6 encapsulation and the OMNI link overlay to provide globally unique addresses for MANET nodes. The discussion raised questions about the integration of MANETs with existing Internet infrastructure and the potential limitations of the current approach. The drafts \href{https://datatracker.ietf.org/doc/draft-templin-6man-aero3/}{draft-templin-6man-aero3} and \href{https://datatracker.ietf.org/doc/draft-templin-6man-omni3/}{draft-templin-6man-omni3} provide further insights.

\subsubsection{New MANET Charter Discussions}
The chairs led a discussion on the new MANET charter, focusing on potential additions related to DLEP management and router state interfaces. Concerns were raised about the appropriateness of these topics within the MANET working group, suggesting possible collaboration with other IETF areas. The discussion also touched on the inclusion of address autoconfiguration work, which may require coordination with the Internet Area. The proposed charter text is available for review \href{https://datatracker.ietf.org/meeting/122/materials/slides-122-manet-draft-charter-text-ietf-122-01}{here}.

Meeting materials and slides are available at \href{https://datatracker.ietf.org/meeting/122/materials/slides-122-manet-chair-slides-00}{MANET IETF 122 Meeting Materials}.

\subsubsection{Wrap-up and Next Steps}
The meeting concluded with a call for increased activity on the mailing list to ensure that important topics receive the necessary attention and feedback. The chairs emphasized the need for community engagement to drive the working group's efforts forward. The next meeting will be held in Madrid, where further discussions on the charter and ongoing projects are anticipated.



\newpage

\section{Measurement and Analysis for Protocol Research Group (MAPRG)}

\subsection{Attendees}
\subsubsection{Overview}
The session was attended by 96 participants, including representatives from prominent organizations such as New York University, KAUST, Cisco, Broadcom, Apple, and Deutsche Telekom. The diverse attendance highlighted the broad interest and collaborative efforts in advancing protocol research.

\subsection{Meeting Discussions}

\subsubsection{Overview and Status}
The session began with an overview by Mirja and Dave, who provided updates on the ongoing Hackathon IPv6 Test Pod Open Testing and the call for HAPPY WG measurements. This initiative aims to enhance IPv6 testing methodologies and foster collaboration with the HAPPY Working Group.

\subsubsection{Measuring Collateral Damage in RPKI ROV}
Weitong Li presented a study on the collateral damage in RPKI Route Origin Validation (ROV), revealing that 85.6\% of RPKI-invalid announcements are vulnerable to such damage. The introduction of ImpROV, a mitigation tool, was discussed, which promises to reduce hijack success ratios with minimal overhead. The presentation sparked a discussion on the interplay between RPKI and BGP path selection, emphasizing the need for comprehensive validation strategies. For more details, refer to the \href{https://datatracker.ietf.org/doc/html/draft-ietf-maprg-rpki-rov}{draft-ietf-maprg-rpki-rov}.

\subsubsection{L4S-Compatible Congestion Control}
Fatih Berkay Sarpkaya explored the performance of L4S-compatible congestion control in partial deployments. The discussion highlighted challenges in coexistence with traditional congestion controls and the potential benefits of adopting L4S protocols. Recommendations for network providers were debated, focusing on safe deployment strategies. The session underscored the importance of understanding network dynamics in mixed environments. More information can be found in the \href{https://datatracker.ietf.org/doc/html/draft-ietf-maprg-l4s-congestion-control}{draft-ietf-maprg-l4s-congestion-control}.

\subsubsection{LEO Satellite Topology Design}
Wenyi Zhang's presentation on LEO satellite networks examined the impact of design parameters on network performance. The analysis revealed that certain configurations could significantly affect latency, prompting discussions on optimizing satellite orbits and inclinations. The insights gained are expected to inform future satellite network designs, enhancing global connectivity.

\subsubsection{Anycast Flipping: Prevalence and Impact}
Xiao Zhang addressed the increasing prevalence of anycast flipping and its impact on web performance. The study's findings indicated a notable increase in flipping incidents, affecting latency and user experience. The session concluded with suggestions for mitigating these effects, including potential labeling of anycast routes.

\subsubsection{Quality of Outcome Scores in Challenging Networks}
Bjørn Ivar Teigen Monclair presented a simulation study on Quality of Outcome (QoO) scores under adverse network conditions. The research emphasized the importance of accurate measurements and appropriate sampling rates to ensure reliable QoO assessments. The discussion highlighted the potential for QoO metrics to guide performance improvements in diverse network environments.

\subsubsection{HTTP Conformance vs. Middleboxes}
Mahmoud Attia's study on HTTP conformance revealed significant deviations from standards due to middlebox interference. The session explored the implications of these findings for protocol compliance and security, advocating for clearer guidelines in RFCs to address middlebox behaviors.

Meeting materials and further details are available at \href{https://meetecho-sin.ietf.org/client/?group=maprg}{Meetecho Client}.




\newpage

\section{MASQUE Working Group (MASQUE)}

\subsection{Attendees Overview}
The MASQUE Working Group session at IETF 122 was attended by 72 participants, including representatives from prominent companies and institutions such as Google, Apple, Cisco, Meta Platforms, Inc., and Ericsson. The diverse attendance underscored the broad interest and collaborative effort in advancing MASQUE-related technologies.

\subsection{Meeting Discussions}

\subsubsection{Proxying Listener UDP in HTTP}
The session began with a discussion on the \href{https://datatracker.ietf.org/doc/html/draft-ietf-masque-connect-udp-listen}{draft-ietf-masque-connect-udp-listen}, led by Abhijit Singh. The group explored the status of implementations, with Christian Huitema from Private Octopus, Inc. indicating progress on his implementation. Eric Kinnear suggested initiating a Working Group Last Call (WGLC) once interoperability is confirmed among multiple implementations.

\subsubsection{QUIC-Aware Proxying Using HTTP}
Eric Rosenberg presented the \href{https://datatracker.ietf.org/doc/html/draft-ietf-masque-quic-proxy}{draft-ietf-masque-quic-proxy}, sparking a debate on server behavior and the necessity of SCONE. The discussion highlighted the need for further interoperability testing, with plans to address remaining issues before the next IETF meeting in Madrid. Participants agreed on the importance of filing issues on GitHub to refine the draft.

\subsubsection{Proxying Ethernet in HTTP}
Alejandro Sedeño's presentation on \href{https://datatracker.ietf.org/doc/html/draft-ietf-masque-connect-ethernet}{draft-ietf-masque-connect-ethernet} focused on seeking interoperability feedback. The group discussed the potential for a WGLC, with a consensus to involve IEEE for a comprehensive review. Craig Taylor raised concerns about MTU, which will be addressed in subsequent discussions.

\subsubsection{DNS Configuration for Proxying IP in HTTP}
Yaroslav Rosomakho led the discussion on \href{https://datatracker.ietf.org/doc/html/draft-ietf-masque-connect-ip-dns}{draft-ietf-masque-connect-ip-dns}, where Ben Schwartz questioned the necessity of DNS request capsules. The dialogue emphasized the need for clarity on DNS configurations, with plans to file issues for further exploration. The group recognized the significance of this draft for site-to-site connectivity.

\subsubsection{The MASQUE Proxy}
David Schinazi presented the \href{https://datatracker.ietf.org/doc/html/draft-schinazi-masque-proxy}{draft-schinazi-masque-proxy}, prompting discussions on privacy and the broader implications of MASQUE. The group considered the potential for collaboration with the Privacy Enhancements and Assessments Research Group (PEARG) and the need for a document clarifying MASQUE's scope and principles. The session concluded with a poll indicating strong interest in further discussions on MASQUE's future direction.

Meeting materials, including notes and minutes, are available at \href{https://notes.ietf.org/notes-ietf-122-masque}{Notes} and \href{https://datatracker.ietf.org/doc/minutes-122-masque/}{Minutes}.



\newpage

\section{Media Type Maintenance (MediaMan)}

\subsection{Attendees}
\subsubsection{Overview}
The meeting was attended by 16 participants, including representatives from prominent organizations such as Google, Huawei Technologies Co., Ltd., Ericsson, and Meta Platforms, Inc. Notable attendees included Harald Alvestrand from Google, Magnus Westerlund from Ericsson, and Murray Kucherawy from Meta Platforms, Inc.

\subsection{Meeting Discussions}

\subsubsection{Introduction and Agenda Review}
The session commenced with a brief introduction and agenda review. An additional item on more extensive negotiation mechanisms was added to the agenda.

\subsubsection{Status of WG Documents}
The working group discussed the current status of their documents, focusing on the \href{https://datatracker.ietf.org/doc/html/draft-ietf-mediaman-6838bis}{draft-ietf-mediaman-6838bis}. Key updates included the merging of pull requests related to standards-tree and toplevel issues. Discussions highlighted the need for a more streamlined process for early allocation of media types, referencing RFC 7120, and the potential involvement of the \href{https://datatracker.ietf.org/wg/ianabis/about/}{ianabis WG} for broader procedural updates.

\subsubsection{RTP Related Information in the Registry}
Magnus Westerlund proposed the inclusion of RTP-related information in the registration template. The group debated the necessity and format of such an inclusion, considering the historical context and potential implications for other top-level types. The discussion concluded with a decision to further explore the idea and develop a concrete proposal.

\subsubsection{Follow-up on Registration Templates}
Rahul Gupta raised concerns about the completeness and accuracy of existing registration templates. The group agreed on the need for a draft to standardize template values and discussed the potential for updates to be reflected in the \href{https://datatracker.ietf.org/doc/html/draft-ietf-mediaman-6838bis}{draft-ietf-mediaman-6838bis}. Alexey Melnikov volunteered to lead this effort, with support from additional volunteers.

\subsubsection{More Extensive Negotiation Mechanisms}
The discussion on negotiation mechanisms centered around the use of suffixes for media types. Rahul Gupta suggested exploring alternatives to structured suffixes, referencing RFC 2912 for potential arithmetic approaches. The group acknowledged the complexity of suffix usage and agreed to continue the conversation on the mailing list.

\subsubsection{Wrap Up and Action Items}
The meeting concluded with a summary of action items, including further discussions on the mailing list and coordination with the ianabis WG for procedural improvements.

Meeting materials are available at \href{https://example.com/meeting-materials}{Meeting Materials}.



\newpage

\section{More Instant Messaging Interoperability (MIMI)}

\subsection{Attendees}
The MIMI working group meeting was attended by 36 participants, including representatives from prominent organizations such as Cisco, Ericsson, Nokia, and the ACLU. Notable attendees included Richard Barnes from Cisco, Alissa Cooper from Knight-Georgetown Institute, and Rohan Mahy from Rohan Mahy Consulting Services.

\subsection{Meeting Discussions}

\subsubsection{MIMI Protocol Overview}
Rohan Mahy and Richard Barnes presented the \href{https://www.ietf.org/archive/id/draft-ietf-mimi-protocol-02.html}{MIMI Protocol}, focusing on its current state and future directions. Discussions highlighted the need for a streamlined approach to options and the introduction of an authentication mechanism to manage quotas. Concerns were raised about the complexity of recommendations, with a consensus to refine the protocol's clarity.

\subsubsection{Minimal Metadata in MIMI Protocol}
Raphael Robert discussed the \href{https://www.ietf.org/archive/id/draft-ietf-mimi-protocol-02.html}{Minimal Metadata} aspect, emphasizing the importance of balancing privacy with functionality. The dialogue underscored the necessity for a well-defined concept of "Connection" and the synchronization of decryption keys, pointing towards potential enhancements in the protocol's security framework.

\subsubsection{Content Format and Room Policy}
Rohan Mahy led the session on \href{https://www.ietf.org/archive/id/draft-ietf-mimi-content-04.html}{Content Format} and \href{https://www.ietf.org/archive/id/draft-mahy-mimi-room-policy-01.html}{Room Policy}, addressing recent updates and unresolved issues. The group agreed on the urgency of starting a Working Group Last Call and the need for further review of delivery reports. The discussions indicated a strategic shift towards more robust content handling and policy implementation.

\subsubsection{Identity and MLS Subgroups}
The \href{https://www.ietf.org/archive/id/draft-mahy-mimi-identity-03.html}{Identity} draft was critiqued for its current state as a survey rather than a solution, prompting a call for a reboot. Brendan McMillion's presentation on \href{https://datatracker.ietf.org/doc/draft-mcmillion-mls-subgroups/}{MLS Subgroups} sparked a debate on identity management and subgroup functionality, with a decision to further explore these issues within the MLS Working Group.

\subsubsection{Next Steps}
The meeting concluded with a consensus on the need for increased implementation efforts and collaboration with students or hackathon participants to accelerate progress. The group emphasized the importance of aligning MIMI's development with MLS standards to ensure interoperability and security.

Meeting materials are available at \href{https://datatracker.ietf.org/meeting/122/session/mimi}{IETF 122 MIMI Session}.



\newpage

\section{Machine Learning for Audio Coding (mlcodec)}

\subsection{Attendees Overview}
The meeting was attended by representatives from prominent organizations such as Xiph.Org, Cisco, Google, Huawei Technologies, Meta, and Ericsson, with a total attendance of 17 participants. Notable attendees included Timothy Terriberry from Xiph.Org, Jean-Marc Valin from Google, and Kyriakos Zarifis from Meta.

\subsection{Meeting Discussions}

\subsubsection{Opus Extension Mechanism}
Timothy Terriberry presented the \href{https://datatracker.ietf.org/doc/html/draft-ietf-mlcodec-opus-extension}{draft-ietf-mlcodec-opus-extension}, focusing on the complexities of extension numbering and the need for clarity in the documentation. The group discussed potential improvements in wording and the necessity of defining use cases for extensions. The consensus was to proceed with a Working Group Last Call (WGLC) after incorporating editorial changes, aiming for submission by mid-April.

\subsubsection{Deep REDundancy}
Jean-Marc Valin discussed the \href{https://datatracker.ietf.org/doc/html/draft-ietf-mlcodec-opus-dred}{draft-ietf-mlcodec-opus-dred}, highlighting the importance of documenting training procedures within the draft. The group debated the necessity of including binary weights and test vectors as normative parts of the specification. The decision was to ensure both are included, with further experiments on quantization effects planned before the next meeting.

\subsubsection{Speech Coding Enhancements}
The \href{https://datatracker.ietf.org/doc/html/draft-ietf-mlcodec-opus-speech-coding-enhancement}{draft-ietf-mlcodec-opus-speech-coding-enhancement} was briefly discussed. Although no presentation was made due to the author's absence, it was noted that ongoing work is expected to be included in the next draft iteration.

\subsubsection{Scalable Quality Extension}
Jean-Marc Valin introduced the \href{https://datatracker.ietf.org/doc/html/draft-valin-opus-scalable-quality-extension}{draft-valin-opus-scalable-quality-extension}, which proposes a scalable quality extension for Opus. The group considered the psychoacoustic rationale for band encoding and the potential for WG adoption. A call for adoption will be initiated, pending no objections from the working group.

Meeting materials are available at \href{https://meetings.conf.meetecho.com/ietf122/?session=33992}{this link}.




\newpage

\section{Messaging Layer Security (MLS) Working Group}

\subsection{Attendees}

The MLS WG session at IETF 122 was attended by 47 participants, including representatives from prominent organizations such as Apple, Cisco, Cloudflare, Google, and the US Naval Postgraduate School. The session provided a platform for discussing advancements and challenges in secure group messaging protocols.

\subsection{Meeting Discussions}

\subsubsection{Mitigations for Insider Replays}

Akshaya Kumar presented on the topic of mitigating insider replay attacks within MLS. The discussion highlighted the need for enhanced integrity and authenticity mechanisms to protect against insider threats. The proposed solutions include MLS-level mitigations such as signing generation numbers and application-level countermeasures. The presentation emphasized the importance of digital signatures and symmetric signcryption models to safeguard against vulnerabilities. For further details, refer to the \href{https://datatracker.ietf.org/doc/html/draft-ietf-mls-architecture}{draft-ietf-mls-architecture}.

\subsubsection{Call for Input on Combiner Draft}

Britta Hale called for input on the MLS Combiner draft, which proposes hybrid combiner models to enhance security considerations. The draft aims to separate post-quantum and traditional cryptographic schedules. Participants were encouraged to review the draft and provide feedback to facilitate progress towards a working group last call. More information can be found in the \href{https://datatracker.ietf.org/doc/html/draft-ietf-mls-combiner}{draft-ietf-mls-combiner}.

\subsubsection{Extensions and Application API}

Rohan Mahy discussed the integration of safe application APIs within MLS extensions. The session explored the challenges of aligning extensions with existing frameworks and proposed a cleaner separation between application logic and MLS behavior. The discussion underscored the need for a structured approach to manage application data and ensure seamless integration with MLS functionalities. The \href{https://datatracker.ietf.org/doc/html/draft-ietf-mls-extensions}{draft-ietf-mls-extensions} provides additional insights.

\subsubsection{Non-Working Group Drafts}

Rohan Mahy presented on the open issues with ML-KEM cipher suites, emphasizing the need for new ciphersuites to accommodate both plain and hybrid models. The session also covered selective disclosure JWTs and CWTs, highlighting their potential to enhance privacy in web token credentials. Discussions on the \href{https://datatracker.ietf.org/doc/html/draft-mahy-mls-pq}{draft-mahy-mls-pq} and \href{https://datatracker.ietf.org/doc/html/draft-mahy-mls-sd-cwt-credential}{draft-mahy-mls-sd-cwt-credential} were pivotal in addressing these advancements.

\subsubsection{Proposed Adoption of MLS Light}

Richard Barnes introduced the MLS Light draft, aimed at optimizing MLS for large groups by reducing download and memory requirements. The draft proposes a model where light clients can join groups with minimal overhead, enhancing scalability and performance. The session debated the trade-offs between authenticity and confidentiality, with a consensus on the potential benefits for large-scale applications. The \href{https://datatracker.ietf.org/doc/html/draft-kiefer-mls-light}{draft-kiefer-mls-light} is under consideration for adoption.

\subsubsection{Distributed and Decentralized MLS}

Mark Xue and Konrad Kohbrok presented on distributed and decentralized MLS approaches, respectively. These discussions focused on enhancing MLS's adaptability to various network conditions and ensuring robust cryptographic state management across distributed systems. The \href{https://datatracker.ietf.org/doc/html/draft-xue-distributed-mls}{draft-xue-distributed-mls} and \href{https://datatracker.ietf.org/doc/html/draft-kohbrok-mls-dmls}{draft-kohbrok-mls-dmls} provide comprehensive frameworks for these innovative approaches.

\subsubsection{Defining End-to-End Encryption (E2EE)}

Mallory Knodel led a discussion on the definition of E2EE, emphasizing the need for a clear, IETF-wide standard. The session explored the challenges and gaps in current E2EE implementations, with a focus on maintaining privacy and security. The \href{https://datatracker.ietf.org/doc/html/draft-knodel-e2ee-definition}{draft-knodel-e2ee-definition} seeks to establish foundational principles for E2EE across various applications.

Meeting materials and further details are available at \href{https://notes.ietf.org/notes-ietf-122-mls}{Notes IETF 122 MLS}.



\newpage

\section{MOPS (Media OPS)}

\subsection{Attendees Overview}

The MOPS session at IETF 122 in Bangkok saw participation from 22 attendees, including representatives from prominent companies and institutions such as Akamai Technologies, Cisco, Comcast, Google, Huawei, and Verizon. The session provided a platform for discussing ongoing work and future directions in media operations.

\subsection{Meeting Discussions}

\subsubsection{Network Overlay Impacts (Sanjay Mishra)}

The presentation on \href{https://datatracker.ietf.org/doc/html/draft-ietf-mops-network-overlay-impacts}{draft-ietf-mops-network-overlay-impacts} highlighted the need for ultra-low latency support in live streaming and addressed issues arising from changed network policies affecting commercial applications. The discussion emphasized the importance of capturing potential issues and soliciting specific examples to enhance the document's robustness. The dialogue also explored the challenges of decoupling applications from protocol dependencies and the necessity of documenting current operational realities to inform future strategies.

\subsubsection{TIPTOP WG -- Taking IP to Other Planets (Padma Pillay-Esnault)}

The TIPTOP WG focuses on adapting IP networking for high-latency environments, such as space. The group aims to leverage existing architectures without creating new ones, addressing energy consumption and computing power limitations in satellites. The work involves adapting QUIC for long-delay networking and coordinating with other working groups to ensure comprehensive solutions.

\subsubsection{SCONE WG -- Standard Communication with Network Elements (Qin Wu)}

SCONE WG's efforts are directed towards establishing a protocol for communication between network elements and endpoints. The discussion covered proposals for MASQUE and QUIC, with an emphasis on narrowing the scope to a single protocol proposal. The group is exploring use cases like mobile networks and throttling policies, with plans to engage with 3GPP for further collaboration.

\subsubsection{SVTA Update (Glenn Deen)}

The SVTA update compared the organization to NANOG, highlighting its role as an inclusive industry body. New working groups on encoding, packaging, and edge computing were introduced, with a focus on achieving streaming latency comparable to broadcast. The update also mentioned upcoming SVTA meetings coinciding with IETF events and the release of a whitepaper on QUIC vs TCP for media delivery.

Meeting materials are available at \href{https://example.com/meeting-materials}{Meeting Materials}.



\newpage

\section{Media over QUIC Working Group (MoQ WG)}

\subsection{Attendees Overview}
\subsubsection{Attendance and Representation}
The MoQ WG meeting was attended by 81 participants, representing a diverse array of prominent companies and institutions. Notable attendees included representatives from Google, Cisco, Akamai, Apple, Ericsson, and China Mobile, among others. The meeting underscored the collaborative efforts of industry leaders and academic institutions in advancing the MoQ standards.

\subsection{Meeting Discussions}
\subsubsection{Administrivia and Agenda Overview}
The meeting commenced with administrivia, where the WG chairs reminded participants of the Note Well and presented the agenda. The session included a review of changes since IETF 121, focusing on the \href{https://datatracker.ietf.org/doc/html/draft-ietf-moq-transport}{MoQT draft} and the \href{https://datatracker.ietf.org/doc/html/draft-ietf-moq-warp}{Warp draft}. The call for adoption of the \href{https://datatracker.ietf.org/doc/html/draft-mzanaty-moq-loc}{LOC draft} was also highlighted.

\subsubsection{Changes Since IETF 121}
Ian Swett presented the significant updates incorporated into the MoQT draft since version -07. The restructuring of sections and the introduction of new components were discussed, with slides available \href{https://datatracker.ietf.org/meeting/122/materials/slides-122-moq-moqt-changes-since-ietf121-00}{here}.

\subsubsection{Publish and Subscribe Tracks}
Ian Swett and Cullen Jennings proposed enhancements for handling initial media tracks, including a new PUBLISH verb and a wildcard subscribe feature. The discussion concluded with a plan to submit two pull requests addressing these proposals.

\subsubsection{Track Alias}
Alan Frindell discussed the Track Alias proposal, noting that further changes are anticipated following the previous discussions. A proposal pull request will be awaited before proceeding.

\subsubsection{MoQT QLog}
Lucas Pardue presented the progress on MoQ QLog, highlighting the development of a scheme proposal and the groundwork for future adoption. The \href{https://datatracker.ietf.org/doc/draft-pardue-moq-qlog-moq-events}{draft} outlines the current state and future directions.

\subsubsection{LOC Update and Adoption Call}
The LOC draft's call for adoption was discussed, with a strong show of support from attendees. The chairs will declare the result post-feedback period. Issues regarding metadata location and potential encryption challenges were raised, with action items assigned for further exploration.

\subsubsection{AUTH for MoQT}
Will Law introduced the initial draft for AUTH in MoQT, proposing a new CWT claim type. The discussion addressed concerns about token management and revocation processes, with a consensus to further refine the proposal before adoption.

\subsection{Meeting Materials}
All meeting materials, including slides and detailed notes, are available \href{https://datatracker.ietf.org/meeting/122/materials/}{here}.

\subsection{Conclusion and Next Steps}
The meeting concluded with a strategic discussion on the future direction of MoQ WG, emphasizing the need for continued collaboration and refinement of proposals. Alan Frindell's transition from chair to co-editor was announced, marking a shift in leadership roles to support ongoing developments.



\newpage

\section{MPLS Working Group (MPLS WG)}

\subsection{Attendees Overview}
The MPLS Working Group meeting was attended by 62 participants, including representatives from prominent companies and institutions such as Ericsson, Juniper Networks, Nokia, Huawei Technologies, and ZTE Corporation. The diverse attendance underscored the broad interest and collaborative effort in advancing MPLS technologies.

\subsection{Meeting Discussions}

\subsubsection{WG Status Update}
The session commenced with a status update from the WG Chairs, Tarek Saad, Nicolai Leymann, and Tony Li. The update included a review of resolved errata and set the stage for the day's discussions.

\subsubsection{LSP Ping/Traceroute for Enabled In-situ OAM Capabilities}
Xiao Min presented on the \href{https://datatracker.ietf.org/doc/draft-xiao-mpls-lsp-ping-ioam-conf-state}{draft-xiao-mpls-lsp-ping-ioam-conf-state}, focusing on mechanisms to maintain LSP paths without allowing changes. Discussions highlighted the importance of scalability and the potential for using management planes to discover node capabilities.

\subsubsection{Simple Two-way Active Measurement Protocol (STAMP) for MPLS LSPs}
Greg Mirsky introduced the \href{https://datatracker.ietf.org/doc/draft-mirsky-mpls-stamp}{draft-mirsky-mpls-stamp}, which proposes a protocol for active measurement in MPLS networks. The dialogue centered on the benefits of stateful versus stateless approaches, with considerations for scalability and security.

\subsubsection{Supporting In Situ Operations, Administration and Maintenance Using MPLS Network Actions}
Greg Mirsky also discussed the \href{https://datatracker.ietf.org/doc/draft-ietf-mpls-mna-ioam}{draft-ietf-mpls-mna-ioam}, which aims to enhance OAM capabilities in MPLS networks. Key points included encoding challenges and the need for updates to address node inspection limits.

\subsubsection{MPLS Label Assignments}
Jeffrey Zhang's presentation on MPLS label assignments sparked a robust debate on label space management and the implications of using global versus context-specific labels. The discussion emphasized the need for clarity in label assignment practices to avoid confusion and ensure efficient network operations.

\noindent Meeting materials, including slides and notes, are available at \href{https://datatracker.ietf.org/meeting/122/session/mpls/}{IETF 122 MPLS WG Meeting Materials}.




\newpage

\section{Network Configuration Working Group (NETCONF)}

\subsection{Attendees}
\subsubsection{Overview}
The NETCONF session was attended by 60 participants, including representatives from notable organizations such as Cisco Systems, Huawei, Nokia, Ericsson, and the National Institute of Technology Karnataka. The session was chaired by Kent Watsen and Per Andersson.

\subsection{Meeting Discussions}

\subsubsection{Introduction}
The session began with an introduction by the chairs, providing an overview of the working group's status and objectives. The focus was on advancing the current drafts and addressing any outstanding issues.

\subsubsection{NETCONF over QUIC}
David Dai led the discussion on \href{https://datatracker.ietf.org/doc/html/draft-ietf-netconf-over-quic-02}{draft-ietf-netconf-over-quic-02}. The conversation highlighted the need for more detailed specifications to ensure implementability. Participants were encouraged to bring discussions to the mailing list to foster broader engagement.

\subsubsection{YANG Groupings for QUIC Clients and Servers}
Per Andersson presented the \href{https://datatracker.ietf.org/doc/html/draft-ietf-netconf-quic-client-server-01}{draft-ietf-netconf-quic-client-server-01}, emphasizing the integration of QUIC as a transport layer. The discussion underscored the importance of aligning with existing protocols to avoid unnecessary updates.

\subsubsection{List Pagination for YANG-driven Protocols}
Qin Wu discussed the \href{https://datatracker.ietf.org/doc/html/draft-ietf-netconf-list-pagination-06}{draft-ietf-netconf-list-pagination-06} and related drafts. The group concluded that all comments had been addressed, and the draft was ready to move forward without further working group last call (WGLC).

\subsubsection{Augmented-by Addition into the IETF-YANG-Library}
Zhuoyau Lin led the discussion on \href{https://datatracker.ietf.org/doc/html/draft-ietf-netconf-yang-library-augmentedby-01}{draft-ietf-netconf-yang-library-augmentedby-01}. The debate focused on simplifying the module by removing unnecessary revision dates, aligning with implementation practices.

\subsubsection{External Trace ID for Configuration Tracing}
Jean Quilbeuf presented \href{https://datatracker.ietf.org/doc/html/draft-ietf-netconf-configuration-tracing-03}{draft-ietf-netconf-configuration-tracing-03}. The group agreed to update the security considerations in line with the latest templates, preparing the draft for WGLC.

\subsubsection{Adaptive Subscription to YANG Notification}
Quifang Ma discussed \href{https://datatracker.ietf.org/doc/html/draft-ietf-netconf-adaptive-subscription-07}{draft-ietf-netconf-adaptive-subscription-07}. The chairs decided to move the draft into WGLC, acknowledging its readiness for further review.

\subsubsection{UDP-based Transport for Configured Subscriptions}
Alex Huang Feng led the discussion on \href{https://datatracker.ietf.org/doc/html/draft-ietf-netconf-udp-notif-19}{draft-ietf-netconf-udp-notif-19}. The session explored the implications of lossy data transport and the need for best practices to guide subscribers.

\subsubsection{YANG Groupings for UDP Clients and Servers}
The \href{https://datatracker.ietf.org/doc/html/draft-ietf-netconf-udp-client-server-06}{draft-ietf-netconf-udp-client-server-06} was discussed with a focus on server port configurations. The consensus was to maintain the current model, with plans to publish the draft.

\subsubsection{Subscription to Distributed Notifications}
Thomas Graf presented \href{https://datatracker.ietf.org/doc/html/draft-ietf-netconf-distributed-notif-13}{draft-ietf-netconf-distributed-notif-13}. The draft was deemed ready for publication, with the chairs tasked with finding a shepherd.

\subsubsection{Non-Chartered Items}
The session also covered non-chartered items, including the \href{https://datatracker.ietf.org/doc/html/draft-netana-netconf-yp-transport-capabilities-01}{YANG Notification Transport Capabilities} and \href{https://datatracker.ietf.org/doc/html/draft-netana-netconf-notif-envelope-02}{Extensible YANG model for YANG-Push Notifications}. These discussions emphasized the importance of implementation and real-world testing to accelerate document progression.

\subsubsection{Meeting Materials}
All meeting materials, including notes and slides, are available at \href{https://datatracker.ietf.org/meeting/122/session/netconf}{NETCONF 122 Meeting Materials}.




\newpage

\section{Network Modeling (NETMOD) Working Group}

\subsection{Attendees Overview}
\subsubsection{Attendance Summary}
The NETMOD Working Group session was attended by 51 participants, representing prominent organizations such as Cisco Systems, Ericsson, Nokia, Huawei, and Deutsche Telekom. The session was chaired by Lou Berger and Kent Watsen, with James Cumming serving as the WG Secretary.

\subsection{Meeting Discussions}

\subsubsection{Session Intro \& WG Status}
The session began with an introduction and status update by the WG Chairs. Key document updates were discussed, highlighting issues with validation tools. Scott Mansfield raised concerns about discrepancies in YANG model warnings, prompting a discussion on tool updates and potential liaison with IEEE.

\subsubsection{RFC 6991 Post Last Call Update}
Mahesh Jethanandani led a discussion on the post last call updates for \href{https://datatracker.ietf.org/doc/html/draft-ietf-netmod-rfc6991-bis}{draft-ietf-netmod-rfc6991-bis}. Key issues included the mac-address typedef regex and URI percent encoding. The group agreed to clarify the mac-address description and considered removing unnecessary encoding text, pending further discussion.

\subsubsection{YANG Versioning – Packages Update}
Rob Wilton presented updates on YANG packages, emphasizing the need for example packages and a package compiler. The group discussed potential collaborations and the importance of validating packages using standard tooling.

\subsubsection{Report on Template Interims}
Kent Watsen reported on YANG template interims, noting collected requirements and ongoing discussions. The group debated the necessity of further interims and the direction of template solutions.

\subsubsection{Populating a List of YANG Data Nodes Using Templates}
Deepak Rajaram presented a framework for using templates in YANG data nodes, emphasizing its informational nature. The group decided not to adopt the draft, focusing instead on a unified template solution.

\subsubsection{Adding Errata to IETF YANG Modules}
Balazs Lengyel discussed the process for updating YANG modules with errata. The group considered mechanisms for WG consensus and the role of IANA in handling complete YANG files.

\subsubsection{DTNMA Application Data Model (ADM) YANG Syntax}
Brian Sipos presented on DTNMA-YANG, sparking a discussion on the potential for a YANG language fork. The group emphasized the need to identify unsolvable problems with current YANG capabilities before considering new approaches.

Meeting materials, including notes and slides, are available at \href{https://datatracker.ietf.org/meeting/122/session/netmod}{NETMOD Session Materials}.


\newpage

\section{Network Management Operations (nmop) WG Minutes}

\subsection{Attendees}
\subsubsection{Overview}
The meeting was attended by 73 participants, including representatives from prominent companies and institutions such as Huawei, Cisco, Deutsche Telekom, Orange, and Swisscom. The diverse attendance underscored the broad interest and collaborative effort in advancing network management operations.

\subsection{Meeting Discussions}

\subsubsection{Agenda Bashing \& Introduction}
The session began with updates from the BBF Spring Meeting on the WT-508/YANG-Push Message Broker liaison. Discussions focused on timestamping and its integration with \href{https://datatracker.ietf.org/doc/html/draft-ietf-nmop-yang-message-broker-integration}{draft-ietf-nmop-yang-message-broker-integration}. A team was formed to review data collection protocol requirements.

\subsubsection{Updates from the Terminology Fairy}
Adrian Farrel presented updates on terminology, with a focus on refining definitions to align with current network management practices. The draft document can be accessed at \href{https://datatracker.ietf.org/doc/draft-ietf-nmop-terminology/}{draft-ietf-nmop-terminology}.

\subsubsection{SIMAP}
Olga Havel discussed the SIMAP project, addressing issues such as network instance partitioning and the optional nature of certain requirements. The discussion highlighted the importance of linking passive and active network elements, with consensus to keep these requirements optional. Relevant documentation is available at \href{https://datatracker.ietf.org/doc/draft-ietf-nmop-digital-map-concept/}{draft-ietf-nmop-digital-map-concept}.

\subsubsection{Message Broker: Extensible YANG Model for Network Telemetry Notifications}
Ahmed Elhassany presented on the need for metadata in data collection, emphasizing its role in differentiating collectors during software upgrades. The session underscored the intent to standardize message formats, with interest from BBF in session metadata for IPFIX. More details can be found in \href{https://datatracker.ietf.org/doc/html/draft-ietf-nmop-yang-message-broker-integration-07}{draft-ietf-nmop-yang-message-broker-integration-07}.

\subsubsection{Anomaly Detection \& Incident Management}
Qin Wu introduced the Incident Management YANG Module, sparking a debate on terminology, particularly the use of "root cause" versus "probable cause." The session concluded with a consensus to refine terminology to better reflect operational realities. The draft is available at \href{https://datatracker.ietf.org/doc/draft-ietf-nmop-network-incident-yang/}{draft-ietf-nmop-network-incident-yang}.

\subsubsection{Bring Your Own Outage}
Holger Keller and Thomas Graf shared insights from Deutsche Telekom and Swisscom network incidents, respectively. Discussions focused on the challenges of verifying maintenance windows and the potential of IPFIX and ForwardingStatus for passive measurement. The session highlighted the need for improved incident detection and response strategies.

\subsubsection{Wrap-up}
The session concluded with a proposal to organize a dedicated interim meeting on knowledge graphs, recognizing their potential to enhance incident management and network design. The meeting materials can be accessed at \href{https://www.network-analytics.org/yp/implementations.html}{Network Analytics Implementations}.

\subsection{Session 2: Hackathon-focused}

\subsubsection{Validate Configured Subscription YANG-Push Publisher Implementations}
Thomas Graf discussed the validation of YANG-Push implementations, with plans to extend support for message broker telemetry messages. The session emphasized the goal of covering the entire architecture, with open-source implementations available at \href{https://www.network-analytics.org/yp/implementations.html}{Network Analytics Implementations}.

\subsubsection{SIMAP for SRv6 and Linking Topology to External Data}
Sherif Mostafa presented on linking SRv6 topology to external data, with discussions on potential improvements and future directions.

\subsubsection{Anomaly Detection Integration Update}
Vincenzo Riccobene provided updates on anomaly detection integration, highlighting the importance of validating internal processes and considering a global confidence score in future iterations.

\subsubsection{YANG Configuration Instance Data to Knowledge Graph}
Michael Mackey explored the use of RDF Mapping Language for converting YANG configuration data to knowledge graphs, discussing challenges and potential collaborations with W3C.

\subsubsection{HTTPS-Notif-Draft Kafka Integration and Bandwidth Analysis for Different Encodings}
The session concluded with a presentation on integrating HTTPS notifications with Kafka, emphasizing the potential for open-source contributions to enhance library support.




\newpage

\section{Network Management Research Group (NMRG)}

\subsection{Attendees Overview}
The 81st meeting of the Network Management Research Group (NMRG) was held in conjunction with IETF 122 in Bangkok and online. The session was attended by 73 participants, including representatives from prominent organizations such as Nokia, Huawei, Deutsche Telekom, Telefonica, and the Internet Society. The diverse attendance underscored the global interest and collaborative efforts in advancing network management research.

\subsection{Meeting Discussions}

\subsubsection{Introduction and Document Status}
The meeting commenced with an introduction by the RG Chairs, who provided updates on document statuses and recent activities within the research group. This set the stage for a series of presentations and discussions focused on emerging technologies and methodologies in network management.

\subsubsection{Network Digital Twin: Concepts and Reference Architecture}
Cheng Zhou presented the \href{https://datatracker.ietf.org/doc/html/draft-irtf-nmrg-network-digital-twin-arch}{draft-irtf-nmrg-network-digital-twin-arch}, highlighting the readiness for IRSG review and the significance of digital twins in network simulations. Discussions emphasized the balance between simulation and emulation, with insights on leveraging digital twins for data generation and model accuracy.

\subsubsection{NDT – Practical Considerations and Examples}
Marco Liebsch discussed practical aspects of Network Digital Twins (NDT), focusing on traffic generation and model stress testing. The dialogue explored the adaptability of NDTs across different network environments and the potential for AI/ML integration in model development.

\subsubsection{Artificial Intelligence for Network Operations}
Reza Rokui presented the \href{https://datatracker.ietf.org/doc/html/draft-king-rokui-ainetops-usecases}{draft-king-rokui-ainetops-usecases}, which outlined AI use cases in network operations. The session highlighted the need for a comprehensive framework that includes both inference and training, with references to ongoing work in related standards organizations.

\subsubsection{Large Model-Based Agents for Network Operation and Maintenance}
Chuyi introduced the \href{https://datatracker.ietf.org/doc/html/draft-chuyi-nmrg-ai-agent-network-00}{draft-chuyi-nmrg-ai-agent-network-00}, focusing on the role of large models in network management. Discussions centered on the challenges of multi-agent collaboration and the integration of state-of-the-art techniques.

\subsubsection{A Framework for LLM-Assisted Network Management}
Mingzhe Xing presented the \href{https://datatracker.ietf.org/doc/html/draft-cui-nmrg-llm-nm}{draft-cui-nmrg-llm-nm}, which proposes a framework for leveraging large language models in network management, emphasizing human-in-the-loop processes.

\subsubsection{Open Discussions on GenAI in Network Management}
The session concluded with an open discussion on the development of Generative AI (GenAI) within NMRG. Participants debated the scope of GenAI research, potential collaborations, and synchronization with other research groups and working groups.

Meeting materials are available at \href{https://datatracker.ietf.org/meeting/122/session/nmrg}{NMRG Meeting Materials}.



\newpage

\section{Network Time Protocol Working Group (NTP WG)}

\subsection{Attendees Overview}
The NTP Working Group meeting at IETF 122 was attended by 17 participants, including representatives from prominent organizations such as Microsoft, Apple, Huawei Technologies, and Nokia. Other notable attendees included members from the Physikalisch-Technische Bundesanstalt and the National ICT Authority.

\subsection{Meeting Discussions}

\subsubsection{NTP/TICTOC WG Document Status Review}
The session began with a review of the current status of various working group documents. The \href{https://datatracker.ietf.org/doc/html/rfc9748}{RFC 9748} has been published, marking a significant milestone in updating NTP registries. The \href{https://datatracker.ietf.org/doc/html/draft-ietf-ntp-interleaved-modes}{draft-ietf-ntp-interleaved-modes} and \href{https://datatracker.ietf.org/doc/html/draft-ietf-tictoc-ptp-enterprise-profile}{draft-ietf-tictoc-ptp-enterprise-profile} remain in the RFC Editor Queue, while the \href{https://datatracker.ietf.org/doc/html/draft-ietf-ntp-over-ptp}{draft-ietf-ntp-over-ptp} is awaiting the completion of the shepherd's writeup.

\subsubsection{NTPv5 Requirements and Protocol Specification}
The \href{https://datatracker.ietf.org/doc/html/draft-ietf-ntp-ntpv5-requirements}{NTPv5 Requirements} document has expired, and the group is seeking additional editorial support. Discussions around the \href{https://datatracker.ietf.org/doc/html/draft-ietf-ntp-ntpv5}{NTPv5 Protocol Specification} focused on the need for further review, particularly concerning the use of Bloom filters for refid. The group emphasized the separation of protocol and algorithm specifications to allow for flexibility in future developments.

\subsubsection{Roughtime and NTS for PTP}
The \href{https://datatracker.ietf.org/doc/html/draft-ietf-ntp-roughtime}{Roughtime} document has been reclassified as experimental, with a Working Group Last Call (WGLC) planned before the next interim meeting. The \href{https://datatracker.ietf.org/doc/html/draft-langer-ntp-nts-for-ptp}{NTS for PTP} draft has undergone editorial changes and added replay protection, with further updates contingent on IEEE 1588 Security Subcommittee decisions.

\subsubsection{IEEE 1588 Update and ITU Leap Second Decisions}
Karen presented updates on IEEE 1588, highlighting ongoing work on enhanced security and the development of a Client-Server PTP model. The group also discussed ITU's considerations on leap seconds, with potential implications for IETF and IEEE 1588. Contributions to the discussion are encouraged, particularly in relation to the BIPM's recommendations.

\subsubsection{AOB and Way Forward}
The meeting concluded with plans for the next interim meeting on April 22 and a hackathon at IETF 123 focusing on NTS and NTPv5. Meeting materials are available at \href{https://www.ietf.org/proceedings/122/ntp.html}{IETF 122 NTP WG Materials}.



\newpage

\section{OAuth Working Group (OAuth WG)}

\subsection{Attendees Overview}
\subsubsection{Attendance and Representation}
The OAuth Working Group meeting was attended by 74 participants, representing prominent organizations such as Ciena, NSA-CCSS, Okta, Huawei, Peraton Labs, Authlete, Yubico, DHS/CISA, New H3C Technologies, Microsoft, Google, and Cisco Systems, among others.

\subsection{Meeting Discussions}

\subsubsection{Chairs Update}
Rifaat Shekh-Yusef and Hannes Tschofenig provided an update on the group's progress, highlighting the publication of RFC9700 Security BCP and RFC9701 JWT Response for Token Introspection. They also discussed ongoing work, including the Protected Resource Metadata and the Browser Based Apps document, which is under IESG review. The chairs encouraged participation in the OAuth Security Workshop and emphasized the importance of Working Group Last Call (WGLC) to facilitate document reviews.

\subsubsection{Token Status List}
Paul presented the \href{https://datatracker.ietf.org/doc/draft-ietf-oauth-status-list/}{Token Status List} draft, which proposes a scalable revocation mechanism using JSON/CBOR encoding. The draft is designed to support various tokens, including SD-JWT and ISO 18013-5 mDoc, and is named in the EIDAS ARF1.4 as a primary revocation mechanism. Key changes include recommendations for key resolution, trust management, and privacy considerations. The group discussed the potential for a separate draft to address status lists for x.509 certificates, with consensus leaning towards collaboration with the LAMPS working group.

\subsubsection{Attestation-based Client Authentication}
Paul and Christian discussed the \href{https://datatracker.ietf.org/doc/draft-ietf-oauth-attestation-based-client-auth/}{Attestation-based Client Authentication} draft, which aims to establish backend attested client authentication through the front channel. The draft introduces a nonce mechanism to prevent replay attacks, with ongoing discussions about the best approach for nonce fetching. The group agreed to continue discussions on the mailing list to resolve outstanding issues.

\subsubsection{OAuth 2.1}
Aaron Parecki presented the \href{https://datatracker.ietf.org/doc/draft-ietf-oauth-v2-1/}{OAuth 2.1} draft, which consolidates best practices from OAuth 2.0 without introducing new behaviors. The document is nearing completion, with recent additions including DPoP and Step Up Authentication. The group discussed the scope of HTTP binding, ultimately deciding to maintain the current HTTP-based protocol to preserve simplicity and clarity.

\subsubsection{OAuth First-Party Apps}
Aaron discussed the \href{https://datatracker.ietf.org/doc/draft-ietf-oauth-first-party-apps/}{OAuth First-Party Apps} draft, which aims to improve user experience by enabling authorization code flow through native app UX. The group debated whether to define this as a PAR extension, with differing opinions on its impact on implementation plans. Further discussions will continue on the mailing list.

\subsubsection{Client ID Scheme}
The \href{https://datatracker.ietf.org/doc/draft-parecki-oauth-client-id-scheme/}{Client ID Scheme} draft was presented by Aaron, addressing the need for client ID schemes in scenarios where preregistration is not feasible. The proposal includes using a URL as a client ID, with discussions focusing on compatibility with existing OAuth tooling and OpenID Federation. The draft is not yet adopted, pending further alignment with related standards.

\subsubsection{Updating Security BCP}
Pedram Hosseyni highlighted the need to update the Security BCP in light of new attacks, such as audience injection and AS mix-up attacks. The group discussed procedural options for updating the BCP, with consensus on creating a focused document to address these issues promptly.

\subsubsection{SD-JWT and SD-JWT VC}
Brian Campbell provided updates on the \href{https://datatracker.ietf.org/doc/draft-ietf-oauth-selective-disclosure-jwt/}{SD-JWT} and \href{https://datatracker.ietf.org/doc/draft-ietf-oauth-sd-jwt-vc/}{SD-JWT VC} drafts, addressing issues related to issuer fields and certificate-based trust. The group discussed the drafts' progress and remaining issues, with an emphasis on resolving key concerns to advance towards completion.

\subsubsection{JWT Profile for Client Authentication and Authorization Grants}
Mike Jones and Brian Campbell presented the \href{https://datatracker.ietf.org/doc/draft-ietf-oauth-rfc7523bis/}{JWT Profile} draft, detailing a vulnerability and proposed solutions. A poll indicated support for Brian's version, guiding the next steps in refining the draft.

\subsubsection{Transaction Token}
Pieter Kasselman discussed the \href{https://datatracker.ietf.org/doc/draft-ietf-oauth-transaction-tokens/}{Transaction Token} draft, focusing on the lifetime of access tokens and transaction tokens. The group emphasized the importance of capturing initiation context and agreed on the need for precise wording to address open issues.

\subsubsection{OAuth Identity Chaining}
Brian Campbell presented the \href{https://datatracker.ietf.org/doc/draft-ietf-oauth-identity-chaining/}{OAuth Identity Chaining} draft, which aims to clarify patterns without introducing new functionality. The group discussed similarities to Kerberos and potential complexities, with Hannes offering to provide additional insights.

\subsubsection{Deferred Key Binding for OAuth}
Justin Richer introduced the \href{https://datatracker.ietf.org/doc/html/draft-richer-oauth-tmb-claim}{Deferred Key Binding for OAuth} draft, proposing an interim meeting to further explore the topic.

\subsubsection{Token Exchange Extensions}
Nick Watson discussed potential extensions for token exchange, focusing on error handling and refresh token expiration. The group considered practical aspects of client interaction and emphasized the need for careful consideration of security implications.

Meeting materials are available at \href{https://datatracker.ietf.org/meeting/122/materials/}{IETF 122 Meeting Materials}.



\newpage

\section{OpenPGP WG}

\subsection{Attendees}
The OpenPGP Working Group meeting was attended by 33 participants, including representatives from prominent organizations such as Nokia, Cisco Systems, Sequoia PGP, MTG AG, ACLU, Red Hat, and the UK NCSC. Notable attendees included Jonathan Sadler (Nokia), Paul Wouters (Aiven), Scott Fluhrer (Cisco Systems), and Daniel Gillmor (ACLU).

\subsection{Meeting Discussions}

\subsubsection{Interop and RFC 9580 Deployment}
Justus Winter presented on the deployment of RFC 9580, highlighting interoperability challenges. Discussions revealed that some clients do not yet support v6 imports, and there is ongoing work to improve compatibility, particularly with the Hockeypuck server. Contributions are sought for a Rust implementation to enhance the ecosystem.

\subsubsection{Post-Quantum Cryptography in OpenPGP}
Falko Strenzke discussed the integration of post-quantum cryptography into OpenPGP, referencing the \href{https://datatracker.ietf.org/doc/draft-ietf-openpgp-pqc/}{draft-ietf-openpgp-pqc}. Scott Fluhrer provided feedback on cryptographic transformations, emphasizing the need for compliance with CNSA 2.0 standards. The dialogue underscored the importance of aligning with emerging cryptographic requirements.

\subsubsection{Persistent Symmetric Keys}
Daniel Huigens introduced the concept of persistent symmetric keys, as detailed in the \href{https://datatracker.ietf.org/doc/draft-ietf-openpgp-persistent-symmetric-keys/}{draft-ietf-openpgp-persistent-symmetric-keys}. The discussion focused on the practical applications and potential drawbacks of using persistent keys, with concerns raised about the security implications of reusing keys for multiple encryptions.

\subsubsection{OpenPGP Key Replacement}
Andrew Gallagher presented on key replacement strategies, referencing the \href{https://datatracker.ietf.org/doc/draft-ietf-openpgp-replacementkey/}{draft-ietf-openpgp-replacementkey}. The session explored the procedural aspects of key replacement and its impact on existing cryptographic practices.

\subsubsection{HKP and Semantic Cleanup}
The meeting also covered the Hypertext Key Protocol (HKP) and semantic cleanup efforts, with Andrew Gallagher leading the discussion. The focus was on improving protocol efficiency and addressing legacy client issues. The group considered the implications of these changes on the broader OpenPGP ecosystem.

Meeting materials, including slide presentations, are available at \href{https://meetings.conf.meetecho.com/ietf122/?session=34014}{IETF 122 Meeting Materials}.




\newpage

\section{Operations and Management Area Working Group (OPSAWG) \& OPS Area Agenda}

\subsection{Attendees}
\subsubsection{Overview}
The session was attended by 71 participants, including representatives from prominent companies and institutions such as Cisco, Huawei, Nokia, Google, and Telefonica. The diverse attendance highlighted the broad interest and collaborative nature of the discussions.

\subsection{Meeting Discussions}

\subsubsection{Publishing End-Site Prefix Lengths}
Presenter: Oliver Gasser \\
The discussion centered around the \href{https://datatracker.ietf.org/doc/html/draft-ietf-opsawg-prefix-lengths}{draft-ietf-opsawg-prefix-lengths}, which aims to standardize the publication of end-site prefix lengths. Feedback was largely positive, with minor nits identified for correction. The draft is nearing readiness for Working Group Last Call (WGLC), pending minor revisions.

\subsubsection{Export of Delay Performance Metrics in IPFIX}
Presenter: Thomas Graf \\
The \href{https://datatracker.ietf.org/doc/html/draft-ietf-opsawg-ipfix-on-path-telemetry}{draft-ietf-opsawg-ipfix-on-path-telemetry} was discussed, focusing on exporting delay performance metrics. The draft has passed WG Last Call and is progressing towards IESG review, with no significant objections raised during the session.

\subsubsection{IPFIX Alternate-Marking Information Elements}
Presenter: Giuseppe Fioccola \\
Discussion on the \href{https://datatracker.ietf.org/doc/html/draft-ietf-opsawg-ipfix-alt-mark}{draft-ietf-opsawg-ipfix-alt-mark} highlighted the need for better operational use case descriptions. The draft's relationship with other IPFIX documents was explored, with suggestions to enhance deployment information for clarity.

\subsubsection{Applying COSE Signatures for YANG Data Provenance}
Presenter: Diego Lopez \\
The \href{https://datatracker.ietf.org/doc/html/draft-lopez-opsawg-yang-provenance}{draft-lopez-opsawg-yang-provenance} was presented, focusing on securing YANG data provenance using COSE signatures. A poll indicated strong support for adoption, with a follow-up poll planned for the mailing list to formalize this.

\subsubsection{A Data Manifest for Contextualized Telemetry Data}
Presenter: Jean Quilbeuf \\
The session reviewed the \href{https://datatracker.ietf.org/doc/html/draft-ietf-opsawg-collected-data-manifest}{draft-ietf-opsawg-collected-data-manifest}, which aims to standardize metadata for telemetry data. Discussions revealed overlaps with other initiatives, prompting a need for coordination to avoid redundancy and enhance interoperability.

\subsubsection{An Information Model for Packet Discard Reporting}
Presenter: John Evans \\
The \href{https://datatracker.ietf.org/doc/html/draft-ietf-opsawg-discardmodel}{draft-ietf-opsawg-discardmodel} and \href{https://datatracker.ietf.org/doc/html/draft-evans-opsawg-ipfix-discard-class-ie}{draft-evans-opsawg-ipfix-discard-class-ie} were discussed, focusing on packet discard reporting. The group debated the structure of the drafts, considering whether to consolidate them for clarity and ease of updates.

\subsubsection{Export of QUIC Information in IPFIX}
Presenter: Yao Liu \\
The \href{https://datatracker.ietf.org/doc/html/draft-lin-opsawg-ipfix-quic-header}{draft-lin-opsawg-ipfix-quic-header} was introduced, with discussions on how to handle QUIC header information within IPFIX. Suggestions were made to include more examples and use cases in future revisions to enhance understanding and implementation.

\subsection{Meeting Materials}
All meeting materials are available at \href{https://datatracker.ietf.org/meeting/122/materials/agenda-122-opsawg}{IETF 122 OPSAWG Materials}.




\newpage

\section{Path Computation Element Working Group (PCE)}

\subsection{Attendees}
\subsubsection{Overview}
The meeting was attended by 44 participants, including representatives from prominent companies and institutions such as Huawei, Nokia, Orange, Cisco Systems, and The University of Tokyo. Notable attendees included Dhruv Dhody from Huawei, Andrew Stone from Nokia, and Olivier Dugeon from Orange.

\subsection{Meeting Discussions}

\subsubsection{SRv6 Policy SID List Optimization}
Zafar Ali presented the \href{https://datatracker.ietf.org/doc/draft-all-pce-srv6-policy-sid-list-optimization/}{draft-all-pce-srv6-policy-sid-list-optimization}, focusing on the optimization of SRv6 policy SID lists. The discussion highlighted the need to clarify the document's orientation towards PCE-Initiated policies and consider PCC-Initiated scenarios. The potential for a common document in the SPRING working group was also discussed, with a call for collaboration between PCE and IDR chairs.

\subsubsection{Stateful Interdomain}
Olivier Dugeon discussed the \href{https://datatracker.ietf.org/doc/draft-ietf-pce-stateful-interdomain/}{draft-ietf-pce-stateful-interdomain}, emphasizing the notification message between PCEs and its broader applicability. The conversation touched on operational aspects, naming conventions, and the need for further reviews due to the document's complexity. Volunteers were sought for additional reviews to ensure comprehensive feedback.

\subsubsection{Topology Filter}
Quan Xiong introduced the \href{https://datatracker.ietf.org/doc/draft-xpbs-pce-topology-filter/}{draft-xpbs-pce-topology-filter}, which aims to enhance topology filtering capabilities. The discussion focused on the use of sub-TLVs for extension and the relationship between topology filters and existing constraints. Clarifications were requested on the overlap with constraints and the optional nature of certain TLVs.

\subsubsection{PCE Operational Clarification \& Amendments to Stateful PCEP}
Andrew Stone presented two drafts: \href{https://datatracker.ietf.org/doc/draft-koldychev-pce-operational/}{draft-koldychev-pce-operational} and \href{https://datatracker.ietf.org/doc/draft-many-pce-stateful-amendment/}{draft-many-pce-stateful-amendment}. The operational document was seen as a valuable resource for clarifications and is expected to move quickly towards adoption. The potential for it to remain a living document was discussed, with consensus on its utility for implementers.

\subsubsection{LSP State Reporting Extensions}
Samuel Sidor discussed the \href{https://datatracker.ietf.org/doc/draft-sidor-pce-lsp-state-reporting-extensions/}{draft-sidor-pce-lsp-state-reporting-extensions}, addressing previous comments and clarifying the use of the D flag for LSP instantiation. The discussion emphasized the need for further updates to address outstanding comments and ensure clarity.

\subsubsection{Precision Availability Metrics}
Luis Contreras presented the \href{https://datatracker.ietf.org/doc/draft-contreras-pce-pam/}{draft-contreras-pce-pam}, focusing on the use of precision metrics for path computation. The discussion revolved around the applicability of these metrics and their role in path selection. Further offline discussions were suggested to explore the integration of these metrics with existing constraints.

\subsubsection{P2MP SR Policy}
Hooman Bidgoli discussed the \href{https://datatracker.ietf.org/doc/draft-ietf-pce-sr-p2mp-policy/}{draft-ietf-pce-sr-p2mp-policy}, with feedback provided on document structure and editorial aspects. The process for shepherd review and the need for volunteers to review the document, particularly those with multicast expertise, was emphasized.

Meeting materials, including slides and recordings, are available at \href{https://datatracker.ietf.org/meeting/122/session/pce}{IETF 122 PCE Session Materials}.




\newpage

\section{Privacy Enhancements and Assessments Research Group (PEARG)}

\subsection{Attendees Overview}
The PEARG meeting at IETF 122 was attended by 53 participants, including representatives from prominent organizations such as Cloudflare, NIC.BR, the Canadian Centre for Cyber Security, and Apple. The diverse attendance underscored the global interest in privacy enhancements and assessments.

\subsection{Meeting Discussions}

\subsubsection{Authentication in MLS and Its Variants}
Keitaro Hashimoto presented on the topic of authenticating application messages in MLS, focusing on efficiency, post-quantum security, and anonymous blocklistability. The discussion highlighted the trade-offs in the Cosmos variant, particularly regarding data transmission and the overhead of establishing one-time keys. Questions from attendees like Guiin Wang and Richard Barnes spurred a deeper dive into the anonymity and efficiency of the proposed scheme. The presentation materials are available \href{https://datatracker.ietf.org/meeting/122/materials/slides-122-pearg-authentication-in-mls-ietf122-v7-00}{here}.

\subsubsection{Guidelines for Performing Safe Measurements on the Internet}
Gurshabad Grover provided insights into guidelines for conducting internet measurements, emphasizing the need for language that avoids negativity and the importance of integrating prior research. The session encouraged feedback to refine these guidelines further, with updates already reflecting some suggestions.

\subsubsection{Future Directions for the Research Group}
The discussion on future directions emphasized the rapidly evolving privacy landscape and the need for a focus on post-quantum cryptography. The group considered whether censorship-related work should become a research focus, with mixed feedback. The conversation also touched on privacy-preserving techniques and the potential for joint workshops to inform IETF privacy reviews. The group is open to community input on future topics, with a call for agenda items for upcoming meetings.

\textbf{Next Steps:} The group will continue discussions on the mailing list, gather input on future research topics, and consider expanding its focus on censorship, post-quantum cryptography, and engagement with IETF privacy standards.

Meeting materials can be accessed \href{https://datatracker.ietf.org/doc/minutes-122-pearg/}{here}.



\newpage

\section{Post-Quantum Use in Protocols (PQUIP)}

\subsection{Attendees Overview}
The PQUIP WG session at IETF 122 in Bangkok saw participation from 138 attendees, including representatives from prominent organizations such as NSA, Cisco Systems, Microsoft, Huawei, and Cloudflare. This diverse attendance underscores the broad interest and collaborative effort in addressing post-quantum cryptography challenges.

\subsection{Meeting Discussions}

\subsubsection{Current Document Status}
The session began with an update on the status of key documents. The \href{https://datatracker.ietf.org/doc/html/draft-ietf-pquip-pqt-hybrid-terminology}{draft-ietf-pquip-pqt-hybrid-terminology} has completed IESG review and is in the RFC Editor's queue. Meanwhile, the \href{https://datatracker.ietf.org/doc/html/draft-ietf-pquip-hybrid-signature-spectrums}{draft-ietf-pquip-hybrid-signature-spectrums} is in IETF Last Call, and the \href{https://datatracker.ietf.org/doc/html/draft-ietf-pquip-pqc-engineers}{draft-ietf-pquip-pqc-engineers} is under review with questions pending from the AD.

\subsubsection{PQC in Certificates at the Hackathon}
Jean-Pierre Fiset presented on the integration of PQC in certificates during the Hackathon. The group, which meets bi-weekly, emphasized the importance of continuous engagement beyond Hackathons to advance PQC adoption.

\subsubsection{Side Meeting on PQC Dialogue with Government Stakeholders}
John Preuß Matsson reported on a side meeting focused on engaging government stakeholders in PQC discussions. The session was recorded, and notes will be circulated to the mailing list, highlighting the importance of governmental collaboration in PQC strategy.

\subsubsection{Hash-based Signatures: State and Backup Management}
Thom Wiggers discussed the management of hash-based signatures, with further discussions planned on the mailing list. Participants were encouraged to share the \href{https://datatracker.ietf.org/doc/draft-wiggers-hbs-state/}{draft} within their organizations to broaden input.

\subsubsection{Adapting HSMs for Post-Quantum Cryptography}
Tiru Reddy's presentation sparked debate on the role of IETF in guiding hardware manufacturers. While some questioned its scope, others, like Deirdre Connolly, supported the initiative, recognizing its potential to define PQC standards for HSMs.

\subsubsection{Efficient PQ through Hybrids}
Britta Hale's presentation on hybrid PQ solutions was well-received, particularly for its benchmarks. The discussion explored the long-term viability of hybrid schemes, with insights into their security and efficiency benefits.

\subsubsection{The Great Private Key War of ‘25}
Mike Ounsworth's presentation addressed the complexities of private key management across protocols. The discussion highlighted the need for cross-protocol compatibility and the challenges posed by differing approaches to key management.

\subsubsection{IETF is Quantum-Fragile}
Deirdre Connolly's presentation underscored the vulnerabilities of current protocols to quantum threats. The session, marked by passionate exchanges, emphasized the urgency of addressing these fragilities to safeguard future communications.

Meeting materials are available at \href{https://datatracker.ietf.org/meeting/122/session/pquip/}{PQUIP Session Materials}.



\newpage

\section{PRIVACYPASS (PP)}

\subsection{Attendees}
\subsubsection{Overview}
The PRIVACYPASS working group meeting was attended by 52 participants, including representatives from prominent companies and institutions such as Google, Meta Platforms, Inc., Cloudflare, Mozilla, and Carnegie Mellon University. The diverse attendance underscored the broad interest and collaborative effort in advancing privacy-enhancing technologies.

\subsection{Meeting Discussions}
\subsubsection{Token Metadata}
Scott Hendrickson presented on the \href{https://datatracker.ietf.org/doc/draft-hendrickson-privacypass-expiration-extension/}{Privacy Pass Token Expiration Extension} and \href{https://datatracker.ietf.org/doc/draft-ietf-privacypass-public-metadata-issuance/}{Privacy Pass Issuance Protocols with Public Metadata}. The discussions highlighted the need for decoupling token expiration from issuer key rotation, enhancing privacy by allowing metadata to be signed by issuers. Feedback was solicited on the proposed extensions, which aim to improve the flexibility and privacy guarantees of token issuance.

\subsubsection{Attester Issuer Protocol}
The \href{https://datatracker.ietf.org/doc/draft-hendrickson-pp-attesterissuer/}{Attester Issuer Protocol} draft was discussed, focusing on defining the interaction between attesters and issuers. This protocol is crucial for ensuring privacy guarantees in deployments where attesters and issuers are distinct entities. The group debated whether the draft should include guidance on validating attesters, such as through mTLS.

\subsubsection{Privacy Pass Reverse Flow}
Thibault Meunier introduced the concept of \href{https://datatracker.ietf.org/doc/draft-meunier-privacypass-reverse-flow/}{Privacy Pass Reverse Flow}, which allows origins to bootstrap token issuance from another token. This approach could enable more flexible token management and potentially reduce the need for new tokens, though it requires further exploration to ensure robust privacy properties.

\subsubsection{Including Privacy Pass Tokens in TLS Handshakes}
Tommy Pauly proposed integrating \href{https://datatracker.ietf.org/doc/draft-pauly-privacypass-for-tls/}{Privacy Pass Tokens in TLS Handshakes}, which could allow TLS terminators to enforce rate limits before HTTP authentication. While this could streamline token verification, concerns were raised about the practicality and necessity of such integration, suggesting further coordination with the TLS working group.

\subsubsection{Privacy Pass with Token Binding Extension}
Wei Guo discussed the \href{https://datatracker.ietf.org/doc/draft-guo-privacypass-token-binding/}{Privacy Pass with Token Binding Extension}, which aims to protect tokens from theft attacks using Schnorr NIZKP. The extension could enhance security for non-interactive use cases, though its applicability to high-value transactions remains a topic for further investigation.

Meeting materials are available at \href{https://notes.ietf.org/notes-ietf-122-privacypass}{notes-ietf-122-privacypass}.




\newpage

\section{QUIC Working Group (QUIC WG)}

\subsection{Attendees}
\subsubsection{Overview}
The QUIC Working Group meeting was attended by 120 participants, including representatives from prominent companies and institutions such as Meta, Huawei, Cloudflare, Deutsche Telekom, Ericsson, Microsoft, Cisco Systems, and Google. The diverse attendance underscores the broad interest and collaborative effort in advancing QUIC-related technologies.

\subsection{Meeting Discussions}

\subsubsection{Introduction / Administrivia}
The meeting commenced with administrative updates, including scribe selection and a review of the \href{https://www.ietf.org/about/note-well.html}{NOTE WELL} and \href{https://www.rfc-editor.org/rfc/rfc7154.html}{Code of Conduct}. The agenda was confirmed, and the meeting materials were made available via \href{https://github.com/quicwg/wg-materials/blob/main/ietf122/chairs.pdf}{slides}.

\subsubsection{Multipath QUIC Update}
Yanmei Liu presented updates on \href{https://datatracker.ietf.org/doc/html/draft-ietf-quic-multipath}{multipath QUIC}, focusing on open issues such as the necessity of `PATH\_CIDS\_BLOCKED`. The discussion highlighted its utility for debugging rather than operational necessity. The consensus was to proceed with the current draft towards Working Group Last Call (WGLC), acknowledging the need for further text on RTT calculation guidance.

\subsubsection{Address Discovery Update}
Marten Seeman discussed updates to \href{https://datatracker.ietf.org/doc/draft-ietf-quic-address-discovery/}{address discovery}, emphasizing privacy concerns and the non-negotiable nature of address announcements. The group agreed on maintaining the current approach, aligning with operator expectations.

\subsubsection{Receive Timestamps}
Joseph Beshay introduced the \href{https://datatracker.ietf.org/doc/draft-smith-quic-receive-ts/}{draft on receive timestamps}, sparking a debate on the extensibility of ACK frames. The group leaned towards adopting a flexible approach to accommodate future extensions, with a strong interest in integrating this feature for congestion control enhancements.

\subsubsection{Extended Key Update}
Yaroslav Rosomakho presented the \href{https://datatracker.ietf.org/doc/draft-rosomakho-quic-extended-key-update/}{extended key update draft}, addressing security concerns in long-lived sessions. The discussion underscored the importance of maintaining TLS context and the potential for periodic key updates to mitigate compromise risks. The group expressed support for adopting this work, recognizing its relevance to 3GPP infrastructure.

\subsubsection{Source Buffer Management}
Stuart Cheshire raised the topic of source buffer management, questioning its applicability to QUIC. The discussion highlighted the importance of addressing queue management, particularly in devices with suboptimal downstream handling.

\subsubsection{QMux (QUIC on Streams)}
Alan Frindell presented \href{https://datatracker.ietf.org/doc/draft-kazuho-quic-quic-on-streams/}{QMux}, advocating for its development within the QUIC WG. The proposal received broad support, emphasizing the need for a consistent API and the potential for scalable deployments. The group acknowledged the necessity of rechartering to accommodate this work.

Meeting materials are available at \href{https://github.com/quicwg/wg-materials/blob/main/ietf122/}{QUIC WG materials}.

The discussions reflected a strategic shift towards enhancing QUIC's capabilities, with a focus on interoperability, security, and scalability. The outcomes suggest a proactive approach to addressing emerging challenges and leveraging QUIC's potential in diverse application scenarios.



\newpage

\section{RADIUS Extensions Working Group (RADEXT)} 

\subsection{Attendees Overview}
The RADEXT Working Group meeting was attended by representatives from prominent companies and institutions such as Cisco, ACLU, Uni Bremen, and Queensland University of Technology, totaling 21 participants. The meeting materials can be accessed \href{https://datatracker.ietf.org/meeting/122/materials/agenda-122-radext}{here}.

\subsection{Meeting Discussions}

\subsubsection{(Datagram) Transport Layer Security (D)TLS Encryption for RADIUS}
The discussion on \href{https://datatracker.ietf.org/doc/html/draft-ietf-radext-radiusdtls-bis}{draft-ietf-radext-radiusdtls-bis} focused on finalizing the document, with all major issues addressed. The group considered transitioning from Reject with Error-Cause to Protocol-Error packets, which are not forwarded by proxies, and emphasized the necessity of mTLS for client authentication. The need for TLS rekeying in long sessions was debated, with a consensus leaning towards a SHOULD requirement. The session concluded with plans for a Working Group Last Call (WGLC) pending final changes.

\subsubsection{Mitigations of Congestion and Load Issues in RADIUS Proxy Fabrics}
The \href{https://datatracker.ietf.org/doc/html/draft-janfred-radext-radius-congestion-control}{draft-janfred-radext-radius-congestion-control} was discussed, highlighting the challenges of congestion in proxy networks. Proposed solutions included introducing Response-Delay or Request-Block attributes to manage traffic. The group debated whether a broader solution was necessary, with a consensus that the current approach could mitigate overload issues, particularly in contexts like Eduroam.

\subsubsection{Update on Ongoing Documents}
The group reviewed the status of ongoing documents, noting that \href{https://datatracker.ietf.org/doc/html/draft-ietf-radext-tls-psk}{draft-ietf-radext-tls-psk} and \href{https://datatracker.ietf.org/doc/html/draft-ietf-radext-radiusv11}{draft-ietf-radext-radiusv11} were submitted to the IESG for publication. The lack of comments on \href{https://datatracker.ietf.org/doc/html/draft-ietf-radext-deprecating-radius}{draft-ietf-radext-deprecating-radius} and \href{https://datatracker.ietf.org/doc/html/draft-ietf-radext-reverse-coa}{draft-ietf-radext-reverse-coa} suggests readiness for WGLC.

\subsubsection{A Syntax for the RADIUS Connect-Info Attribute Used in Wi-Fi Networks}
The \href{https://datatracker.ietf.org/doc/html/draft-grayson-connectinfo}{draft-grayson-connectinfo} was presented, discussing its potential as an informational document. The group considered its publication as a standard, with reference to RFC2866 and RFC6929 regarding attribute modifications.

\subsubsection{RADIUS Attributes for National Security and Emergency Preparedness Service}
The \href{https://datatracker.ietf.org/doc/html/draft-gundavelli-radepcs}{draft-gundavelli-radepcs} was reviewed, with discussions on IANA considerations and document adoption. The group debated the merits of independent submission versus working group adoption, considering the implications for document control.

\subsubsection{Report from the Conference in Tampere}
The report highlighted vendor and operator discussions on practical issues, with suggestions for making Protocol-Error a standards track document.

\subsubsection{Suggestions for Next Steps for RADEXT}
The chairs outlined the completion of current work items and the potential rechartering of the group. Discussions included maintaining the group for ongoing RADIUS maintenance, with input from Area Directors needed.

\subsubsection{Open Mic}
The session concluded with an open mic, allowing attendees to voice additional concerns and suggestions.




\newpage

\section{Remote Attestation Procedures (RATS)}

\subsection{Attendees Overview}
\subsubsection{Participants}
The RATS session was attended by 60 participants, including representatives from prominent organizations such as Intel, Sandelman Software Works Inc, Linaro, Transforming Information Security LLC, and Politecnico di Torino. The session also saw participation from various academic institutions and industry leaders, reflecting a diverse and engaged audience.

\subsection{Meeting Discussions}

\subsubsection{Conceptual Message Wrappers (CMW)}
Thomas Fossati presented the \href{https://datatracker.ietf.org/doc/draft-ietf-rats-msg-wrap/}{draft-ietf-rats-msg-wrap}, which is currently in its second Working Group Last Call (WGLC). The discussion focused on the semantic precision of claims and privacy concerns related to X.509 certificates. The group debated the necessity of defining Object Identifiers (OIDs) for evidence and attestation results separately. The RATS chairs will continue to monitor discussions on the mailing list to address unresolved issues.

\subsubsection{EAT Measured Component}
The \href{https://datatracker.ietf.org/doc/draft-ietf-rats-eat-measured-component/}{draft-ietf-rats-eat-measured-component} was discussed, with Carsten Bormann emphasizing the importance of clear data representation. Monty Wiseman agreed to review the draft for compatibility with the Trusted Computing Group's canonical event log. Further discussions will be conducted on the mailing list.

\subsubsection{Concise Reference Integrity Manifest (CoRIM)}
Yogesh Deshpande's presentation on the \href{https://datatracker.ietf.org/doc/draft-ietf-rats-corim/}{draft-ietf-rats-corim} highlighted the need for reviews before proceeding with WGLC. The RATS chairs will initiate a WGLC based on reviewer feedback, with a completion target set for April 2025.

\subsubsection{PKIX Key Attestation}
Mike Ounsworth discussed the \href{https://datatracker.ietf.org/doc/draft-ietf-rats-pkix-key-attestation/}{draft-ietf-rats-pkix-key-attestation}, focusing on its scope and terminology. The session explored the differences between key and workload attestation, with Kathleen Moriarty offering assistance in aligning terminology.

\subsubsection{Reference Interaction Models}
Henk Birkholz presented the \href{https://datatracker.ietf.org/doc/draft-ietf-rats-reference-interaction-models/}{draft-ietf-rats-reference-interaction-models}, which is in WGLC. The session called for additional reviews, with a WGLC to follow based on feedback.

\subsubsection{Event Stream Subscription}
The \href{https://datatracker.ietf.org/doc/draft-ietf-rats-network-device-subscription/}{draft-ietf-rats-network-device-subscription} was discussed, with consensus to proceed with WGLC. The RATS chairs will initiate this process on the mailing list.

\subsubsection{Evidence Transformations}
Ned Smith's \href{https://datatracker.ietf.org/doc/draft-smith-rats-evidence-trans/}{draft-smith-rats-evidence-trans} was considered for adoption. The draft aims to provide semantics for CoRIM verifiers, with discussions on scope and future document strategies. An adoption poll showed strong support, and the chairs will start an adoption call.

\subsubsection{MUD-Based RATS Resources Discovery}
Henk Birkholz's \href{https://datatracker.ietf.org/doc/draft-birkholz-rats-mud/}{draft-birkholz-rats-mud} was discussed for adoption. The session showed mixed opinions on timing, and further discussions will continue on the mailing list.

\subsubsection{EAT Attestation Results}
Thomas Fossati presented the \href{https://datatracker.ietf.org/doc/draft-fv-rats-ear/}{draft-fv-rats-ear}, with discussions on its integration with AR4SI. The adoption poll indicated strong support, and the chairs will initiate an adoption call.

Meeting materials are available at \href{https://datatracker.ietf.org/meeting/122/materials/agenda-rats-00}{RATS Meeting Materials}.



\newpage

\section{Registration Protocols Extensions (REGEXT)}

\subsection{Attendees Overview}
\subsubsection{Prominent Attendees and Total Attendance}
The REGEXT session at IETF 122 was attended by 35 participants, including representatives from major organizations such as Verisign, ICANN, GoDaddy, Liberty Global, and Apple. Notable attendees included Scott Hollenbeck from Verisign, James Galvin from ICANN Board - Identity Digital, and Antoin Verschuren from Liberty Global.

\subsection{Meeting Discussions}
\subsubsection{RFC Ed Queue}
The session discussed the \href{https://datatracker.ietf.org/doc/draft-ietf-regext-epp-ttl/}{EPP mapping for DNS Time-To-Live (TTL) values}, which is currently in the RFC Editor Queue. Jim Galvin highlighted the importance of this document as a precursor to related RDAP queries, indicating that further progress is contingent on its completion.

\subsubsection{Submitted to IESG}
Several drafts were submitted to the IESG, including \href{https://datatracker.ietf.org/doc/draft-ietf-regext-epp-delete-bcp/}{Best Practices for Deletion of Domain and Host Objects in EPP} and \href{https://datatracker.ietf.org/doc/draft-ietf-regext-rdap-rir-search/}{RDAP RIR Search}. Discussions emphasized the need for revised drafts and the anticipation of AD write-ups. The \href{https://datatracker.ietf.org/doc/draft-ietf-regext-rdap-geofeed/}{RDAP Extension for Geofeed Data} is currently in IETF last call, reflecting its advanced stage in the process.

\subsubsection{Existing Work}
The session reviewed ongoing work such as \href{https://datatracker.ietf.org/doc/draft-ietf-regext-rdap-versioning/}{Versioning in RDAP} and \href{https://datatracker.ietf.org/doc/draft-ietf-regext-epp-https/}{EPP Transport over HTTPS}. Discussions focused on potential changes to draft assumptions and the need for community feedback. Notably, the \href{https://datatracker.ietf.org/doc/draft-ietf-regext-rdap-rpki/}{RDAP extension for RPKI Registration Data} was presented to CIDROPS, garnering positive feedback and suggestions for further discussion.

\subsubsection{New Work Presentations}
Gavin Brown presented \href{https://datatracker.ietf.org/doc/draft-brown-rdap-referrals/}{Efficient RDAP Referrals}, proposing a method to streamline RDAP client-server interactions. The presentation sparked a lively discussion on the practicality and implementation of HTTP HEAD requests for RDAP referrals. Victor Zhou introduced \href{https://github.com/xinbenlv/rfc-draft-authcodesec/blob/main/draft-zzn-authcodesec-00.txt}{Domain Transfer Authorization Using Cryptographic Signatures}, which aims to enhance security in domain transfers through cryptographic methods.

\subsubsection{Proposed Charter Change}
Jim Galvin proposed a charter change to clarify the scope of the working group, particularly regarding EPP over QUIC and HTTPS. The proposal suggests expanding the charter to include protocol extensions for transporting EPP or RDAP elements, with feedback from Andy Newton and others suggesting further refinements.

Meeting materials are available at \href{https://datatracker.ietf.org/meeting/122/materials/slides-122-regext-sessa-chair-slides-tuesday-01}{Chair slides}.

\subsubsection{Next Steps}
The session concluded with a call for further discussion on the proposed charter changes and the potential for interim meetings to address open drafts. The discussions and outcomes from this session are expected to significantly influence the technical direction and strategic focus of the REGEXT working group.



\newpage

\section{Registration Protocols Parameters Working Group (RPP WG)}

\subsection{Attendees Overview}
The RPP WG meeting was attended by 56 participants, including representatives from prominent organizations such as ICANN, SIDN, Verisign, Ericsson, and Cloudflare. The diverse attendance underscored the broad interest and collaborative effort in advancing registration protocols.

\subsection{Meeting Discussions}

\subsubsection{WG Operation}
The session on working group operation focused on the use of GitHub for collaboration, with a strong preference expressed for maintaining discussions on the mailing list as per RFC 8874. The community wiki and GitHub repository were highlighted as key resources, with ongoing debates about the balance between these platforms. The discussion emphasized the importance of clear communication channels to facilitate effective collaboration.

\subsubsection{Various Updates}
Several updates were presented, including a \href{https://datatracker.ietf.org/meeting/122/materials/slides-122-rpp-hackathon-rpp-01}{Hackathon Recap} by M. Wullink, which showcased functional code for RPP domain creation. P. Kowalik shared insights from a \href{https://datatracker.ietf.org/meeting/122/materials/slides-122-rpp-epp-extensions-analysis-centr-survey-results-01}{CENTR Survey}, highlighting the widespread use of DNSSEC and RGP extensions. J. Gould provided a progress update on EPP Extensibility and Extensions, emphasizing the need for comprehensive analysis of existing extensions.

\subsubsection{Requirements}
M. Wullink led a discussion on RPP requirements, exploring data representation and API validation. The conversation touched on the potential use of JSON and YAML formats, with a consensus on the need for flexibility while ensuring at least one mandatory data format. Discoverability and EPP compatibility were debated, with suggestions to separate data format considerations from transport layers.

\subsubsection{Architecture}
P. Kowalik presented initial thoughts on RPP architecture, sparking discussions on JSON schema usage and the handling of RPP extensions. The dialogue underscored the importance of structured approaches to extension management and the potential challenges posed by the lack of JSON namespaces.

\subsubsection{Deliverables and Milestones}
Gavin presented revised milestones, proposing a target for WG consensus on requirements by September 2025. The timeline was deemed ambitious yet necessary to drive progress, with suggestions for interim meetings to maintain momentum.

Meeting materials are available at \href{https://datatracker.ietf.org/meeting/122/materials/slides-122-rpp-1-chair-slides-02}{Chair Slides}.

The discussions during the RPP WG meeting highlighted critical areas for development and set the stage for future work, with a focus on refining requirements and architectural considerations to advance the registration protocols landscape.



\newpage

\section{Routing Area Working Group (RTGWG)}

\subsection{Attendees}
\subsubsection{Overview}
The RTGWG meeting was attended by 145 participants, including representatives from prominent companies and institutions such as Cisco, Juniper Networks, Huawei Technologies, Deutsche Telekom, and China Telecom. The diverse attendance underscored the broad interest and collaborative efforts in addressing routing challenges.

\subsection{Meeting Discussions}

\subsubsection{Dynamic Networks to Hybrid Cloud DCs: Problems and Mitigation Practices}
Linda Dunbar presented the \href{https://datatracker.ietf.org/doc/html/draft-ietf-rtgwg-net2cloud-problem-statement}{draft-ietf-rtgwg-net2cloud-problem-statement}, highlighting the challenges in dynamic networks interfacing with hybrid cloud data centers. The discussion focused on the need for another last call and potential directorate review to refine the draft further.

\subsubsection{Multi-segment SD-WAN via Cloud DCs}
The \href{https://datatracker.ietf.org/doc/html/draft-ietf-rtgwg-multisegment-sdwan}{draft-ietf-rtgwg-multisegment-sdwan} was discussed by Linda Dunbar, emphasizing its stability and readiness for a working group last call. The conversation explored its differentiation from service chaining, focusing on traffic authentication and forwarding mechanisms.

\subsubsection{YANG Data Model for IPv6 Neighbor Discovery}
Fan Zhang introduced the \href{https://datatracker.ietf.org/doc/html/draft-ietf-rtgwg-ipv6-address-resolution-yang}{draft-ietf-rtgwg-ipv6-address-resolution-yang}, which aims to cover IPv6 neighbor discovery. The dialogue centered on the scope of the model and the need for comprehensive coverage of neighbor discovery protocols.

\subsubsection{Fast Failure Detection in VRRP with Point to Point BFD}
Aditya Dogra discussed the \href{https://datatracker.ietf.org/doc/html/draft-ietf-rtgwg-vrrp-bfd-p2p}{draft-ietf-rtgwg-vrrp-bfd-p2p}, focusing on enhancing VRRP with BFD for rapid failure detection. The potential use of seamless BFD was considered to simplify state management and improve scalability.

\subsubsection{SR Policy Programming RPC}
Zafar Ali presented the \href{https://datatracker.ietf.org/doc/html/draft-ali-spring-sr-policy-programming-rpc}{draft-ali-spring-sr-policy-programming-rpc}, advocating for RPC-based SR policy programming. The discussion questioned the necessity of new tools over existing YANG models and emphasized the need for a comparison of approaches.

\subsubsection{The Challenges and Requirements for Routing in Computing Cluster Networks}
Yizhou Li's \href{https://datatracker.ietf.org/doc/html/draft-li-rtgwg-computing-network-routing}{draft-li-rtgwg-computing-network-routing} was reviewed, aiming to document the unique routing requirements in computing clusters. The group discussed the potential for new strategies and the need for a detailed specification of missing elements.

\subsubsection{Routing in Satellite Networks: Challenges \& Considerations}
Tianji Jiang's \href{https://datatracker.ietf.org/doc/html/draft-lj-rtgwg-sat-routing-consideration}{draft-lj-rtgwg-sat-routing-consideration} explored the unique challenges of satellite network routing. The conversation highlighted the need for innovative routing technologies and the potential integration with terrestrial networks.

\subsubsection{IGP Color-Aware Shortcut}
Changwang Lin discussed the \href{https://datatracker.ietf.org/doc/html/draft-cheng-lsr-igp-shortcut-enhancement}{draft-cheng-lsr-igp-shortcut-enhancement}, focusing on enhancing IGP shortcuts with color-awareness. The necessity of formalizing local behaviors into an RFC was debated.

\subsubsection{Artificial Intelligence (AI) for Network Operations}
Cheng Li presented the \href{https://datatracker.ietf.org/doc/html/draft-king-rokui-ainetops-usecases}{draft-king-rokui-ainetops-usecases}, exploring AI applications in network operations. The discussion emphasized leveraging existing management technologies and the importance of event correlation in AI-driven network management.

Meeting materials are available at \href{https://datatracker.ietf.org/meeting/122/session/rtgwg}{RTGWG Meeting Materials}.




\newpage

\section{Secure Asset Transfer Protocol (SATP)}

\subsection{Attendees}
\subsubsection{Overview}
The SATP working group meeting was attended by representatives from prominent institutions such as MIT, IBM, Huawei, and ETH Zürich, totaling 21 participants. Notable attendees included Thomas Hardjono from MIT, Venkatraman Ramakrishna from IBM, and Wes Hardaker from USC/ISI and ICANN Board.

\subsection{Meeting Discussions}

\subsubsection{Chair Introduction}
Wes Hardaker and Claire Facer opened the meeting, setting the stage for discussions on the current status and future direction of SATP documents.

\subsubsection{Status Update on SATP Documents}
The chairs provided an update on the SATP documents, highlighting ongoing efforts to address critical issues such as public key encoding and the specification of acceptable values. The discussion referenced the \href{https://datatracker.ietf.org/doc/html/draft-ietf-satp-documents}{draft-ietf-satp-documents} for further details.

\subsubsection{Chair Review Results}
Wes Hardaker led a review of the SATP documents, identifying areas needing clarification, such as the creation of an IANA table for enumerations. The dialogue emphasized the importance of ensuring all fields are fully specified to facilitate interoperability.

\subsubsection{Recharter Open Discussion}
The recharter discussion, led by the chairs, explored the need for multiple implementations to support new charter items. Participants debated the scope of data sharing and metadata attachment, with references to \href{https://datatracker.ietf.org/doc/html/draft-ietf-satp-recharter}{draft-ietf-satp-recharter}. The conversation also touched on the requirements for network and gateway operations, suggesting a potential shift towards more comprehensive documentation.

\subsubsection{Stage-0 Considerations}
Thomas Hardjono presented on Stage-0 considerations, focusing on the pre-transfer context setup and its implications for asset identification and network interaction. The discussion highlighted the need for a global asset ID standard and the challenges of maintaining unique identifiers across diverse networks.

\subsubsection{SATP Implementation Guide}
Hyojin Song introduced the SATP Implementation Guide, emphasizing the importance of open-source implementations for cross-border payments. The guide aims to provide a framework for integrating SATP into existing systems, with a draft available at \href{https://datatracker.ietf.org/doc/html/draft-song-satp-implementation-guide}{draft-song-satp-implementation-guide}.

Meeting materials, including the agenda and video recordings, can be accessed via the \href{https://datatracker.ietf.org/meeting/122/session/satp}{datatracker link}.



\newpage

\section{Source Address Validation Networking (SAVNET)}

\subsection{Attendees}
\subsubsection{Overview}
The SAVNET working group meeting was attended by 62 participants, including representatives from prominent organizations such as China Telecom, Cisco Systems, Huawei, and Tsinghua University. The diverse attendance underscored the broad interest and collaborative effort in addressing source address validation challenges.

\subsection{Meeting Discussions}

\subsubsection{Intra-Domain and General Components}
The session began with Shuai Wang presenting the \href{https://datatracker.ietf.org/doc/draft-wang-sav-deployment-status/}{Source Address Validation Deployment Status}, highlighting the methodologies used to evaluate ISAV deployment in access networks. Discussions emphasized the need for diverse evaluation methods to ensure accurate results. Lancheng Qin's presentation on \href{https://datatracker.ietf.org/doc/draft-ietf-savnet-intra-domain-architecture/}{intra-domain SAVNET architecture} sparked dialogue on deployment flexibility and traffic engineering, with Jeffrey Haas noting the complexity of network-wide SAV procedures. Wei Wang's \href{https://datatracker.ietf.org/doc/draft-wang-savnet-intra-domain-solution-bm-spf/}{new intra-domain SAV solution} was critiqued for its assumptions about symmetric routing, leading to a consensus on the need for more adaptable solutions.

\subsubsection{Inter-Domain Architecture and Solutions}
Libin Liu presented a \href{https://datatracker.ietf.org/doc/draft-ietf-savnet-inter-domain-problem-statement/}{gap analysis} of inter-domain networks, prompting discussions on feasible path usage and the challenges of source spoofing. K. Sriram's update on \href{https://datatracker.ietf.org/doc/draft-ietf-sidrops-bar-sav/}{SAV using BGP UPDATE Messages} highlighted the need for directional accuracy in path selection. Nan Geng's \href{https://datatracker.ietf.org/doc/draft-geng-idr-bgp-savnet/}{inter-domain SAVNET solution} raised security concerns about SPD message origins, suggesting further offline discussions to address these issues.

\subsubsection{OAM \& Management}
Jing Zhao's presentation on \href{https://datatracker.ietf.org/doc/draft-tong-idr-bgp-ls-sav-rule/}{advertisement of multi-sourced SAV rules} using BGP Link-State was met with skepticism regarding protocol extensions within the WG. The discussion concluded with suggestions to refine the approach to distributing SAV rules. The session ended with Haiyang Zhang and Changwang Lin's \href{https://datatracker.ietf.org/doc/draft-li-savnet-sav-yang/}{YANG Data Model} presentation, which unfortunately ran out of time for comments.

Meeting materials are available at \href{https://datatracker.ietf.org/meeting/122/session/savnet}{SAVNET Meeting Materials}.



\newpage

\section{System for Cross-domain Identity Management (SCIM)}

\subsection{Attendees Overview}

The SCIM working group meeting was attended by 26 participants, including representatives from prominent organizations such as Cisco Systems, Okta, Amazon Web Services (AWS), Microsoft, and Ping Identity. The diverse attendance underscores the broad interest and collaborative effort in advancing identity management standards.

\subsection{Meeting Discussions}

\subsubsection{Chairs Intro}

The meeting commenced with a brief introduction by the chairs, covering the agenda and procedural notes. A query was raised regarding the adoption of Delta Queries, with a suggestion to seek updates via the mailing list.

\subsubsection{SCIM Use Cases}

Pam and Paulo presented an overview of the \href{https://datatracker.ietf.org/doc/html/draft-ietf-scim-use-cases}{draft-ietf-scim-use-cases}, emphasizing the need for common terminology and concepts to ensure consistency across drafts. The presentation included generic and specialized use cases, with a call for volunteers to review the document. Meeting materials are available \href{https://datatracker.ietf.org/meeting/122/materials/slides-122-scim-use-cases}{here}.

\subsubsection{SCIM Roles and Entitlements}

Aaron discussed the \href{https://datatracker.ietf.org/doc/html/draft-ietf-scim-roles-entitlements}{draft-ietf-scim-roles-entitlements}, highlighting its adoption and current implementation by Okta. The draft extends core schema roles and entitlements, with proposed changes pending review. The session concluded with a call for community feedback to advance the draft.

\subsubsection{Update from the OpenID IPSIE Working Group}

Aaron provided insights into the OpenID IPSIE Working Group's focus on enterprise identity challenges, emphasizing interoperability and security. The discussion highlighted the overlap with SCIM drafts and the importance of collaboration to avoid isolated profiling efforts. Feedback from SCIM implementers was solicited to align IPSIE's identity lifecycle goals with SCIM's work.

\subsubsection{AOB}

The meeting concluded with an open floor for additional business, with no further topics raised.

\subsection{Conclusion}

The SCIM working group continues to make strides in refining identity management standards, with active participation from key industry players. The discussions underscored the importance of collaboration and alignment across related initiatives, setting the stage for future advancements in the field.



\newpage

\section{Supply Chain Integrity, Transparency, and Trust (SCITT)}

\subsection{Attendees Overview}
\subsubsection{Attendees}
The SCITT working group meeting was attended by 40 participants, including representatives from prominent organizations such as Carnegie Mellon University, Microsoft, Cloudflare, and Keio University. Notable attendees included Christopher Inacio from Carnegie Mellon University, Nicole Bates from Microsoft, and Thibault Meunier from Cloudflare.

\subsection{Meeting Discussions}

\subsubsection{SCITT Overview}
Henk provided an update on the SCITT framework, emphasizing the support for multiple vendors to submit statements about an artifact. The discussion included a review of update streams and the architectural overview, highlighting improvements such as a more intuitive diagram and technical refinements to the architecture. The \href{https://datatracker.ietf.org/doc/html/draft-ietf-scitt-scrapi}{draft-ietf-scitt-scrapi} was discussed, with a focus on the adoption of Merkle tree proofs by ISEG.

\subsubsection{Introduction to Transparency 201}
Jon presented on the importance of quality statements within the SCITT framework, ensuring statements cannot be backdated and emphasizing the need for verifiable record-keeping. The discussion covered scenarios such as tamper-proofing and pre-commitment, with a focus on business use cases rather than personal identities. The presentation underscored the role of transparency services in providing non-equivocation and maintaining the integrity of statements.

\subsubsection{SCRAPI and SCITT Transparency Services}
Steve discussed the architecture of SCITT transparency services, including the append-only log and the choices around metadata protection. The conversation addressed the implications of unprotected headers and the potential for selective disclosure. The \href{https://datatracker.ietf.org/doc/html/draft-ietf-scitt-scrapi}{draft-ietf-scitt-scrapi} was referenced to clarify what should be stored in the transparency log.

\subsubsection{Hackathon Report}
Jon and Nobuo Aoki reported on their hackathon efforts, which included transitioning from JSON to CBOR and addressing challenges related to asynchronous post/receipt processes. The session highlighted the successful implementation of a YANG model for SBOMs and the potential for making statements about statements.

\subsubsection{Next Steps}
The group discussed the readiness of the architecture for last call and the possibility of scheduling a virtual interim meeting to review feedback. The conversation also touched on the future of use case documents within the IETF framework.

\subsubsection{AOB Open Mic}
The open mic session invited discussions on scheduling future meetings and the publication of use case documents. Chris clarified the IETF's current stance on use case documents, and Henk proposed posting new documents outside the IETF framework.

\subsubsection{Wrap-up and Conclusion}
The meeting concluded with a summary of the discussions and a reiteration of the next steps, including the potential for a virtual interim meeting to address any significant feedback received.

Meeting materials are available at \href{https://www.ietf.org/proceedings/122/scitt.html}{IETF-122 SCITT Meeting Materials}.



\newpage

\section{Standard Communication with Network Elements (SCONE) WG}

\subsection{Attendees}

The SCONE working group meeting at IETF 122 in Bangkok saw participation from 97 attendees, including representatives from prominent companies and institutions such as Google, Meta, Ericsson, and Verizon. Notable attendees included Brian Trammell (Google), Qin Wu (Huawei), Martin Thomson (Mozilla), and Sanjay Mishra (Verizon).

\subsection{Meeting Discussions}

\subsubsection{Introduction}

The meeting commenced with the chairs, Brian Trammell and Qin Wu, emphasizing the importance of progressing with experimentation to refine the working group's objectives and deliverables.

\subsubsection{Work Item 1: Protocol Framing}

The discussion on the Merged Protocol Proposal: TRONE protocol, led by Marcus Ihlar and Martin Thomson, focused on the negotiability of parameters such as \texttt{b\_min} and \texttt{b\_max}, and the encoding of units. The dialogue highlighted the complexity of dynamic encoding and the potential for ossification with new versions. The proposal aims to provide guidance on throughput measurement and address concerns about network element injection. The group considered the implications of TRONE hints for network classification and the potential for deployment challenges with existing DPI practices. The discussion concluded with a poll indicating strong support for the proposal as a basis for a WG protocol specification, pending further security considerations.

\subsubsection{Work Item 2: Applicability and Manageability}

Sanjay Mishra presented the SCONE Mobile Network Use Case, which sparked a discussion on whether TRONE meets all the outlined requirements. The group acknowledged that while some requirements might not be fully addressed, they could be refined through ongoing discussions. Kyriakos Zarifis's presentation on Throttle Policies Taxonomy and Impact emphasized the need for subscriber-based throttling, which was noted as a critical aspect for consideration.

\subsubsection{Conclusion}

The chairs concluded the meeting by outlining an aggressive schedule for virtual interim meetings to ensure significant progress before the next IETF meeting in Madrid. The group plans to focus on Hackathon goals and appoint editors for the TRONE document. The discussion also touched on the potential need for additional documents to capture comprehensive requirements and semantics.

Meeting materials are available at \href{https://meetings.conf.meetecho.com/ietf122/?session=33930}{SCONE Meeting Materials}.



\newpage

\section{SECDISPATCH/DISPATCH Hybrid Meeting (SECDISPATCH)}

\subsection{Attendees Overview}
The SECDISPATCH/DISPATCH Hybrid Meeting at IETF122 saw participation from 164 attendees, including representatives from prominent organizations such as Google, Cisco Systems, Cloudflare, and the ACLU. The diverse attendance underscored the broad interest in the topics discussed, reflecting a wide array of expertise and perspectives.

\subsection{Meeting Materials}
Access the meeting materials via \href{https://meetings.conf.meetecho.com/ietf122/?session=33976}{MeetEcho}, \href{https://notes.ietf.org/notes-ietf-122-secdispatch}{Notes}, and \href{https://zulip.ietf.org/#narrow/stream/secdispatch}{Zulip}.

\subsection{Meeting Discussions}

\subsubsection{Recommendations for Key Directories over HTTP}
Presenter: Thibault Meunier (Remote)  
The discussion centered around the \href{https://datatracker.ietf.org/doc/draft-darling-key-directory-over-http/}{draft-darling-key-directory-over-http}, focusing on the integrity and security of passing public keys over HTTP. Key points included the necessity of tying public keys to their intended use and the potential dispatch to the JOSE working group. The conversation highlighted the need for a unified approach to key rotation and caching, with suggestions to continue discussions on the secdispatch mailing list.

\subsubsection{Organization Trust Relationship Protocol}
Presenter: Ralph W. Brown (Remote)  
The \href{https://datatracker.ietf.org/doc/draft-org-trust-relationship-protocol/}{draft-org-trust-relationship-protocol} sparked debate over its complexity and the appropriateness of its AD sponsorship. Concerns were raised about the document's broad scope and the lack of a clear security model. The discussion concluded with a suggestion to explore interest in creating a dedicated mailing list for further deliberation.

\subsubsection{LDAP Additional Syntaxes}
Presenter: Carl Eric Codère (Remote)  
The presentation on \href{https://datatracker.ietf.org/doc/draft-codere-ldapsyntax/}{draft-codere-ldapsyntax} addressed the need for updated syntax in line with modern ASN1 standards. The proposal was well-received, with suggestions to coordinate discussions with ART and WIT ADs to determine the appropriate working group for further development.

\subsection{Summary of Dispatch Actions}
The meeting concluded with a consensus to continue discussions on the secdispatch mailing list for the Key Directories over HTTP draft, while the Trust Relationship Protocol and LDAP Additional Syntaxes drafts were directed towards the ART area for further exploration. These actions aim to refine the proposals and align them with existing standards and practices, potentially influencing future protocol developments.



\newpage

\section{Secure Inter-Domain Routing Operations (SIDROPS)}

\subsection{Attendees Overview}
\subsubsection{Attendance}
The SIDROPS session at IETF-122 was attended by 80 participants, featuring representatives from prominent organizations such as the Number Resource Organization, BIRD, Vigil Security, Arrcus, US NIST, RIPE NCC, Verisign, Keio University, and Juniper Networks, among others.

\subsection{Meeting Discussions}

\subsubsection{NRO RPKI Program for 2025}
Sofia Silva Berenguer presented the \href{https://datatracker.ietf.org/meeting/122/materials/slides-122-sidrops-nro-rpki-program-update-sofia-00}{NRO RPKI Program Update}, outlining strategic plans for the upcoming year. The presentation was well-received, with no questions from the audience, indicating a clear understanding and alignment with the proposed direction.

\subsubsection{ASPA-based AS Path Verification Examples}
K Sriram discussed the \href{https://datatracker.ietf.org/meeting/122/materials/slides-122-sidrops-aspa-based-as-path-verification-examples-and-unit-tests-00}{ASPA-based AS Path Verification Examples and Unit Tests}, reporting four implementations of the \href{https://datatracker.ietf.org/doc/html/draft-ietf-sidrops-aspa-verification-20}{draft-ietf-sidrops-aspa-verification-20}. The session concluded without questions, suggesting consensus on the approach.

\subsubsection{RPKI Repository Problem Statement and Requirements}
Yingying Su's presentation on the \href{https://datatracker.ietf.org/meeting/122/materials/slides-122-sidrops-rpki-repository-problem-statement-and-requirements-00}{RPKI Repository Problem Statement and Requirements} sparked a discussion about the proposal's intent. It was clarified that the subordinate subject, not the CA, determines the publication point, aiming to enhance cooperation among repository operators.

\subsubsection{YANG Data Model for RPKI to Router Protocol}
Jishnu Roy updated attendees on the \href{https://datatracker.ietf.org/meeting/122/materials/slides-122-sidrops-yang-data-model-for-rpki-to-router-protocol-00}{YANG Data Model for RPKI to Router Protocol}, with plans for a call for adoption post-IETF 122. The discussion included whether the model should be AFI-distinguished, indicating ongoing refinement.

\subsubsection{ASPA-based AS\_PATH Verification for BGP Export}
Jia Zhang proposed \href{https://datatracker.ietf.org/meeting/122/materials/slides-122-sidrops-aspa-based-as-path-verification-for-bgp-export-slides-122-sidrops-aspa-egress-01-00}{ASPA-based AS\_PATH Verification for BGP Export}, suggesting its necessity for an RFC. The proposal's utility was debated, with some questioning its practical application.

\subsubsection{RPKI Terminology}
Tim Bruijnzeels introduced the \href{https://datatracker.ietf.org/meeting/122/materials/slides-122-sidrops-07-rpki-terminology-00}{RPKI Terminology} draft, proposing a call for adoption soon. This indicates a move towards standardizing terminology within the RPKI framework.

\subsubsection{RDAP Extension for RPKI Registration Data}
Andy Newton's \href{https://datatracker.ietf.org/meeting/122/materials/slides-122-sidrops-an-rdap-extension-for-rpki-registration-data-00}{RDAP Extension for RPKI Registration Data} presentation led to discussions on including message digests in RDAP reports and the scope of RDAP tools, highlighting the need for clarity in data handling.

\subsubsection{Source Address Validation Using SOAs}
Minglin Jia's presentation on \href{https://datatracker.ietf.org/meeting/122/materials/slides-122-sidrops-source-address-validation-using-source-origin-authorizations-soas-00}{Source Address Validation Using SOAs} raised concerns about operational complexity, suggesting further refinement is needed.

\subsubsection{Using Forwarding Commitments to Verify BGP AS\_PATH}
Yangfei Guo discussed \href{https://datatracker.ietf.org/meeting/122/materials/slides-122-sidrops-using-forwarding-commitments-fcs-to-verify-bgp-as-path-00}{Using Forwarding Commitments (FCs) to Verify BGP AS\_PATH}, with questions about data volume and deployment scope, indicating the need for further exploration.

\subsubsection{RPKI Relying Party with Distributed Systems of Synchronization Nodes}
Di Ma's \href{https://datatracker.ietf.org/meeting/122/materials/slides-122-sidrops-rpki-relying-party-with-distributed-systems-of-synchronization-nodes-00}{RPKI Relying Party with Distributed Systems of Synchronization Nodes} presentation discussed task assignment suitability, suggesting a focus on optimizing synchronization processes.

Meeting materials are available at \href{https://datatracker.ietf.org/meeting/122/materials/agenda-122-sidrops-00}{IETF-122 SIDROPS Meeting Materials}.



\newpage

\section{Structured Messaging Language (SML) Working Group}

\subsection{Attendees}
The SML Working Group meeting was attended by 26 participants, including representatives from prominent companies and institutions such as Google, Fastmail, Proton, and the Max-Planck Institute for Informatics. Notable attendees included Hans-Jörg Happel from audriga GmbH, Alexey Melnikov from Isode Limited, and Daniel Huigens from Proton.

\subsection{Meeting Discussions}

\subsubsection{Structured Vacation Notices}
Hans-Jörg Happel presented the draft \href{https://datatracker.ietf.org/doc/html/draft-ietf-sml-structured-vacation-notices}{draft-ietf-sml-structured-vacation-notices}, focusing on the scope of "Vacation notices" or "Out-of-office replies." The discussion centered around whether to keep the proposal simple with limited use cases or to expand it. The consensus was to avoid complexity that might hinder industry adoption, such as supporting multiple absence periods. The proposal to use schema.org types and an optional "opening hours" field was debated, with concerns about feature creep. The group agreed on the potential need for a registry of predefined values and considered the appropriateness of the term "vacation notice."

\subsubsection{Structured Email}
The draft \href{https://datatracker.ietf.org/doc/html/draft-ietf-sml-structured-email}{draft-ietf-sml-structured-email} was discussed, with Hans-Jörg Happel highlighting the need for an IANA registry for RDF vocabularies used in SML. The group explored the possibility of creating email-specific templates outside the RFC document and the challenges of using https URLs versus URNs. The discussion also covered multipart email structures and the handling of json-ld parts by email clients. The potential for using SML for email signatures and the need for error/status codes in structured emails were considered, with suggestions to use existing headers like SUPERSEDES for indicating outdated data.

Meeting materials are available at \href{https://meetings.conf.meetecho.com/ietf122/?group=sml&short=&item=1}{Meetecho} and \href{https://notes.ietf.org/notes-ietf-122-sml}{Notes}.

\subsubsection{Outcomes and Next Steps}
The meeting concluded with several action items, including Hans-Jörg Happel's task to raise the question of extending the schema with an "unavailabilityType" field on the mailing list. The group recognized the need for further experimentation and discussion on handling structured data in emails, particularly regarding json-ld fragments and the use of content-id. The potential impact of these discussions could lead to significant advancements in how structured data is managed and utilized in email communications, influencing future standards and implementations.



\newpage

\section{Stub Network Auto Configuration (SNAC)}

\subsection{Attendees}
\subsubsection{Overview}
The SNAC working group meeting was attended by 23 participants, including representatives from prominent companies and institutions such as Cisco, Huawei Technologies, Apple, Google, and Siemens. Notable attendees included Darren Dukes (Cisco), David Lou (Huawei Technologies), Ted Lemon (Apple), and Jen Linkova (Google).

\subsection{Meeting Discussions}

\subsubsection{Administrative \& Agenda Bashing}
The session commenced with a brief administrative overview and agenda bashing led by the chairs, Darren Dukes and David Lou.

\subsubsection{Multiple Prefixes for Delegation}
The discussion on multiple IA\_PREFIX options in an IA\_PD response highlighted the complexities of prefix selection in both unconstrained and constrained networks. Ted Lemon (Apple) emphasized the importance of using both ULA and GUA prefixes, noting that while GUA provides global connectivity, ULA ensures local network stability. Éric Vyncke (Cisco) proposed selecting one ULA and one GUA due to their distinct properties. The debate underscored the need for a balanced approach, considering the constraints of SNAC routers and the potential for different behaviors across stub network technologies. The conversation referenced \href{https://github.com/ietf-wg-snac/draft-ietf-snac-simple/issues/99}{draft-ietf-snac-simple/issues/99} for further technical details.

\subsubsection{Automatically Connecting Stub Networks to Unmanaged Infrastructure}
The session addressed updates to the \href{https://datatracker.ietf.org/doc/draft-ietf-snac-simple/}{draft-ietf-snac-simple} document, focusing on resolving open issues. Key topics included the clarification of RA Guard in Appendix A and the behavior of stub routers. The discussion concluded with a consensus on the need for interim meetings every two weeks to expedite the document's finalization.

\subsubsection{Any Other Business (AOB)}
The meeting concluded with an open floor for any other business, allowing participants to raise additional topics or concerns.

Meeting materials are available at \href{https://meetings.conf.meetecho.com/ietf122/?session=34032}{IETF 122 SNAC Session}.



\newpage

\section{Secure Provisioning of Identities in Constrained Environments (SPICE)}

\subsection{Attendees Overview}

The SPICE working group meeting was attended by 47 participants, including representatives from prominent organizations such as Nokia, Microsoft, Mozilla, Cloudflare, and Cisco Systems. The diverse attendance underscores the broad interest and collaborative effort in advancing identity provisioning standards.

\subsection{Meeting Discussions}

\subsubsection{Welcome and Introduction}

The session commenced with an introduction to SPICE, highlighting its objectives and the significance of secure identity provisioning in constrained environments. The \href{https://datatracker.ietf.org/meeting/122/materials/slides-122-spice-chair-slides-00}{meeting materials} are available for further details.

\subsubsection{SPICE SD-CWT - Rohan Mahy}

Rohan Mahy presented updates on the \href{https://datatracker.ietf.org/doc/html/draft-ietf-spice-sd-cwt}{draft-ietf-spice-sd-cwt}, focusing on selective disclosure mechanisms. Discussions revolved around the integration of decoys and the necessity for nested disclosures, which could enhance privacy while maintaining data integrity. The session concluded with a call for continued work on open issues and interoperability testing.

\subsubsection{SPICE GLUE - Brent Zundel}

Brent Zundel introduced the concept of GLobal Unique Enterprise Identifiers through the \href{https://datatracker.ietf.org/doc/html/draft-zundel-spice-glue-id}{draft-zundel-spice-glue-id}. The proposal aims to standardize organizational identifiers using URNs, addressing the need for consistency across various ID schemes. The group discussed namespace considerations, with a consensus to explore working group adoption.

\subsubsection{OpenID Connect Claims for CBOR Web Tokens - Beltram Maldant}

Beltram Maldant discussed the registration of OpenID Connect claims for CBOR Web Tokens, as detailed in the \href{https://datatracker.ietf.org/doc/html/draft-maldant-spice-oidc-cwt}{draft-maldant-spice-oidc-cwt}. The proposal received positive feedback, with participants supporting its adoption due to its practical utility in real-world JWT applications.

\subsubsection{Public Key Service Provider - Donghui Wang}

Donghui Wang presented a framework for a Public Key Service Provider, outlined in the \href{https://datatracker.ietf.org/doc/html/draft-wang-spice-public-key-service-provider}{draft-wang-spice-public-key-service-provider}. The discussion highlighted the complexities of managing public keys across multiple issuers and verifiers, with suggestions to evaluate existing mechanisms before pursuing new designs.

\subsubsection{Updating Use Case Draft - Brent Zundel}

Brent Zundel provided an update on the use case draft, emphasizing the need for comprehensive examples to guide implementation. The \href{https://datatracker.ietf.org/doc/draft-ietf-spice-use-cases/}{draft-ietf-spice-use-cases} has been revised to incorporate new scenarios, inviting feedback from the community.

\subsubsection{Any Other Business (AOB)}

The session concluded with a discussion on the potential development of an architecture document, which could serve as a valuable resource for both current and future SPICE initiatives. Participants acknowledged the challenges but recognized the importance of establishing a clear protocol framework.



\newpage

\section{SPRING (Source Packet Routing in Networking)}

\subsection{Attendees}

The SPRING working group meeting at IETF-122 was attended by 116 participants, including representatives from prominent companies and institutions such as Juniper Networks, Cisco Systems, Nokia, Huawei Technologies, and China Mobile. The meeting was chaired by Alvaro Retana, Bruno Decraene, and Joel M. Halpern, with Shuping Peng serving as the secretary and minutes taker.

\subsection{Meeting Discussions}

The meeting materials can be accessed via \href{https://meetings.conf.meetecho.com/ietf122/?session=33982}{MeetEcho}, \href{https://notes.ietf.org/notes-ietf-122-spring}{Notes}, and \href{https://zulip.ietf.org/#narrow/stream/spring}{Zulip}.

\subsubsection{SRv6 Security Considerations}

Presenter: Tal Mizrahi  
\href{https://datatracker.ietf.org/doc/draft-ietf-spring-srv6-security/}{draft-ietf-spring-srv6-security}

The discussion focused on the security considerations for SRv6, highlighting the need for infrastructure address range filtering and the role of network management systems (NMS) in control plane operations. The IESG has requested progress on this work, emphasizing the need for working group engagement and public discussions on the mailing list.

\subsubsection{Eligibility Concept in Segment Routing Policies}

Presenter: Himanshu Shah  
\href{https://datatracker.ietf.org/doc/draft-karboubi-spring-sr-policy-eligibility/}{draft-karboubi-spring-sr-policy-eligibility}

The eligibility concept introduces an additional check for segment routing policies, with discussions on the interaction between local and PCE-based eligibility checks. The need for consistency with existing RFCs and the potential for a new RFC9256bis were debated, with a call for further discussion on the mailing list.

\subsubsection{Circuit Style Segment Routing Policies with Optimized SID List Depth}

Presenter: Himanshu Shah  
\href{https://datatracker.ietf.org/doc/draft-karboubi-spring-sidlist-optimized-cs-sr/}{draft-karboubi-spring-sidlist-optimized-cs-sr}

This presentation explored optimized SID list depth for circuit-style segment routing policies. The discussion centered on implementation specifics and the architectural implications of using node versus adjacency SIDs, with a recommendation to continue the conversation on the mailing list.

\subsubsection{Flexible Candidate Path Selection of SR Policy}

Presenter: Yisong Liu  
\href{https://datatracker.ietf.org/doc/draft-liu-spring-sr-policy-flexible-path-selection/}{draft-liu-spring-sr-policy-flexible-path-selection}

The proposal for flexible candidate path selection was discussed, with emphasis on aligning eligibility criteria across use cases. The need for consistency with RFC9256 and the potential for updating it were highlighted, with a suggestion to summarize the discussions on the mailing list.

\subsubsection{SR-MPLS Aggregation Segment}

Presenter: Bruno Decraene  
\href{https://datatracker.ietf.org/doc/draft-decraene-spring-sr-mpls-aggregation-segment/}{draft-decraene-spring-sr-mpls-aggregation-segment}

The SR-MPLS aggregation segment was presented, focusing on the scalability benefits of double push operations in PEs. The discussion included considerations for compatibility with LDP switching and the need for further exploration of this aspect.

\subsubsection{SRv6 SFC Architecture with SR-aware Functions}

Presenter: Wataru Mishima  
\href{https://datatracker.ietf.org/doc/draft-watal-spring-srv6-sfc-sr-aware-functions/}{draft-watal-spring-srv6-sfc-sr-aware-functions}

The architecture for SRv6 SFC with SR-aware functions was discussed, with an emphasis on collaboration and input from the Hackathon experience. The need for alignment with related drafts was noted.

\subsubsection{Advertise SRv6 Locator Information by IPv6 Neighbor Discovery}

Presenter: Liyan Gong / Changwang Lin  
\href{https://datatracker.ietf.org/doc/draft-gong-spring-nd-advertise-srv6-locator/}{draft-gong-spring-nd-advertise-srv6-locator}

The proposal to advertise SRv6 locator information via IPv6 Neighbor Discovery raised questions about motivation, security considerations, and the trust model, particularly in data center environments. The need for further discussion and review by related working groups was emphasized.

\subsubsection{SRv6 Path Verification}

Presenter: Feng Yang  
\href{https://datatracker.ietf.org/doc/draft-yang-spring-srv6-verification/}{draft-yang-spring-srv6-verification}

The SRv6 path verification draft was discussed, with comparisons to IOAM Proof of Transit and the complexity of path verification. The importance of considering enhancements to RFC8754 and engaging with the 6man working group was highlighted.

\subsubsection{Benchmarking Methodology for Segment Routing}

Presenter: Paolo Volpato  
\href{https://datatracker.ietf.org/doc/draft-ietf-bmwg-sr-bench-meth/}{draft-ietf-bmwg-sr-bench-meth}

The benchmarking methodology for segment routing was presented, with a call for comments and feedback from the working group to refine the draft.

\subsubsection{Export of Path Segment Identifier Information in IPFIX}

Presenter: Yao Liu  
\href{https://datatracker.ietf.org/doc/draft-liu-opsawg-ipfix-path-segment/}{draft-liu-opsawg-ipfix-path-segment}

The export of path segment identifier information in IPFIX was discussed, with support expressed for the draft. Although not a SPRING draft, engagement and review on the mailing list were encouraged.

Overall, the meeting underscored the importance of working group engagement, consistency across drafts, and the need for public discussions to advance the work effectively.



\newpage

\section{Segment Routing over IPv6 Operations (SRv6OPS)}

\subsection{Attendees Overview}
\subsubsection{Attendance and Representation}
The SRv6OPS working group meeting was attended by 88 participants, featuring representatives from prominent companies and institutions such as Turkcell, China Telecom, Ericsson, Cisco, Huawei, and Deutsche Telekom. The diverse attendance underscored the broad interest and collaborative effort in advancing SRv6 technologies.

\subsection{Meeting Discussions}

\subsubsection{Operator Presentations}
\textbf{SRv6 at Turkcell:} Mehmet Durmus presented Turkcell's strategic shift towards SRv6 to enhance network scalability and flexibility. The discussion highlighted Turkcell's successful field tests with two vendors and plans for multi-vendor testing and live deployment. The transition from SR-MPLS to SRv6 was driven by the need for better scalability during the pandemic-induced traffic surge. Turkcell's roadmap includes testing compressed SIDs and deploying SR controllers by 2027.

\textbf{SRv6 Practice in China Telecom:} Yongqing Zhu detailed China Telecom's SRv6 deployment, focusing on SID block allocation and inter-AS deployment. The choice of BGP-SR over PCEP for SR policy provisioning was discussed, alongside the use of both HBH and E2E telemetry modes. The presentation emphasized the operational considerations and reliability solutions in place.

\subsubsection{Internet-Draft Presentations}
\textbf{Problem Summary:} Yisong Liu introduced the \href{https://datatracker.ietf.org/doc/html/draft-liu-srv6ops-problem-summary}{draft-liu-srv6ops-problem-summary}, outlining current deployment problems (DOPs) without solutions. Feedback suggested refining the draft to clearly distinguish between problems and requirements.

\textbf{Deployment Options:} Michael McBride presented the \href{https://datatracker.ietf.org/doc/html/draft-mcbride-srv6ops-srv6-deployment}{draft-mcbride-srv6ops-srv6-deployment}, which explores SRv6 deployment strategies. The draft was encouraged to include more illustrative content and address migration from various network technologies.

\textbf{IPv6 Address Assignment for SRv6:} Yisong Liu's \href{https://datatracker.ietf.org/doc/html/draft-liu-srv6ops-sid-address-assignment}{draft-liu-srv6ops-sid-address-assignment} was presented without questions, indicating consensus or clarity on the topic.

\textbf{SID Space Inter-domain Addressing:} Erik Kline discussed the \href{https://datatracker.ietf.org/doc/html/draft-eknb-srv6ops-interdomain-sidspace}{draft-eknb-srv6ops-interdomain-sidspace}, focusing on inter-provider collaboration using ASNs. The draft remains informational, with ongoing discussions about its experimental status.

\textbf{Benchmarking Methodology:} Paolo Volpato's \href{https://datatracker.ietf.org/doc/html/draft-ietf-bmwg-sr-bench-meth}{draft-ietf-bmwg-sr-bench-meth} was briefly discussed, with emphasis on the need for expert review of C-SID aspects.

\textbf{Best Practices for Protection of SRv6 Networks:} Although time constraints limited discussion, Changwang Lin's \href{https://datatracker.ietf.org/doc/html/draft-liu-srv6ops-sr-protection}{draft-liu-srv6ops-sr-protection} remains a critical area for future exploration.

Meeting materials, including slides and recordings, are available at \href{https://datatracker.ietf.org/meeting/122/session/srv6ops}{IETF 122 SRv6OPS Session Materials}.




\newpage

\section{Secure Shell Maintenance (SSHM) WG}

\subsection{Attendees}
\subsubsection{Overview}
The SSHM working group meeting was attended by 63 participants, including representatives from prominent organizations such as NSA-CCSS, Cisco Systems, Red Hat, Google, and Huawei. Notable attendees included Job Snijders, Stephen Farrell, Damien Miller, and Eric Rescorla.

\subsection{Meeting Discussions}

\subsubsection{Administrivia}
The meeting commenced with a reminder of the IETF's Note Well principles, emphasizing the importance of respectful and constructive interactions among participants.

\subsubsection{Document Status}
The discussion on document status focused on the \href{https://datatracker.ietf.org/doc/html/draft-ietf-sshm-ntruprime}{draft-ietf-sshm-ntruprime}, which has completed the Working Group Last Call (WGLC) phase. The chairs concluded that the IANA registry entry should be marked as SHOULD, despite an appeal from Eric Rescorla regarding consensus clarity. The decision was revisited through an email poll, resulting in a rough consensus for the SHOULD marking. The draft is pending another revision, with Job Snijders addressing the shepherd writeup and ensuring all comments are considered.

\subsubsection{Terrapin Attack and Strict KEX}
Damien Miller presented on the Terrapin attack and the proposed Strict KEX solution. The discussion highlighted the need for cryptographic review and potential formal analysis of the fix. Participants debated the implications of client-server agreement on strict KEX protection, with Damien suggesting that connections should be terminated if protections are not agreed upon. The group acknowledged the necessity of documenting the current workaround while considering future enhancements, such as a full transcript hash and real Key Derivation Function (KDF).

\subsubsection{OpenSSH Certificate Format}
Damien Miller also introduced the OpenSSH certificate format, discussing its structure and the need for a registry to track extensions. The format includes critical options and non-critical extensions, with the latter being non-essential for operation. The discussion touched on the flexibility of signature algorithms and the potential benefits of using a hash of the public key for efficiency.

\subsubsection{AOB}
The meeting concluded slightly ahead of schedule, with a brief discussion on additional business as time permitted.

Meeting materials and slides are available at \href{https://meetings.conf.meetecho.com/ietf122/?session=33762}{Meetecho IETF 122 Session 33762}.




\newpage

\section{Secure Telephone Identity Revisited (STIR)}

\subsection{Attendees Overview}
\subsubsection{Attendance}
The STIR working group meeting at IETF 122 was attended by 24 participants, including representatives from prominent organizations such as Cisco, Deutsche Telekom, Meta Platforms, Inc., and TransUnion. Notable attendees included Russ Housley from Vigil Security, Eric Rescorla, and Jon Peterson from TransUnion.

\subsection{Meeting Discussions}

\subsubsection{Certificates}
Chris Wendt presented the \href{https://datatracker.ietf.org/doc/html/draft-wendt-stir-certificate-transparency-05}{draft-wendt-stir-certificate-transparency-05}, focusing on the application of Certificate Transparency (CT) within STIR. The discussion revolved around whether CT could be utilized without protocol modifications and how verifiers could ensure certificates are logged. The consensus was to restructure the document to clarify the reliance on CT without protocol changes, with an adoption call pending the revised document.

Chris also introduced the \href{https://datatracker.ietf.org/doc/html/draft-wendt-acme-authority-token-jwtclaimcon-00}{draft-wendt-acme-authority-token-jwtclaimcon-00}, which will be formally presented in the ACME working group.

\subsubsection{VESPER - Verifiable STI Personas}
The \href{https://datatracker.ietf.org/doc/html/draft-wendt-stir-vesper-use-cases-00}{draft-wendt-stir-vesper-use-cases-00} was presented by Chris Wendt, highlighting use cases for VESPER. Discussions focused on the differentiation between STIR and WebPKI in terms of display names and caller details. The group agreed to refine use cases and trust models for further discussion, emphasizing the need for accountable verification authorities.

\subsubsection{Caller ID Verification (CIV)}
Feng Hao presented the \href{https://datatracker.ietf.org/doc/html/draft-hao-civ-00}{draft-hao-civ-00}, initially submitted to DISPATCH but redirected to STIR. The group debated the draft's premise, with some objections regarding its assumptions about STIR's capabilities. The discussion was cut short due to time constraints, leaving no action items.

\subsubsection{Any Other Business (AoB)}
Orie Steele announced Robert Sparks' departure as chair, with Russ Housley and Ben Campbell continuing as co-chairs. The group expressed gratitude for Robert's contributions.

Meeting materials can be accessed via \href{https://www.ietf.org/proceedings/122/stir.html}{IETF 122 STIR Meeting Materials}.




\newpage

\section{Software Updates for Internet of Things (SUIT)}

\subsection{Attendees}
\subsubsection{Overview}
The SUIT working group meeting at IETF 122 was attended by 30 participants, including representatives from prominent organizations such as Crypto4A Inc., Arm Ltd., Ericsson, and the University of Applied Sciences Bonn-Rhein-Sieg. The diverse attendance underscored the broad interest and collaborative effort in advancing secure firmware updates for IoT devices.

\subsection{Meeting Discussions}

\subsubsection{Firmware Encryption with SUIT Manifests}
The discussion on \href{https://datatracker.ietf.org/doc/html/draft-ietf-suit-firmware-encryption-23}{draft-ietf-suit-firmware-encryption-23} focused on addressing feedback from the previous meeting in Dublin. The draft is ready for a new submission, pending a quick Working Group Last Call (WGLC) if no objections arise. The emphasis was on ensuring all comments, particularly those from Orie, were adequately addressed.

\subsubsection{SUIT Manifest Format}
The \href{https://datatracker.ietf.org/doc/html/draft-ietf-suit-manifest-33}{draft-ietf-suit-manifest-33} has undergone significant updates, including moving the conformance matrix to the main body and updating the UUID to RFC2562. The draft is in the editor's queue, with IANA feedback pending resolution.

\subsubsection{SUIT Manifest Extensions for Multiple Trust Domains}
Minor updates were made to the \href{https://datatracker.ietf.org/doc/html/draft-ietf-suit-trust-domains-10}{draft-ietf-suit-trust-domains-10}, focusing on clarifying the intended audience and typical isolation levels. The draft is scheduled for a telechat on April 17.

\subsubsection{Secure Reporting of Update Status}
The \href{https://datatracker.ietf.org/doc/html/draft-ietf-suit-report-11}{draft-ietf-suit-report-11} has been elevated from an informative to a standard status, with a publication request submitted. IANA feedback is being addressed.

\subsubsection{Strong Assertions of IoT Network Access Requirements}
The \href{https://datatracker.ietf.org/doc/html/draft-ietf-suit-mud-10}{draft-ietf-suit-mud-10} draft, now in the RFC editor's queue, highlights the pros and cons of using SUIT-MUD, with a focus on device certificates and configuration requirements. An IANA action was triggered by the March 3 update.

\subsubsection{Mandatory-to-Implement Algorithms for SUIT Manifests}
The \href{https://datatracker.ietf.org/doc/html/draft-ietf-suit-mti-13}{draft-ietf-suit-mti-13} clarifies profiles for constrained node use cases and expands on defenses against chosen plaintext attacks. The draft will align with the jose-fully-spec-algorithms draft before proceeding to an IETF Last Call.

\subsubsection{Update Management Extensions for SUIT Manifests}
The \href{https://datatracker.ietf.org/doc/html/draft-ietf-suit-update-management-09}{draft-ietf-suit-update-management-09} is ready for progression, with references to CoRIM removed. The chairs are prepared to advance the draft.

\subsubsection{Other Discussions}
Time permitting, discussions included syncing I.D. on the datatracker and GitHub, and best practices for maintaining drafts requiring CDDL files. Akira raised a query about CDDL practices, with Brendan suggesting post-publication challenges and the preference for full CDDL at the end of drafts.

Meeting materials are available at \href{https://notes.ietf.org/notes-ietf-122-suit}{notes.ietf.org/notes-ietf-122-suit}.



\newpage

\section{TCP Maintenance and Minor Extensions (TCPM)}

\subsection{Attendees Overview}
The TCPM meeting at IETF-122 in Bangkok saw participation from 43 attendees, including representatives from prominent organizations such as Google, Microsoft, Apple, Ericsson, and Nokia. The diverse attendance underscored the broad interest and collaborative effort in advancing TCP protocols.

\subsection{Meeting Discussions}

\subsubsection{WG Status Update}
The session began with a status update on the working group, highlighting the progress of the \href{https://datatracker.ietf.org/doc/html/draft-ietf-tcpm-accurateecn}{AccurateECN} document, which has roots tracing back to the ConEx initiative.

\subsubsection{Proportional Rate Reduction for TCP}
The discussion on \href{https://datatracker.ietf.org/doc/html/draft-ietf-tcpm-prr-rfc6937bis-13}{Proportional Rate Reduction for TCP} featured positive feedback on recent nomenclature changes. Implementations in FreeBSD and Linux were noted, with Neal Cardwell from Google confirming Linux's long-standing adoption of similar concepts.

\subsubsection{TCP ACK Rate Request Option}
The \href{https://datatracker.ietf.org/doc/html/draft-ietf-tcpm-ack-rate-request-07}{TCP ACK Rate Request Option} draft sparked a detailed dialogue on its utility for constrained devices and the need for further review before proceeding to Working Group Last Call (WGLC). The discussion emphasized the importance of addressing middlebox issues and ensuring comprehensive document reviews.

\subsubsection{TCP RST Diagnostic Payload}
The \href{https://datatracker.ietf.org/doc/html/draft-boucadair-tcpm-rst-diagnostic-payload-11}{TCP RST Diagnostic Payload} draft was debated with a focus on encoding strategies and security considerations. The potential for human-readable strings in diagnostics was discussed, with a consensus on the need for further testing and refinement.

\subsubsection{Source Buffer Management (TCP\_REPLENISH\_TIME)}
Stuart Cheshire presented the \href{https://datatracker.ietf.org/doc/html/draft-cheshire-sbm-01}{Source Buffer Management} draft, advocating for a standardized API to ensure consistent behavior across platforms. The discussion highlighted the importance of defining algorithms for uniform application performance.

\subsubsection{Behavior of close() for TCP Sockets with Non-Zero Linger Time}
Michael Tuexen led a discussion on the behavior of \texttt{close()} for TCP sockets with non-zero linger time, questioning the practical benefits of waiting periods and the implications for application-level acknowledgments.

Meeting materials and additional details can be accessed via the \href{https://www.ietf.org/proceedings/122/tcpm.html}{IETF proceedings page}.




\newpage

\section{Traffic Engineering Architecture and Signaling (TEAS)}

\subsection{Attendees Overview}
\subsubsection{Attendance}
The TEAS working group session at IETF 122 was attended by 67 participants, including representatives from prominent companies and institutions such as Telefonica Innovacion Digital, Ericsson, Nokia, Cisco Systems, Huawei, and Juniper Networks, among others.

\subsection{Meeting Discussions}
The meeting materials are available at \href{https://datatracker.ietf.org/meeting/122/session/teas}{TEAS Session Materials}.

\subsubsection{Administrivia \& WG Status}
The session commenced with an overview of the working group's status, presented by the chairs. A key point of discussion was the draft \href{https://datatracker.ietf.org/doc/html/draft-ietf-teas-actn-vn-yang}{draft-ietf-teas-actn-vn-yang}, which is in the AUTH-48 state. The chairs noted a delay in reaching one of the authors, prompting a decision to potentially reclassify the author as a contributor if no response is received shortly.

\subsubsection{WG Draft Updates}
The chairs provided updates on various working group drafts, emphasizing the need for further discussion on \href{https://datatracker.ietf.org/doc/html/draft-ietf-ns-ip-mpls}{draft-ietf-ns-ip-mpls} due to concerns about the sample size in mailing list discussions. The chairs encouraged participants to share any reservations on the mailing list.

\subsubsection{Scalability Considerations for Network Resource Partition}
Jie Dong presented the draft \href{https://datatracker.ietf.org/doc/draft-ietf-teas-nrp-scalability/07/}{draft-ietf-teas-nrp-scalability}, focusing on scalability issues in network resource partitioning. The discussion highlighted the need to clarify the expected scale of NRP instances, with considerations ranging from tens to thousands, depending on deployment scenarios. The draft aims to provide guidance on scalability properties and potential consequences of different deployment choices.

\subsubsection{Applicability of ACTN for POI Service Assurance}
Paolo Volpato discussed the draft \href{https://datatracker.ietf.org/doc/draft-poidt-teas-actn-poi-assurance/05/}{draft-poidt-teas-actn-poi-assurance}, which explores the applicability of ACTN for service assurance in points of interest (POI). The document aims to identify existing YANG models and address any gaps. The working group plans to consider the document for adoption, pending further analysis and feedback.

\subsubsection{DC Aware TE Topology Model}
Luis Contreras presented the draft \href{https://datatracker.ietf.org/doc/draft-llc-teas-dc-aware-topo-model/04}{draft-llc-teas-dc-aware-topo-model}, which proposes a model for data center-aware traffic engineering topology. The discussion centered on the model's applicability to both TE and non-TE networks and its potential integration with existing TE models. The group agreed to continue discussions offline to determine the appropriate working group for this work.

\subsubsection{OAM for Network Resource Partition NRP in SR}
Liyan Gong introduced the draft \href{https://datatracker.ietf.org/doc/draft-gong-teas-spring-nrp-oam/00/}{draft-gong-teas-spring-nrp-oam}, focusing on operations, administration, and maintenance (OAM) for network resource partitioning in segment routing. The discussion raised concerns about the use of BFD and STAMP for troubleshooting NRP selector programming, with suggestions to continue the conversation on the mailing list.

\subsubsection{In-Place Bandwidth Update for MPLS RSVP-TE LSPs}
Zafar Ali presented the draft \href{https://datatracker.ietf.org/doc/draft-alibee-teas-rsvp-inplace-lsp-bw-update/01/}{draft-alibee-teas-rsvp-inplace-lsp-bw-update}, which addresses in-place bandwidth updates for MPLS RSVP-TE LSPs. The discussion focused on the challenges of implementing in-place updates and the potential need for standardization to address inconsistencies in current implementations.

\subsubsection{Multipath Traffic Engineering}
Kireeti Kompella discussed the draft \href{https://datatracker.ietf.org/doc/draft-kompella-teas-mpte/00/}{draft-kompella-teas-mpte}, which explores multipath traffic engineering. The presentation emphasized the concept of slack in computation and its implications for traffic engineering. The group acknowledged the complexity of the topic and agreed to continue exploring the space in future sessions.



\newpage

\section{Transport in Interplanetary Protocols (TIPTOP)}

\subsection{Attendees Overview}
The TIPTOP working group session was attended by 116 participants, including representatives from prominent organizations such as Amazon, Cisco, Juniper Networks, and the University of Sheffield. The diverse attendance underscored the broad interest in developing protocols for interplanetary communication.

\subsection{Meeting Discussions}

\subsubsection{IP in Deep Space: Key Characteristics, Use Cases, and Requirements}
Presented by Wesley Eddy, this session focused on the draft \href{https://datatracker.ietf.org/doc/html/draft-many-tiptop-usecase}{draft-many-tiptop-usecase}. The discussion highlighted the need for feedback on security considerations and packet loss characterization. The dialogue emphasized accommodating both human and robotic communication needs, suggesting a nuanced approach to traffic classification.

\subsubsection{An Architecture for IP in Deep Space}
Tony Li presented the \href{https://datatracker.ietf.org/doc/html/draft-many-tiptop-ip-architecture}{draft-many-tiptop-ip-architecture}, addressing the architecture's capability to support real-time communication within celestial networks. The session explored the scope of limited domains and the challenges of multihoming, with a consensus on the need for further exploration of addressing schemes.

\subsubsection{QUIC Simulation Results and Profile}
Marc Blanchet discussed the \href{https://datatracker.ietf.org/doc/html/draft-many-tiptop-quic-profile}{draft-many-tiptop-quic-profile}, focusing on the adaptation of QUIC for space communication. The session raised questions about flow control and the necessity of simulations, with a call for collaboration with other groups to advance the research.

\subsubsection{Key Exchange Customization for TIPTOP}
Britta Hale introduced the concept of implementing MLS within QUIC, although no draft was available. The discussion centered on the importance of addressing key management in long-delay environments, with suggestions to engage with the TLS working group for further development.

\subsection{Meeting Materials}
All meeting materials, including slides and drafts, are available at \href{https://datatracker.ietf.org/meeting/122/materials.html}{IETF 122 Materials}.

\subsubsection{Wrap-up and Next Steps}
The session concluded with a discussion on the proposed contribution model via GitHub, which received mixed feedback. The group agreed on the importance of flexibility in draft development and the potential for leveraging existing IETF processes. Future work will focus on refining the drafts and engaging with related working groups to ensure comprehensive coverage of interplanetary communication challenges.



\newpage

\section{Transport Layer Security (TLS) Working Group}

\subsection{Attendees Overview}
\subsubsection{Attendance and Representation}
The TLS working group meeting was attended by 130 participants, representing a diverse array of prominent companies and institutions. Notable attendees included representatives from Broadcom, Huawei, Mozilla, Venafi, Dell Technologies, Akamai, Google, and Cloudflare, among others.

\subsection{Meeting Discussions}

\subsubsection{Administrivia}
The meeting commenced with administrative updates, including a review of the Note Well and scribe assignments. The status of various drafts was discussed, including the moderation of mailing lists and updates on the FATT status. The group also reviewed the progress of drafts such as \href{https://datatracker.ietf.org/doc/html/draft-ietf-tls-ech}{draft-ietf-tls-ech} and \href{https://datatracker.ietf.org/doc/html/draft-ietf-tls-rfc8446bis}{draft-ietf-tls-rfc8446bis}.

\subsubsection{Working Group Drafts}
\paragraph{-rfc8773bis FATT Report}
Britta Hale presented the FATT report, emphasizing the use of PSKs in TLS for quantum resilience. The discussion highlighted the need for external reviews and the potential revisions to security claims. The working group considered whether to publish the draft as is, revise it, or wait for formal analysis.

\subsubsection{Non-Working Group Drafts}
\paragraph{PAKEs}
The session on PAKEs included presentations by Chris Wood and Liang Xia on their respective drafts. Discussions focused on the integration of PAKEs into TLS, with a consensus to explore mechanisms for PAKEs in the TLS handshake, despite historical challenges.

\paragraph{Implicit ECH}
Nick Sullivan discussed the potential for implicit ECH, with considerations on making ECH universal to avoid detection. The group agreed to further explore this topic without delaying current ECH progress.

\paragraph{ECH Public Names}
Martin Thomson addressed the use of public names in ECH, considering the implications for routing and censorship. The discussion emphasized the need for secure key access and the potential for using public names to enhance privacy.

\paragraph{Anonymity Sets in ECH}
Jonathan Hoyland presented on anonymity sets in ECH, with discussions on fallback flows and the importance of maintaining DNS independence. The group agreed to continue discussions on the mailing list.

\paragraph{Identity Crisis in Attested TLS}
Muhammad Usama Sardar introduced the concept of identity crisis in attested TLS, sparking interest in the use of TLS and DevID certificates. The discussion acknowledged existing standards and the need for further exploration.

Meeting materials and additional information can be accessed via \href{https://notes.ietf.org/notes-ietf-122-tls}{meeting notes}.




\newpage

\section{Transport and Services Working Group (tsvwg)}

\subsection{Attendees}
\subsubsection{Overview}
The TSVWG session at IETF 122 in Bangkok saw participation from 88 attendees, including representatives from prominent organizations such as Ericsson, Google, Cisco, Nokia, and Apple. The diverse attendance underscored the collaborative effort in addressing transport and service challenges.

\subsection{Meeting Discussions}

\subsubsection{WG Status and Draft Updates}
The session began with an update on the working group's status and draft progress. Key drafts under IESG review include \href{https://datatracker.ietf.org/doc/html/draft-ietf-tsvwg-udp-options-dplpmtud}{draft-ietf-tsvwg-udp-options-dplpmtud} and \href{https://datatracker.ietf.org/doc/html/draft-ietf-tsvwg-multipath-dccp}{draft-ietf-tsvwg-multipath-dccp}. Discussions highlighted the need for clearer communication with other SDOs, with suggestions to modify normative language to informative recommendations. The \href{https://datatracker.ietf.org/doc/html/draft-ietf-tsvwg-careful-resume}{draft-ietf-tsvwg-careful-resume} has concluded its last call, and the group is preparing for further reviews.

\subsubsection{Transport Drafts}
Martin Duke presented on \href{https://datatracker.ietf.org/doc/html/draft-ietf-tsvwg-udp-ecn}{draft-ietf-tsvwg-udp-ecn}, discussing the implications of OS version dependencies on ECN capabilities. The session explored the potential for documenting older OS capabilities and the challenges of sending CE marks from the sender side. The presentation emphasized the importance of understanding dual-stack socket behaviors.

\subsubsection{WG Differentiated Services: L4S \& NQB}
Jason Livingood provided an update on L4S deployment, noting significant progress with Comcast deploying to millions of homes. The discussion focused on the operational guidance for L4S, with Greg White inviting contributions to address security considerations in \href{https://datatracker.ietf.org/doc/html/draft-ietf-tsvwg-l4sops}{draft-ietf-tsvwg-l4sops}. The session underscored the transformative potential of L4S in reducing latency and improving application performance.

\subsubsection{SCTP Encryption}
Magnus Westerlund's presentation on DTLS protection for SCTP explored various options for implementation, considering IPR concerns and deployment feasibility. The working group debated the merits of different options, with a consensus leaning towards further exploration of options 3 and 4. The session concluded with a call for a design team to refine these options before the next meeting.

\subsubsection{Individual Drafts and New Work}
The session also covered new work such as \href{https://datatracker.ietf.org/doc/html/draft-tahiliani-tsvwg-fq-pie}{draft-tahiliani-tsvwg-fq-pie}, which proposes modifications to PIE for queue delay calculation. Discussions on UDP fragmentation and source buffer management highlighted ongoing efforts to address fragmentation control inconsistencies and latency issues. The group expressed interest in further exploring these areas, with a focus on practical implementations and cross-platform consistency.

Meeting materials and additional details can be accessed via the \href{https://datatracker.ietf.org/wg/tsvwg/documents/}{TSVWG document tracker}.



\newpage

\section{Time Variant Routing (TVR) Working Group}

\subsection{Attendees Overview}
The TVR Working Group meeting at IETF 122 in Bangkok was attended by 58 participants, including representatives from prominent organizations such as Johns Hopkins University Applied Physics Laboratory, NASA, Juniper Networks, Huawei Technologies, and Cisco Systems. The diverse attendance underscored the broad interest and collaborative effort in advancing time variant routing technologies.

\subsection{Meeting Discussions}

\subsubsection{Introduction, Note Well, Milestones}
The session commenced with the chairs, Ed Birrane and Tony Li, outlining the agenda and emphasizing the importance of adhering to IETF's "Note Well" guidelines. The introduction set the stage for a focused discussion on the group's milestones and upcoming deliverables.

\subsubsection{TVR Requirements}
Daniel King presented the \href{https://datatracker.ietf.org/doc/html/draft-ietf-tvr-requirements-05}{draft-ietf-tvr-requirements-05}, highlighting recent refinements and the need for further feedback. The discussion centered on the scope of technology requirements and the connection between requirements and applicability. The group agreed that the requirements draft should precede the YANG model development, ensuring a solid foundation for subsequent work.

\subsubsection{TVR ALTO Exposure}
Luis Mieguel Contreras Murillo discussed the \href{https://datatracker.ietf.org/doc/html/draft-ietf-tvr-alto-exposure-01}{draft-ietf-tvr-alto-exposure-01}, focusing on implementation status and the need for a more generic draft. The conversation explored the potential renaming of the draft to better reflect its scope and the proposal for a new document addressing "off-path solutions for TVR." The group considered the integration of open-source references and the implications for future work.

\subsubsection{TVR Applicability}
Li Zhang presented the \href{https://datatracker.ietf.org/doc/html/draft-zdm-tvr-applicability-01}{draft-zdm-tvr-applicability-01}, prompting an interactive discussion on use cases and the depth of detail required. Participants debated the balance between comprehensive coverage and detailed exploration of specific cases. The dialogue emphasized the importance of aligning the applicability draft with existing YANG models and the potential for leveraging SDN controllers to enhance the document's utility.

\subsubsection{Open Mic}
The open mic session provided an opportunity for attendees to voice additional concerns and suggestions. Topics included the integration of time synchronization considerations in space applications and the criticality of control versus data planes in TVR implementations. The session concluded with a consensus on the need for continued collaboration and refinement of the working group's outputs.

Meeting materials and additional resources are available at \href{https://www.ietf.org/proceedings/122/tvr.html}{IETF 122 TVR Meeting Materials}.




\newpage

\section{Virtual Conferencing (vCon) Working Group}

\subsection{Attendees Overview}
The vCon working group meeting was attended by 27 participants, including representatives from prominent organizations such as Microsoft, Meta Platforms, Inc., and ICANN. Notable attendees included Chris Wendt from Somos, Inc., Alan Jowett from Microsoft, and Murray Kucherawy from Meta Platforms, Inc.

\subsection{Meeting Discussions}

\subsubsection{Dan Petrie - The JSON Format for vCon - Conversation Data Container}
Dan Petrie presented the latest updates on the \href{https://datatracker.ietf.org/doc/draft-ietf-vcon-vcon-container/}{draft-ietf-vcon-vcon-container}, focusing on restructuring the document and proposing a vCon core framework with extensions. The discussion included feedback on defining a core schema, separating contact center functionalities, and addressing media types. The group considered the implications of versioning and the potential for an IANA registry for extensions. The presentation highlighted the need for further discussion on use cases and the maturity of the document.

\subsubsection{Rohan Mahy - Discussions on Updates for -02}
Rohan Mahy discussed updates to the \href{https://datatracker.ietf.org/doc/html/draft-mahy-vcon-mimi-messages}{draft-mahy-vcon-mimi-messages}, which focuses on instant messaging use cases such as backup and message forwarding. The updates since version -01 were reviewed, and the potential for adopting the extension without delay was debated. The discussion emphasized the importance of a format framework for extensions and the utility of the draft in improving the extension document.

\subsubsection{Discussion about Adoption}
The group deliberated on the adoption of the \href{https://datatracker.ietf.org/doc/draft-james-privacy-primer-vcon/}{draft-james-privacy-primer-vcon}, considering its relevance to vCon and broader privacy implications. The consensus was to adopt the document, with discussions on its potential as an informational RFC. The document's utility in guiding the use of sensitive PII information was acknowledged, and a call for adoption was supported by the attendees.

\subsubsection{Wrap Up and Future Directions}
The meeting concluded with a discussion on document organization and the strategic direction of the working group. The need for practical documents addressing SCITT and vCons was highlighted, alongside considerations for combining related documents to streamline the working group's efforts. The group agreed on the importance of aligning with the charter's focus and ensuring that extensions and privacy considerations are adequately addressed.

Meeting materials and further details can be accessed at \href{https://datatracker.ietf.org/meeting/122/materials/agenda-122-vcon}{Meeting Materials}.




\newpage

\section{WebTransport (webtrans) Working Group}

\subsection{Attendees}
\subsubsection{Overview}
The WebTransport (webtrans) Working Group meeting at IETF 122 was attended by 35 participants, including representatives from prominent companies and institutions such as Google, Apple, Meta Platforms, Inc., Akamai, Microsoft, and The University of Tokyo.

\subsection{Meeting Discussions}

\subsubsection{Preliminaries}
The session began with the chairs addressing the \href{https://www.ietf.org/about/note-well/}{Note Well} and \href{https://www.rfc-editor.org/rfc/rfc7154.html}{Code of Conduct}. A tribute was paid to Bernard Aboba, highlighting his contributions to the field.

\subsubsection{W3C WebTransport Update}
Will Law provided an update on the W3C WebTransport, noting that the group is nearing completion with a candidate recommendation expected in 1-2 months. Key updates included changes to network error handling and protocol constructors. Discussions centered around issues such as CORS requirements and the potential removal of serverCertificateHashes, with significant input from the room regarding the implications of these changes.

\subsubsection{WebTransport over HTTP/2 and HTTP/3}
Victor Vasiliev and Eric Kinnear led a detailed discussion on the \href{https://datatracker.ietf.org/doc/html/draft-ietf-webtrans-http3}{draft-ietf-webtrans-http3} and \href{https://datatracker.ietf.org/doc/html/draft-ietf-webtrans-http2}{draft-ietf-webtrans-http2}. The group debated the necessity of mandatory subprotocol negotiation and flow control mechanisms. Consensus leaned towards optional implementation to facilitate broader adoption. The discussion highlighted the complexity of flow control and its impact on interoperability, with plans to test implementations before finalizing decisions.

\subsubsection{Forward and Reverse HTTP/3 over WebTransport}
Ben Schwartz introduced the concept of \href{https://datatracker.ietf.org/doc/html/draft-various-httpbis-h3-webtrans/}{draft-various-httpbis-h3-webtrans}, exploring the interface between HTTP/3 and QUIC. The group discussed potential adoption and the technical challenges of integrating stream IDs. While there was interest in the problem space, the consensus was that further exploration in a dedicated working group might be necessary.

\subsubsection{Wrap up and Summary}
The meeting concluded with a summary of the discussions and a plan for interoperability testing. The chairs thanked the attendees for their contributions and outlined the next steps, emphasizing the importance of continued collaboration to address the technical challenges identified.

Meeting materials are available at \href{https://datatracker.ietf.org/meeting/122/agenda/?show=webtrans}{IETF Agenda Link}.




\newpage

\section{WIMSE Working Group (WIMSE)}

\subsection{Attendees Overview}
\subsubsection{Prominent Attendees and Total Attendance}
The WIMSE working group meeting at IETF 122 was attended by 44 participants, including representatives from notable organizations such as Cisco, JPMorgan Chase \& Co., NSA-CCSS, UberEther, Intel, Microsoft, and Google LLC. 

\subsection{Meeting Discussions}

\subsubsection{Welcome and Chair Update}
The meeting commenced with a welcome and agenda overview by the chairs, Justin and Pieter. The slides, which include various links to the WIMSE repository, website, and drafts, are available \href{https://datatracker.ietf.org/meeting/122/materials/slides-122-wimse-workload-to-workload-protocol-00}{here}.

\subsubsection{Architecture Draft Update}
Joe Saloway presented the \href{https://datatracker.ietf.org/doc/draft-ietf-wimse-arch/}{architecture draft}, highlighting discussions on impersonation and terminology, such as the suggestion to rename "WIMSE Token" to "WIMSE artifact." The group deliberated on scalability and the distributed issuance of identities, emphasizing the need for clear definitions and alignment with other communities' terminology.

\subsubsection{Workload to Workload Draft Update}
Yaron discussed the \href{https://datatracker.ietf.org/doc/draft-ietf-wimse-s2s-protocol/03/}{workload to workload draft}, focusing on trust bundles and private key provisioning. The group identified open issues, including audience claim verification and terminology mapping across different groups.

\subsubsection{Workload Identity Practices Draft Update}
Yaroslav presented updates on the \href{https://datatracker.ietf.org/doc/draft-ietf-wimse-workload-identity-practices/}{workload identity practices draft}, discussing credential issuance and provisioning. The document aims to capture the current state and is not exhaustive. New reviewers were solicited to further refine the draft.

\subsubsection{Credential Exchange}
Arndt introduced the \href{https://datatracker.ietf.org/doc/draft-schwenkschuster-wimse-credential-exchange/}{credential exchange draft}, addressing issues like revocation and provisioning. The discussion explored the potential for standardizing patterns and the relevance of remote attestation in credential exchanges.

\subsubsection{Identity Crisis in Attested TLS}
Usama's presentation on attested TLS highlighted the need for a draft to address security assumptions and platform trust. The group encouraged further exploration and documentation of these issues.

\subsubsection{WIMSE Requirements for Confidential Computing}
Hannes discussed the intersection of WIMSE with confidential computing, emphasizing the need for isolation and trust in hardware. The conversation touched on industry trends and the scope of technologies like Intel and ARM.

\subsubsection{WIMSE Impact on Data Residency Requirements}
Ramki's presentation proposed binding workload, domain, and platform identities to address data residency requirements. The group encouraged drafting a document to elaborate on these ideas and gather feedback.

\subsubsection{Any Other Business}
The session concluded with discussions on workload identity definitions and ongoing work in related areas. Participants were encouraged to continue these conversations offline and contribute to the development of comprehensive definitions and threat models.

Meeting materials and slides are available \href{https://datatracker.ietf.org/meeting/122/materials/slides-122-wimse-wimse-architecture-00.pdf}{here}.



\newpage

\section{WebRTC Ingest Signaling Harmonization (WISH)}

\subsection{Attendees}
The WISH working group meeting at IETF 122 was attended by representatives from several prominent companies and institutions, including Cloudflare, Apple, Akamai, and Alibaba Cloud. In total, there were 9 attendees, featuring notable participants such as Nils Ohlmeier from Cloudflare, Sergio Garcia Murillo, and Sean Turner.

\subsection{Meeting Discussions}

\subsubsection{Working Group Drafts}
The session began with updates on the status of key working group drafts. Sean Turner presented the \href{https://datatracker.ietf.org/doc/html/draft-ietf-wish-whip}{draft-ietf-wish-whip}, which is currently in the AUTH48 state, indicating it is nearing publication. The discussion highlighted the importance of finalizing the WebRTC HTTP Ingest Protocol (WHIP) to streamline media ingestion processes. The chairs also discussed the \href{https://datatracker.ietf.org/doc/html/draft-ietf-wish-whep}{draft-ietf-wish-whep}, focusing on server-side strategies and potential improvements, as outlined in recent communications.

\subsubsection{Key Discussion Points}
The group deliberated on the readiness of Pull Request 31 for merging, with Lorenzo Miniero and Dan offering assistance if needed. Sergio Garcia Murillo expressed his preference to remain an author rather than an editor, emphasizing his ongoing commitment to the project. Discoverability issues were also addressed, particularly the complexities of URL schema work, with no definitive guidance from the HTTP Working Group. Dapeng Liu from Alibaba Cloud volunteered to support efforts related to server-side event issues, underscoring the collaborative spirit of the group.

\subsubsection{Future Directions}
The meeting concluded with a consensus that the working group should remain active as long as progress continues. This reflects a strategic decision to ensure ongoing development and refinement of protocols that are critical to the WebRTC ecosystem.

Meeting materials and further details can be accessed via the \href{https://notes.ietf.org/notes-ietf-122-wish}{meeting notes}.




\newpage
\end{document}